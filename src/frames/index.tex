% SPDX-License-Identifier: CC-BY-SA-4.0
% Author: Matthieu Perrin

\begingroup

%\part{Introduction}
% 
% 
%\section{Présentation du module}
% 
%\subsection{Généralités} 
%% SPDX-License-Identifier: CC-BY-SA-4.0
% Author: Matthieu Perrin
% Part: 
% Section: 
% Sub-section: 
% Frame: 

\begingroup

\begin{frame}{Généralités}

  \begin{block}{Objectifs théoriques}
    \begin{itemize}
    \item Comprendre les fondements mathématiques de l'informatique théorique
      \begin{itemize}
      \item Structures de mots et de langages
      \item Notion de non-déterminisme
      \item Frontières entre types de langages
      \end{itemize}
    \item Comprendre le fonctionnement de la partie avant des compilateurs
      \begin{itemize}
      \item Analyse lexicale : théorie des automates finis
      \item Analyse syntaxique : théorie des automates à pile 
      \end{itemize}
    \end{itemize}
  \end{block}

  \begin{block}{Objectifs pratiques}
    \begin{itemize}
    \item Créer un interpréteur en Java pour le langage d'algorithmique 
      \begin{itemize}
      \item JFlex et expressions rationnelles
      \item CUP et grammaires algébriques
      \end{itemize}
    \end{itemize}
  \end{block}
  
  \begin{block}{Prérequis}
    \begin{itemize}
    \item Mathématiques : théorie des ensembles, démonstrations
    \item Informatique : algorithmique, programmation impérative 
    \end{itemize}
  \end{block}
\end{frame}

\endgroup

%% SPDX-License-Identifier: CC-BY-SA-4.0
% Author: Matthieu Perrin
% Part: 
% Section: 
% Sub-section: 
% Frame: 

\begingroup

\begin{frame}{Bibliographie}
  \begin{block}{Livres}
    \footnotesize
    \begin{itemize}
    \item K. D. Cooper \& L. Torczon. \textit{\textbf{Engineering a compiler} -- $2^\text{nd}$ Ed.}, Morgan Kaufmann 2013
    \item A. Aho, R. Sethi \& J. Ullman. \textit{Compilateurs : principes, techniques et outils}, InterEditions, 1991. Voir aussi la seconde édition en anglais.
    \item P. Wolper. \textit{\textbf{Introduction à la calculabilité} -- $3^\text{nd}$ Ed.}, Dunod 2006
    \item O. Carton. \textit{Langages Formels -- Calculabilité et complexité}, Vuibert 2008
    \item J.-M. Autebert, \textit{Théorie des langages et des automates}, Masson 1994
    \item J.-M. Autebert, \textit{Langages algébriques}, Masson 1989
    \item P. Linz, \textit{Formal Languages and Automata}, Jones and Barnett Publishers, 2006.
    \item P. Séébold, \textit{Fondamentaux de la théorie des automates}, Ellipses, 2020.
    \end{itemize}
  \end{block}
  
  \begin{block}{Outil pédagogique}
    \footnotesize
    \begin{itemize}
    \item S. H. Rodger. \textit{JFLAP}. \url{http://www.jflap.org/} 
    \end{itemize}
  \end{block}
\end{frame}
\endgroup

%% SPDX-License-Identifier: CC-BY-SA-4.0
% Author: Matthieu Perrin
% Part: 
% Section: 
% Sub-section: 
% Frame: 

\begingroup

%\begin{frame}{Test}
% 
%  \alt<2>{vrai}{faux}
%  
%  \alert<2>{alert}   ~~~~  \example<2>{example}   ~~~~~~  \structure<2>{structure}   ~~~~~   \fade<2>{fade}           
% 
%  \alertb<2>{alertb} ~~~~  \exampleb<2>{exampleb} ~~~~~~  \structureb<2>{structureb} ~~~~~   \fadeb<2>{fadeb}         
% 
%  \Alert<2>{Alert}   ~~~~  \Example<2>{Example}   ~~~~~~  \Structure<2>{Structure}   ~~~~~   \Fade<2>{Fade}           
% 
%  \Alertb<2>{Alertb} ~~~~  \Exampleb<2>{Exampleb} ~~~~~~  \Structureb<2>{Structureb} ~~~~~   \Fadeb<2>{Fadeb}         
% 
% 
%$$  \alert<2>{alert}   ~~~~  \example<2>{example}   ~~~~~~  \structure<2>{structure}   ~~~~~   \fade<2>{fade}           $$
%$$  \alertb<2>{alertb} ~~~~  \exampleb<2>{exampleb} ~~~~~~  \structureb<2>{structureb} ~~~~~   \fadeb<2>{fadeb}         $$
%$$  \Alert<2>{Alert}   ~~~~  \Example<2>{Example}   ~~~~~~  \Structure<2>{Structure}   ~~~~~   \Fade<2>{Fade}           $$
%$$  \Alertb<2>{Alertb} ~~~~  \Exampleb<2>{Exampleb} ~~~~~~  \Structureb<2>{Structureb} ~~~~~   \Fadeb<2>{Fadeb}         $$
% 
% 
%  \alt<1>{vrai}{faux}
%  
%\end{frame}


\begin{frame}{Objet d'étude}
 
  \begin{block}{Théorie des langages formels}
    Se place à l'intersection de plusieurs disciplines
      \begin{description}[Mathématiques :]
      \item[Linguistique :]  Modélisation abstraite des \alert{langues naturelles}
      \item[Mathématiques :] S'ancre dans la \alert{théorie des ensembles}
      \item[Logique :]       Définit un cadre d'étude des \alert{prédicats logiques}
      \item[Informatique :]  Produit des \alert{algorithmes sur les chaînes de caractères}
      \end{description}
  \end{block}
 
  \vspace{2mm}
  \begin{block}{Définitions clés}
    \begin{description}
    \item[Alphabet :] \alert{Ensemble fini non vide} de \structure{symboles}\hspace\fill        \example{Exemple : $\{a, b, ..., z\}$}
    \item[Mot :] \alert{Suite finie de symboles} de l'alphabet\hspace\fill        \example{Exemple : ``langage''}
    \item[Langage :] \alert{Ensemble de mots} \hspace\fill \example{Exemple : $\{\text{``langage''}, \text{``langue''}\}$}
    \end{description}
  \end{block}
  
\end{frame}

\endgroup

% 
%\subsection{Exemples de langages}
% 
%% SPDX-License-Identifier: CC-BY-SA-4.0
% Author: Matthieu Perrin
% Part: 
% Section: 
% Sub-section: 
% Frame: 

\begingroup

\begin{frame}{Exemple : un langage naturel}

  \begin{block}{Lexique français}
    \begin{description}
    \item[Langage :] L'ensemble des mots du \structure{dictionnaire}
    \item[Alphabet :] $\Sigma_{LF} \eqdef \{\text{`a'}, ...,  \text{`z'}, \text{`à'}, ..., \text{`ü'}, \text{`ç'}, \text{`-'}, \text{`\textquotesingle'}\}$
    \item[Mots :] $\{\text{``a''}, \text{``à''}, \text{``abaissement''}, ..., \text{``zut''}, \text{``zygomatique''}, \text{``zygote''}\}$
    \end{description}
  \end{block}
  
  \pause
  \begin{block}{Syntaxe française}
    \begin{description}
    \item[Langage :] L'ensemble des phrases/textes \structure{grammaticalement} corrects
    \item[Alphabet :] $\Sigma_{SF} \eqdef \Sigma_{LF} \cup \{\text{`A'}, ..., \text{`Z'}, \text{`~'}, \text{`.'}, \text{`?'}, ...\}$
    \item[Mots :] $\{\text{``A beau mentir qui vient de loin''}, \text{``À bon chat, bon rat''}, ...\}$
    \end{description}
  \end{block}
  
  \pause
  \begin{alertblock}{Sémantique française}
    \begin{itemize}
    \item \alert{Attention :} `a' (lettre) $\neq$ ``a'' (mot) $\neq$ a (conjugaison de avoir)
    \item On s'intéressera peu à la sémantique dans ce cours
    \item On n'écrira généralement pas les guillemets
    \end{itemize}
  \end{alertblock}

\end{frame}

\endgroup

%% SPDX-License-Identifier: CC-BY-SA-4.0
% Author: Matthieu Perrin
% Part: 
% Section: 
% Sub-section: 
% Frame: 

\begingroup

\begin{frame}{Exemple : un langage de programmation}

  C++ a également un lexique et une syntaxe.
  \begin{block}{Lexique du C++}
    \begin{description}
    \item[Mots-clés] \lstinline{if}, \lstinline{while}, \lstinline{class}, \lstinline{typedef}, ...
    \item[Identifiants :] \lstinline{x}, \lstinline{age}, \lstinline{number\_of\_participants}
    \item[Constantes :] \lstinline{5}, \lstinline{-3.14}, \lstinline{0x3A88C6}, \lstinline{'\\t'}, \lstinline{"Hello world"}, ...
    \item[Symboles :] \lstinline{(}, \lstinline{)}, \lstinline{\;}, \lstinline{+}, ...
    \end{description}
    \alert{Remarque :} le lexique du C++ est infini... mais ``simple'' $\rightarrow$ \structure{langage rationnel}
  \end{block}
  
  \begin{block}{Syntaxe du C++}
    \begin{description}
    \item[Expressions :] \lstinline{(-b + std::sqrt(b * b - 4 * a * c)) / (2 * a)}
    \item[Structures de contrôle :] \lstinline{if(condition()) \{p\_true()\;\} else \{p\_false()\;\}}
    \item[Structures de données :] \lstinline{struct pair \{int x\; int y;\}\;}
    \end{description}
    \alert{Remarque :} syntaxe définie récursivement $\rightarrow$ \structure{langage algébrique}
  \end{block}

\end{frame}

\endgroup

%% SPDX-License-Identifier: CC-BY-SA-4.0
% Author: Matthieu Perrin
% Part: 
% Section: 
% Sub-section: 
% Frame: 

\begingroup

\begin{frame}{Architecture d'un compilateur}
  
  \tf[top]{
    \begin{tikzpicture}[anchor=mid]
      \draw[rounded corners, fill=example!15] (0.7,0.4) rectangle (3.1,4.0);
      \draw[rounded corners, fill=example!15] (3.6,0.4) rectangle (5.4,4.0);
      \draw[rounded corners, fill=example!15] (5.9,0.4) rectangle (8.3,4.0);
      \draw (1.9,3.75) node{Front end};
      \draw (4.5,3.75) node{Optimiseur};
      \draw (7.1,3.75) node{Back end};

      \footnotesize
      \draw[rounded corners, fill=structure!20] (1.0,2) +(-.2,-1.5) rectangle +(.2,1.5) +(0,0) node[rotate=90]{\footnotesize Analyse lexicale};
      \draw[rounded corners, fill=structure!20] (1.6,2) +(-.2,-1.5) rectangle +(.2,1.5) +(0,0) node[rotate=90]{\footnotesize Analyse syntaxique};
      \draw[rounded corners, fill=structure!00] (2.2,2) +(-.2,-1.5) rectangle +(.2,1.5) +(0,0) node[rotate=90]{\footnotesize Analyse contextuelle};
      \draw[rounded corners, fill=structure!00] (2.8,2) +(-.2,-1.5) rectangle +(.2,1.5) +(0,0) node[rotate=90]{\footnotesize Élaboration};
      \draw[rounded corners, fill=structure!00] (3.9,2) +(-.2,-1.5) rectangle +(.2,1.5) +(0,0) node[rotate=90]{\footnotesize Optimisation $1$};
      \draw[rounded corners, fill=structure!00] (5.1,2) +(-.2,-1.5) rectangle +(.2,1.5) +(0,0) node[rotate=90]{\footnotesize Optimisation $n$};
      \draw[rounded corners, fill=structure!00] (6.2,2) +(-.2,-1.5) rectangle +(.2,1.5) +(0,0) node[rotate=90]{\footnotesize Sélection d'instructions};
      \draw[rounded corners, fill=structure!00] (6.8,2) +(-.2,-1.5) rectangle +(.2,1.5) +(0,0) node[rotate=90]{\footnotesize Ordonnancement};
      \draw[rounded corners, fill=structure!00] (7.4,2) +(-.2,-1.5) rectangle +(.2,1.5) +(0,0) node[rotate=90]{\footnotesize Allocation de registres};
      \draw[rounded corners, fill=structure!00] (8.0,2) +(-.2,-1.5) rectangle +(.2,1.5) +(0,0) node[rotate=90]{\footnotesize Génération};
      \draw[rounded corners, fill=structure!00] (4.5,2) +(0,.1) node{...};
      
      \draw[-latex] (-.2,2) -- (.8,2);
      \draw[-latex] (1.2,2) -- (1.4,2);
      \draw[-latex] (1.8,2) -- (2.0,2);
      \draw[-latex] (2.4,2) -- (2.6,2);
      \draw[-latex] (3.0,2) -- (3.7,2);
      \draw[-latex] (4.1,2) -- (4.9,2);
      \draw[-latex] (5.3,2) -- (6.0,2);
      \draw[-latex] (6.4,2) -- (6.6,2);
      \draw[-latex] (6.9,2) -- (7.2,2);
      \draw[-latex] (7.6,2) -- (7.8,2);
      \draw[-latex] (8.2,2) -- (9.2,2);
      
      \draw (.2,2.15) node {\scriptsize Langage};
      \draw (.2,1.85) node {\scriptsize source};
      
      \draw (8.8,2.15) node {\scriptsize Langage};
      \draw (8.8,1.85) node {\scriptsize cible};
    \end{tikzpicture}
  }

  \tfBlock[bottom]{Objectifs du cours}{
    \begin{itemize}
    \item \vspace{-2mm} Comprendre comment un compilateur ``lit'' le code source
    \item Concevoir un interpréteur pour le langage algorithmique
    \end{itemize}
  }

  \tf[y=-17mm]{
    \begin{tikzpicture}
      \node                                                                (start)   at (0,   3.0) {};
      \node[smBox, minimum width=2cm, minimum height=1.1cm]                (lexer)   at (2.5, 3.0) {Analyseur\\lexical\vspace{2mm}\\\textit{Lexer}};
      \node[smBox, minimum width=2cm, minimum height=1.1cm]                (parser)  at (7,   3.0) {Analyseur\\syntaxique\vspace{2mm}\\\textit{Parser}};
      \node                                                                (end)     at (10,  3.0) {};

      \scriptsize
      \path[-latex] (start) edge
      node[above,align=center]{Langage\\ source}
      node[below, example]{$15+2 * x$}
      (lexer);

      \path[-latex] (lexer) edge
      node[above,align=center]{Chaîne de tokens}
      node[below, example]{$\textsc{int}~\textsc{plus}~\textsc{int}~\textsc{times}~\textsc{var}$}
      (parser);

      \path[-latex] (parser) edge
      node[above,align=center]{Arbre de la syntaxe\\abstraite (AST)}
      node[below, example]{\begin{tikzpicture}[example]
          \node (a) at (2.0,1.8) {$+$};
          \node (b) at (1.5,1.4) {$15$}; \path (a) edge (b); 
          \node (c) at (2.5,1.4) {$*$};  \path (a) edge (c); 
          \node (d) at (2.0,1.0) {$2$};  \path (c) edge (d); 
          \node (e) at (3.0,1.0) {$x$};  \path (c) edge (e); 
      \end{tikzpicture}}
      (end);
    \end{tikzpicture}
  }
    
\end{frame}

\endgroup

%% SPDX-License-Identifier: CC-BY-SA-4.0
% Author: Matthieu Perrin
% Part: 
% Section: 
% Sub-section: 
% Frame: 

\begingroup

\begin{frame}{Exemple : langages sur les nombres}
  
  \begin{block}{Représentation décimale des nombres}
    Soit $\Sigma = \{\text{`0'}, \text{`1'}, \text{`2'}, ..., \text{`9'}\}$ l'ensemble des chiffres.
    
    Pour un entier $n \in \mathbb{N}$, on note \structure{$(n)_{10}$} sa représentation décimale. 
    
    \begin{description}
    \item[\alert{Attention :}] `3' (chiffre) $\neq$ ``3'' (représentation décimale) $\neq$ 3 (nombre)
    \end{description}
    
    \begin{itemize}
    \item $\mathcal{L}_{\mathbb{N}} = \{(n)_{10} \mid n \in \mathbb{N}\} = \{\text{``0''},\text{``1''},\text{``2''},\text{``3''}, \text{``4''}, \text{``5''}, ...\}$
    \item<2-> $\mathcal{L}_{\mathbb{P}} = \{(n)_{10} \mid n \in \mathbb{N} \land n \text{ est premier}\} = \{\text{``2''},\text{``3''},\text{``5''}, \text{``7''}, \text{``11''}, ...\}$
    \end{itemize}
  \end{block}
  
  \vspace{3mm} 
  
  \begin{exampleblock}{Exercice}
    \begin{description}
    \item[Question :] $\text{``1547''} \in \mathcal{L}_{\mathbb{N}}$ ?
    \item<2->[Question :] $\text{``1547''} \in \mathcal{L}_{\mathbb{P}}$ ?
    \item<3->[Indication :] $1547 = 7 \times 13 \times 17$ 
    \end{description}
  \end{exampleblock}
  
  \uncover<3->{
    \centering
    \alert{Pour savoir si $(n)_{10}\in \mathcal{L}_{\mathbb{P}}$, il faut et il suffit de savoir si $n$ est premier }
  }  

\end{frame}

\endgroup

%% SPDX-License-Identifier: CC-BY-SA-4.0
% Author: Matthieu Perrin
% Part: 
% Section: 
% Sub-section: 
% Frame: 

\begingroup

\begin{frame}{Problème binaire et problème de décision}
  
  Un \structure{problème binaire} est une question dont la question est \alert{oui} ou \alert{non} en fonction de ses \structure{entrées}.
  
  \begin{exampleblock}{Exemples}
    \vspace{-2mm}
    \begin{itemize}
    \item \example{Primalité :} un entier est-il premier ?  
    \item \example{Classification :} une image donnée contient-elle un chat ? 
    \item \example{Base de données :} une requête donnée retourne-t-elle des résultats ?  
    \item \example{Sudoku :} une grille donnée peut-elle être remplie ? 
    \item \example{SAT :} une formule en forme normale conjonctive est-elle satisfiable ? 
    \item \example{Preuve :} étant donné un énoncé de théorème, peut-on le démontrer ? 
    \item \example{Décision :} étant donnés un mot $u$ et un langage $L$, a-t-on $u\in L$ ? 
    \end{itemize}
  \end{exampleblock}
  \pause
  \begin{alertblock}{Observations}
    \vspace{-2mm}
    \begin{itemize}
    \item En informatique, les entrées sont toujours des suites de 0 et de 1
    \item \alert{Les langages sont une métaphore des problèmes binaires}
    \item On peut généraliser aux problèmes qui ne sont pas binaires
    \end{itemize}
  \end{alertblock}
  
\end{frame}

\endgroup

%% SPDX-License-Identifier: CC-BY-SA-4.0
% Author: Matthieu Perrin
% Part: 
% Section: 
% Sub-section: 
% Frame: 

\begingroup


\begin{frame}{Exemple : représentation d'un nombre réel}
  \begin{block}{Langage représentant le calcul des décimales de $\pi$}
    On étend la notation \structure{$(n)_{10}$} aux nombres $n$ décimaux. 
    \begin{itemize}
    \item Pour $x\in \mathbb{R}$, on note $\alert{\mathcal{L}_{x}}$ le langage des troncatures de $x$
      $$\alert{\mathcal{L}_{x} \eqdef \left\{\left(\frac{\left\lfloor 10^n x \right\rfloor}{10^n} \right)_{10} \Big| n \in \mathbb{N}\right\}}$$
    \item Par exemple, $\example{\mathcal{L}_{\pi} = \left\{\text{``3''}, \text{``3.1''}, \text{``3.14''}, \text{``3.141''}, \text{``3.1415''}, ...   \right\}}$
    \end{itemize}
  \end{block}
  \pause
  \vspace{-2mm}
  \begin{alertblock}{Remarques}
    \vspace{-1mm}
    \begin{itemize}
    \item L'ensemble des langages est indénombrable
      \begin{itemize}
      \item Au moins autant de langages $\alert{\mathcal{L}_{x}}$ que de réel $\alert{x}$
      \end{itemize}
    \item Tout langage est dénombrable
      \begin{itemize}
      \item Par exemple, le langage des mots sur $\{1, 2\}$ est inclus dans $\alert{\mathcal{L}_{\mathbb{N}}}$
      \end{itemize}
    \item Aucun langage ne peut exprimer tous les langages 
      \begin{itemize}
      \item Même les langages de programmation $\rightarrow$ \alert{problèmes indécidables}
      \item Même le langage des mathématiques ~$\rightarrow$ \alert{problèmes indéfinissables}
      \end{itemize}
    \end{itemize}
  \end{alertblock}
\end{frame}

\endgroup

%% SPDX-License-Identifier: CC-BY-SA-4.0
% Author: Matthieu Perrin
% Part: 
% Section: 
% Sub-section: 
% Frame: 

\begingroup


\begin{frame}{Hiérarchie des langages}

  \tfBlock[top]{Formalismes de description d'un langage}{
    \begin{itemize}
    \item Comment représenter un langage \structure{infini} de façon \structure{finie} ?
    \item Il doit avoir une \alert{structure} interne
    \item La ``\structure{complexité}'' de sa structure entraine la ``\structure{difficulté}'' à le décider
    \end{itemize}
  }

  \tf[y=-17mm]{\footnotesize
    \begin{tikzpicture}
      \draw[structure, fill=structure!20] (12mm,12mm) ellipse(40mm and 21mm); \draw (12mm,30mm) node{définissable};
      \draw[structure, fill=structure!10] (10mm,10mm) ellipse(35mm and 18mm); \draw (10mm,25mm) node{récursivement énumérable};
      \draw[structure, fill=structure!20] (08mm,08mm) ellipse(30mm and 15mm); \draw (08mm,20mm) node{décidable};
      \draw[structure, fill=structure!10] (06mm,06mm) ellipse(25mm and 12mm); \draw (06mm,15mm) node{contextuel};
      \draw[structure, fill=structure!20] (04mm,04mm) ellipse(20mm and 09mm); \draw (04mm,10mm) node{algébrique};
      \draw[structure, fill=structure!10] (02mm,02mm) ellipse(15mm and 06mm); \draw (02mm,05mm) node{rationnel};
      \draw[structure, fill=structure!20] (00mm,00mm) ellipse(10mm and 03mm); \draw (00mm,00mm) node{fini};
    \end{tikzpicture}
  }
  
\end{frame}


\endgroup

% 
%\subsection{Objectifs du cours}
%% SPDX-License-Identifier: CC-BY-SA-4.0
% Author: Matthieu Perrin
% Part: 
% Section: 
% Sub-section: 
% Frame: 

\begingroup

\begin{frame}{Résumé}
  \begin{block}{Objectifs pratiques}
    \begin{itemize}
    \item Créer un compilateur pour un DSL 
      \begin{itemize}
      \item Flex et expressions rationnelles
      \item Bison et grammaires algébriques
      \end{itemize}
    \end{itemize}
  \end{block}
 
  \begin{block}{Objectifs théoriques}
    \begin{itemize}
    \item Comprendre le fonctionnement des compilateurs
      \begin{itemize}
      \item Analyse lexicale : théorie des automates finis
      \item Analyse syntaxique : théorie des automates à pile 
      \end{itemize}
    \item Comprendre les bases de la calculabilité
      \begin{itemize}
      \item Lemmes de l'étoile et de pompage 
      \item Langages contextuels et généraux
      \item Hiérarchie de Chomsky
      \end{itemize}
    \end{itemize}
  \end{block}
\end{frame}


\endgroup

% 
%\section{Rappels logiques et mathématiques}
% 
%\subsection{Logique du premier ordre}
%% SPDX-License-Identifier: CC-BY-SA-4.0
% Author: Matthieu Perrin
% Part: 
% Section: 
% Sub-section: 
% Frame: 

\begingroup

\begin{frame}{Calcul des prédicats}
  \begin{block}{Définition -- prédicat}
    Un \structure{prédicat} est une \alert{formule logique paramétrée par un/des paramètre(s)} 
    \begin{itemize}
    \item En français, groupe verbal à l'infinitif 
    \item Les paramètres sont des variables 
    \item Instanciation possible des paramètres par des \alert{arguments}
    \item La \alert{valeur de vérité} dépend des arguments
    \end{itemize}
  \end{block}
  \begin{exampleblock}{Exemples}
    \begin{description}
    \item [Être bleu] $ \begin{array}[t]{lclcl}
      \mathit{Bleu}(x) &\eqdef& x \text{ est bleu}&&\\
      \mathit{Bleu}(\text{le ciel}) &\equiv& \text{le ciel est bleu} &\equiv& \text{vrai} \\
      \mathit{Bleu}(\text{le tableau}) &\equiv& \text{le tableau est bleu} &\equiv& \text{faux}
    \end{array}$
    \item [Être inférieur à] $\begin{array}[t]{lclcl}
      \mathit{Inf}(x, y)  &\eqdef& x < y &&\\
      \mathit{Inf}(3, 7)  &\equiv& 3 < 7  &\equiv& \text{vrai} \\
      \mathit{Inf}(12, 7) &\equiv& 12 < 7 &\equiv& \text{faux}
    \end{array}$
    \end{description}
  \end{exampleblock}  
  
\end{frame}


\endgroup

%% SPDX-License-Identifier: CC-BY-SA-4.0
% Author: Matthieu Perrin
% Part: 
% Section: 
% Sub-section: 
% Frame: 

\begingroup

\begin{frame}{Combinaison logique de prédicats}
  \begin{block}{Opérateurs}
    \begin{description}
    \item[Négation : ] \alert{$\lnot f$} (\structure{non $f$}) : vraie si $f$ est fausse, et fausse sinon
    \item[Conjonction : ] \alert{$f \land g$} (\structure{$f$ et $g$}) : vraie si $f$ et $g$ sont toutes les deux vraies
    \item[Disjonction : ] \alert{$f \lor g$} (\structure{$f$ ou $g$}) : vraie si au moins l'une des deux est vraie 
    \item[Implication : ] \alert{$f \Rightarrow g$} (\structure{$f$ implique $g$}) : vraie si $f$ est fausse ou $g$ est vraie 
    \item[Équivalence : ] $\alert{f \Leftrightarrow g} \eqdef (f \Rightarrow g) \land (g \Rightarrow f)$ : \structure{$f$ équivalent à $g$}
    \end{description}
  \end{block}
  \begin{block}{Quantificateurs}
    \begin{description}
    \item[Universel : ] \alert{$\forall x : f(x)$} : \structure{pour tout $x$, $f(x)$}
    \item[Existentiel : ] \alert{$\exists x : f(x)$} : \structure{il existe $x$ tel que $f(x)$}
    \item[Unicité conditionnelle: ] $\alert{\unique x : f(x)} \eqdef \forall x \forall y\, f(x) \land f(y) \Rightarrow x=y$\\
      \structure{S'il existe $x$ tel que $f(x)$, $x$ est unique}
    \item[Existence-unicité : ] $\alert{\existsunique x : f(x)} \eqdef \exists x : f(x) \land \unique x : f(x)$\\
   %   $\hspace{-2mm}\begin{array}[t]{ccl}
   %   \alert{\existsunique x : f(x)} &\eqdef& \exists x : (f(x) \land \forall y\, f(y) \Rightarrow y=x)\\
   %   &\equiv& (\exists x : f(x)) \land (\unique x : f(x)) \\
   % \end{array}$\\
      \structure{il existe un unique $x$ tel que $f(x)$}
    \end{description}
  \end{block}
\end{frame}


\endgroup

% 
%\subsection{Notion d'ensembles}
%% SPDX-License-Identifier: CC-BY-SA-4.0
% Author: Matthieu Perrin
% Part: 
% Section: 
% Sub-section: 
% Frame: 

\begingroup

\begin{frame}{Ensembles}
  \begin{block}{Définition -- ensemble}
    \begin{itemize}
    \item Un \structure{ensemble} est une collection non-ordonnée d'objets uniques.
    \item Chaque objet d'un ensemble est appelé un \structure{élément} de cet ensemble.
    \item Soit $E$ un ensemble, un élément $a$ de $E$ \structure{appartient à} $E$, noté \structure{$a \in E$}
    \end{itemize}
  \end{block}
  \begin{exampleblock}{Exemples}
    \begin{itemize}
    \item \example{$\mathbb{N} = \{0, 1, 2, 3, \dots\}$} : l'ensemble des entiers naturels
    \item \example{$3\in \mathbb{N}$} est vrai, $\pi \in \mathbb{N}$ est faux, noté \example{$\pi \notin \mathbb{N}$}
    \item \example{$\forall x, x\notin\emptyset$} : $\emptyset$ est appelé l'\structure{ensemble vide}
    \end{itemize}
  \end{exampleblock}
  \begin{alertblock}{Attention}
    Il y a des règles sur quelles collections sont ou non des ensembles !
    \begin{itemize}
    \item Ces règles sont appelées \alert{axiomes} (par exemple, ZFC)
    \item On admet l'existence des ensembles usuels ($\mathbb{N}$, $\mathbb{R}$, fonctions, etc)
    \item L'étude des axiomes n'est ni l'objet de ce cours, ni un prérequis
    \end{itemize}
  \end{alertblock}
\end{frame}
\endgroup

%% SPDX-License-Identifier: CC-BY-SA-4.0
% Author: Matthieu Perrin
% Part: 
% Section: 
% Sub-section: 
% Frame: 

\begingroup

\begin{frame}{Sous-ensembles}
  \begin{block}{Définition -- inclusion}
    Soient $A$ et $B$ deux ensembles.
    \begin{itemize}
    \item $A$ est \structure{inclus} dans $B$, noté \structure{$A \subseteq B$}, si tout élément de $A$ appartient à $B$
      $$\alert{\forall x, x \in A \Rightarrow x\in B}  \hspace{1cm}\text{noté} \hspace{1cm} \alert{\forall x \in A, x\in B}$$
    \item $A$ est \structure{strictement inclus} dans $B$, noté \structure{$A \varsubsetneq B$}, si \alert{$A \subseteq B$ et $A \neq B$}
    \item Si $A\subseteq B$, on dit que A est un \structure{sous-ensemble}, ou une \structure{partie} de $B$
    \item On note \structure{$\mathscr{P}(B)$} l'ensemble des parties de $B$ :
      $$\forall A\forall B, \alert{A \in \mathscr{P}(B) \Leftrightarrow A\subseteq B}$$
    \end{itemize}
  \end{block}

  \vspace{-3mm}
  \begin{exampleblock}{Exemple}
    \begin{itemize}
    \item $\mathscr{P}(\{1, 2\}) = \{ \emptyset, \{1\}, \{2\}, \{1, 2\}\}$
    \item $\emptyset \subseteq \{1\}$ et $\{1\} \subseteq \{1, 2\}$
    \item $\{1\} \nsubseteq \{2\}$ et $\{2\} \nsubseteq \{1\}$
    \end{itemize}
  \end{exampleblock}
\end{frame}

\endgroup

%% SPDX-License-Identifier: CC-BY-SA-4.0
% Author: Matthieu Perrin
% Part: 
% Section: 
% Sub-section: 
% Frame: 

\begingroup



\begin{frame}{Définition par compréhension}
  \begin{block}{Définition -- sous-ensemble défini en intension}
    Soient $A$ un ensemble et $P$ un prédicat.
    Il existe un ensemble \structure{$\{x \in A \,|\, P(x)\}$} qui contient les éléments de $A$ qui vérifient $P$

    \vspace{-2mm}
    $$\forall y, \left(\alert{y\in \{x \in A \,|\, P(x)\}}\right) \alert{\Leftrightarrow} \left(\alert{y\in A \land P(y)}\right)$$
  \end{block}
  %  \vspace{-3mm}
  \begin{exampleblock}{Exemple -- prédicat ``être pair''}
    \begin{itemize}
    \item $\mathit{estPair}(x) \eqdef \exists k\in \mathbb{Z} : x = 2\times k$  \hspace{\fill} $\mathit{pair} \eqdef \{n\in \mathbb{Z} \,|\, \mathit{estPair}(n)\}$
    \item \example{$\mathit{estPair}(6)$} car \example{$3\in \mathbb{Z} \land 6 = 2\times 3$} \hspace{\fill}  \example{$6 \in \mathit{pair}$} car \example{$6 \in \mathbb{Z}$} et \example{$\mathit{estPair}(6)$}
    \item \example{$5 \notin \mathit{pair}$} car \example{$\lnot \mathit{estPair}(5)$ \hspace{\fill} \example{$\pi \notin \mathit{pair}$} car \example{$\pi \notin \mathbb{Z}$}}
    \end{itemize}
  \end{exampleblock}

  \begin{block}{Cas particulier -- sous-ensemble défini par extension}
    Soient $A$ un ensemble et $a_1, a_2, ..., a_n \in A$. On note :
    
    \vspace{-2mm}
    $$\alert{\{a_1, a_2, ..., a_n\} \eqdef \{x\in A | x=a_1 \lor x=a_2 \lor ... \lor x=a_n\}}$$
  \end{block}

\end{frame}

\endgroup

%% SPDX-License-Identifier: CC-BY-SA-4.0
% Author: Matthieu Perrin
% Part: 
% Section: 
% Sub-section: 
% Frame: 

\begingroup


\begin{frame}{Extensionalité}
  \begin{block}{Définition -- égalité entre ensembles}
    Deux ensembles sont \structure{égaux} s'ils contiennent les mêmes éléments : 
    $$\alert{\forall A \forall B, (A=B) \Leftrightarrow (\forall x, x \in A \Leftrightarrow x\in B)}$$
  \end{block}

  \begin{exampleblock}{\structure{Preuve par double inclusion}}
    Pour montrer que \alert{$A=B$}, il suffit de montrer \alert{$A \subseteq B$ et $B\subseteq A$}

    \pause
    \begin{itemize}
    \item \example{Exemple :} montrons que \example{$\{1, 2, 3\} = \{3, 3, 1, 2\}$}
      \begin{enumerate}
      \item \example{Mq. $\{1, 2, 3\} \subseteq \{3, 3, 1, 2\}$}, c'est-à-dire \structure{$\forall x\in \{1, 2, 3\}, x \in \{3, 3, 1, 2\}$} \\
        Soit $x \in \{1, 2, 3\}$. %Mq. $x \in \{3, 3, 1, 2\}$ \\%(soit $x$, supposons $x\in \{1, 2, 3\}$)\\
        Par définition de $\{1, 2, 3\}$, $x=1 \lor x=2 \lor x=3$\\
        Il y a 3 cas possibles : 
        \begin{enumerate}
        \item Si $x=1$, alors $x=3 \lor x=3 \lor x=1 \lor x=2$, donc $x\in \{3, 3, 1, 2\}$
        \item Si $x=2$, alors $x=3 \lor x=3 \lor x=1 \lor x=2$, donc $x\in \{3, 3, 1, 2\}$
        \item Si $x=3$, alors $x=3 \lor x=3 \lor x=1 \lor x=2$, donc $x\in \{3, 3, 1, 2\}$
        \end{enumerate}
        Dans tous les cas, on a bien $x\in \{3, 3, 1, 2\}$
      \item \example{Mq. $\{3, 3, 1, 2\} \subseteq \{1, 2, 3\}$} : preuve similaire
      \end{enumerate}
      Donc, par double inclusion, on a bien $\{1, 2, 3\} = \{3, 3, 1, 2\}$.
    \end{itemize}
  \end{exampleblock}
\end{frame}
\endgroup

%% SPDX-License-Identifier: CC-BY-SA-4.0
% Author: Matthieu Perrin
% Part: 
% Section: 
% Sub-section: 
% Frame: 

\begingroup


\begin{frame}{Opérations ensemblistes (1)}
  Soit $E$ un ensemble. Soient $A, B, C \in \mathscr{P}(E)$
  \begin{block}{Définition -- union et intersection}
    \begin{description}
    \item[Union :] $A \cup B \eqdef \{x \in E \,|\, x \in A \lor x \in B\}$
    \item[Intersection :] $A \cap B \eqdef \{x \in E \,|\, x \in A \land x \in B\}$
    \end{description}
  \end{block}
  \begin{block}{Propriétés}
    \begin{description}
    \item[Associativité :] \alert{$(A \cup B) \cup C = A \cup (B \cup C)$} et \alert{$(A \cap B) \cap C = A \cap (B \cap C)$}
    \item[Commutativité :] \alert{$A \cup B = B \cup A$} et \alert{$A \cap B = B \cap A$}
    \item[Idempotence :] \alert{$A \cup A = A = A \cap A$}
    \item[Neutralité :] \alert{$A \cup \emptyset = A = A \cap E$}
    \item[Absorption :] \alert{$A \cup E = E$} et \alert{$A \cap \emptyset = \emptyset$}
    \item[Distributivité :] \alert{$A \cup (B \cap C) = (A \cup B) \cap (A \cup C)$} et \alert{$A \cap (B \cup C) = (A \cap B) \cup (A \cap C)$}
    \end{description}
  \end{block}
\end{frame}
\endgroup

%% SPDX-License-Identifier: CC-BY-SA-4.0
% Author: Matthieu Perrin
% Part: 
% Section: 
% Sub-section: 
% Frame: 

\begingroup

\begin{frame}{Opérations ensemblistes (2)}

  Soit $E$ un ensemble. Soient $A, B \in \mathscr{P}(E)$
  \begin{block}{Définition -- différence ensembliste et complément}
    \begin{description}
    \item[Différence :] $A \setminus B \eqdef \{x \in A \,|\, x \notin B\}$
    \item[Complément :] $\overline{A} \eqdef E \setminus A$
    \end{description}
  \end{block}
  \begin{block}{Propriétés}
    \begin{description}
    \item[Involution :] \alert{$\overline{\overline{A}} = A$}
    \item[Lois de De Morgan :] \alert{$\overline{(A \cup B)} = \overline{A} \cap \overline{B}$} et \alert{$\overline{(A \cap B)} = \overline{A} \cup \overline{B}$}
    \end{description}
  \end{block}
  \begin{alertblock}{Attention}
    La définition du complément de $A$ dépend de l'univers $E$ !
    \begin{itemize}
    \item $\pi \notin \overline{\mathit{pair}}$ si $E = \mathbb{Z}$
    \item $\pi \in \overline{\mathit{pair}}$ si $E = \mathbb{R}$
    \end{itemize}
  \end{alertblock}

\end{frame}

\endgroup

% 
%\subsection{Relations}
%% SPDX-License-Identifier: CC-BY-SA-4.0
% Author: Matthieu Perrin
% Part: 
% Section: 
% Sub-section: 
% Frame: 

\begingroup

\begin{frame}{Couple et produit cartésien}
  \begin{block}{Définition}
    \begin{itemize}
    \item Un \structure{couple} $\langle a, b \rangle$ est une \alert{collection ordonnée de deux éléments} $a$ et $b$
    \item Deux couples sont égaux si leurs éléments sont égaux deux à deux 
      \vspace{-2mm}
      $$\forall a \forall b \forall c \forall d, \alert{\langle a, b \rangle =  \langle c, d \rangle \Leftrightarrow a = c \land b = d}$$
    \item Généralisation : un \structure{$n$-uplet} est une \alert{collection ordonnée de $n$ éléments}
    \item Le \structure{produit cartésien} de deux ensembles $A$ et $B$, noté \structure{$A\times B$}, est \alert{l'ensemble des couples formé d'un élément de $A$ et d'un élément de $B$}
      \vspace{-2mm}
      $$\forall A \forall B \forall a \forall b, \alert{\langle a, b \rangle \in A \times B \Leftrightarrow a \in A \land b \in B}$$
    \end{itemize}
  \end{block}
  \vspace{-2mm}
  \begin{exampleblock}{Exemple}
    Soient $A = \{1, 2, 3\}$ et $B = \{5, 6\}$. On a
    \vspace{-2mm}
    $$A\times B = \{\langle 1, 5 \rangle, \langle 1, 6 \rangle, \langle 2, 5 \rangle, \langle 2, 6 \rangle, \langle 3, 5 \rangle, \langle 3, 6 \rangle\}$$ 
  \end{exampleblock}
\end{frame}
\endgroup

%% SPDX-License-Identifier: CC-BY-SA-4.0
% Author: Matthieu Perrin
% Part: 
% Section: 
% Sub-section: 
% Frame: 

\begingroup

\begin{frame}{Relation}
  Soient $A$ et $B$ deux ensembles. 
  \begin{block}{Définition}
    \begin{itemize}
    \item Une \structure{relation (binaire)} est un \alert{sous-ensemble de $A \times B$}. 
    \item Une relation relation est \structure{homogène} si $A = B$. 
    \item Soit $\bowtie$ une relation binaire, on note parfois \structure{$a \,\bowtie\, b$} si \alert{$\langle a, b \rangle \in \bowtie$}. 
    \end{itemize}
  \end{block}
  \begin{block}{Remarque : définition par compréhension}
    Une relation binaire peut être vue comme un prédicat sur deux arguments
  \end{block}
  
  \begin{exampleblock}{Exemple}
    Soit $E$ un ensemble. L'inclusion est une relation homogène sur $\mathscr{P}(E)$ : 
    $$\subseteq \,\eqdef \{\langle A, B \rangle \in \mathscr{P}(E) \times \mathscr{P}(E) \,|\, A \subseteq B \}$$ 
  \end{exampleblock}

  \vspace{-2mm}
  \begin{itemize}
  \item Généralisation : Une \structure{relation $n$-aire} est un \alert{ensemble de $n$-uplets}.
  \end{itemize}
\end{frame}


\endgroup

%% SPDX-License-Identifier: CC-BY-SA-4.0
% Author: Matthieu Perrin
% Part: 
% Section: 
% Sub-section: 
% Frame: 

\begingroup


\begin{frame}{Propriétés des relations homogènes}
  Soit $\bowtie$ une relation binaire homogène sur un ensemble $E$.
  \begin{block}{Définitions -- Propriétés usuelles}
    \vspace{-1mm}
    \begin{description}
    \item[Reflexivité]  $\forall x\in E, \alert{x\bowtie x}$
    \item[Symétrie]     $\forall x\in E, \forall y\in E, \alert{x \bowtie y \Rightarrow y \bowtie x}$
    \item[Asymétrie]    $\forall x\in E, \forall y\in E, \alert{x \bowtie y \Rightarrow \lnot(y \bowtie x)}$
    \item[Antisymétrie] $\forall x\in E, \forall y\in E, \alert{x \bowtie y \land y \bowtie x \Rightarrow x=y}$
    \item[Transitivité] $\forall x\in E, \forall y\in E, \forall z\in E, \alert{x \bowtie y \land y \bowtie z \Rightarrow x \bowtie z}$
    \item[Totalité]     $\forall x\in E, \forall y\in E, \alert{x \bowtie y \lor y \bowtie x}$
    \end{description}
  \end{block}

  \vspace{-1mm}
  \begin{block}{Définition -- Fermeture réflexive et transitive}
    La \structure{fermeture réflexive et transitive} de $\bowtie$, notée $\bowtie^\star$, est la plus petite relation réflexive et transitive contenant $\bowtie$. 

    \vspace{-2mm}
    $$\alert{\bowtie^\star = \bigcap \{ \mathrel{R} \in\mathscr{P}(E\times E) | \bowtie \subseteq \mathrel{R} \land \mathrel{R} \text{ réflexive } \land \mathrel{R} \text{ transitive}\}}$$

    Informellement, \example{$x  \bowtie^\star y$} si \example{$x = y$} ou \example{$x = a_1 \bowtie a_2 \bowtie ... \bowtie a_n = y$}.
  \end{block}
\end{frame}

\endgroup

%% SPDX-License-Identifier: CC-BY-SA-4.0
% Author: Matthieu Perrin
% Part: 
% Section: 
% Sub-section: 
% Frame: 

\begingroup

\begin{frame}{Relation d'ordre}
  Soit $\bowtie$ une relation binaire homogène sur un ensemble $E$.

  \begin{block}{Définition -- Relation d'ordre}
    $\bowtie$ est une \structure{relation d'ordre} si elle est \alert{réflexive}, \alert{transitive} et \alert{antisymétrique}.
  \end{block}
  \begin{exampleblock}{Exemple}
    Soit $E$ un ensemble. 
    Montrons que $\subseteq$ est une relation d'ordre sur $\mathscr{P}(E)$.
    \begin{description}
    \item[Réflexivité]  Montrons que \alert{$\forall A\in \mathscr{P}(E), A\subseteq A$}
      \begin{itemize}
      \item Soit \structure{$A \in \mathscr{P}(E)$}. Montrons que $\alert{A\subseteq A}$.
      \item Soit $x\in A$. On a $x\in A$.
      \end{itemize}
    \item[Transitivité] Montrons que \alert{$\forall A, B, C \in \mathscr{P}(E), A \subseteq B \land B \subseteq C \Rightarrow A \subseteq C$}
      \begin{itemize}
      \item Soient $\structure{A, B, C \in \mathscr{P}(E)}$, tels que $\structure{A \subseteq B}$ et $\structure{B \subseteq C}$. 
      \item Soit $x \in A$. Donc $x\in B$, donc $x\in C$. 
      \end{itemize}
    \item[Antisymétrie] Montrons que \alert{$\forall A,B \in \mathscr{P}(E), A \subseteq B \land B \subseteq A \Rightarrow A=B$}
      \begin{itemize}
      \item Soient $\structure{A, B \in \mathscr{P}(E)}$, tels que $\structure{A \subseteq B}$ et $\structure{B \subseteq A}$. 
      \item Par double inclusion, on a $A=B$. 
      \end{itemize}
    \end{description}
  \end{exampleblock}
\end{frame}
\endgroup

%% SPDX-License-Identifier: CC-BY-SA-4.0
% Author: Matthieu Perrin
% Part: 
% Section: 
% Sub-section: 
% Frame: 

\begingroup

\begin{frame}{Équivalence et isomorphisme entre automates}

  \tf[text, top]{
    Soient $A = \langle \Sigma, Q_A, I_A, F_A, \rightarrow_A \rangle$ et $B = \langle \Sigma, Q_B, I_B, F_B, \rightarrow_B \rangle$ deux automates. 
  }

  \tfBlock[y=17mm]{Définition -- Automates équivalents}{
    $A$ et $B$ sont \structure{équivalents}, noté $\alert{A \equiv B}$, s'ils acceptent le même langage :
    $$\alert{A_1 \equiv A_2  \eqdef \mathcal{L}(A_1) = \mathcal{L}(A_2)}.$$
  }

  \tfBlock[bottom=2mm]{Définition -- Automates isomorphes}{
    $A$ et $B$ sont \structure{isomorphes}, noté $\alert{A \simeq B}$, s'ils ne diffèrent que par le nom de leurs états, \textit{c.-à-d.}
    s'il existe une fonction bijective $f : Q_A \rightarrow Q_B$ telle que :

    \vspace{2mm}
    $\structure{\forall a\in \Sigma, \forall q, q',}$\\
    $\structure{\begin{array}[t]{@{\quad\quad}lll}
        q\in Q_A &\Leftrightarrow& f(q) \in Q_B \\
        q\in I_A &\Leftrightarrow& f(q) \in I_B \\
        q\in F_A &\Leftrightarrow& f(q) \in F_B \\
        q \xrightarrow{a}_A q' &\Leftrightarrow& f(q) \xrightarrow{a}_B f(q')
      \end{array}
    }$
    \vspace{2mm}

    Deux automates isomorphes sont équivalents. 
  }

  \tf[bottom=2mm, x=3cm]{\small
    \begin{tikzpicture}[smAutomaton]
      \draw (1,0.75) node{\normalsize $\simeq$};

      \smState[\smStructure \smInitial]   (a0) at (0.0, 1.5) {$A$};
      \smState[\smStructure \smAccepting] (a1) at (0.0, 0.0) {$B$};
      \smState[\smExample \smInitial]     (b0) at (2.0, 1.5) {$0$};
      \smState[\smExample \smAccepting]   (b1) at (2.0, 0.0) {$1$};

      \smPath[\smStructure] (a0) edge[bend left]  node {$a$} (a1);
      \smPath[\smStructure] (a1) edge[bend left]  node {$a$} (a0);
      \smPath[\smStructure] (a1) edge[loop right] node {$b$} (a1);
      \smPath[\smStructure] (a0) edge[loop right] node {$b$} (a0);
      
      \smPath[\smExample]   (b0) edge[bend left]  node {$a$} (b1);
      \smPath[\smExample]   (b1) edge[bend left]  node {$a$} (b0);
      \smPath[\smExample]   (b1) edge[loop right] node {$b$} (b1);
      \smPath[\smExample]   (b0) edge[loop right] node {$b$} (b0);
    \end{tikzpicture}
  }
\end{frame}

\endgroup

%% SPDX-License-Identifier: CC-BY-SA-4.0
% Author: Matthieu Perrin
% Part: 
% Section: 
% Sub-section: 
% Frame: 

\begingroup

\begin{frame}{Exemple : congruence sur les entiers}
  \small
  Soit $k\in \mathbb{Z}$. Définissons la relation $\sim_k$ sur $\mathbb{Z}$ comme suit:

  $$\alert{\sim_k \eqdef \{\langle a, b\rangle \in \mathbb{Z}\times \mathbb{Z} | \exists i\in \mathbb{Z}, a = b + i k \}}.$$

  
  Montrons que $\sim_k$ est une relation d'équivalence sur $\mathbb{Z}$.
  \begin{description}
  \item[Réflexivité]  Montrons que \alert{$\forall a \in \mathbb{Z}, a \sim_k a$}
    \begin{itemize}
    \item Soit \structure{$a\in \mathbb{Z}$}.
    \item On a $a=a + 0k$, donc $\structure{a\sim_k a}$.
    \end{itemize}
  \item[Transitivité] Montrons que \alert{$\forall a, b, c \in \mathbb{Z}, a \sim_k b \land b \sim_k c \Rightarrow a \sim_k c$}
    \begin{itemize}
    \item Soient $\structure{a, b, c \in \mathbb{Z}}$, tels que $\structure{a \sim_k b}$ et $\structure{b \sim_k c}$. 
    \item Il existe $i, j \in \mathbb{Z}$ tels que $a = b + i k$ et $b = c + j k$.
    \item Donc $a = (c + j k) + i k = c + (i+j) k$, donc $\structure{a \sim_k c}$.
    \end{itemize}
  \item[Symétrie] Montrons que \alert{$\forall a, b \in \mathbb{Z}, a \sim_k b \Rightarrow b \sim_k a$}
    \begin{itemize}
    \item Soient $\structure{a, b \in \mathbb{Z}}$, tels que $\structure{a \sim_k b}$. 
    \item Il existe $i \in \mathbb{Z}$ tels que $a = b + i k$.
    \item Donc $b = a + (-i) k$, donc $\structure{b \sim_k a}$.
    \end{itemize}
  \end{description}
\end{frame}

\endgroup

% 
%\subsection{Fonctions}
%% SPDX-License-Identifier: CC-BY-SA-4.0
% Author: Matthieu Perrin
% Part: 
% Section: 
% Sub-section: 
% Frame: 

\begingroup

\begin{frame}{Fonction}
  Soient $E$ et $F$ deux ensembles.

  \begin{block}{Définition -- Fonction}
    \begin{itemize}
    \item Une \structure{fonction de $E$ dans $F$} est une relation $f \in \mathscr{P}(E \times F)$ qui vérifie la propriété suivante :
      $\alert{\forall x\in E, \existsunique y\in F, \langle x, y \rangle \in f}$
    \item L'\structure{ensemble des fonctions} de $E$ dans $F$ est noté $\alert{\mathscr{F}(E, F)}$.

      $$\alert{\mathscr{F}(E, F) \eqdef \{f\in \mathscr{P}(E \times F) | \forall x\in E, \existsunique y\in F, \langle x, y \rangle \in f\}}$$

    \item Pour $f\in \mathscr{F}(E, F)$ et $x\in E$,
      on note $\alert{f(x)}$ \structure{l'unique $y$ tel que $\langle x, y \rangle \in f$}.

    \item Pour $f\in \mathscr{F}(E, F)$ et $E' \in \mathscr{P}(E)$, on note
      
      $$\alert{f(E') \eqdef \{f(x) | x\in E'\} \eqdef \{y\in F | \exists x\in E', y = f(x)\}}$$

    \item \structure{Définition par une expression :} si on a une expression $\mathit{expr}[x]$ telle que, 
      $\forall x\in E, \mathit{expr}[x] \in F$, on peut définir une fonction : 

      $$\alert{\left\{\begin{array}{rcl}
      E &\rightarrow& F \\
      x &\mapsto& \mathit{expr}[x] \\
      \end{array}\right.  \eqdef \{c \in E\times F | \exists x\in E, c = \langle x, \mathit{expr}[x] \rangle\}}$$
    \end{itemize}
  \end{block}
  
\end{frame}

\endgroup

%% SPDX-License-Identifier: CC-BY-SA-4.0
% Author: Matthieu Perrin
% Part: 
% Section: 
% Sub-section: 
% Frame: 

\begingroup


\begin{frame}{Bijection}
  Soient $E$ et $F$ deux ensembles, et $f\in \mathscr{F}(E, F)$ une fonction de $E$ dans $F$.
  \begin{block}{Définition -- Bijection}
    \begin{itemize}
    \item $f$ est une \structure{bijection} si $\alert{\forall y\in F, \existsunique x\in E, \langle x, y \rangle \in f}$. 
    \item Si $f$ est une bijection, on note $f^{-1} = \{\langle y, x \rangle | \langle x, y \rangle \in f\}$. 
    \item Si $f$ est une bijection, alors $f^{-1}$ est une bijection de $F$ dans $E$. 
    \end{itemize}
  \end{block}
  
  \begin{exampleblock}{Exemple}
    Les deux fonctions suivantes sont des bijections, et $f = g^{-1}$.
    $$
    f \eqdef \left\{\begin{array}{rcl}
    \mathbb{Z} &\rightarrow& \mathbb{Z}\\
    n &\mapsto & n+1
    \end{array}\right.
    \hspace{1cm}
    g \eqdef \left\{\begin{array}{rcl}
    \mathbb{Z} &\rightarrow& \mathbb{Z}\\
    n &\mapsto & n-1
    \end{array}\right.
    $$
  \end{exampleblock}

\end{frame}

\endgroup

% 
%\subsection{Récurrence}
%% SPDX-License-Identifier: CC-BY-SA-4.0
% Author: Matthieu Perrin
% Part: 
% Section: 
% Sub-section: 
% Frame: 

\begingroup

\begin{frame}{Récurrence simple}

  \structure{Principe de la récurrence simple.}\\
  Soit $P$ un prédicat s'appliquant sur tous les nombres entiers. On a : 

  $$\structure{(\alert{P(0)} \land \alert{\forall k\in \mathbb{N}, P(k) \Rightarrow P(k+1)}) \Rightarrow \alert{\forall n\in \mathbb{N}, P(n)}}$$

  \pause
  \vspace{3mm}
  \example{Exemple :} 
  considérons le prédicat $\example{\displaystyle P(n) \eqdef \sum_{i = 1}^n i = \frac{n(n+1)}{2}}$.\\

  \begin{description}
  \item[\example{Initialisation :}] Montrons \alert{$P(0)$}.
    \uncover<3->{
      \begin{itemize}
      \item[\example{\myRec}] On a bien $\sum_{i = 1}^0 i = 0 = \frac{0 (0+1)}{2}$.
      \end{itemize}
    }
  \item[\example{Hérédité :}] Montrons \alert{$\forall k\in \mathbb{N}, P(k) \Rightarrow P(k+1)$}.
    \uncover<4->{
      \begin{itemize}
      \item[\example{\myRec}] Soit \structure{$k\in \mathbb{N}$}. Supposons \structure{$P(k)$}. Montrons \alert{$P(k+1)$}.
      \item[\example{\myRec}] $\begin{array}[t]{rcll}
        \sum_{i = 1}^{k+1} i & = & { \sum_{i = 1}^{k} i} + (k+1) & \text{\footnotesize Isolement de $k+1$}\\
        & = &\frac{k(k+1)}{2} + \frac{2(k+1)}{2}  & \text{\footnotesize hypothèse de récurrence}\\
        & = &\frac{(k+1)(k+2)}{2}  & \text{\footnotesize factorisation de $\frac{k+1}{2}$}\\
      \end{array}$
      \item[\example{\myRec}] Donc $P(k+1)$, donc $\forall k\in \mathbb{N}, P(k) \Rightarrow P(k+1)$.
      \end{itemize}
    }
  \item[\example{Conclusion :}] Donc par récurrence simple, on a bien $\alert{\forall n\in \mathbb{N}, P(n)}$
  \end{description}
\end{frame}

\endgroup

%% SPDX-License-Identifier: CC-BY-SA-4.0
% Author: Matthieu Perrin
% Part: 
% Section: 
% Sub-section: 
% Frame: 

\begingroup

\begin{frame}{Récurrence forte}

  \begin{block}{Principe de la récurrence forte.}
    Soit $P$ un prédicat s'appliquant sur tous les nombres entiers. On a : 

    $$\structure{(\alert{\forall k\in \mathbb{N}, (\forall i < k,  P(i)) \Rightarrow P(k)}) \Rightarrow \alert{\forall n\in \mathbb{N}, P(n)}}$$
  \end{block}
  
  \pause

  \begin{exampleblock}{Exemple :}

    \example{$\displaystyle P(n) \eqdef n\ge 2 \Rightarrow n$ est un produit de nombres premiers}.\\

    \example{Récurrence :} Montrons \alert{$\alert{\forall k\in \mathbb{N}, (\forall i < k,  P(i)) \Rightarrow P(k)}$}.\\
    \uncover<3-|handout>{
      Soit \structure{$k\in \mathbb{N}$}. Supposons \structure{$\forall i < k,  P(i)$}. Montrons \alert{$P(k)$}.
      \begin{itemize}
      \item[\example{\myRec}] Cas 1 : $k$ est premier. (\example{Couvre le cas $k=2$}.)
        \begin{itemize}
        \item[\example{\myRec}] Donc $k = k$, donc $P(k)$.
        \end{itemize}
      \item[\example{\myRec}] Cas 2 : $k$ n'est pas premier. 
        \begin{itemize}
        \item[\example{\myRec}] Il existe $a, b < k$ tels que $k=ab$. 
        \item[\example{\myRec}] Par l'hypothèse de récurrence,  $a = a_1 a_2 ... a_{x}$ et $b = b_1 b_2 ... b_{y}$.
        \item[\example{\myRec}] Donc $k = a b = a_1 a_2 ... a_{x} b_1 b_2 ... b_{y}$, donc $P(k)$.
        \end{itemize}
      \end{itemize}
    }
    \example{Conclusion :} Donc par récurrence forte, on a bien $\alert{\forall n\in \mathbb{N}, P(n)}$

  \end{exampleblock}
\end{frame}

\endgroup

% 
%\section{Mots et langages formels}
% 
%\subsection{Notion d'alphabet}
%% SPDX-License-Identifier: CC-BY-SA-4.0
% Author: Matthieu Perrin
% Part: 
% Section: 
% Sub-section: 
% Frame: 

\begingroup

\begin{frame}{Alphabet}
  
  Soit $n\in \mathbb{N}$. On note $\alert{\llbracket n\rrbracket \eqdef \{x\in \mathbb{N} \,|\, 1 \le x \le n\} = \{1, ..., n\}}$.
  
  \begin{block}{Définition -- alphabet}
    Soit $E$ un ensemble.
    \begin{itemize}
    \item On dit que $E$ est \structure{fini} s'il existe\footnote{Si un tel $n$ existe, il est unique.} $n\in \mathbb{N}$ et une bijection de $\llbracket n\rrbracket$ dans $E$. 
    \item $n$ est appelé le \structure{cardinal} de $E$, noté \alert{$|E|$}.
    \item Un \structure{alphabet} est un \alert{ensemble fini non-vide}. 
    \end{itemize}
  \end{block}

  \begin{exampleblock}{Exemple}
    $\{a, b, c\}$ est un alphabet de cardinal 3, d'après la bijection suivante :

    $$
    \left\{\begin{array}{ccl}
    \llbracket 3\rrbracket &\rightarrow& \{a, b, c\}\\
    1 &\mapsto& a\\ 
    2 &\mapsto& b\\ 
    3 &\mapsto& c\\ 
    \end{array}\right.
    $$
  \end{exampleblock}

\end{frame}

\endgroup

% 
%\subsection{Notion de mot}
%% SPDX-License-Identifier: CC-BY-SA-4.0
% Author: Matthieu Perrin
% Part: 
% Section: 
% Sub-section: 
% Frame: 

\begingroup

\begin{frame}{Mot}

  Soit $\Sigma$ un alphabet et $n\in \mathbb{N}$.
  \begin{itemize}
  \item Un \structure{mot} de longueur $n$ sur $\Sigma$ est une \alert{fonction de $\llbracket n\rrbracket$ dans $\Sigma$}.
  \item L'ensemble des mots de longueur $n$ sur $\Sigma$ est \alert{$\Sigma^n = \mathscr{F}\left(\llbracket n\rrbracket, \Sigma\right)$}.
  \item L'ensemble des mots sur $\Sigma$, noté \structure{$\Sigma^\star$}, est :
    $$\alert{\displaystyle \Sigma^\star \eqdef \bigcup_{n\in \mathbb{N}} \Sigma^n = \Sigma^0 \cup \Sigma^1 \cup \Sigma^2 \cup ...}$$
  \item L'ensemble des mots \structure{non-vides} sur $\Sigma$ est
    $\alert{\Sigma^+ \eqdef \bigcup_{n \neq 0 } \Sigma^n}$
  \end{itemize}

  \begin{exampleblock}{Exemples}
    Soit $\Sigma = \{a, b, ..., z\}$

    \begin{minipage}[t]{.5\textwidth}
      \begin{itemize}
      \item $\varepsilon = \left\{\begin{array}{ccc}
        \llbracket 0\rrbracket & \rightarrow & \Sigma\\
        x & \mapsto & \_
      \end{array}\right. \in \Sigma^0$
      \item $\text{``a''} = \left\{\begin{array}{ccc}
        \llbracket 1\rrbracket & \rightarrow & \Sigma\\
        x & \mapsto & \text{`a'}
      \end{array}\right. \in \Sigma^1$
      \end{itemize}
    \end{minipage}%
    \begin{minipage}[t]{.5\textwidth}
      \begin{itemize}
      \item $\text{``bon''} = \left\{\begin{array}{ccc}
        \llbracket 3\rrbracket& \rightarrow & \Sigma\\
        1 & \mapsto & \text{`b'}\\
        2 & \mapsto & \text{`o'}\\
        3 & \mapsto & \text{`n'}\\
      \end{array}\right. \in \Sigma^3$
      \end{itemize}
    \end{minipage}%
    \begin{itemize}
    \item Par abus de notation, on fera l'amalgame entre $\Sigma$ et $\Sigma^1$. 
    \end{itemize}
  \end{exampleblock}

\end{frame}

\endgroup

%% SPDX-License-Identifier: CC-BY-SA-4.0
% Author: Matthieu Perrin
% Part: 
% Section: 
% Sub-section: 
% Frame: 

\begingroup

\begin{frame}{Longueur et occurrences}
  
  Soient $\Sigma$ un alphabet, $a\in\Sigma$, $u\in \Sigma^\star$ et $i\in \llbracket |u|\rrbracket$.
  \begin{itemize}
  \item La \structure{longueur} de $u$, notée $\alert{|u|}$, est l'unique $n$ tel que $u\in \Sigma^n$
  \item On note \alert{$u_i = u(i)$} le $i^\text{ème}$ caractère de $u$
  \item Une \structure{occurrence de $a$ dans $u$} est un indice $i\in \llbracket |u|\rrbracket$ tel que $u_i = a$
  \item Le nombre d'occurrences de $a$ dans $u$ est noté $\alert{|u|_a}$
  \end{itemize}

  \begin{exampleblock}{Exemples}
    \begin{itemize}
    \item $|\text{``banane''}| = 6$
    \item $\forall x\in \Sigma^\star, |x| = 0 \Leftrightarrow x = \varepsilon$ car $\Sigma^0 = \{\varepsilon\}$
    \item Les occurrences de `a' dans ``banane'' sont les indices 2 et 4
    \item $|\text{``banane''}|_{\text{`a'}} = 2$
    \end{itemize}
  \end{exampleblock}

\end{frame}

\endgroup

%% SPDX-License-Identifier: CC-BY-SA-4.0
% Author: Matthieu Perrin
% Part: 
% Section: 
% Sub-section: 
% Frame: 

\begingroup

\begin{frame}{Concaténation}

  \tfBlock[top=-5mm]{Définition -- Concaténation de deux mots}{
    Soient $\Sigma$ un alphabet, $u, v \in \Sigma^\star$, et $n\in \mathbb{N}$.

    \begin{itemize}
    \item La \structure{concaténation} de $u$ et $v$, notée \structure{$u \cdot v$} (ou \structure{$uv$}), est le mot
      de longueur $|u| + |v| $ dont les premiers symboles forment $u$ et les derniers forment $v$
      $$\alert{
        u \cdot v  \quad\eqdef\quad  \left\{\begin{array}{@{}r@{~}c@{~}ll@{}}
        \llbracket |u| + |v| \rrbracket& \rightarrow & \Sigma\\
         i   & \mapsto & u_i & \text{pour } i \in \llbracket |u| \rrbracket \\
        |u| + j & \mapsto & v_j & \text{pour } j \in \llbracket |v| \rrbracket
        \end{array}\right.
        \quad=\quad u_1 \cdots u_{|u|} \cdot v_1 \cdots v_{|v|}
      }$$
    \item La \structure{puissance $n$ du mot $x$} est le mot \alert{$x^n = x\cdots x$} (répété $n$ fois)
      $$\alert{u^0 \eqdef \varepsilon} \quad \text{ et } \quad \alert{u^{n+1} \eqdef u \cdot u^n}$$
    \end{itemize}
  }

  \tfExampleBlock[bottom]{Exemple}{\small\vspace{-7mm}
    $$\text{``bon''} \cdot \text{``jour''} = \left\{
    \begin{array}{@{~}r@{~~}c@{~~}l@{~~}c@{~~}c@{~}}
    \llbracket 3 + 4 \rrbracket & \rightarrow & \Sigma\\
    1 & \mapsto & \text{``bon''}_{1}  & = &\text{`b'}\\
    2 & \mapsto & \text{``bon''}_{2}  & = &\text{`o'}\\
    3 & \mapsto & \text{``bon''}_{3}  & = &\text{`n'}\\
    3+1 & \mapsto & \text{``jour''}_{1} & = &\text{`j'}\\
    3+2 & \mapsto & \text{``jour''}_{2} & = &\text{`o'}\\
    3+3 & \mapsto & \text{``jour''}_{3} & = &\text{`u'}\\
    3+4 & \mapsto & \text{``jour''}_{4} & = &\text{`r'}\\
    \end{array}
    \right. = \text{``bonjour''}$$
  }

\end{frame}

\endgroup

%% SPDX-License-Identifier: CC-BY-SA-4.0
% Author: Matthieu Perrin
% Part: 
% Section: 
% Sub-section: 
% Frame: 

\begingroup



\begin{frame}{Structure de monoïde}
  Soit $\Sigma$ un alphabet. 
  
  \begin{block}{Propriétés de la concaténation}
  \begin{description}
  \item[Associativité :] $\forall u, v, w \in \Sigma^\star, \alert{(u\cdot v)\cdot w = u\cdot (v\cdot w)}$
    \begin{itemize}
    \item On notera $\structure{u\cdot v\cdot w}$
    \end{itemize}
  \item[Neutralité :] $\forall u \in \Sigma^\star, \alert{\varepsilon \cdot u = u = u \cdot \varepsilon}$
  \item[Simplifiabilité :] $\forall u, v, w \in \Sigma^\star,$
    \begin{description}
    \item[À gauche :] $\alert{u \cdot v = u \cdot w \Rightarrow v=w}$
    \item[À droite :] $\alert{v \cdot u = w \cdot u \Rightarrow v=w}$
    \end{description}
  \item[Non-commutativité :] $ \exists u,v \in \Sigma^\star, \alert{u\cdot v \neq v\cdot u}$
  \end{description}
  \end{block}

  \begin{block}{Remarque}
    Un \structure{monoïde} est un ensemble muni d'une loi de composition interne associative, et d'un élément neutre. \\
    Le monoïde \structure{$\langle \Sigma^\star, \cdot , \varepsilon \rangle$} est appelé \alert{monoïde libre engendré par $\Sigma$}.
  \end{block}
\end{frame}


\endgroup

%% SPDX-License-Identifier: CC-BY-SA-4.0
% Author: Matthieu Perrin
% Part: 
% Section: 
% Sub-section: 
% Frame: 

\begingroup

\begin{frame}{Facteur, préfixe, suffixe}
  
  \begin{block}{Définitions}
    Soient $\Sigma$ un alphabet, $u, v \in \Sigma^\star$.
    \begin{itemize}
    \item $v$ est un \structure{facteur} \phantom{propr}\,\,\, de $u$ si\quad \alert{$\exists x, y           \in \Sigma^\star, u = x \cdot v \cdot y$}
    \item $v$ est un \structure{préfixe} \phantom{propre}      de $u$ si\quad \alert{$\exists y\phantom{, x} \in \Sigma^\star, u = \phantom {w \cdot }\; v \cdot y$}
    \item $v$ est un \structure{préfixe propre}                de $u$ si\quad \alert{$\exists y\phantom{, x} \in \Sigma^+,    u = \phantom{ w \cdot }\; v \cdot y$}
    \item $v$ est un \structure{suffixe} \phantom{propre}\,    de $u$ si\quad \alert{$\exists x\phantom{, y} \in \Sigma^\star, u = x \cdot v$}
    \item $v$ est un \structure{suffixe propre}\,              de $u$ si\quad \alert{$\exists x\phantom{, y} \in \Sigma^+,    u = x \cdot v$}
      
    \item L'ensemble des préfixes de $u$ est noté \structure{$\mathit{Pref}(u)$}
    \item L'ensemble des préfixes propres de $u$ est noté \structure{$\mathit{PrefProp}(u)$}
    \item L'ensemble des suffixes de $u$ est noté \structure{$\mathit{Suff}(u)$}
    \item L'ensemble des suffixes propres de $u$ est noté \structure{$\mathit{SuffProp}(u)$}
    \end{itemize}
  \end{block}

  \begin{exampleblock}{Exemple}\vspace{-2mm}
    $$\begin{array}{cc}
    \mathit{Pref}(abc) = \example{\{\varepsilon, a, ab, abc\}} & \mathit{PrefProp}(abc) = \example{\{\varepsilon, a, ab\}}\\
    \mathit{Suff}(abc) = \example{\{\varepsilon, c, bc, abc\}} & \mathit{SuffProp}(abc) = \example{\{\varepsilon, c, bc\}}
    \end{array}$$
  \end{exampleblock}

\end{frame}

\endgroup

%% SPDX-License-Identifier: CC-BY-SA-4.0
% Author: Matthieu Perrin
% Part: 
% Section: 
% Sub-section: 
% Frame: 

\begingroup

\begin{frame}{Relation d'ordre}
  Soit $\bowtie$ une relation binaire homogène sur un ensemble $E$.

  \begin{block}{Définition -- Relation d'ordre}
    $\bowtie$ est une \structure{relation d'ordre} si elle est \alert{réflexive}, \alert{transitive} et \alert{antisymétrique}.
  \end{block}
  \begin{exampleblock}{Exemple}
    Soit $E$ un ensemble. 
    Montrons que $\subseteq$ est une relation d'ordre sur $\mathscr{P}(E)$.
    \begin{description}
    \item[Réflexivité]  Montrons que \alert{$\forall A\in \mathscr{P}(E), A\subseteq A$}
      \begin{itemize}
      \item Soit \structure{$A \in \mathscr{P}(E)$}. Montrons que $\alert{A\subseteq A}$.
      \item Soit $x\in A$. On a $x\in A$.
      \end{itemize}
    \item[Transitivité] Montrons que \alert{$\forall A, B, C \in \mathscr{P}(E), A \subseteq B \land B \subseteq C \Rightarrow A \subseteq C$}
      \begin{itemize}
      \item Soient $\structure{A, B, C \in \mathscr{P}(E)}$, tels que $\structure{A \subseteq B}$ et $\structure{B \subseteq C}$. 
      \item Soit $x \in A$. Donc $x\in B$, donc $x\in C$. 
      \end{itemize}
    \item[Antisymétrie] Montrons que \alert{$\forall A,B \in \mathscr{P}(E), A \subseteq B \land B \subseteq A \Rightarrow A=B$}
      \begin{itemize}
      \item Soient $\structure{A, B \in \mathscr{P}(E)}$, tels que $\structure{A \subseteq B}$ et $\structure{B \subseteq A}$. 
      \item Par double inclusion, on a $A=B$. 
      \end{itemize}
    \end{description}
  \end{exampleblock}
\end{frame}
\endgroup

% 
%\subsection{Notion de langage}
%% SPDX-License-Identifier: CC-BY-SA-4.0
% Author: Matthieu Perrin
% Part: 
% Section: 
% Sub-section: 
% Frame: 

\begingroup

\begin{frame}{Langage reconnu par un AFN}

  \on[text, top]{
    Soit $A=\langle \Sigma, Q, I, F, \rightarrow \rangle$ un AFN.

    \begin{block}{Définition -- langage reconnu (ou accepté)}
      Un mot $u$ est \structure{reconnu} par $A$ s'il mène d'un état initial à un état final
      $$\exists i\in I, \exists f\in F,~  \alert{\langle u, i \rangle \leadsto_A^\star \langle \varepsilon, f\rangle}$$
      Le langage \structure{reconnu} par $A$ est l'ensemble $\alert{\mathcal{L}(A)}$ des mots reconnus par $A$
      $$\alert{\mathcal{L}(A) \eqdef \left\{u \in \Sigma^\star \,\middle\mid\, \exists i\in I, \exists f\in F,  \langle u, i \rangle \leadsto_A^\star \langle \varepsilon, f\rangle\right\}}$$
    \end{block}
  }

  \onExampleBlock[bottom]{Exemple}{
    Pour \example{$i=3$}, \example{$f=2$} et \example{$u=bbc$}, on a : 
    \begin{enumerate}
    \item $\langle bbc, 1 \rangle \leadsto^\star \langle \varepsilon, 2 \rangle$
    \item $1 \in I$
    \item $2 \in F$
    \end{enumerate}
    Donc \example{$bbc \in \mathcal{L}(A)$}
  }

  \on[bottom, x=.25\textwidth]{
    \begin{tikzpicture}[automaton, grid size=15mm]
      \state[initial]       (1) at (0,1) {$1$}; 
      \state[accepting]     (2) at (1,1) {$2$}; 
      \state[initial above] (3) at (0,0) {$3$}; 
      \state                (4) at (1,0) {$4$}; 

      \path (1) edge             node       {$a$}           (2);
      \path (1) edge[loop above] node       {$a$}           (1);
      \path (3) edge             node[swap] {$b$}           (4);
      \path (3) edge[loop left ] node       {$b$}           (3);
      \path (4) edge[loop right] node       {$b$}           (4);
      \path (2) edge[loop above] node       {$c$}           (2);
      \path (4) edge             node[swap] {$\varepsilon$} (2);
    \end{tikzpicture}
  }
  
\end{frame}


\endgroup

%% SPDX-License-Identifier: CC-BY-SA-4.0
% Author: Matthieu Perrin
% Part: 
% Section: 
% Sub-section: 
% Frame: 

\begingroup

\begin{frame}{Opérations sur les langages}

  Soit $\Sigma$ un alphabet, et $L, M \in \mathscr{P}(\Sigma^\star)$ deux langages. 
  
  \begin{block}{Rappel -- Opérations ensemblistes}
    \begin{description}[Fermeture de Kleene :]
    \item[Union :] $\alert{L \cup M} \eqdef \{ x \in \Sigma^\star \mid x \in L \lor x \in M\}$
    \item[Intersection :] $\alert{L \cap M} \eqdef \{ x \in \Sigma^\star \mid x \in L \land x \in M\}$
    \item[Différence :] $\alert{L \setminus M} \eqdef \{ x \in \Sigma^\star \mid x \in L \land x \notin M\}$
    \item[Complémentaire :] $\alert{\overline{L}} \eqdef \{ x \in \Sigma^\star \mid x \notin L\} = \Sigma^\star \setminus L$
    \end{description}
  \end{block}

  \begin{block}{Définition -- Concaténation des langages}
    \begin{description}[Fermeture de Kleene :]
    \item[Concaténation :] $\alert{L \cdot M} \eqdef \{ x \cdot y \mid x \in L \land y \in M\}$ (parfois noté $LM$)
    \item[Puissance :] $L^n = L \cdots L$ ($L$ est répété $n$ fois) :
      $$\alert{L^0} \eqdef \{\varepsilon\} \quad\text{et}\quad \alert{L^{n+1}} \eqdef L^n \cdot L$$
    \item[Fermeture de Kleene :] $\alert{L^\star} \eqdef \bigcup_{i\in \mathbb{N}} L^i$
    \item[Fermeture positive :] $\alert{L^+} \eqdef \bigcup_{i>0} L^i = L^\star \cdot L = L \cdot L^\star$
    \end{description}
  \end{block}

\end{frame}

\endgroup

%% SPDX-License-Identifier: CC-BY-SA-4.0
% Author: Matthieu Perrin
% Part: 
% Section: 
% Sub-section: 
% Frame: 

\begingroup

\begin{frame}{Structure d'algèbre de Kleene}
  Soient $\Sigma$ un alphabet et $L, L_1, L_2, L_3 \in \mathscr{P}(\Sigma)$ des langages. 
 
  \begin{itemize}
  \item \textbf{Union} \hspace{\fill} $\langle \mathscr{P}(\Sigma), \cup, \emptyset \rangle$ est un monoïde commutatif
    \begin{description}
    \item[neutralité :] $\emptyset \cup L = L = L \cup \emptyset$ 
    \item[associativité :] $L_1 \cup (L_2 \cup L_3) = (L_1 \cup L_2) \cup L_3$
    \item[commutativité :] $L_1 \cup L_2 = L_2 \cup L_1$
    \item[idempotence :] $L \cup L = L$
    \end{description}
  \item \textbf{Concaténation} \hspace{\fill} $\langle \mathscr{P}(\Sigma), \cdot, \{\varepsilon\} \rangle$ est un monoïde
    \begin{description}
    \item[neutralité :] $\{\varepsilon\} \cdot L = L = L \cdot \{\varepsilon\}$
    \item[associativité :] $L_1 \cdot (L_2 \cdot L_3) = (L_1 \cdot L_2) \cdot L_3$
    \item[absorption :] $L \cdot \emptyset = \emptyset = \emptyset \cdot L$
    \end{description}
  \item \textbf{Mise à l'étoile} 
    \begin{description}
    \item[neutralité :] $\emptyset^\star = \{\varepsilon\}$
    \item[idempotence :] $L^{\star^\star} = L^\star$
    \end{description}
  \item \textbf{Intéraction entre opérateurs} 
    \begin{description}
    \item[distributivité :] $L_1 \cdot (L_2 \cup L_3) = (L_1\cdot L_2) \cup (L_1 \cdot L_3)$\\
      $(L_2 \cup L_3) \cdot L_1 = (L_2\cdot L_1) \cup (L_3 \cdot L_1)$
    \item[absorption :] $L \cup L^\star = L^\star = L^\star \cup L$
    \item[fermetures :] $L^\star = \{ \varepsilon \} \cup L^+$
    \end{description}
  \end{itemize}
\end{frame}


\endgroup

% 
% 
%\part{Analyse Lexicale}
% 
% 
%\section{Langages et expressions rationnels}
% 
%\subsection{Introduction}
%% SPDX-License-Identifier: CC-BY-SA-4.0
% Author: Matthieu Perrin
% Part: 
% Section: 
% Sub-section: 
% Frame: 

\begingroup

\begin{frame}{Reconnaissance d'un langage quelconque}
  \begin{block}{Algorithme de recherche ascendante par force brute} 

    \begin{description}
    \item [Entrées :]
      \begin{itemize}
      \item Une grammaire $G$, si possible contextuelle
      \item Un mot $u$
      \end{itemize}

    \item [Sortie :] une réponse booléenne sur \structure{$u \in \mathcal{L}(G)$}
      %  Certaines (si réponse oui) / toutes (si réponse non) les listes de symboles terminaux/non-terminaux dérivant $m$

    \item [Teminaison :] garantie si $G$ est contextuelle
    \item [Complexité :] exponentielle par rapport à $|u|$
    \end{description}
  \end{block}
\end{frame}

\endgroup

%% SPDX-License-Identifier: CC-BY-SA-4.0
% Author: Matthieu Perrin
% Part: 
% Section: 
% Sub-section: 
% Frame: 

\begingroup

\begin{frame}{Exemple d'utilisation du lemme de pompage}
  
  \tfBlock[top=-5mm]{Montrer que $L \eqdef \{a^nb^nc^n \mid n\in \mathbb{N}\}$ n'est pas algébrique}{
    Soit $\Sigma \eqdef \{a, b, c\}$.%
    \only<2-|handout>{
      Si $L$ est algébrique, $L$ vérifie le lemme de pompage :

      \vspace{-4mm}
      $$
      \begin{array}{c}
        \structure{\exists N\in \mathbb{N}}, \alert{\forall u\in L, |u| \ge N} \Rightarrow (\structure{\exists v, w, x, y, z\in \Sigma^\star}, \\
        u = v \cdot w \cdot x\cdot y \cdot z \land w \cdot y\neq \varepsilon \land |w\cdot x\cdot y| \le N \land \alert{\forall i \in \mathbb{N}}, v\cdot w^i\cdot x\cdot y^i \cdot z \in L)
      \end{array}
      $$
      \vspace{-2mm}
      
      \structure{Soit $N$ donné par le lemme pompage}.
    }
    
    \only<3-|handout>{%
      \alert{Posons $u = a^N b^N c^N$. On a bien $u\in L$ et $|u| = 3N \ge N$}.\\
      \structure{Soit $v \cdot w \cdot x\cdot y \cdot z$ la décomposition de $u$ donnée par le lemme de pompage}. 
    }
  }
  
  \tf<4-|handout>[y=-10mm]{
    \begin{smArray}[width=4mm, height=3mm, name={$u=$}]
      \smCell[\smNone]{\alert{$a$}}      \smCoord{(a1)}
      \smCell[\smNone]{\alert{$\cdots$}}               
      \smCell[\smNone]{\alert{$a$}}      \smCoord{(an)}
      \smCell[\smNone]{\alert{$\cdot$}}               
      \smCell[\smNone]{\alert{$b$}}      \smCoord{(b1)}
      \smCell[\smNone]{\alert{$\cdots$}}               
      \smCell[\smNone]{\alert{$b$}}      \smCoord{(bn)}
      \smCell[\smNone]{\alert{$\cdot$}}               
      \smCell[\smNone]{\alert{$c$}}      \smCoord{(c1)}
      \smCell[\smNone]{\alert{$\cdots$}}               
      \smCell[\smNone]{\alert{$c$}}      \smCoord{(cn)}
      
      \draw [decorate, decoration={brace, amplitude=5pt}] ([xshift=1mm]a1.north west) -- ([xshift=-1mm]an.north east) node[midway,yshift=4mm]{$N$};
      \draw [decorate, decoration={brace, amplitude=5pt}] ([xshift=1mm]b1.north west) -- ([xshift=-1mm]bn.north east) node[midway,yshift=4mm]{$N$};
      \draw [decorate, decoration={brace, amplitude=5pt}] ([xshift=1mm]c1.north west) -- ([xshift=-1mm]cn.north east) node[midway,yshift=4mm]{$N$};

      \draw [decorate, decoration={brace, amplitude=5pt, mirror}] ([xshift=1mm]an.south west) -- ([xshift=-1mm]b1.south east) node[midway,yshift=-4mm]{$|wxy|\le N$};
      \draw [decorate, decoration={brace, amplitude=5pt, mirror}] ([xshift=1mm]bn.south west) -- ([xshift=-1mm]c1.south east) node[midway,yshift=-4mm]{$|wxy|\le N$};
    \end{smArray}       
  }

  \tf<5-|handout>[text, bottom=-1mm]{
    \begin{itemize}
    \item Comme $wy \neq \varepsilon$, $\alpha = wy[1] \in \Sigma$ est une lettre de $wy$.
    \item Comme $|wxy| \le N$, il existe $\beta \in \Sigma$ tel que $\beta$ n'est pas une lettre de $wy$. 
    \item \alert{Posons $i=2$}. $|v \cdot w^2 \cdot x\cdot  y^2\cdot   z|_\alpha > |v\cdot  w^2\cdot  x\cdot  y^2\cdot   z|_\beta$, donc $v\cdot  w^2\cdot  x\cdot  y^2\cdot   z\notin L$. 
    \end{itemize}
    Absurde ! Donc $L$ n'est pas algébrique. 
  }

  \tfExampleBlock<-4>[y=-25mm]{On sait}{}

  \tf<1>[y=-25mm, anchor=north, text]{
    \begin{itemize}
    \item $\begin{array}[t]{l}
      \alert{\forall L \in \textsc{alg}_\Sigma}, \structure{\exists N\in \mathbb{N}}, \forall u\in L, |u| \ge N \Rightarrow (\exists v, w, x, y, z\in \Sigma^\star, \\
      u = v w x y z \land w y\neq \varepsilon \land |wxy| \le N \land \forall i \in \mathbb{N}, v w^i x y^i z \in L)
    \end{array}$
    \end{itemize}
  }

  \tf<2-4>[y=-25mm, anchor=north, left=.35\textwidth]{
    \begin{itemize}
    \item $L \in \textsc{alg}_\Sigma$
    \item<3-> $v, w, x, y, z\in \Sigma^\star$
    \end{itemize}
  }
  
  \tf<2-4>[y=-25mm, anchor=north, width=.35\textwidth]{
    \begin{itemize}
    \item $N \in \mathbb{N}$
    \item<3-> $wy\neq \varepsilon$
    \end{itemize}
  }
  
  \tf<3-4>[y=-25mm, anchor=north, right=.35\textwidth]{
    \begin{itemize}
    \item $v \cdot w \cdot x \cdot y \cdot z = a^N b^N c^N$
    \item $\alert{|wxy| \le N}$
    \end{itemize}
  }
  
\end{frame}

\endgroup

%% SPDX-License-Identifier: CC-BY-SA-4.0
% Author: Matthieu Perrin
% Part: 
% Section: 
% Sub-section: 
% Frame: 

\begingroup

\begin{frame}{Construction des expressions rationnelles}
  
  Les expressions rationnelles sur un alphabet $\Sigma$ sont construites par induction
  
  \begin{description}[xxxxxxxxxxx]
  \item[$\emptyset$ :] représente le langage vide \alert{$\emptyset$}
  \item[$\varepsilon$ :] représente le langage neutre \alert{$\{\varepsilon\}$}
  \item[$a$ :] représente le langage \alert{$\{a\}$}, pour tout $a\in \Sigma$
  \item[$\mathit{reg}_1 \cdot \mathit{reg}_2$ :] représente le langage \alert{$\mathcal{L}(\mathit{reg}_1) \cdot \mathcal{L}(\mathit{reg}_2)$}
  \item[$\mathit{reg}_1 \mid \mathit{reg}_2$ :] représente le langage \alert{$\mathcal{L}(\mathit{reg}_1) \cup \mathcal{L}(\mathit{reg}_2)$}
  \item[$\mathit{reg}^\star$ :] représente le langage \alert{$\mathcal{L}(\mathit{reg})^\star$}
  \end{description}

  Où \structure{$\mathcal{L}(\mathit{reg})$} est le langage représenté par l'expression rationnelle $\mathit{reg}$

  \pause
  \begin{block}{Notations}
    \begin{itemize}
    \item L'opérateur \alert{$\cdot$} est le plus souvent \structure{omis}
    \item On utilise les règles de \alert{préséance} : \quad
      \structure{1.} \alert{${}^\star$} \quad
      \structure{2.} \alert{$\cdot$}   \quad
      \structure{3.} \alert{$|$}
      \begin{center}
        Par exemple, \example{$\mathcal{L}(ab^\star\mid cd^\star) = \mathcal{L}(((a \cdot b^\star) \mid (c \cdot d^\star)))$}
      \end{center}
    \item D'autres constructions peuvent être ajoutées \emph{(sucre syntaxique)}
      \begin{center}
        \structure{$\mathit{reg}^+ \eqdef (\mathit{reg} \cdot \mathit{reg}^\star)$} \quad\quad
        \structure{$\mathit{reg}^? \eqdef (\mathit{reg} \mid \varepsilon)$}
      \end{center}
    \item Une expression rationnelle \structure{canonique} n'utilise que les opérateurs \alert{$.$}, \alert{$|$} et \alert{${}^\star$}
    \end{itemize}
  \end{block}
  
\end{frame}

\endgroup

% 
%\subsection{Expressions rationnelles}
%% SPDX-License-Identifier: CC-BY-SA-4.0
% Author: Matthieu Perrin
% Part: 
% Section: 
% Sub-section: 
% Frame: 

\begingroup

\begin{frame}{Langage des expressions rationnelles}
  
  Soit $\Sigma$ un alphabet. \\
  Les \structure{expressions rationnelles} sur $\Sigma$
  sont des \textit{expressions}, donc des mots, sur l'alphabet
  $\tilde{\Sigma} = \Sigma \sqcup \{
  \text{`\example{\underline{$\emptyset$}}'},
  \text{`\example{\underline{$\varepsilon$}}'},
  \text{`\example{\underline{$($}}'},
  \text{`\example{\underline{$)$}}'},
  \text{`\example{\underline{$|$}}'},
  \text{`\example{\underline{$\cdot$}}'},
  \text{`\example{\underline{${}^\star$}}'}
  \}$ \footnote[frame, 1]{$\sqcup$ désigne l'union disjointe. Parfois, on note $0$, $1$, $+$ et $\times$ au lieu de $\emptyset$, $\varepsilon$, $|$ et $\cdot$}.

  L'ensemble des expressions rationnelles sur $\Sigma$ forme le langage $\alert{\textsc{regex}_\Sigma}$ définie
  récursivement de la façon suivante.

  \begin{block}{Définition --- Expression rationnelle}
    \alert{$\textsc{regex}_\Sigma$} est le \alert{plus petit sous-ensemble}\footnote[frame,2]{Au sens de l'inclusion} de $\tilde{\Sigma}^\star$ tel que

    \vspace{1mm}
    \begin{tabular}{rlrl}
      \vspace{.5mm}\structure{\myRec}& &$\text{``\example{\underline{$\emptyset$}}''} \alert{\in \textsc{regex}_\Sigma}$\\
      \vspace{.5mm}\structure{\myRec} & &$\text{``\example{\underline{$\varepsilon$}}''} \alert{\in \textsc{regex}_\Sigma}$\\
      \vspace{.5mm}\structure{\myRec} & $\forall \structure{a}\in \Sigma$&\alert{$\structure{a} \in \textsc{regex}_\Sigma$}\\
      \vspace{.5mm}\structure{\myRec}& $\forall \structure{u}, \structure{v} \in \textsc{regex}_\Sigma$&
      $\text{``\example{\underline{$($}}''} \cdot \structure{u} \cdot \text{``\example{\underline{$|$}}''} \cdot \structure{v} \cdot \text{``\example{\underline{$)$}}''}
      \alert{\in \textsc{regex}_\Sigma}$
      & \example{par ex. $(a|b)$}\\
      \vspace{.5mm}\structure{\myRec}& $\forall \structure{u}, \structure{v} \in \textsc{regex}_\Sigma$&
      $\text{``\example{\underline{$($}}''} \cdot \structure{u} \cdot \text{``\example{\underline{$\cdot$}}''} \cdot \structure{v} \cdot \text{``\example{\underline{$)$}}''}
      \alert{\in \textsc{regex}_\Sigma}$
      & \example{par ex. $(a\cdot (b | a))$}\\
      \vspace{.5mm}\structure{\myRec}& $\forall \structure{u} \in \textsc{regex}_\Sigma$&
      $\structure{u} \cdot \text{``\example{\underline{${}^\star$}}''} \alert{\in \textsc{regex}_\Sigma}$
      & \example{par ex. $(a|b)^\star$}\\
    \end{tabular}
  \end{block}

  On donnera une définition plus propre dans le chapitre suivant (page \ref{slide:grammaireRegex}). 
\end{frame}


\endgroup

%% SPDX-License-Identifier: CC-BY-SA-4.0
% Author: Matthieu Perrin
% Part: 
% Section: 
% Sub-section: 
% Frame: 

\begingroup


\begin{frame}{Sémantique des expressions rationnelles}
  Soit $\Sigma$ un alphabet.\\
  Toute expression rationnelle \alert{$r$} sur $\Sigma$ \alert{décrit} un langage \alert{$\mathcal{S}(r) \in \mathscr{P}(\Sigma^\star)$}.

  \begin{block}{Définition -- Sémantique d'une expression rationnelle}
    
    La fonction \alert{$\mathcal{S} : \textsc{regex}_\Sigma \rightarrow \mathscr{P}(\Sigma^\star)$}, appelée \structure{la fonction sémantique} des expressions rationnelle,
    est définie inductivement de la façon suivante.

    $$\structure{
      \mathcal{S} \eqdef \left\{\begin{array}{rcll}
      \textsc{regex}_\Sigma &\rightarrow& \mathscr{P}(\Sigma^\star)\\
      \example{\emptyset} &\mapsto& \alert{\emptyset}\\
      \example{\varepsilon} &\mapsto& \alert{\{\varepsilon\}}\\
      \example{a} &\mapsto& \alert{\{a\}} & \text{si } a \in \Sigma \\
      \example{(u | v)} &\mapsto& \alert{\mathcal{S}(u) \cup \mathcal{S}(v)}  & \text{si } u, v \in \textsc{regex}_\Sigma\\
      \example{(u \cdot v)} &\mapsto& \alert{\mathcal{S}(u) \cdot \mathcal{S}(v)}  & \text{si } u, v \in \textsc{regex}_\Sigma \\
      \example{u^\star} &\mapsto& \alert{\mathcal{S}(u)^\star}  & \text{si } u \in \textsc{regex}_\Sigma\\
      \end{array}\right.}
    $$
  \end{block}

  
  \vspace{-3mm}
  \begin{exampleblock}{Exemples}
  \vspace{-2mm}
  \footnotesize
  \begin{itemize}
    \item $\mathcal{S}(\mathstring{\example{a | b}}) = \mathcal{S}(\mathstring{\example{a}}) \alert{\cup} \mathcal{S}(\mathstring{\example{b}}) = \alert{\{a\} \cup \{b\}}  = \alert{\{a, b\}}$
    \item\vspace{-1mm} $\mathcal{S}(\mathstring{\example{a \cdot (a | b) \cdot b}}) = \mathcal{S}(\example{\mathstring{a}}) \alert{\cdot} \mathcal{S}(\example{\mathstring{a | b}}) \alert{\cdot} \mathcal{S}(\mathstring{\example{b}}) = \alert{\{a\} \cdot \{a, b\} \cdot \{b\}}  = \alert{\{aab, abb\}}$
    \end{itemize}
  \end{exampleblock}
\end{frame}

\endgroup

%% SPDX-License-Identifier: CC-BY-SA-4.0
% Author: Matthieu Perrin
% Part: 
% Section: 
% Sub-section: 
% Frame: 

\begingroup

\begin{frame}{Abus de notation}
  \begin{block}{Simplifications}
    \begin{itemize}
    \item L'opérateur \alert{$\cdot$} est le plus souvent \structure{omis}
    \item On utilise les règles de \alert{préséance} pour éviter les parenthèses inutiles
      \begin{center}
        \structure{1.} \alert{${}^\star$} \hspace{1cm}
        \structure{2.} \alert{$\cdot$} \hspace{1cm}
        \structure{3.} \alert{$|$}
      \end{center}
      \item Par exemple, \example{$ab^\star|cd^\star$} représente \example{$((a \cdot b^\star) | (c \cdot d^\star))$}
    \end{itemize}
  \end{block}

  \begin{block}{Sucre syntaxique}
    \begin{itemize}
    \item Ajout de  l'opérateur $\alert{+}$ : $u^+ \eqdef (u \cdot u^\star)$
    \item Ajout de  l'opérateur $\alert{?}$ : $u^? \eqdef ( u | \varepsilon)$
    \item Beaucoup d'autres, selon les outils
    \end{itemize}
  \end{block}

  \begin{block}{Définition -- Expression rationnelle standard}
    Une expression rationnelle est dite \structure{standard}
    si, et seulement si, les seuls opérateurs utilisés sont les opérateurs $.$, $|$ et ${}^\star$
  \end{block}
\end{frame}


\endgroup

% 
%\subsection{Langages rationnels}
%% SPDX-License-Identifier: CC-BY-SA-4.0
% Author: Matthieu Perrin
% Part: 
% Section: 
% Sub-section: 
% Frame: 

\begingroup

\begin{frame}{Définition des langages rationnels}
  Soient $\Sigma$ un alphabet, et $L \in \mathscr{P}(\Sigma^\star)$ un langage sur $\Sigma$.

  \begin{block}{Définition --- Langage rationnel}
    $L$ est \structure{rationnel}, noté $\alert{L\in \textsc{rat}_\Sigma}$ s'il est décrit par une expression rationnelle :
    $$\alert{\textsc{rat}_\Sigma = \{\mathcal{S}(u) \mid u \in \textsc{regex}_\Sigma\}}$$
  \end{block}

  \begin{exampleblock}{Conséquences}
    \vspace{2mm}
    \begin{tabular}{rlr}
      \vspace{1mm}\example{Langage vide}& &\alert{$\emptyset \in \textsc{rat}_\Sigma$}\\
      \vspace{1mm}\example{Mot vide}& &\alert{$\{ \varepsilon \} \in \textsc{rat}_\Sigma$}\\
      \vspace{1mm}\example{Caractères}& $\forall a\in \Sigma$&\alert{$ \{ a \} \in \textsc{rat}_\Sigma$}\\
      \vspace{1mm}\example{Union}& $\forall L_1, L_2 \in \textsc{rat}_\Sigma$&\alert{$ L_1 \cup L_2 \in \textsc{rat}_\Sigma$}\\
      \vspace{1mm}\example{Produit}& $\forall L_1, L_2 \in \textsc{rat}_\Sigma$&\alert{$ L_1 \cdot L_2 \in \textsc{rat}_\Sigma$}\\
      \vspace{1mm}\example{Fermeture}& $\forall L_1 \in \textsc{rat}_\Sigma$&\alert{$ L_1^\star \in \textsc{rat}_\Sigma$}\\
    \end{tabular}
  \end{exampleblock}
\end{frame}

\endgroup

%% SPDX-License-Identifier: CC-BY-SA-4.0
% Author: Matthieu Perrin
% Part: 
% Section: 
% Sub-section: 
% Frame: 

\begingroup

\begin{frame}{Exemple d'utilisation du lemme de pompage}
  
  \tfBlock[top=-5mm]{Montrer que $L \eqdef \{a^nb^nc^n \mid n\in \mathbb{N}\}$ n'est pas algébrique}{
    Soit $\Sigma \eqdef \{a, b, c\}$.%
    \only<2-|handout>{
      Si $L$ est algébrique, $L$ vérifie le lemme de pompage :

      \vspace{-4mm}
      $$
      \begin{array}{c}
        \structure{\exists N\in \mathbb{N}}, \alert{\forall u\in L, |u| \ge N} \Rightarrow (\structure{\exists v, w, x, y, z\in \Sigma^\star}, \\
        u = v \cdot w \cdot x\cdot y \cdot z \land w \cdot y\neq \varepsilon \land |w\cdot x\cdot y| \le N \land \alert{\forall i \in \mathbb{N}}, v\cdot w^i\cdot x\cdot y^i \cdot z \in L)
      \end{array}
      $$
      \vspace{-2mm}
      
      \structure{Soit $N$ donné par le lemme pompage}.
    }
    
    \only<3-|handout>{%
      \alert{Posons $u = a^N b^N c^N$. On a bien $u\in L$ et $|u| = 3N \ge N$}.\\
      \structure{Soit $v \cdot w \cdot x\cdot y \cdot z$ la décomposition de $u$ donnée par le lemme de pompage}. 
    }
  }
  
  \tf<4-|handout>[y=-10mm]{
    \begin{smArray}[width=4mm, height=3mm, name={$u=$}]
      \smCell[\smNone]{\alert{$a$}}      \smCoord{(a1)}
      \smCell[\smNone]{\alert{$\cdots$}}               
      \smCell[\smNone]{\alert{$a$}}      \smCoord{(an)}
      \smCell[\smNone]{\alert{$\cdot$}}               
      \smCell[\smNone]{\alert{$b$}}      \smCoord{(b1)}
      \smCell[\smNone]{\alert{$\cdots$}}               
      \smCell[\smNone]{\alert{$b$}}      \smCoord{(bn)}
      \smCell[\smNone]{\alert{$\cdot$}}               
      \smCell[\smNone]{\alert{$c$}}      \smCoord{(c1)}
      \smCell[\smNone]{\alert{$\cdots$}}               
      \smCell[\smNone]{\alert{$c$}}      \smCoord{(cn)}
      
      \draw [decorate, decoration={brace, amplitude=5pt}] ([xshift=1mm]a1.north west) -- ([xshift=-1mm]an.north east) node[midway,yshift=4mm]{$N$};
      \draw [decorate, decoration={brace, amplitude=5pt}] ([xshift=1mm]b1.north west) -- ([xshift=-1mm]bn.north east) node[midway,yshift=4mm]{$N$};
      \draw [decorate, decoration={brace, amplitude=5pt}] ([xshift=1mm]c1.north west) -- ([xshift=-1mm]cn.north east) node[midway,yshift=4mm]{$N$};

      \draw [decorate, decoration={brace, amplitude=5pt, mirror}] ([xshift=1mm]an.south west) -- ([xshift=-1mm]b1.south east) node[midway,yshift=-4mm]{$|wxy|\le N$};
      \draw [decorate, decoration={brace, amplitude=5pt, mirror}] ([xshift=1mm]bn.south west) -- ([xshift=-1mm]c1.south east) node[midway,yshift=-4mm]{$|wxy|\le N$};
    \end{smArray}       
  }

  \tf<5-|handout>[text, bottom=-1mm]{
    \begin{itemize}
    \item Comme $wy \neq \varepsilon$, $\alpha = wy[1] \in \Sigma$ est une lettre de $wy$.
    \item Comme $|wxy| \le N$, il existe $\beta \in \Sigma$ tel que $\beta$ n'est pas une lettre de $wy$. 
    \item \alert{Posons $i=2$}. $|v \cdot w^2 \cdot x\cdot  y^2\cdot   z|_\alpha > |v\cdot  w^2\cdot  x\cdot  y^2\cdot   z|_\beta$, donc $v\cdot  w^2\cdot  x\cdot  y^2\cdot   z\notin L$. 
    \end{itemize}
    Absurde ! Donc $L$ n'est pas algébrique. 
  }

  \tfExampleBlock<-4>[y=-25mm]{On sait}{}

  \tf<1>[y=-25mm, anchor=north, text]{
    \begin{itemize}
    \item $\begin{array}[t]{l}
      \alert{\forall L \in \textsc{alg}_\Sigma}, \structure{\exists N\in \mathbb{N}}, \forall u\in L, |u| \ge N \Rightarrow (\exists v, w, x, y, z\in \Sigma^\star, \\
      u = v w x y z \land w y\neq \varepsilon \land |wxy| \le N \land \forall i \in \mathbb{N}, v w^i x y^i z \in L)
    \end{array}$
    \end{itemize}
  }

  \tf<2-4>[y=-25mm, anchor=north, left=.35\textwidth]{
    \begin{itemize}
    \item $L \in \textsc{alg}_\Sigma$
    \item<3-> $v, w, x, y, z\in \Sigma^\star$
    \end{itemize}
  }
  
  \tf<2-4>[y=-25mm, anchor=north, width=.35\textwidth]{
    \begin{itemize}
    \item $N \in \mathbb{N}$
    \item<3-> $wy\neq \varepsilon$
    \end{itemize}
  }
  
  \tf<3-4>[y=-25mm, anchor=north, right=.35\textwidth]{
    \begin{itemize}
    \item $v \cdot w \cdot x \cdot y \cdot z = a^N b^N c^N$
    \item $\alert{|wxy| \le N}$
    \end{itemize}
  }
  
\end{frame}

\endgroup

%% SPDX-License-Identifier: CC-BY-SA-4.0
% Author: Matthieu Perrin
% Part: 
% Section: 
% Sub-section: 
% Frame: 

\begingroup

\begin{frame}{Équivalence entre expressions rationnelles}

  \begin{block}{Définition -- Équivalence entre expressions rationnelles}
    Deux expressions rationnelles \alert{$u$} et \alert{$v$} sont dites \structure{équivalentes}
    (ou par abus de langage ``égales''), noté \alert{$u \equiv v$},
    si elles décrivent le même langage.
    $$\alert{u \equiv v ~\eqdef~ \mathcal{L}(u) = \mathcal{L}(v)}$$
  \end{block}

  \vspace{-2mm}
  \begin{exampleblock}{Exemples}
    \begin{itemize}
    \item \example{$a^\star \mid b^\star \not\equiv (a \mid b)^\star$} : $ab \in \mathcal{L}((a \mid b)^\star)$ mais $ab \not\in \mathcal{L}(a^\star \mid  b^\star)$
    \item \example{$(a^\star \mid b^\star)^\star \equiv (a \mid b)^\star$} : $\mathcal{L}((a^\star \mid b^\star)^\star) = \mathcal{L}((a \mid b)^\star) = \{a, b\}^\star$
    \end{itemize}
  \end{exampleblock}

  \begin{block}{Rappel -- Relation d'équivalence}
    L'équivalence est bien une relation d'équivalence, car elle vérifie :
    \begin{description}
    \item[Réflexivité :] $\forall u\in\textsc{regex}_\Sigma, u\equiv u$
    \item[Symétrie :]    $\forall u, v\in\textsc{regex}_\Sigma, u\equiv v \Rightarrow v\equiv u$
    \item[Transitivité :]$\forall u, v, w\in\textsc{regex}_\Sigma, (u\equiv v \land v \equiv w) \Rightarrow u\equiv w$
    \end{description}
  \end{block}
  
\end{frame}

\endgroup

%% SPDX-License-Identifier: CC-BY-SA-4.0
% Author: Matthieu Perrin
% Part: 
% Section: 
% Sub-section: 
% Frame: 

\begingroup

\begin{frame}{Équivalence entre expressions rationnelles}

  Mêmes propriétés que les langages.
  
  Soient $\Sigma$ un alphabet et $u$, $v$ et $w$ des expressions rationnelles. 
  
  \begin{itemize}
  \item Union
    \begin{description}[commutativité :]
    \item[neutralité :] $\emptyset \mid  u \equiv u \equiv u \mid  \emptyset$ 
    \item[associativité :] $u \mid  (v \mid  w) \equiv (u \mid  v) \mid  w$ \quad \structure{noté $u\mid v\mid w$} 
    \item[commutativité :] $u \mid  v \equiv v \mid  u$
    \item[idempotence :] $u \mid  u \equiv u$
    \end{description}
  \item Concaténation 
    \begin{description}[commutativité :]
    \item[neutralité :] $\varepsilon \cdot u \equiv u \equiv u \cdot \varepsilon$
    \item[associativité :] $u \cdot (v \cdot w) \equiv (u \cdot v) \cdot w$  \quad \structure{noté $u\cdot v\cdot w$ ou $uvw$}
    \item[absorption :] $u \cdot \emptyset \equiv \emptyset \equiv \emptyset \cdot u$
    \end{description}
  \item Mise à l'étoile
    \begin{description}[commutativité :]
    \item[neutralité :] $\emptyset^\star \equiv \varepsilon$
    \item[idempotence :] $u^{\star^\star} \equiv u^\star$
    \end{description}
  \item Intéraction entre opérateurs 
    \begin{description}[commutativité :]
    \item[distributivité :] $u \cdot (v \mid  w) \equiv u\cdot v \mid  u \cdot w$ \quad \structure{et} \quad $(v \mid  w)\cdot u \equiv v\cdot u \mid  w\cdot u$
    \item[absorption :] $u \mid  u^\star \equiv u^\star \equiv u^\star \mid  u$
    \item[fermetures :] $u^\star \equiv \varepsilon \mid  u^+$
    \end{description}
  \end{itemize}

\end{frame}


\endgroup

% 
%\subsection{Définitions rationnelles}
%% SPDX-License-Identifier: CC-BY-SA-4.0
% Author: Matthieu Perrin
% Part: 
% Section: 
% Sub-section: 
% Frame: 

\begingroup

\begin{frame}{Exemple d'utilisation du lemme de pompage}
  
  \tfBlock[top=-5mm]{Montrer que $L \eqdef \{a^nb^nc^n \mid n\in \mathbb{N}\}$ n'est pas algébrique}{
    Soit $\Sigma \eqdef \{a, b, c\}$.%
    \only<2-|handout>{
      Si $L$ est algébrique, $L$ vérifie le lemme de pompage :

      \vspace{-4mm}
      $$
      \begin{array}{c}
        \structure{\exists N\in \mathbb{N}}, \alert{\forall u\in L, |u| \ge N} \Rightarrow (\structure{\exists v, w, x, y, z\in \Sigma^\star}, \\
        u = v \cdot w \cdot x\cdot y \cdot z \land w \cdot y\neq \varepsilon \land |w\cdot x\cdot y| \le N \land \alert{\forall i \in \mathbb{N}}, v\cdot w^i\cdot x\cdot y^i \cdot z \in L)
      \end{array}
      $$
      \vspace{-2mm}
      
      \structure{Soit $N$ donné par le lemme pompage}.
    }
    
    \only<3-|handout>{%
      \alert{Posons $u = a^N b^N c^N$. On a bien $u\in L$ et $|u| = 3N \ge N$}.\\
      \structure{Soit $v \cdot w \cdot x\cdot y \cdot z$ la décomposition de $u$ donnée par le lemme de pompage}. 
    }
  }
  
  \tf<4-|handout>[y=-10mm]{
    \begin{smArray}[width=4mm, height=3mm, name={$u=$}]
      \smCell[\smNone]{\alert{$a$}}      \smCoord{(a1)}
      \smCell[\smNone]{\alert{$\cdots$}}               
      \smCell[\smNone]{\alert{$a$}}      \smCoord{(an)}
      \smCell[\smNone]{\alert{$\cdot$}}               
      \smCell[\smNone]{\alert{$b$}}      \smCoord{(b1)}
      \smCell[\smNone]{\alert{$\cdots$}}               
      \smCell[\smNone]{\alert{$b$}}      \smCoord{(bn)}
      \smCell[\smNone]{\alert{$\cdot$}}               
      \smCell[\smNone]{\alert{$c$}}      \smCoord{(c1)}
      \smCell[\smNone]{\alert{$\cdots$}}               
      \smCell[\smNone]{\alert{$c$}}      \smCoord{(cn)}
      
      \draw [decorate, decoration={brace, amplitude=5pt}] ([xshift=1mm]a1.north west) -- ([xshift=-1mm]an.north east) node[midway,yshift=4mm]{$N$};
      \draw [decorate, decoration={brace, amplitude=5pt}] ([xshift=1mm]b1.north west) -- ([xshift=-1mm]bn.north east) node[midway,yshift=4mm]{$N$};
      \draw [decorate, decoration={brace, amplitude=5pt}] ([xshift=1mm]c1.north west) -- ([xshift=-1mm]cn.north east) node[midway,yshift=4mm]{$N$};

      \draw [decorate, decoration={brace, amplitude=5pt, mirror}] ([xshift=1mm]an.south west) -- ([xshift=-1mm]b1.south east) node[midway,yshift=-4mm]{$|wxy|\le N$};
      \draw [decorate, decoration={brace, amplitude=5pt, mirror}] ([xshift=1mm]bn.south west) -- ([xshift=-1mm]c1.south east) node[midway,yshift=-4mm]{$|wxy|\le N$};
    \end{smArray}       
  }

  \tf<5-|handout>[text, bottom=-1mm]{
    \begin{itemize}
    \item Comme $wy \neq \varepsilon$, $\alpha = wy[1] \in \Sigma$ est une lettre de $wy$.
    \item Comme $|wxy| \le N$, il existe $\beta \in \Sigma$ tel que $\beta$ n'est pas une lettre de $wy$. 
    \item \alert{Posons $i=2$}. $|v \cdot w^2 \cdot x\cdot  y^2\cdot   z|_\alpha > |v\cdot  w^2\cdot  x\cdot  y^2\cdot   z|_\beta$, donc $v\cdot  w^2\cdot  x\cdot  y^2\cdot   z\notin L$. 
    \end{itemize}
    Absurde ! Donc $L$ n'est pas algébrique. 
  }

  \tfExampleBlock<-4>[y=-25mm]{On sait}{}

  \tf<1>[y=-25mm, anchor=north, text]{
    \begin{itemize}
    \item $\begin{array}[t]{l}
      \alert{\forall L \in \textsc{alg}_\Sigma}, \structure{\exists N\in \mathbb{N}}, \forall u\in L, |u| \ge N \Rightarrow (\exists v, w, x, y, z\in \Sigma^\star, \\
      u = v w x y z \land w y\neq \varepsilon \land |wxy| \le N \land \forall i \in \mathbb{N}, v w^i x y^i z \in L)
    \end{array}$
    \end{itemize}
  }

  \tf<2-4>[y=-25mm, anchor=north, left=.35\textwidth]{
    \begin{itemize}
    \item $L \in \textsc{alg}_\Sigma$
    \item<3-> $v, w, x, y, z\in \Sigma^\star$
    \end{itemize}
  }
  
  \tf<2-4>[y=-25mm, anchor=north, width=.35\textwidth]{
    \begin{itemize}
    \item $N \in \mathbb{N}$
    \item<3-> $wy\neq \varepsilon$
    \end{itemize}
  }
  
  \tf<3-4>[y=-25mm, anchor=north, right=.35\textwidth]{
    \begin{itemize}
    \item $v \cdot w \cdot x \cdot y \cdot z = a^N b^N c^N$
    \item $\alert{|wxy| \le N}$
    \end{itemize}
  }
  
\end{frame}

\endgroup

%% SPDX-License-Identifier: CC-BY-SA-4.0
% Author: Matthieu Perrin
% Part: 
% Section: 
% Sub-section: 
% Frame: 

\begingroup

\begin{frame}{Définition des langages rationnels}
  Soient $\Sigma$ un alphabet, et $L \in \mathscr{P}(\Sigma^\star)$ un langage sur $\Sigma$.

  \begin{block}{Définition --- Langage rationnel}
    $L$ est \structure{rationnel}, noté $\alert{L\in \textsc{rat}_\Sigma}$ s'il est décrit par une expression rationnelle :
    $$\alert{\textsc{rat}_\Sigma = \{\mathcal{S}(u) \mid u \in \textsc{regex}_\Sigma\}}$$
  \end{block}

  \begin{exampleblock}{Conséquences}
    \vspace{2mm}
    \begin{tabular}{rlr}
      \vspace{1mm}\example{Langage vide}& &\alert{$\emptyset \in \textsc{rat}_\Sigma$}\\
      \vspace{1mm}\example{Mot vide}& &\alert{$\{ \varepsilon \} \in \textsc{rat}_\Sigma$}\\
      \vspace{1mm}\example{Caractères}& $\forall a\in \Sigma$&\alert{$ \{ a \} \in \textsc{rat}_\Sigma$}\\
      \vspace{1mm}\example{Union}& $\forall L_1, L_2 \in \textsc{rat}_\Sigma$&\alert{$ L_1 \cup L_2 \in \textsc{rat}_\Sigma$}\\
      \vspace{1mm}\example{Produit}& $\forall L_1, L_2 \in \textsc{rat}_\Sigma$&\alert{$ L_1 \cdot L_2 \in \textsc{rat}_\Sigma$}\\
      \vspace{1mm}\example{Fermeture}& $\forall L_1 \in \textsc{rat}_\Sigma$&\alert{$ L_1^\star \in \textsc{rat}_\Sigma$}\\
    \end{tabular}
  \end{exampleblock}
\end{frame}

\endgroup

% 
\section{Automates finis}
 
%\subsection{Reconnaissance d'un langage rationnel}
%% SPDX-License-Identifier: CC-BY-SA-4.0
% Author: Matthieu Perrin
% Part: 
% Section: 
% Sub-section: 
% Frame: 

\begingroup

\begin{frame}{Génération d'analyseur lexical}
  \begin{block}{Problème}
    \vspace{-2mm}
    \begin{description}
    \item[Entrée :] une expression rationnelle $r$
    \item[Sortie :] un \alert{analyseur lexical} pour le langage $\mathcal{S}(r)$
      \begin{itemize}
      \item Programme qui décide si son entrée appartient à $\mathcal{S}(r)$
      \end{itemize}
    \end{description}
  \end{block}
  
  \centering
  \scalebox{.9}{\begin{tikzpicture}

      \draw[white] (-1.8,0) rectangle (9.8,6.2);
      
      \draw[rounded corners, structure,fill=structure!20] (0,5) +(-1.2,-.5) rectangle +(1.2,.5) +(0,0) node{\small \begin{tabular}{c}Expression\\ rationnelle \end{tabular}};
      \draw[rounded corners, structure,fill=structure!20] (8,5) +(-1.2,-.5) rectangle +(1.2,.5) +(0,0) node{\small \begin{tabular}{c}Analyseur\\ lexical \end{tabular}};
      \draw[rounded corners, structure,fill=structure!20] (0,3) +(-1.2,-.5) rectangle +(1.2,.5) +(0,0) node{\small \begin{tabular}{c}Automate fini\\ non-déterministe \end{tabular}};
      \draw[rounded corners, alert,fill=alert!20] (4,3) +(-1.2,-.5) rectangle +(1.2,.5) +(0,0) node{\small \begin{tabular}{c}Automate fini\\ déterministe \end{tabular}};
      \draw[rounded corners, structure,fill=structure!20] (8,3) +(-1.2,-.5) rectangle +(1.2,.5) +(0,0) node{\small \begin{tabular}{c}Automate fini\\ minimal \end{tabular}};
      
      \draw[dashed, -latex] (1.2,5) -- (6.8,5);
      \draw[-latex, structure] (0,4.5) -- (0,3.5);
      \draw[-latex,alert] (1.2,3) -- (2.8,3);
      \draw[-latex] (5.2,3) -- (6.8,3);
      \draw[-latex, structure] (8,3.5) -- (8,4.5);

      \draw[example] (0,5.5) node[above]{$a (b|c)^\star$};
      \draw[example] (8,5.5) node[above]{$abc \rightarrow \cmark$, $bac \rightarrow \xmark$ };

      \draw[structure] (0,4.1) node[right]{\tiny Algorithme de};
      \draw[structure] (0,3.9) node[right]{\tiny Thompson};

      \draw[alert] (2,3.15) node{\tiny Sous-ensembles de};
      \draw[alert] (2,2.85) node{\tiny Rabin \& Scott};

      \draw (6,3.15) node{\tiny Méthode de};
      \draw (6,2.85) node{\tiny Moore};

      \draw[structure] (8,4) node[left]{\tiny Transcription};

      \draw[example] (0,2.5) node[below]{\scalebox{.5}{\begin{tikzpicture}
            \node (a) {};
            \node (b) [above=of a] {};
            \node (c) [right=of b] {};
            \node (d) [below=of c] {};
            \node (e) [above right=of d] {};
            \node (f) [right=of e] {};
            \node (g) [below right=of d] {};
            \node (h) [right=of g] {};
            \node (i) [below right=of f] {};
            \node (j) [below=of d] {};

            \node[fill=example!10,state,initial, initial text=] (a1) at (a) {$1$};
            \node[fill=example!10,state] (b1) at (b) {$2$};
            \node[fill=example!10,state] (e1) at (e) {$3$};
            \node[fill=example!10,state] (f1) at (f) {$4$};
            \node[fill=example!10,state] (g1) at (g) {$5$};
            \node[fill=example!10,state] (h1) at (h) {$6$};
            \node[fill=example!10,state] (d1) at (d) {$7$};
            \node[fill=example!10,state] (i1) at (i) {$8$};
            \node[fill=example!10,state] (c1) at (c) {$9$};
            \node[fill=example!10,state, accepting] (j1) at (j) {$10$};

            \path[->]  (a1) edge node[left] {$a$} (b1);
            \path[->]  (e1) edge node[below] {$b$} (f1);
            \path[->]  (g1) edge node[above] {$c$} (h1);
            \path[->]  (d1) edge node[right] {$\varepsilon$} (e1);
            \path[->]  (d1) edge node[right] {$\varepsilon$} (g1);
            \path[->]  (f1) edge node[left ] {$\varepsilon$} (i1);
            \path[->]  (h1) edge node[left ] {$\varepsilon$} (i1);
            \path[->]  (i1) edge node[above] {$\varepsilon$} (d1);
            \path[->]  (c1) edge node[left] {$\varepsilon$} (d1);
            \path[->]  (d1) edge node[left ] {$\varepsilon$} (j1);
            \path[->]  (b1) edge node[above] {$\varepsilon$} (c1);
      \end{tikzpicture}}};

      \draw[example] (4,2.5) node[below]{\scalebox{.5}{\begin{tikzpicture}
            \node[fill=example!10,state,initial, initial text=] (a) {$1$};
            \node[fill=example!10,state, accepting] (b) [below=of a] {$2$};
            \node[fill=example!10,state, accepting] (c) [above right=of b] {$3$};
            \node[fill=example!10,state, accepting] (d) [below right=of b] {$4$};

            \path[->]  (a) edge node[left] {$a$} (b);
            \path[->]  (b) edge node[above] {$b$} (c);
            \path[->]  (b) edge node[below] {$c$} (d);
            \path[->]  (c) edge[bend right=5mm] node[left] {$c$} (d);
            \path[->]  (d) edge[bend right=5mm] node[right] {$b$} (c);
            \path[->]  (c) edge[loop right, looseness=5] node {$b$} (c);
            \path[->]  (d) edge[loop right, looseness=5] node {$c$} (d);
      \end{tikzpicture}}};
      \draw[example] (8,2.5) node[below]{\scalebox{.5}{\begin{tikzpicture}
            \node[fill=example!10,state,initial, initial text=] (a) {$1$};
            \node[fill=example!10,state, accepting] (b) [below=of a] {$2$};

            \path[->]  (a) edge node[left] {$a$} (b);
            \path[->]  (b) edge[loop left, looseness=5] node {$b$} (b);
            \path[->]  (b) edge[loop right, looseness=5] node {$c$} (b);
      \end{tikzpicture}}};
  \end{tikzpicture}}
\end{frame}

\endgroup

%% SPDX-License-Identifier: CC-BY-SA-4.0
% Author: Matthieu Perrin
% Part: 
% Section: 
% Sub-section: 
% Frame: 

\begingroup

\SetKwFunction{Decide}{decider}
\SetKwData{Mot}{u}
\SetKwData{VuA}{vuA}
\SetKwData{VuB}{vuB}

\begin{frame}{Exemple d'algorithme d'analyse lexicale}

  \tf[text,y=1.5]{
    \begin{algorithm}[H]
      \Fn{\Decide(\Mot : mot) : booléen}{
        $\tfAlert<2>{\alert<2->\VuA} \leftarrow \tfAlert<3>{\False}$, $\tfExample<2>{\example<2->\VuB} \leftarrow \tfAlert<3>{\False}$\;
        \Pour{$i$ de $1$ à $|\Mot|$}{
          \uSi{\tfStructure<4,5>{$\Mot[i] = \text{`a'} \land \lnot \VuB$}}{
            \tfAlert<4>{$\VuA \leftarrow \True$}\;
          }\uSinonSi{\tfStructure<6-8>{$\Mot[i]=\text{`b'} \land \VuA$}}{
            \tfAlert<6>{$\VuB \leftarrow \True$}\;
          }\lSinon{\Retourner \False}
        }
        \Retourner \tfAlert<9>{$\VuA \land \VuB$}\; 
      }
    \end{algorithm}
  }

  \tfBlock[text,y=-2.3]{Représentation graphique}{
    \begin{tikzpicture}[smAutomaton]
      \uncover<2-|handout>{
        \smState               at  (1.1,1.8) {$q$}; 
        \node[right]           at  (1.5,1.8) {état de l'algorithme}; 
      }
      \uncover<3-|handout>{
        \smState[\smInitial]   at (1.1,0.9) {$i$}; 
        \node[right]           at (1.5,0.9) {initial}; 
      }
      \uncover<4-|handout>{
        \smState       (start) at (1.1,0)   {$q_1$}; 
        \smState       (end)   at (2.7,0)   {$q_2$}; 
        \node[right]           at (3.1,0)   {transition}; 
        \smPath  (start) edge node {a} (end);
      }
      \uncover<9-|handout>{
        \smState[\smAccepting] at (3.6,0.9) {$f$}; 
        \node[right]           at (4.0,0.9) {final}; 
      }
    \end{tikzpicture}
  }
  
  \tfBlock[right=4cm, y=3] {Expression rationnelle} {
    \vspace{-4mm}
    \[\textstyle a^+ b^+ \]
  }
  
  \tfBlock<2|handout>[right=4cm,y=2,anchor=north]{Variables}{
    \begin{itemize}
    \item \alert{\VuA} : a-t-on vu un $a$ ?
    \item \example{\VuB} : a-t-on vu un $b$ ?
    \end{itemize}
  }

  
  \tfBlock<3->[right=4cm,y=2,anchor=north]{Exemple d'exécution}{
    \centering
    \begin{smArray}[size=5mm, headsize=2mm, name={\Mot :}]
      \smHead<3> 
      \smCell         {a} \smHead<4> 
      \smCell         {a} \smHead<5> 
      \smCell         {b} \smHead<6> 
      \smCell         {b} \smHead<7> 
      \smCell         {b} \smHead<8> 
      \smCell[\smNone]{ } \smHead<9> 
    \end{smArray}
    \uncover<9>{\alert{Mot reconnu}}
  }

  \tf<2-|handout>[x=3,y=-2.3cm]{
    \begin{smAutomaton}
      \draw[opacity=0] (-1.5,-1) rectangle (3,3);
      
      \smState[\smInitial<3-|handout> \smAlert<3>  \smStructure<4>] (00) at (0.0,1.5) {\alert{0}\example{0}}; 
      \smState[                       \smAlert<4-5>\smStructure<6>] (10) at (1.5,1.5) {\alert{1}\example{0}}; 
      \smState[\smAccepting<9|handout>\smAlert<6->                ] (11) at (1.5,0.0) {\alert{1}\example{1}};
      \smState[                                                   ] (01) at (0.0,0.0) {\alert{0}\example{1}};
      
      \smPath<4->[\smStructure<4>]   (00) edge             node {$a$} (10);
      \smPath<5->[\smStructure<5>]   (10) edge[loop above] node {$a$} (10); 
      \smPath<6->[\smStructure<6>]   (10) edge             node {$b$} (11); 
      \smPath<7->[\smStructure<7-8>] (11) edge[loop right] node {$b$} (11); 
    \end{smAutomaton}
  }
  
\end{frame}

\endgroup

%% SPDX-License-Identifier: CC-BY-SA-4.0
% Author: Matthieu Perrin
% Part: 
% Section: 
% Sub-section: 
% Frame: 

\begingroup

\SetKwFunction{Decider}{decider}
\SetKwFunction{Afficher}{Ecrire}
\SetKwData{Mot}{u}
\SetKwData{Init}{initial\_state}
\SetKwData{Final}{final\_state}
\SetKwData{Transition}{transitions}

\begin{frame}{Implémentation d'un automate déterministe complet}

  \tfBlock[top=-5mm]{Modélisation de l'automate par une structure de données}{
    \begin{description}
    \item[$\Init$] : entier (l'état au début de la reconnaissance)
    \item[{$\Final$}] : tableau de booléens
      \begin{itemize}
      \item $\Final[q] = \True$ si le mot est accepté dans l'état $q$
      \end{itemize}
    \item[{$\Transition$}] : tableau d'entiers à deux dimensions
      \begin{itemize}
      \item $\Transition[c, q]$ : état obtenu quand $c$ est lu dans $q$
      \end{itemize}
    \end{description}
  }

  \tfExampleBlock[y=-2mm]{Exemple}{}

  \tf[text, bottom=-2mm]{
    \scalebox{.85}{\begin{algorithm}[H]
        \Algo{exemple}{
          $A \leftarrow \left\{\begin{array}{lcl}
          \Init &\leftarrow& 0 \\
          \Final &\leftarrow& \left[\True, \True, \False \right]\\
          \Transition &\leftarrow& \left[ \begin{array}{ll} a :& \left[0, 2, 2 \right] \\b :& \left[1, 1, 2 \right] \end{array}\right]\\
          \end{array}
          \right\}
          $\;
          $\Afficher(\Decider(\text{``aabb''}, A))$\;
        }
 
        \vspace{1mm}
        \Fn{\Decider($u$ : mot, $A$ : automate) : booléen}{
          $q \leftarrow A.\Init$\;
          \lPour{$i$ de $1$ à $|u|$}{
            $q \leftarrow A.\Transition[u[i], q]$%\;
          }
          \Retourner $A.\Final[q]$\;
        }
      \end{algorithm}}
    }
     
    \tf[x=.3\textwidth, y=-1cm]{
      \begin{tikzpicture}[smAutomaton, node distance=1.5cm]
        \smState[\smInitial, \smAccepting] (0)              {0}; 
        \smState[\smAccepting]             (1) [right=of 0] {1}; 
        \smState                           (2) [right=of 1] {2}; 
        
        \smPath (0) edge[loop above] node{$a$}    (0);
        \smPath (0) edge             node{$b$}    (1);
        \smPath (1) edge             node{$a$}    (2);
        \smPath (1) edge[loop above] node{$b$}    (1);
        \smPath (2) edge[loop above] node{$a, b$} (2);
      \end{tikzpicture}
    }

\end{frame}


\endgroup

%% SPDX-License-Identifier: CC-BY-SA-4.0
% Author: Matthieu Perrin
% Part: 
% Section: 
% Sub-section: 
% Frame: 

\begingroup

\begin{frame}{Génération d'analyseur lexical}

  \onBlock[top=-5mm]{Problème}{
    \begin{description}
    \item[Entrée :] une expression rationnelle $r$
    \item[Sortie :] un \alert{analyseur lexical} pour le langage $\mathcal{L}(r)$
      \begin{itemize}
      \item Programme qui décide si son entrée appartient à $\mathcal{L}(r)$
      \end{itemize}
    \end{description}
  }

  \on[y=-5mm]{\footnotesize
    \begin{tikzpicture}
      \small
      \node[smBox, minimum width=2.4cm, minimum height=1cm] (exp) at (0,1.7) {Expression\\rationnelle};
      \node[smBox, minimum width=2.4cm, minimum height=1cm] (lex) at (8,1.7) {Analyseur\\lexical};
      \node[smBox, minimum width=2.4cm, minimum height=1cm, alert, fill=alert!20] (dfa) at (4,0.0) {Automate fini\\déterministe};
      \node[example, above] at (exp.north) {$a (b|c)^\star$};
      \node[example, above] at (lex.north) {$abc \rightarrow \cmark$, $bac \rightarrow \xmark$ };

      \uncoverb<1>{
        \draw[-latex, alert, rounded corners] (exp.south) -- (0,0) -- node[above]{??} (dfa.west);
        \draw[-latex, alert, rounded corners] (dfa.east)  -- node[above]{??} (8,0) -- (lex.south);
      }
      
      \uncover<2>{
        \small
        \node[smBox, minimum width=2.4cm, minimum height=1cm] (nfa) at (0,0.0) {Automate fini\\non-déterministe};
        \node[smBox, minimum width=2.4cm, minimum height=1cm] (min) at (8,0.0) {Automate fini\\minimal};

        \tiny
        \path[-latex,dashed] (exp) edge                                                                        (lex);
        \path[-latex]        (exp) edge[bend left] node[right, align=left] {Algorithme de\\Thompson}           (nfa);
        \path[-latex]        (nfa) edge[bend left] node[left, align=right] {Lemme\\d'Arden}                    (exp);
        \path[-latex]        (nfa) edge            node[align=center]      {Sous-ensembles\\de Rabin \& Scott} (dfa);
        \path[-latex]        (dfa) edge            node[align=center]      {Méthode de\\Moore}                 (min);
        \path[-latex]        (min) edge            node[left, align=right] {Transcription}                     (lex);
      }
    \end{tikzpicture}
  }

  \onBlock<2>[bottom]{Nécessité de monter en abstraction}{
    \begin{itemize}
    \item Notion d'automate non-déterministe
    \end{itemize}
  }

\end{frame}

\endgroup

% 
%\subsection{Définition des automates finis}
%% SPDX-License-Identifier: CC-BY-SA-4.0
% Author: Matthieu Perrin
% Part: 
% Section: 
% Sub-section: 
% Frame: 

\begingroup

\begin{frame}{Généralisation : automates finis non-déterministes}

  Pour un mot donné, il peut exister plusieurs chemins dans l'automate
  \begin{enumerate}
  \item \structure{Plusieurs états initiaux} : l'automate \textit{décide} dans lequel il commence.
  \item \alert{Plusieurs transitions avec même origine et étiquette} : l'automate \textit{décide} laquelle il tire.
  \item {\color{example} $\varepsilon$-transition} : transition qui n'utilise aucune entrée.
    C'est une transition \textit{spontanée} : l'automate \textit{décide} de changer d'état sans lire de symbole.
  \end{enumerate}
  
  \begin{exampleblock}{Exemple}% : $(a^+ | b^+) c^\star$}
    \centering  
    \begin{tikzpicture}[smAutomaton]
      \smState[\smInitial \smStructure] (s1) at (0,1.5)  {$s_1$}; 
      \smState                          (s2) at (2,1.5)  {$s_2$}; 
      \smState[\smAccepting]            (s3) at (4,.75) {$s_3$}; 
      \smState[\smInitial \smStructure] (s4) at (0,0.0)  {$s_4$}; 
      \smState                          (s5) at (2,0.0)  {$s_5$}; 

      \smPath[\smAlert] (s1) edge             node[swap]  {$a$}           (s2);
      \smPath[\smAlert] (s1) edge[loop above] node        {$a$}           (s1);
      \smPath (s2) edge[loop above] node        {$a$}           (s2);
      \smPath[\smAlert] (s4) edge             node[swap]  {$b$}           (s5);
      \smPath[\smAlert] (s4) edge[loop below] node        {$b$}           (s4);
      \smPath (s5) edge[loop below] node        {$b$}           (s5);
      \smPath (s3) edge[loop right] node        {$c$}           (s3);
      \smPath[\smExample] (s2) edge             node        {$\varepsilon$} (s3);
      \smPath[\smExample] (s5) edge             node[swap]  {$\varepsilon$} (s3);
    \end{tikzpicture}
  \end{exampleblock}

\end{frame}

\endgroup

%% SPDX-License-Identifier: CC-BY-SA-4.0
% Author: Matthieu Perrin
% Part: 
% Section: 
% Sub-section: 
% Frame: 

\begingroup

\begin{frame}{Interprétation ubiquitaire du non-déterminisme}

  \onBlock[top=-3mm]{Interprétation du non-déterminisme comme de l'ubiquité}{
    \begin{itemize}
    \item L'automate se trouve dans un sous-ensemble des états
    \item Le mot est reconnu si l'un des états du sous-ensemble est final
    \item Ces sous-ensembles forment un nouvel automate, qui est déterministe
    \end{itemize}
  }

  \onExampleBlock[y=1mm]{Exemple : reconnaissance de $\alert{abc}$ par l'automate suivant}{
    \vspace{-2mm}
    $$
    \alertb<1>{\{1\}}
    \uncover<2->{\xrightarrow{a} \alert<2> {\{2, 3, 5, 7, 9, 10\}}}
    \uncover<3->{\xrightarrow{b} \alertb<3>{\{3, 4, 5, 7, 8, 10\}}}
    \uncover<4->{\xrightarrow{c} \alertb<4>{\{3, 5, 6, 7, 8, 10\}}}
    $$
  }
  
  \on[bottom] {
    \begin{tikzpicture}[automaton]
      \state[initial,   alert ob=<1> ] (1)  at (0,2) {$1$};    
      \state[           alert on=<2> ] (2)  at (1,2) {$2$};    
      \state[           alert on=<2->] (3)  at (4,2) {$3$};    
      \state[           alert ob=<3> ] (4)  at (5,2) {$4$};    
      \state[           alert on=<2->] (5)  at (4,0) {$5$};    
      \state[           alert ob=<4->] (6)  at (5,0) {$6$};    
      \state[           alert on=<2->] (7)  at (3,1) {$7$};    
      \state[           alert ob=<3->] (8)  at (6,1) {$8$};    
      \state[           alert on=<2> ] (9)  at (2,2) {$9$};    
      \state[accepting, alert on=<2->] (10) at (2,0) {$10$};   

      \path (1) edge node {$a$}           (2);
      \path (3) edge node {$b$}           (4);
      \path (5) edge node {$c$}           (6);
      \path (7) edge node {$\varepsilon$} (3);
      \path (7) edge node {$\varepsilon$} (5);
      \path (4) edge node {$\varepsilon$} (8);
      \path (6) edge node {$\varepsilon$} (8);
      \path (9) edge node {$\varepsilon$} (7);
      \path (8) edge node {$\varepsilon$} (7);
      \path (7) edge node {$\varepsilon$} (10);
      \path (2) edge node {$\varepsilon$} (9);
    \end{tikzpicture}
  }

\end{frame}

\endgroup

%% SPDX-License-Identifier: CC-BY-SA-4.0
% Author: Matthieu Perrin
% Part: 
% Section: 
% Sub-section: 
% Frame: 

\begingroup

\begin{frame}{Modélisation mathématique}

  \begin{block}{Définition -- Automate fini non-déterministe (AFN)}
    \vspace{3mm}
    Un \structure{automate fini} est un quintuplet \alert{$\langle \Sigma, Q, I, F, \rightarrow \rangle$} tel que :
    \begin{description}[xxxxx]
    \item[\alert{$\Sigma$}] ensemble fini non vide : \structure{l'alphabet}
    \item[\alert{$Q$}] ensemble fini non vide : \structure{les états}
    \item[\alert{$I$}] $\subseteq Q$ : \structure{les états initiaux}
    \item[\alert{$F$}] $\subseteq Q$ : \structure{les états finaux (ou accepteurs)}
    \item[\alert{$\rightarrow$}] $\subseteq  Q \times (\Sigma \cup \{\varepsilon\}) \times Q$ : \structure{la relation de transition}
    \end{description}

    \vspace{3mm}
    Une \structure{transition} est un triplet \alert{$\langle q, a, q' \rangle \in \rightarrow$}, que l'on note \alert{$q\xrightarrow{a} q'$}, tel que :
    \begin{description}[xxxxx]
    \item[\alert{$q$}] $\in Q$ : \structure{l'état de départ}
    \item[\alert{$a$}] $\in \Sigma \cup \{\varepsilon\}$ : \structure{l'étiquette}
    \item[\alert{$q'$}] $\in Q$ : \structure{l'état d'arrivée}
    \end{description}
  \end{block}

  \on[y=-25mm, x=30mm]{
    \begin{tikzpicture}[automaton]
      \state (q)  at (0,0) {$q$}; 
      \state (q1) at (1,0) {$q'$}; 
      \path  (q) edge node {$a$} (q1);
    \end{tikzpicture}
  }

\end{frame}

\endgroup

%% SPDX-License-Identifier: CC-BY-SA-4.0
% Author: Matthieu Perrin
% Part: 
% Section: 
% Sub-section: 
% Frame: 

\begingroup

\begin{frame}{Exemple d'utilisation du lemme de pompage}
  
  \tfBlock[top=-5mm]{Montrer que $L \eqdef \{a^nb^nc^n \mid n\in \mathbb{N}\}$ n'est pas algébrique}{
    Soit $\Sigma \eqdef \{a, b, c\}$.%
    \only<2-|handout>{
      Si $L$ est algébrique, $L$ vérifie le lemme de pompage :

      \vspace{-4mm}
      $$
      \begin{array}{c}
        \structure{\exists N\in \mathbb{N}}, \alert{\forall u\in L, |u| \ge N} \Rightarrow (\structure{\exists v, w, x, y, z\in \Sigma^\star}, \\
        u = v \cdot w \cdot x\cdot y \cdot z \land w \cdot y\neq \varepsilon \land |w\cdot x\cdot y| \le N \land \alert{\forall i \in \mathbb{N}}, v\cdot w^i\cdot x\cdot y^i \cdot z \in L)
      \end{array}
      $$
      \vspace{-2mm}
      
      \structure{Soit $N$ donné par le lemme pompage}.
    }
    
    \only<3-|handout>{%
      \alert{Posons $u = a^N b^N c^N$. On a bien $u\in L$ et $|u| = 3N \ge N$}.\\
      \structure{Soit $v \cdot w \cdot x\cdot y \cdot z$ la décomposition de $u$ donnée par le lemme de pompage}. 
    }
  }
  
  \tf<4-|handout>[y=-10mm]{
    \begin{smArray}[width=4mm, height=3mm, name={$u=$}]
      \smCell[\smNone]{\alert{$a$}}      \smCoord{(a1)}
      \smCell[\smNone]{\alert{$\cdots$}}               
      \smCell[\smNone]{\alert{$a$}}      \smCoord{(an)}
      \smCell[\smNone]{\alert{$\cdot$}}               
      \smCell[\smNone]{\alert{$b$}}      \smCoord{(b1)}
      \smCell[\smNone]{\alert{$\cdots$}}               
      \smCell[\smNone]{\alert{$b$}}      \smCoord{(bn)}
      \smCell[\smNone]{\alert{$\cdot$}}               
      \smCell[\smNone]{\alert{$c$}}      \smCoord{(c1)}
      \smCell[\smNone]{\alert{$\cdots$}}               
      \smCell[\smNone]{\alert{$c$}}      \smCoord{(cn)}
      
      \draw [decorate, decoration={brace, amplitude=5pt}] ([xshift=1mm]a1.north west) -- ([xshift=-1mm]an.north east) node[midway,yshift=4mm]{$N$};
      \draw [decorate, decoration={brace, amplitude=5pt}] ([xshift=1mm]b1.north west) -- ([xshift=-1mm]bn.north east) node[midway,yshift=4mm]{$N$};
      \draw [decorate, decoration={brace, amplitude=5pt}] ([xshift=1mm]c1.north west) -- ([xshift=-1mm]cn.north east) node[midway,yshift=4mm]{$N$};

      \draw [decorate, decoration={brace, amplitude=5pt, mirror}] ([xshift=1mm]an.south west) -- ([xshift=-1mm]b1.south east) node[midway,yshift=-4mm]{$|wxy|\le N$};
      \draw [decorate, decoration={brace, amplitude=5pt, mirror}] ([xshift=1mm]bn.south west) -- ([xshift=-1mm]c1.south east) node[midway,yshift=-4mm]{$|wxy|\le N$};
    \end{smArray}       
  }

  \tf<5-|handout>[text, bottom=-1mm]{
    \begin{itemize}
    \item Comme $wy \neq \varepsilon$, $\alpha = wy[1] \in \Sigma$ est une lettre de $wy$.
    \item Comme $|wxy| \le N$, il existe $\beta \in \Sigma$ tel que $\beta$ n'est pas une lettre de $wy$. 
    \item \alert{Posons $i=2$}. $|v \cdot w^2 \cdot x\cdot  y^2\cdot   z|_\alpha > |v\cdot  w^2\cdot  x\cdot  y^2\cdot   z|_\beta$, donc $v\cdot  w^2\cdot  x\cdot  y^2\cdot   z\notin L$. 
    \end{itemize}
    Absurde ! Donc $L$ n'est pas algébrique. 
  }

  \tfExampleBlock<-4>[y=-25mm]{On sait}{}

  \tf<1>[y=-25mm, anchor=north, text]{
    \begin{itemize}
    \item $\begin{array}[t]{l}
      \alert{\forall L \in \textsc{alg}_\Sigma}, \structure{\exists N\in \mathbb{N}}, \forall u\in L, |u| \ge N \Rightarrow (\exists v, w, x, y, z\in \Sigma^\star, \\
      u = v w x y z \land w y\neq \varepsilon \land |wxy| \le N \land \forall i \in \mathbb{N}, v w^i x y^i z \in L)
    \end{array}$
    \end{itemize}
  }

  \tf<2-4>[y=-25mm, anchor=north, left=.35\textwidth]{
    \begin{itemize}
    \item $L \in \textsc{alg}_\Sigma$
    \item<3-> $v, w, x, y, z\in \Sigma^\star$
    \end{itemize}
  }
  
  \tf<2-4>[y=-25mm, anchor=north, width=.35\textwidth]{
    \begin{itemize}
    \item $N \in \mathbb{N}$
    \item<3-> $wy\neq \varepsilon$
    \end{itemize}
  }
  
  \tf<3-4>[y=-25mm, anchor=north, right=.35\textwidth]{
    \begin{itemize}
    \item $v \cdot w \cdot x \cdot y \cdot z = a^N b^N c^N$
    \item $\alert{|wxy| \le N}$
    \end{itemize}
  }
  
\end{frame}

\endgroup

%% SPDX-License-Identifier: CC-BY-SA-4.0
% Author: Matthieu Perrin
% Part: 
% Section: 
% Sub-section: 
% Frame: 

\begingroup

\begin{frame}{Automates déterministes et complets}

  \vspace{-1mm}

  Soit $A = \langle \Sigma, Q, I, F, \mu \rangle$ un AFN.

  \vspace{-1mm}
  \begin{block}{Définition -- AFN complet}
    On dit que $A$ est \structure{complet} si pour chaque état $q$ et chaque symbole $a$, il existe au moins une transition sortant de $q$ étiquetée $a$ :

    \vspace{-3mm}
    $$\alert{\forall q\in Q,  \forall a\in \Sigma, \exists q' \in Q : \langle q,a, q' \rangle \in \mu}$$
  \end{block}
  
  \vspace{-3mm}
  \begin{block}{Définition -- Automate fini déterministe (AFD)}
    On dit que $A$ est \structure{déterministe} si toutes les conditions sont vérifiées : 
    \begin{description}
    \item[$\varepsilon$-libre] $A$ ne possède pas d'$\varepsilon$-transition : \alert{$\mu \subseteq Q \times \Sigma \times Q$}
    \item[unitaire] $A$ ne possède pas qu'un seul état initial : \alert{$|I| = 1$}
    \item[fonction partielle]  pour chaque état $q$ et chaque symbole $a$, il existe au plus une transition sortant de $q$ étiquetée $a$ : 
    \end{description}
    $$\alert{\forall q\in Q,  \forall a\in \Sigma, \unique q' \in Q : \langle q, a, q' \rangle \in \mu}$$
  \end{block}


  \vspace{-2mm}
  \begin{alertblock}{Remarque}
    \begin{itemize}
    \item \vspace{-3mm} Si $A$ est déterministe et complet, $\mu$ est une fonction.
    \item \vspace{-1mm} Dans ce cas, on note $q' = \mu(q, a)$ pour $\langle q, a, q' \rangle \in \mu$
    \end{itemize}
  \end{alertblock}
\end{frame}

\endgroup

 
\subsection{Langage reconnu par un automate}
% SPDX-License-Identifier: CC-BY-SA-4.0
% Author: Matthieu Perrin
% Part: 
% Section: 
% Sub-section: 
% Frame: 

\begingroup

\begin{frame}{Configurations d'un AFN}
  
  \tfBlock[top=-3mm]{Définition -- Configuration d'un AFN}{
    Soit $A=\langle \Sigma, Q, I, F, \rightarrow \rangle$ un AFN. \\
    Une \structure{configuration} de $A$ est un couple \alert{$\langle u, q \rangle$} tel que :
    \begin{description}[xxxx]
    \item[\alert{$u$}] $\in \Sigma^\star$ : \structure{mot restant à reconnaître}
    \item[\alert{$q$}] $\in Q$ \,: \structure{l'état courant dans la simulation}
    \end{description}
    Une configuration $\langle u, q \rangle$ est dite \structure{initiale} si \alert{$q\in I$} et \structure{acceptante} si \alert{$u=\varepsilon \land q \in F$}
  }

  \tfExampleBlock[y=-1mm]{Exemples}{}

  \tf[bottom, left=.33\textwidth]{
    \centering
    \example{\large $\langle bbbcc, 3 \rangle$}
    \\\vspace{1mm}
    \begin{smArray}[size=5mm]
      \smCell[\smExample]{b} \smHead[example]
      \smCell[\smExample]{b}
      \smCell[\smExample]{b}
      \smCell[\smExample]{c}
      \smCell[\smExample]{c}
    \end{smArray}
    \\\vspace{1mm}
    \begin{tikzpicture}[smAutomaton]\footnotesize
      \smState[\smInitial]                (1) at (0.0,1.2) {$1$}; 
      \smState[\smAccepting]              (2) at (1.5,1.2) {$2$}; 
      \smState[\smExample\smInitialAbove] (3) at (0.0,0.0) {$3$}; 
      \smState[]                          (4) at (1.5,0.0) {$4$}; 

      \smPath (1) edge             node[swap] {$a$}           (2);
      \smPath (1) edge[loop above] node       {$a$}           (1);
      \smPath (3) edge             node       {$b$}           (4);
      \smPath (3) edge[loop left ] node       {$b$}           (3);
      \smPath (4) edge[loop right] node       {$b$}           (4);
      \smPath (2) edge[loop above] node       {$c$}           (2);
      \smPath (4) edge             node       {$\varepsilon$} (2);
    \end{tikzpicture}
  }

  \tf[bottom, width=.33\textwidth]{
    \centering
    \example{\large $\langle bcc, 4 \rangle$}
    \\\vspace{1mm}
    \begin{smArray}[size=5mm]
      \smCell{b}
      \smCell{b}
      \smCell[\smExample]{b} \smHead[example]
      \smCell[\smExample]{c}
      \smCell[\smExample]{c}
    \end{smArray}
    \\\vspace{1mm}
    \begin{tikzpicture}[smAutomaton]\footnotesize
      \smState[\smInitial]      (1) at (0.0,1.2) {$1$}; 
      \smState[\smAccepting]    (2) at (1.5,1.2) {$2$}; 
      \smState[\smInitialAbove] (3) at (0.0,0.0) {$3$}; 
      \smState[\smExample]      (4) at (1.5,0.0) {$4$}; 

      \smPath (1) edge             node[swap] {$a$}           (2);
      \smPath (1) edge[loop above] node       {$a$}           (1);
      \smPath (3) edge             node       {$b$}           (4);
      \smPath (3) edge[loop left ] node       {$b$}           (3);
      \smPath (4) edge[loop right] node       {$b$}           (4);
      \smPath (2) edge[loop above] node       {$c$}           (2);
      \smPath (4) edge             node       {$\varepsilon$} (2);
    \end{tikzpicture}
  }

  \tf[bottom, right=.33\textwidth]{
    \centering
    \example{\large $\langle \varepsilon, 2 \rangle$}
    \\\vspace{1mm}
    \begin{smArray}[size=5mm]
      \smCell[\smNone]{}
      \smCell{b}
      \smCell{b}
      \smCell{b} 
      \smCell{c}
      \smCell{c}
      \smCell[\smNone]{}\smHead[example]
    \end{smArray}
    \\\vspace{1mm}
    \begin{tikzpicture}[smAutomaton]\footnotesize
      \smState[\smInitial]             (1) at (0.0,1.2) {$1$}; 
      \smState[\smAccepting\smExample] (2) at (1.5,1.2) {$2$}; 
      \smState[\smInitialAbove]        (3) at (0.0,0.0) {$3$}; 
      \smState[]                       (4) at (1.5,0.0) {$4$}; 

      \smPath (1) edge             node[swap] {$a$}           (2);
      \smPath (1) edge[loop above] node       {$a$}           (1);
      \smPath (3) edge             node       {$b$}           (4);
      \smPath (3) edge[loop left ] node       {$b$}           (3);
      \smPath (4) edge[loop right] node       {$b$}           (4);
      \smPath (2) edge[loop above] node       {$c$}           (2);
      \smPath (4) edge             node       {$\varepsilon$} (2);
    \end{tikzpicture}
  }
  
\end{frame}

\endgroup

% SPDX-License-Identifier: CC-BY-SA-4.0
% Author: Matthieu Perrin
% Part: 
% Section: 
% Sub-section: 
% Frame: 

\begingroup

\begin{frame}{Langage entre états}

  \tfExampleBlock[y=-10mm]{Exemple}{}

  \tf[text, top]{
    Soit $A=\langle \Sigma, Q, I, F, \mu \rangle$ un AFN.
    \begin{block}{Définition -- Langage entre états}
      Le \structure{langage entre deux états} $q$ et $q'$ de $Q$, noté \alert{$\mathcal{L}_A(q, q')$},
      est l'ensemble des mots $u$ tels qu'il existe un chemin d'actions menant de $\langle u, q \rangle$ à $\langle \varepsilon, q'\rangle$.

      \vspace{-2mm}
      $$\alert{\mathcal{L}_A \eqdef \left\{\begin{array}{ccc}
        Q \times Q &\rightarrow& \mathcal{P}\left(\Sigma^\star\right)\\
        q, q' & \mapsto & \{u \in \Sigma^\star | \langle u,q\rangle \leadsto_A^\star \langle\varepsilon,q'\rangle\}
        \end{array}\right.}$$

      \begin{description}
      \item[Remarque :] $\displaystyle\mathcal{L}_A(q, q') = \{u \in \Sigma^\star | \alert{\forall v \in \Sigma^\star}, \langle u\alert{v},q\rangle \leadsto_A^\star \langle \alert{v},q'\rangle\}$
      \end{description}
    \end{block}
  }


  \tf[bottom, x=.4\textwidth]{
    \begin{tikzpicture}[smAutomaton]
      \smState[\smInitial]   (s1) at (0.0,1.2) {$s_1$}; 
      \smState[\smAccepting] (s2) at (1.5,1.2) {$s_2$}; 
      \smState[\smInitial]   (s3) at (0.0,0,0) {$s_3$}; 
      \smState               (s4) at (1.5,0,0) {$s_4$}; 

      \smPath (s1) edge             node[above] {$a$}           (s2);
      \smPath (s1) edge[loop above] node        {$a$}           (s1);
      \smPath (s3) edge             node[swap]  {$b$}           (s4);
      \smPath (s3) edge[loop below] node        {$b$}           (s3);
      \smPath (s4) edge[loop below] node        {$b$}           (s4);
      \smPath (s2) edge[loop right] node        {$c$}           (s2);
      \smPath (s4) edge             node        {$\varepsilon$} (s2);
    \end{tikzpicture}
  }

  \tf[bottom, text=.4\textwidth, x=2mm]{
    $$\begin{array}{rcl}
      \langle aac, s_1 \rangle &\leadsto & \langle ac, s_1 \rangle\\
      \langle ac, s_1 \rangle &\leadsto & \langle c, s_2 \rangle\\
      \langle c, s_2 \rangle &\leadsto & \langle \varepsilon, s_2 \rangle\\
      \\
      \example{\langle aac, s_1 \rangle} & \example{\leadsto^\star} & \example{\langle \varepsilon, s_2 \rangle}
    \end{array}$$
    $$\example{aac \in \mathcal{L}_A(s_1, s_2)}$$
  }

  \tf[bottom, left=.3\textwidth]{
    $$\begin{array}{rcl}
      \langle bbb, s_3 \rangle &\leadsto & \langle bb, s_3 \rangle\\
      \langle bb, s_3 \rangle &\leadsto & \langle b, s_4 \rangle\\
      \langle b, s_4 \rangle &\leadsto & \langle \varepsilon, s_4 \rangle\\
      \\
      \example{\langle bbb, s_3 \rangle} & \example{\leadsto^\star} & \example{\langle \varepsilon, s_4 \rangle}
    \end{array}$$
    $$\example{bbb \in \mathcal{L}_A(s_3, s_4)}$$
  }

\end{frame}


\endgroup

% SPDX-License-Identifier: CC-BY-SA-4.0
% Author: Matthieu Perrin
% Part: 
% Section: 
% Sub-section: 
% Frame: 

\begingroup

\begin{frame}{Langage reconnu par un AFN}
  \small
  
  Soit $A=\langle \Sigma, Q, I, F, \mu \rangle$ un AFN.

  \vspace{-2mm}
  \begin{block}{Définition -- langage reconnu (ou accepté)}

    Un mot d'un langage entre un état initial et un état final est dit \structure{reconnu} par $A$.

    Le langage \structure{reconnu} par $A$ est l'ensemble $\alert{\mathcal{L}(A)}$ des mots reconnus par $A$.

    \alert{$$\begin{array}{lll}
        \mathcal{L}(A) &\eqdef& \displaystyle \bigcup_{i\in I} \bigcup_{f\in F} \mathcal{L}_A(i,f)\\
        &=& \{u \in \Sigma^\star \mid \exists i\in I, \exists f\in F,  \langle u, i \rangle \leadsto_A^\star \langle \varepsilon, f\rangle\}
      \end{array}$$}
  \end{block}

  \begin{exampleblock}{Exemple}
    \noindent\begin{minipage}{.4\textwidth}%
    \scalebox{.8}{\begin{tikzpicture}[shorten >=1pt,node distance=1.5cm,on grid,auto]
        \node [state,initial, initial text=] (s1)   {$s_1$}; 
        \node [state,accepting] (s2) [right=of s1]  {$s_2$}; 

        \node [state,initial, initial text=] (s4) [below=of s1]  {$s_3$}; 
        \node [state=] (s5) [right=of s4]  {$s_4$}; 

        \path [->]    (s1) edge node[above] {a} (s2);
        \path [->]    (s1) edge[loop above, looseness=5] node {a} (s1);

        \path [->]    (s4) edge node[above] {b} (s5);
        \path [->]    (s4) edge[loop below, looseness=5] node {b} (s4);
        \path [->]    (s5) edge[loop below, looseness=5] node {b} (s5);

        \path [->]    (s2) edge[loop right, looseness=5] node {c} (s2);

        \path [->]    (s5) edge node {$\varepsilon$} (s2);
    \end{tikzpicture}}\end{minipage}%
    \begin{minipage}{.6\textwidth}
      Pour \example{$i=s_1$}, \example{$f=s_2$} et \example{$u=aac$}, on a : 
      \begin{enumerate}
      \item $s_1 \in I$
      \item $s_2 \in F$
      \item $aac \in \mathcal{L}_A(s_1, s_2)$
      \end{enumerate}
      Donc \example{$aac \in \mathcal{L}(A)$}
    \end{minipage}
  \end{exampleblock}
\end{frame}


\endgroup

% SPDX-License-Identifier: CC-BY-SA-4.0
% Author: Matthieu Perrin
% Part: 
% Section: 
% Sub-section: 
% Frame: 

\begingroup

\begin{frame}{Équivalence et isomorphisme entre automates}

  \tf[text, top]{
    Soient $A = \langle \Sigma, Q_A, I_A, F_A, \rightarrow_A \rangle$ et $B = \langle \Sigma, Q_B, I_B, F_B, \rightarrow_B \rangle$ deux automates. 
  }

  \tfBlock[y=17mm]{Définition -- Automates équivalents}{
    $A$ et $B$ sont \structure{équivalents}, noté $\alert{A \equiv B}$, s'ils acceptent le même langage :
    $$\alert{A_1 \equiv A_2  \eqdef \mathcal{L}(A_1) = \mathcal{L}(A_2)}.$$
  }

  \tfBlock[bottom=2mm]{Définition -- Automates isomorphes}{
    $A$ et $B$ sont \structure{isomorphes}, noté $\alert{A \simeq B}$, s'ils ne diffèrent que par le nom de leurs états, \textit{c.-à-d.}
    s'il existe une fonction bijective $f : Q_A \rightarrow Q_B$ telle que :

    \vspace{2mm}
    $\structure{\forall a\in \Sigma, \forall q, q',}$\\
    $\structure{\begin{array}[t]{@{\quad\quad}lll}
        q\in Q_A &\Leftrightarrow& f(q) \in Q_B \\
        q\in I_A &\Leftrightarrow& f(q) \in I_B \\
        q\in F_A &\Leftrightarrow& f(q) \in F_B \\
        q \xrightarrow{a}_A q' &\Leftrightarrow& f(q) \xrightarrow{a}_B f(q')
      \end{array}
    }$
    \vspace{2mm}

    Deux automates isomorphes sont équivalents. 
  }

  \tf[bottom=2mm, x=3cm]{\small
    \begin{tikzpicture}[smAutomaton]
      \draw (1,0.75) node{\normalsize $\simeq$};

      \smState[\smStructure \smInitial]   (a0) at (0.0, 1.5) {$A$};
      \smState[\smStructure \smAccepting] (a1) at (0.0, 0.0) {$B$};
      \smState[\smExample \smInitial]     (b0) at (2.0, 1.5) {$0$};
      \smState[\smExample \smAccepting]   (b1) at (2.0, 0.0) {$1$};

      \smPath[\smStructure] (a0) edge[bend left]  node {$a$} (a1);
      \smPath[\smStructure] (a1) edge[bend left]  node {$a$} (a0);
      \smPath[\smStructure] (a1) edge[loop right] node {$b$} (a1);
      \smPath[\smStructure] (a0) edge[loop right] node {$b$} (a0);
      
      \smPath[\smExample]   (b0) edge[bend left]  node {$a$} (b1);
      \smPath[\smExample]   (b1) edge[bend left]  node {$a$} (b0);
      \smPath[\smExample]   (b1) edge[loop right] node {$b$} (b1);
      \smPath[\smExample]   (b0) edge[loop right] node {$b$} (b0);
    \end{tikzpicture}
  }
\end{frame}

\endgroup

% SPDX-License-Identifier: CC-BY-SA-4.0
% Author: Matthieu Perrin
% Part: 
% Section: 
% Sub-section: 
% Frame: 

\begingroup

\begin{frame}{Langage droit d'un état}
  Soit $A=\langle \Sigma, Q, I, F, \mu \rangle$ un AFN.
  \begin{block}{Définition -- Langage droit}
    Le \structure{langage droit} d'un état $q \in Q$, noté \alert{$\mathcal{LD}_A(q)$},
    est l'ensemble des mots reconnus par un chemin d'actions menant de $q$ à un état final de $A$. 
    
    $$\alert{\mathcal{LD}_A(q) \eqdef \bigcup_{f\in F} \mathcal{L}_A(q, f)}$$

    \begin{description}
    \item[Remarques :] $\displaystyle\mathcal{L}(A) = \bigcup_{i\in I} \mathcal{LD}_A(i)$ \hspace\fill $\displaystyle\mathcal{LD}_A(q) = \mathcal{L}\left(\langle \Sigma, Q, \{q\}, F, \mu \rangle\right)$
    \end{description}
  \end{block}

  \vspace{-3.65mm}
  \begin{exampleblock}{Exemple}
    \noindent\begin{minipage}{.35\textwidth}%
    \scalebox{.75}{\begin{tikzpicture}[shorten >=1pt,node distance=1.5cm,on grid,auto]
        \node [state,initial, initial text=] (s1)   {$q_1$}; 
        \node [state,accepting] (s2) [right=of s1]  {$q_2$}; 

        \node [state,initial, initial text=] (s4) [below=of s1]  {$q_3$}; 
        \node [state, example, fill=example!20] (s5) [right=of s4]  {$q_4$}; 

        \path [->]    (s1) edge node[above] {a} (s2);
        \path [->]    (s1) edge[loop above, looseness=5] node {a} (s1);

        \path [->]    (s4) edge node[above] {b} (s5);
        \path [->]    (s4) edge[loop below, looseness=5] node {b} (s4);
        \path [->]    (s5) edge[loop below, looseness=5] node {b} (s5);

        \path [->]    (s2) edge[loop right, looseness=5] node {c} (s2);

        \path [->]    (s5) edge node {$\varepsilon$} (s2);
    \end{tikzpicture}}\end{minipage}%
    \begin{minipage}{.65\textwidth}
      \vspace{-3mm}
      \example{$\begin{array}{rcl}%
          \mathcal{LD}_A(q_4) &=& \mathcal{L}_A\left(q_4, q_3\right)\\
          &=& b^\star c^\star\\
        \end{array}$}
    \end{minipage}
  \end{exampleblock}
\end{frame}


\endgroup

% SPDX-License-Identifier: CC-BY-SA-4.0
% Author: Matthieu Perrin
% Part: 
% Section: 
% Sub-section: 
% Frame: 

\begingroup

\begin{frame}{Langage gauche d'un état}
  Soit $A=\langle \Sigma, Q, I, F, \mu \rangle$ un AFN.
  \begin{block}{Définition -- Langage gauche}
    Le \structure{langage gauche} d'un état $q \in Q$, noté \alert{$\mathcal{LG}_A(q)$},
    est l'ensemble des mots reconnus par un chemin d'actions menant d'un état initial de $A$ à $q$. 
    
    $$\alert{\mathcal{LG}_A(q) \eqdef \bigcup_{i\in I} \mathcal{L}_A(i, q)}$$

    \begin{description}
    \item[Remarques :] $\displaystyle\mathcal{L}(A) = \bigcup_{f\in F} \mathcal{LG}_A(f)$ \hspace\fill $\displaystyle\mathcal{LG}_A(q) = \mathcal{L}\left(\langle \Sigma, Q, I, \{q\}, \mu \rangle\right)$
      \\\hspace\fill $\displaystyle\mathcal{L}(A) = \bigcup_{q\in Q} \mathcal{LG}_A(q) \cdot \mathcal{LD}_A(q)$
    \end{description}
  \end{block}

  \vspace{-13mm}
  \noindent\begin{minipage}{.35\textwidth}%
  \begin{exampleblock}{Exemple}
    \scalebox{.75}{\begin{tikzpicture}[shorten >=1pt,node distance=1.5cm,on grid,auto]
        \node [state,initial, initial text=] (s1)   {$q_1$}; 
        \node [state,accepting] (s2) [right=of s1]  {$q_2$}; 

        \node [state,initial, initial text=] (s4) [below=of s1]  {$q_3$}; 
        \node [state, example, fill=example!20] (s5) [right=of s4]  {$q_4$}; 

        \path [->]    (s1) edge node[above] {a} (s2);
        \path [->]    (s1) edge[loop above, looseness=5] node {a} (s1);

        \path [->]    (s4) edge node[above] {b} (s5);
        \path [->]    (s4) edge[loop below, looseness=5] node {b} (s4);
        \path [->]    (s5) edge[loop below, looseness=5] node {b} (s5);

        \path [->]    (s2) edge[loop right, looseness=5] node {c} (s2);

        \path [->]    (s5) edge node {$\varepsilon$} (s2);
    \end{tikzpicture}}
  \end{exampleblock}
  \end{minipage}%
  \begin{minipage}{.65\textwidth}%
    \vspace{10mm}%
    \example{$\begin{array}{rclll}%
        \mathcal{LG}_A(q_4) &=& \mathcal{L}_A(q_1, q_4) &\cup& \mathcal{L}_A(q_3, q_4)\\
        &=& \emptyset &\cup& b^+\\
        &=& b^+\\
      \end{array}$}
  \end{minipage}
\end{frame}

\endgroup

% SPDX-License-Identifier: CC-BY-SA-4.0
% Author: Matthieu Perrin
% Part: 
% Section: 
% Sub-section: 
% Frame: 

\begingroup

\begin{frame}{Automate émondé}

  \vspace{-2mm}
  \begin{block}{Définition -- Propriétés des états}
    Soit $A=\langle \Sigma, Q, I, F, \rightarrow \rangle$ un AFN. Un état $q \in Q$ est dit
    \begin{description}[co-accessible]
    \item[accessible] s'il peut être atteint à partir d'un état initial \hspace\fill  \example{ex : 2, 3, 4, 5, 6}

      \vspace{-3mm}
      $$\alert{\exists i\in I, \exists u\in \Sigma^\star, \langle u, i\rangle \leadsto^\star \langle \varepsilon, q\rangle}$$
      \vspace{-6mm}

    \item[inaccessible] s'il n'est pas accessible  \hspace\fill  \example{ex : 0, 1}
    \item[co-accessible] si on peut atteindre un état final à partir de $q$  \hspace\fill  \example{ex : 0,1, 2, 3, 6}

      \vspace{-3mm}
      $$\alert{\exists f\in F, \exists u\in \Sigma^\star, \langle u, q\rangle \leadsto^\star \langle \varepsilon, f\rangle}$$
      \vspace{-6mm}

    \item[stérile] s'il n'est pas co-accessible  \hspace\fill  \example{ex : 4, 5}
    \item[utile] s'il est accessible et co-accessible  \hspace\fill  \example{ex : 3, 6}
    \item[inutile] s'il n'est pas utile  \hspace\fill  \example{ex : 0, 1, 4, 5}
    \end{description}
  \end{block}
  
  \begin{block}{Définition -- Automate émondé}
    Un AFN est dit \structure{émondé} si tous ses états sont utiles. 
  \end{block}

  \begin{block}{Théorème -- Émondage}
    Tout AFN est équivalent à un automate émondé.
  \end{block}

  \on[bottom, x=.35\textwidth]{
    \begin{tikzpicture}[automaton, example, y=10mm]
      \state[accepting] (0) at (0,2) {$0$}; 
      \state            (1) at (1,2) {$1$}; 
      \state[initial  ] (2) at (0,1) {$2$}; 
      \state            (3) at (1,1) {$3$}; 
      \state[initial  ] (4) at (0,0) {$4$}; 
      \state            (5) at (1,0) {$5$}; 
      \state[accepting] (6) at (2,1) {$6$}; 

      \path (0) edge node{$a$} (1);
      \path (1) edge node{$b$} (3);
      \path (2) edge node{$a$} (3);
      \path (3) edge node{$b$} (5);
      \path (4) edge node{$a$} (5);
      \path (3) edge node{$b$} (6);
    \end{tikzpicture}
  }

  
\end{frame}


\endgroup

% SPDX-License-Identifier: CC-BY-SA-4.0
% Author: Matthieu Perrin
% Part: 
% Section: 
% Sub-section: 
% Frame: 

\begingroup

\begin{frame}[fragile]{Algorithme de l'émondage}

  \vspace{-3mm}
  \begin{block}{Théorème -- Émondage}
    Soit $A=\langle \Sigma, Q, I, F, \mu \rangle$ un AFN. Il existe un automate émondé $B \equiv A$.
  \end{block}
  
  \vspace{2mm}
  \begin{tikzpicture}
    
    \fill<2>[rounded corners, structure!20]      (-7.8,1.1)   rectangle (-5.6,1.45);
    \fill<3-5>[rounded corners, structure!20]    (-7.8,.4)    rectangle (-0.2,1.1);
    \fill<6>[rounded corners, example!20]   (-7.8,.05)   rectangle (-5.1,.4);
    \fill<7-9>[rounded corners, example!20] (-7.8,-.65)  rectangle (0.6,.05);
    \fill<10>[rounded corners, alert!20]    (-7.8,-1.35) rectangle (-3.2,-.65);
    
    \draw (0,0) node{\begin{minipage}{1.5\textwidth}\scalebox{.75}{\begin{algorithm}[H]
            \SetKwFunction{Emondage}{emondage}
            \SetKwFunction{Accessible}{accessible}
            \SetKwFunction{Coaccessible}{co-accessible}
            \SetKwData{Adresse}{u}
            \SetKwData{HasAt}{vu@}
            \SetKwData{isEmpty}{vu$\beta$}
            \SetKwData{HasDot}{vu$\bullet$}

            \Fn{\Emondage($A = \langle \Sigma, Q, I, F, \mu \rangle$ : automate) : automate}{
              $\Accessible \leftarrow I$\;
              \Tantque{$\exists \langle q, a, q' \rangle \in \mu, q\in \Accessible \land q'\notin\Accessible$}{
                $\Accessible \leftarrow \Accessible \cup \{q'\}$\;
              }
              $\Coaccessible \leftarrow F$\;
              \Tantque{$\exists \langle q, a, q' \rangle \in \mu, q\not\in \Coaccessible \land q'\in\Coaccessible$}{
                $\Coaccessible \leftarrow \Coaccessible \cup \{q\}$\;
              }
              $Q' \leftarrow \Accessible \cap \Coaccessible$\;
              $\mu' \leftarrow \{\langle q, a, q' \rangle \in \mu | q\in Q' \land q'\in Q' \}$\;
              \Retourner $\langle \Sigma, Q', I \cap Q', F \cap Q', \mu'\rangle$\;
            }
        \end{algorithm}}
    \end{minipage}};

    \draw (0.5,-2.7) node{\scalebox{.85}{\begin{tikzpicture}[shorten >=1pt,node distance=1.5cm,on grid,auto]
          \node (s0) {}; 
          \node (s1) [right=of s0] {}; 
          \node (s2) [below=of s0] {}; 
          \node (s3) [right=of s2] {}; 
          \node (s4) [below=of s2] {}; 
          \node (s5) [right=of s4] {}; 
          \node (s6) [right=of s3] {}; 

          \node<-5>[state, accepting]              (0) at (s0) {0}; 
          \node<-7>[state]                         (1) at (s1) {1}; 
          \node<-1,5-8>[state, initial, initial text=] (2) at (s2) {2}; 
          \node<-2,5-6>[state]                         (3) at (s3) {3}; 
          \node<-1,5-9>[state, initial, initial text=] (4) at (s4) {4}; 
          \node<-3,5-9>[state]                         (5) at (s5) {5}; 
          \node<-4,5>[state, accepting]              (6) at (s6) {6}; 

          \node<2-5>[state, initial, initial text=, structure, fill=structure!20] (2) at (s2) {2}; 
          \node<3-5>[state, structure, fill=structure!20] (3) at (s3) {3}; 
          \node<2-5>[state, initial, initial text=, structure, fill=structure!20] (4) at (s4) {4}; 
          \node<4-5>[state, structure, fill=structure!20] (5) at (s5) {5}; 
          \node<5>  [state, accepting, structure, fill=structure!20] (6) at (s6) {6}; 

          \node<6-9>[state, accepting, example, fill=example!20] (6) at (s6) {6}; 
          \node<6-9>[state, accepting, example, fill=example!20] (0) at (s0) {0}; 
          \node<7-9>[state, example, fill=example!20] (3) at (s3) {3}; 
          \node<8-9>[state, example, fill=example!20] (1) at (s1) {1}; 
          \node<9>  [state, initial, initial text=, example, fill=example!20] (2) at (s2) {2}; 

          \node<10>[state, initial, initial text=, alert, fill=alert!20] (2) at (s2) {2}; 
          \node<10>[state, alert, fill=alert!20]                         (3) at (s3) {3}; 
          \node<10>[state, accepting, alert, fill=alert!20]              (6) at (s6) {6}; 
          \node<10>[state, accepting, black!30]              (0) at (s0) {0}; 
          \node<10>[state, black!30]                         (1) at (s1) {1}; 
          \node<10>[state, initial, initial text=, black!30] (4) at (s4) {4}; 
          \node<10>[state, black!30]                         (5) at (s5) {5}; 

          \node<11>[state, initial, initial text=] (2) at (s2) {2}; 
          \node<11>[state]                         (3) at (s3) {3}; 
          \node<11>[state, accepting]              (6) at (s6) {6}; 

          \path<-9>      [->]              (0) edge node{$a$} (1);
          \path<-7,9>    [->]              (1) edge node{$b$} (3);
          \path<8>       [example,->] (1) edge node{$b$} (3);
          \path<-2,4-8>  [->]              (2) edge node{$a$} (3);
          \path<3>       [structure, ->]   (2) edge node{$a$} (3);
          \path<9>       [example,->] (2) edge node{$a$} (3);
          \path<-3,5-9>  [->]              (3) edge node{$b$} (5);
          \path<4>       [structure, ->]   (3) edge node{$b$} (5);
          \path<-3,5-9>  [->]              (4) edge node{$a$} (5);
          \path<4>       [structure, ->]   (4) edge node{$a$} (5);
          \path<-4,6,8,9>[->]              (3) edge node{$b$} (6);
          \path<5>       [structure, ->]   (3) edge node{$b$} (6);
          \path<7>       [example,->] (3) edge node{$b$} (6);
          
          \path<10>[->, black!30]   (0) edge node{$a$} (1);
          \path<10>[->, black!30]   (1) edge node{$b$} (3);
          \path<10>[->, alert] (2) edge node{$a$} (3);
          \path<10>[->, black!30]   (3) edge node{$b$} (5);
          \path<10>[->, black!30]   (4) edge node{$a$} (5);
          \path<10>[->, alert] (3) edge node{$b$} (6);

          \path<11>[->] (2) edge node{$a$} (3);
          \path<11>[->] (3) edge node{$b$} (6);

    \end{tikzpicture}}};

    \draw (-5,-1.75) node[below]{\begin{minipage}{.55\textwidth}
        \begin{exampleblock}{Exemple}
          \begin{itemize}
          \item<2->  \structure{$\Accessible = \{2, 4\only<3->{, 3}\only<4->{, 5}\only<5->{, 6}\}$}
          \item<6->  \example{$\Coaccessible = \{0, 6\only<7->{, 3}\only<8->{, 1}\only<9->{, 2}\}$}
          \item<10-> \alert{$Q' = \{2, 3, 6\}$}
          \end{itemize}
        \end{exampleblock}
    \end{minipage}};

  \end{tikzpicture}

\end{frame}


\endgroup

 
\subsection{Des automates aux expressions rationnelles}
% SPDX-License-Identifier: CC-BY-SA-4.0
% Author: Matthieu Perrin
% Part: 
% Section: 
% Sub-section: 
% Frame: 

\begingroup

\begin{frame}{Langage reconnu par un AFN}

  \on[text, top]{
    Soit $A=\langle \Sigma, Q, I, F, \rightarrow \rangle$ un AFN.

    \begin{block}{Définition -- langage reconnu (ou accepté)}
      Un mot $u$ est \structure{reconnu} par $A$ s'il mène d'un état initial à un état final
      $$\exists i\in I, \exists f\in F,~  \alert{\langle u, i \rangle \leadsto_A^\star \langle \varepsilon, f\rangle}$$
      Le langage \structure{reconnu} par $A$ est l'ensemble $\alert{\mathcal{L}(A)}$ des mots reconnus par $A$
      $$\alert{\mathcal{L}(A) \eqdef \left\{u \in \Sigma^\star \,\middle\mid\, \exists i\in I, \exists f\in F,  \langle u, i \rangle \leadsto_A^\star \langle \varepsilon, f\rangle\right\}}$$
    \end{block}
  }

  \onExampleBlock[bottom]{Exemple}{
    Pour \example{$i=3$}, \example{$f=2$} et \example{$u=bbc$}, on a : 
    \begin{enumerate}
    \item $\langle bbc, 1 \rangle \leadsto^\star \langle \varepsilon, 2 \rangle$
    \item $1 \in I$
    \item $2 \in F$
    \end{enumerate}
    Donc \example{$bbc \in \mathcal{L}(A)$}
  }

  \on[bottom, x=.25\textwidth]{
    \begin{tikzpicture}[automaton, grid size=15mm]
      \state[initial]       (1) at (0,1) {$1$}; 
      \state[accepting]     (2) at (1,1) {$2$}; 
      \state[initial above] (3) at (0,0) {$3$}; 
      \state                (4) at (1,0) {$4$}; 

      \path (1) edge             node       {$a$}           (2);
      \path (1) edge[loop above] node       {$a$}           (1);
      \path (3) edge             node[swap] {$b$}           (4);
      \path (3) edge[loop left ] node       {$b$}           (3);
      \path (4) edge[loop right] node       {$b$}           (4);
      \path (2) edge[loop above] node       {$c$}           (2);
      \path (4) edge             node[swap] {$\varepsilon$} (2);
    \end{tikzpicture}
  }
  
\end{frame}


\endgroup

% SPDX-License-Identifier: CC-BY-SA-4.0
% Author: Matthieu Perrin
% Part: 
% Section: 
% Sub-section: 
% Frame: 

\begingroup

\begin{frame}{Lien entre langages droits}

  \onExampleBlock[bottom, left=.55\textwidth]{Exemple}{
    Pour $i \in \{0, 1, 2, 3\}$, on pose $L_i = \mathcal{L}_{A}(q_i)$
    $$
    \left\{
    \begin{array}{rclcl}
      \mathcal{L}(A) = L_0 &=& \left(\example{\{b\} \cdot L_0} \right) \cup \left(\structure{\{a\} \cdot L_1} \right) &=& \example{b\cdot L_0} \mid \structure{a\cdot L_1} \\
      L_1 &=& \left(\{b\} \cdot L_1 \right) \cup \left(\{b\} \cdot L_2 \right) &=& b\cdot L_1 \mid b\cdot L_2 \\
      L_2 &=& \left(\{\varepsilon\} \cdot L_3 \right) \cup \{\varepsilon\} &=& L_3 \mid \varepsilon \\
      L_3 &=& \left(\{a\} \cdot L_3 \right) \cup \left(\{b\} \cdot L_1 \right) &=& a \cdot L_3 \mid b\cdot L_1 \\
    \end{array}
    \right.
    $$
  }
  \onBlock[top=-5mm]{Analyse}{
    Soient $A=\langle \Sigma, Q, I, F, \rightarrow \rangle$ un AFN, $a\in \Sigma \cup \{\varepsilon\}$, $u\in \Sigma^\star$, et $q\in Q$. On a :
    {\small
      $$
      \begin{array}{rcl}
        a u \in \mathcal{L}_{A}(q) &\Leftrightarrow& \exists f\in F, \structure{\langle a u, q\rangle \leadsto^\star \langle \varepsilon, f \rangle}\\
        &\Leftrightarrow& \exists f\in F, \exists q', \example{\langle a u, q\rangle \leadsto \langle u, q' \rangle} \structure{\leadsto^\star \langle \varepsilon, f \rangle}\\
        &\Leftrightarrow& \exists q', \example{q \xrightarrow{a} q'} \land \structure{u \in \mathcal{L}_{A}(q')}\\
      \end{array}$$
    }
    De plus, $\varepsilon \in \mathcal{L}_A(q)$ si $q \in F$. Ainsi, pour tout $q$ :
      $$\alert{\mathcal{L}_{A}(q) = \bigcup_{q \xrightarrow{b} q'} \{b\} \cdot \mathcal{L}_{A}(q') \quad\cup \left\{\begin{array}{@{\,}cl}\{\varepsilon\}&\text{si }q \in F\\\emptyset&\text{sinon}\end{array}\right.}$$
  }

  \on[bottom, x=45mm]{
    \begin{tikzpicture}[automaton]
      \state[initial]   (q0) at (0,2)  {$q_0$};
      \state            (q1) at (0,1)  {$q_1$};
      \state[accepting] (q2) at (0,0)  {$q_2$}; 
      \state            (q3) at (1,1)  {$q_3$}; 

      \path[structure] (q0) edge             node[swap] {$a$} (q1);
      \path            (q1) edge             node[swap] {$b$} (q2);
      \path            (q2) edge             node[swap] {$\varepsilon$} (q3);
      \path            (q3) edge             node[swap] {$b$} (q1);
      \path[example]   (q0) edge[loop right] node       {$b$} (q0);
      \path            (q1) edge[loop left ] node       {$b$} (q1);
      \path            (q3) edge[loop above] node       {$a$} (q3);
    \end{tikzpicture}
  }

\end{frame}

\endgroup

% SPDX-License-Identifier: CC-BY-SA-4.0
% Author: Matthieu Perrin
% Part: 
% Section: 
% Sub-section: 
% Frame: 

\begingroup

\begin{frame}{Lemme d'Arden}
  Soient $\Sigma$ un alphabet, et $A, B \in \mathscr{P}(\Sigma^\star)$ deux langages.
  \begin{block}{Lemme -- Lemme d'Arden}
    \begin{itemize}
    \item \alert{$A^\star B$} est la \structure{plus petite solution} de l'équation \structure{$X = AX | B$}
    \item si \structure{$\varepsilon \notin A$}, alors \alert{$A^\star B$} est l'\alert{unique} solution de l'équation \structure{$X = AX | B$}
    \end{itemize}
  \end{block}
  \begin{exampleblock}{Exemple}
    \noindent\begin{minipage}{.4\textwidth}
    \begin{tikzpicture}[shorten >=1pt,node distance=1.5cm,on grid,auto]
      \node [state,initial, initial text=] (s1)   {$0$}; 
      \node [state,accepting] (s2) [right=of s1]  {$1$};
      \path [->]    (s1) edge[loop above, looseness=5] node {a} (s1);
      \path [->]    (s1) edge node[above] {b} (s2);
    \end{tikzpicture}
    \end{minipage}%
    \begin{minipage}{.6\textwidth}
      $ L = a L | b = a^\star b$
    \end{minipage}
  \end{exampleblock}
  \begin{block}{Remarques}
    \begin{itemize}
    \item {\small L'équation $X = AX | B$ est appelée \structure{équation rationnelle linéaire à droite}}
    \item {\small Pour l'équation $X = XA | B$ linéaire à gauche, mêmes résultats avec $B A^\star$}
    \item {\small Si \structure{$\varepsilon \in A$}, $\Sigma^\star$ est également solution des deux équations}
    \end{itemize}
  \end{block}
\end{frame}

\endgroup

% SPDX-License-Identifier: CC-BY-SA-4.0
% Author: Matthieu Perrin
% Part: 
% Section: 
% Sub-section: 
% Frame: 

\begingroup

\begin{frame}{Démonstration du lemme de l'étoile}

  \vspace{-2mm}  
  \onlyb<-3>{\hspace\fill\example{\textbf{Exemple}}}%
  \only<4->{\hspace\fill\alert{\textbf{Lemme de l'étoile}}}

  Soient $\Sigma$ un alphabet, et $L$ un langage rationnel sur $\Sigma$.
  \onlyb<-3>{\hspace\fill\example{$L = a(bc)^\star a$}}
  \only<4->{\hspace\fill\alert{$\forall \Sigma, \forall L\in \textsc{rat}_\Sigma,$}}

  Soit $A$ son automate minimal. Posons $N = |A|$.  
  \onlyb<-3>{\hspace\fill\example{$N=4$}}
  \only<4->{\hspace\fill\alert{$\exists N\in \mathbb{N},$}}
  
  Soit $u \in L$ tel que $|u| \ge N$. 
  \onlyb<-3>{\hspace\fill\example{$u = abca$}}
  \only<4->{\hspace\fill\alert{$\forall u\in L, |u| \ge N \Rightarrow ($}}

  \begin{center}
    \begin{tikzpicture}[automaton, y=5mm]
      \state[initial]   (0) at (0, 2) {$0$};
      \state[alert]     (1) at (1, 1) {$1$};
      \state[example]   (2) at (2, 1) {$2$};
      \state[accepting] (3) at (0, 0) {$3$};

      \path[structure]  (0) edge node {$a$} (1);
      \path[example]    (1) edge[bend left] node {$b$} (2);
      \path[example]    (2) edge[bend left] node {$c$} (1);
      \path[structure]  (1) edge node {$a$} (3);
    \end{tikzpicture}

    \pause
    \structure{Pour reconnaître un mot de 4 lettres ou plus, il faut suivre une boucle}
  \end{center}

  Il existe un état $q$ que l'on visite deux fois. Posons :
  \onlyb<-3>{\hspace\fill\example{$q = 1$}}%
  
  \begin{itemize}
  \item $x$ le préfixe de $u$ avant la première visite de $q$
    \onlyb<-3>{\hspace\fill\example{$x = a$}}
    \only<4-|handout>{\hspace\fill\alert{$\exists x, y, z \in \Sigma^\star,$}}
  \item $y$ le facteur de $u$ entre les deux visites de $q$
    \onlyb<-3>{\hspace\fill\example{$y = bc$}}
    \only<4-|handout>{\hspace\fill\alert{$y \neq \varepsilon$}}
  \item $z$ le suffixe de $u$ après la deuxième visite de $q$
    \onlyb<-3>{\hspace\fill\example{$z = a$}}
    \only<4-|handout>{\hspace\fill\alert{$\land\; u = x\cdot y \cdot z$}}
  \end{itemize}

  \pause

  \begin{center}\structure{Si on suit une boucle une fois, on peut la suivre plusieurs fois}\end{center}

  On a :
  \begin{itemize}
  \item On peut prendre la boucle $0$, $2$, $3$, ... fois
    \onlyb<-3>{\hspace\fill\example{$aa \in L, abcbca\in L$}}
    \only<4-|handout>{\hspace\fill\alert{$\land\; \forall i\in \mathbb{N}, x \cdot y^i \cdot z \in L$}}
  \item $N$ lettres suffisent pour repasser dans un état
    \onlyb<-3>{\hspace\fill\example{$|abc| \le 4$}}
    \only<4-|handout>{\hspace\fill\alert{$\land\; |x\cdot y| \le N)$}}
  \end{itemize}

  \phantom{.}
\end{frame}


\endgroup

% SPDX-License-Identifier: CC-BY-SA-4.0
% Author: Matthieu Perrin
% Part: 
% Section: 
% Sub-section: 
% Frame: 

\begingroup

\begin{frame}{Résolution de système d'équations rationnelles}
    \tfBlock[top]{Système d'équations rationnelles}{
    $\left\{\begin{array}{rcl}
    L_0 &=& bL_0 \mid a L_1 \\
    L_1 &=& b L_1 \mid b L_2 \\
    L_2 &=& L_3 \mid \varepsilon \\
    L_3 &=& a L_3 \mid b L_1 \\
    \end{array}\right.$
  }

  \tf[top,x=.25\textwidth]{\small
    \begin{tikzpicture}[smAutomaton, node distance=1.3cm]
      \smState[\smInitial]   (q0)   {$q_0$};
      \smState               (q1) [right=of q0]  {$q_1$};
      \smState[\smAccepting] (q2) [below=of q1]  {$q_2$}; 
      \smState               (q3) [right=of q1]  {$q_3$}; 

      \smPath[\smStructure] (q0) edge             node {$a$} (q1);
      \smPath               (q1) edge             node[swap] {$b$} (q2);
      \smPath               (q2) edge             node[swap] {$\varepsilon$} (q3);
      \smPath               (q3) edge             node[swap] {$b$} (q1);
      \smPath[\smExample]   (q0) edge[loop above] node       {$b$} (q0);
      \smPath               (q1) edge[loop above] node       {$b$} (q1);
      \smPath               (q3) edge[loop above] node       {$a$} (q3);
    \end{tikzpicture}
  }

  \tf[text,bottom]{
    \begin{enumerate}
    \item \structure{$L_1 = b L_1 \mid b L_2$}. $\varepsilon \notin \{b\}$, donc d'après Arden, $L_1 = b^\star b L_2 = b^+ L_2$. 
    \item \structure{$L_0 = b L_0 \mid a L_1$}. $\varepsilon \notin \{b\}$, donc d'après Arden, $L_0 = b^\star a L_1 = b^\star a b^+ L_2$. 
    \item \structure{$L_3 = a L_3 \mid b L_1$}. $\varepsilon \notin \{a\}$, donc d'après Arden, $L_3 = a^\star b L_1 = a^\star b b^+ L_2$. 
    \item $\structure{L_2 = L_3 \mid \varepsilon} = a^\star b b^+ L_2 \mid \varepsilon$. $\varepsilon \notin a^\star b b^+$, donc $L_2 = (a^\star b.b^+)^\star$. 
    \end{enumerate}

    $$
    \left\{
    \begin{array}{rclclcl}
      L_0 &=& b^\star a b^+ (a^\star bb^+)^\star\\
      L_1 &=& b^+ (a^\star bb^+)^\star \\
      L_2 &=& (a^\star bb^+)^\star \\
      L_3 &=& (a^\star bb^+)^+\\
    \end{array}
    \right.
    $$

    \alert{L'automate reconnaît le langage $\mathcal{L}(A) = b^\star a b^+ (a^\star bb^+)^\star$.}
  }
\end{frame}
\endgroup

% SPDX-License-Identifier: CC-BY-SA-4.0
% Author: Matthieu Perrin
% Part: 
% Section: 
% Sub-section: 
% Frame: 

\begingroup

\begin{frame}{Application du lemme d'Arden}

  Soient $A$ et $B$ deux expressions rationnelles. \\On a les deux identités remarquables : 
  \begin{enumerate}
  \item\vspace{3mm} \structure{$(AB)^\star \equiv \varepsilon \mid  A(BA)^\star B$}
    
    \begin{center}
      \alert{Idée : montrer que $(AB)^\star$ et $\varepsilon \mid  A(BA)^\star B$ sont solutions de $X = AB X \mid \varepsilon$}
    \end{center}
    
    \begin{itemize}
    \item Posons $C = A \setminus \{\varepsilon\}$ et $D = B \setminus \{\varepsilon\}$.\\
      \begin{itemize}
      \item On a $(A B)^\star \equiv (C D)^\star$  \hspace\fill car $\forall L, L^\star = (L \mid  \varepsilon)^\star$
      \item et $\varepsilon  \mid  A (B A)^\star B \equiv \varepsilon \mid  C (D C)^\star D$
      \end{itemize}
    \item Montrons que $\varepsilon \mid  C (D C)^\star D$ est solution de $X = CD X \mid  \varepsilon$.
      \begin{itemize}
      \item $\begin{array}[t]{rclrcl}
        \structure{CD (\example{\varepsilon \mid  C (D C)^\star D}) \mid  \varepsilon}
        & = & CD \mid  CD C (D C)^\star D \mid  \varepsilon\\
        & = & CD \mid  C (D C)^+ D \mid  \varepsilon\\
        & = & C (D C)^\star D \mid  \varepsilon\\
        & = & \example{\varepsilon \mid  C (D C)^\star D}\\
      \end{array}
        $
      \end{itemize}
    \item Par le lemme d'Arden, $(CD)^\star$ est \alert{l'unique} solution de $X = CD X \mid  \varepsilon$
    \item Donc $(CD)^\star \equiv \varepsilon \mid  C (D C)^\star D$, donc $(AB)^\star \equiv \varepsilon \mid  A (BA)^\star B$
    \end{itemize}
  \item\vspace{3mm} \structure{$(A \mid  B)^\star \equiv A^\star (B A^\star)^\star  \equiv (A^\star B)^\star A^\star$}
    \begin{itemize}
    \item Idem, avec l'équation $X = (C\mid D) X \mid  \varepsilon$
    \end{itemize}
  \end{enumerate}
\end{frame}

\endgroup

 
%\section{Des expressions rationnelles aux automates}
% 
%\subsection{Génération d'un automate fini non-déterministe}
%% SPDX-License-Identifier: CC-BY-SA-4.0
% Author: Matthieu Perrin
% Part: 
% Section: 
% Sub-section: 
% Frame: 

\begingroup

\begin{frame}{Génération d'analyseur lexical}
  \begin{block}{Problème}
    \vspace{-2mm}
    \begin{description}
    \item[Entrée :] une expression rationnelle $r$
    \item[Sortie :] un \alert{analyseur lexical} pour le langage $\mathcal{S}(r)$
      \begin{itemize}
      \item Programme qui décide si son entrée appartient à $\mathcal{S}(r)$
      \end{itemize}
    \end{description}
  \end{block}
  
  \centering
  \scalebox{.9}{\begin{tikzpicture}

      \draw[white] (-1.8,0) rectangle (9.8,6.2);
      
      \draw[rounded corners, structure,fill=structure!20] (0,5) +(-1.2,-.5) rectangle +(1.2,.5) +(0,0) node{\small \begin{tabular}{c}Expression\\ rationnelle \end{tabular}};
      \draw[rounded corners, structure,fill=structure!20] (8,5) +(-1.2,-.5) rectangle +(1.2,.5) +(0,0) node{\small \begin{tabular}{c}Analyseur\\ lexical \end{tabular}};
      \draw[rounded corners, structure,fill=structure!20] (0,3) +(-1.2,-.5) rectangle +(1.2,.5) +(0,0) node{\small \begin{tabular}{c}Automate fini\\ non-déterministe \end{tabular}};
      \draw[rounded corners, alert,fill=alert!20] (4,3) +(-1.2,-.5) rectangle +(1.2,.5) +(0,0) node{\small \begin{tabular}{c}Automate fini\\ déterministe \end{tabular}};
      \draw[rounded corners, structure,fill=structure!20] (8,3) +(-1.2,-.5) rectangle +(1.2,.5) +(0,0) node{\small \begin{tabular}{c}Automate fini\\ minimal \end{tabular}};
      
      \draw[dashed, -latex] (1.2,5) -- (6.8,5);
      \draw[-latex, structure] (0,4.5) -- (0,3.5);
      \draw[-latex,alert] (1.2,3) -- (2.8,3);
      \draw[-latex] (5.2,3) -- (6.8,3);
      \draw[-latex, structure] (8,3.5) -- (8,4.5);

      \draw[example] (0,5.5) node[above]{$a (b|c)^\star$};
      \draw[example] (8,5.5) node[above]{$abc \rightarrow \cmark$, $bac \rightarrow \xmark$ };

      \draw[structure] (0,4.1) node[right]{\tiny Algorithme de};
      \draw[structure] (0,3.9) node[right]{\tiny Thompson};

      \draw[alert] (2,3.15) node{\tiny Sous-ensembles de};
      \draw[alert] (2,2.85) node{\tiny Rabin \& Scott};

      \draw (6,3.15) node{\tiny Méthode de};
      \draw (6,2.85) node{\tiny Moore};

      \draw[structure] (8,4) node[left]{\tiny Transcription};

      \draw[example] (0,2.5) node[below]{\scalebox{.5}{\begin{tikzpicture}
            \node (a) {};
            \node (b) [above=of a] {};
            \node (c) [right=of b] {};
            \node (d) [below=of c] {};
            \node (e) [above right=of d] {};
            \node (f) [right=of e] {};
            \node (g) [below right=of d] {};
            \node (h) [right=of g] {};
            \node (i) [below right=of f] {};
            \node (j) [below=of d] {};

            \node[fill=example!10,state,initial, initial text=] (a1) at (a) {$1$};
            \node[fill=example!10,state] (b1) at (b) {$2$};
            \node[fill=example!10,state] (e1) at (e) {$3$};
            \node[fill=example!10,state] (f1) at (f) {$4$};
            \node[fill=example!10,state] (g1) at (g) {$5$};
            \node[fill=example!10,state] (h1) at (h) {$6$};
            \node[fill=example!10,state] (d1) at (d) {$7$};
            \node[fill=example!10,state] (i1) at (i) {$8$};
            \node[fill=example!10,state] (c1) at (c) {$9$};
            \node[fill=example!10,state, accepting] (j1) at (j) {$10$};

            \path[->]  (a1) edge node[left] {$a$} (b1);
            \path[->]  (e1) edge node[below] {$b$} (f1);
            \path[->]  (g1) edge node[above] {$c$} (h1);
            \path[->]  (d1) edge node[right] {$\varepsilon$} (e1);
            \path[->]  (d1) edge node[right] {$\varepsilon$} (g1);
            \path[->]  (f1) edge node[left ] {$\varepsilon$} (i1);
            \path[->]  (h1) edge node[left ] {$\varepsilon$} (i1);
            \path[->]  (i1) edge node[above] {$\varepsilon$} (d1);
            \path[->]  (c1) edge node[left] {$\varepsilon$} (d1);
            \path[->]  (d1) edge node[left ] {$\varepsilon$} (j1);
            \path[->]  (b1) edge node[above] {$\varepsilon$} (c1);
      \end{tikzpicture}}};

      \draw[example] (4,2.5) node[below]{\scalebox{.5}{\begin{tikzpicture}
            \node[fill=example!10,state,initial, initial text=] (a) {$1$};
            \node[fill=example!10,state, accepting] (b) [below=of a] {$2$};
            \node[fill=example!10,state, accepting] (c) [above right=of b] {$3$};
            \node[fill=example!10,state, accepting] (d) [below right=of b] {$4$};

            \path[->]  (a) edge node[left] {$a$} (b);
            \path[->]  (b) edge node[above] {$b$} (c);
            \path[->]  (b) edge node[below] {$c$} (d);
            \path[->]  (c) edge[bend right=5mm] node[left] {$c$} (d);
            \path[->]  (d) edge[bend right=5mm] node[right] {$b$} (c);
            \path[->]  (c) edge[loop right, looseness=5] node {$b$} (c);
            \path[->]  (d) edge[loop right, looseness=5] node {$c$} (d);
      \end{tikzpicture}}};
      \draw[example] (8,2.5) node[below]{\scalebox{.5}{\begin{tikzpicture}
            \node[fill=example!10,state,initial, initial text=] (a) {$1$};
            \node[fill=example!10,state, accepting] (b) [below=of a] {$2$};

            \path[->]  (a) edge node[left] {$a$} (b);
            \path[->]  (b) edge[loop left, looseness=5] node {$b$} (b);
            \path[->]  (b) edge[loop right, looseness=5] node {$c$} (b);
      \end{tikzpicture}}};
  \end{tikzpicture}}
\end{frame}

\endgroup

%% SPDX-License-Identifier: CC-BY-SA-4.0
% Author: Matthieu Perrin
% Part: 
% Section: 
% Sub-section: 
% Frame: 

\begingroup

\begin{frame}{Génération d'un automate}
  \SetKwData{Input}{motif}

  \vspace{-2cm}
  Soit $\Sigma$ un alphabet.

  \begin{minipage}{8cm}
    \begin{block}{Théorème}
      Tout langage rationnel
      est reconnaissable par un automate fini non-déterministe
      $$ \alert{\textsc{rat}_\Sigma \subseteq  \textsc{rec}_\Sigma} $$
    \end{block}
  \end{minipage}
  \begin{block}{Démonstration}
    Algorithme de Thompson
    \begin{description}
    \item [Entrée :] expression rationnelle \structure{$\Input \in \textsc{rat}_\Sigma$}
    \item [Sortie :] automate fini non-déterministe \structure{$A$} tel que $\mathcal{L}(A) = \mathcal{S}(\Input)$ 
      \begin{itemize}
      \item par récurrence sur la structure de $\Input$
      \item construit des automates \alert{normalisés}
      \end{itemize}
    \end{description}
  \end{block}

  \vspace{-6.7cm}\hspace\fill
  \begin{minipage}{2.5cm}
    \centering
    \includegraphics[width=2.5cm]{img/Thompson}
    
    Ken Thompson\footnote[frame,1]{Prix Turing 1983 avec Dennis Ritchie pour le développement d'Unix}
  \end{minipage}\hspace{5mm}
\end{frame}
\endgroup

%% SPDX-License-Identifier: CC-BY-SA-4.0
% Author: Matthieu Perrin
% Part: 
% Section: 
% Sub-section: 
% Frame: 

\begingroup

\begin{frame}{Automate normalisé}

  Soit $A = \langle \Sigma, Q, I, F, \rightarrow\rangle$ un automate fini. 

  \begin{block}{Définition -- Automate unitaire}
    $A$ est dit \structure{unitaire} s'il possède un unique état initial \hspace\fill
    $\alert{\exists q_0 \in Q,~ I = \{q_0\}}$
  \end{block}

  \begin{block}{Définition -- Automate standard}
    $A$ est dit \structure{standard} si :
    \begin{enumerate}
    \item il est \alert{unitaire}
    \item aucune transition n'arrive sur l'état initial \hspace\fill
    $\alert{\nexists q\in Q,~ \nexists a\in \Sigma,~ q \xrightarrow{a} q_0}$
    \end{enumerate}
  \end{block}

  \begin{block}{Définition -- Automate normalisé}
    $A$ est dit \structure{normalisé} si :
    \begin{enumerate}
    \item il est \alert{standard}
    \item il possède un unique état final \hspace\fill $\alert{\exists q_f \in Q,~ F = \{q_f\}}$
    \item aucune transition ne sort de l'état final \hspace\fill
    $\alert{\nexists q\in Q,~ \nexists a\in \Sigma,~ q_f \xrightarrow{a} q}$
    \end{enumerate}
  \end{block}
  
\end{frame}

\endgroup

%% SPDX-License-Identifier: CC-BY-SA-4.0
% Author: Matthieu Perrin
% Part: 
% Section: 
% Sub-section: 
% Frame: 

\begingroup

\begin{frame}{Normalisation d'un automate fini}

  \onBlock[top=-3mm]{Théorème -- Normalisation}{
      Soit $\Sigma$ un alphabet et $L \in \textsc{rec}_\Sigma$.
      $L$ est reconnu par un automate normalisé.
  }

  \onBlock[y=5mm]{Démonstration}{
    Soit $\langle \Sigma, Q, I, F, \rightarrow \rangle$ un automate qui reconnaît $L$.
    \begin{itemize}
    \item Soient $i, f \notin Q$
    \item Soit $\rightarrow' \,=\, \rightarrow \,\cup\,
      \structure{\{\langle i, \varepsilon, q\rangle \mid q \in I\}} \,\cup\,
      \alert{\{\langle q, \varepsilon, f\rangle \mid q \in F\}}$
    \end{itemize}
    Alors $L$ est reconnu par l'automate normalisé :
    $$\langle \Sigma, Q \cup \{i, f\}, \structure{\{i\}}, \alert{\{f\}}, \rightarrow' \rangle$$
  }

  \onExampleBlock[y=-15mm]{Exemple}{}
  
  \on[bottom,x=-.25\textwidth]{
    \begin{tikzpicture}[automaton]
      \state            (2)                     {$2$};
      \state[initial]   (0) [above left =of 2]  {$0$};
      \state[initial]   (1) [below left =of 2]  {$1$};
      \state[accepting] (3) [above right=of 2]  {$3$};
      \state[accepting] (4) [below right=of 2]  {$4$};
      
      \path  (0) edge node {$a$} (3);
      \path  (0) edge node {$b$} (2);
      \path  (1) edge node {$a$} (2);
      \path  (1) edge node {$b$} (4);
      \path  (2) edge node {$a$} (3);
      \path  (2) edge node {$b$} (4);
    \end{tikzpicture}
  }
  
  \on[bottom,x=.25\textwidth]{
    \begin{tikzpicture}[automaton]
      \state                     (2)                     {$2$};
      \state                     (0) [above left =of 2]  {$0$};
      \state                     (1) [below left =of 2]  {$1$};
      \state                     (3) [above right=of 2]  {$3$};
      \state                     (4) [below right=of 2]  {$4$};
      \state[initial, structure] (i) [above left =of 1]  {$i$};
      \state[accepting, alert]   (f) [above right=of 4]  {$f$};

      \path            (0) edge node {$a$} (3);
      \path            (0) edge node {$b$} (2);
      \path            (1) edge node {$a$} (2);
      \path            (1) edge node {$b$} (4);
      \path            (2) edge node {$a$} (3);
      \path            (2) edge node {$b$} (4);
      \path[structure] (i) edge node {$\varepsilon$} (0);
      \path[structure] (i) edge node {$\varepsilon$} (1);
      \path[alert]     (3) edge node {$\varepsilon$} (f);
      \path[alert]     (4) edge node {$\varepsilon$} (f);
    \end{tikzpicture}
  }
  
\end{frame}


\endgroup

%% SPDX-License-Identifier: CC-BY-SA-4.0
% Author: Matthieu Perrin
% Part: 
% Section: 
% Sub-section: 
% Frame: 

\begingroup

\SetKwFunction{Thompson}{thompson}
\SetKwData{Input}{motif}

\begin{frame}{Algorithme de Thompson}

  \tf[y=-3mm]{\footnotesize
    \begin{algorithm}[H]
      
      \Fn{\Thompson(\Input : \textsc{regex}, $\Sigma$ : alphabet) : automate}{
        \Lets \alert{$q_0$}, \alert{$q_f$} deux nouveaux états\;
        \uSi{$\Input = \emptyset$}{
          \vspace{.5mm}
          \uncover<2-|handout>{
            $\begin{array}{@{}l@{\,\gets\,}l@{}}
              Q & \{\alert{q_0}, \alert{q_f}\};\\
              \tau & \emptyset;
            \end{array}$
          }
          \vspace{.5mm}
        }\uSinonSi{$\Input = a\in \Sigma \cup \{\varepsilon\}$}{
          \vspace{.5mm}
          \uncover<3-|handout>{
            $\begin{array}{@{}l@{\,\gets\,}l@{}}
              Q & \{\alert{q_0}, \alert{q_f}\};\\
              \tau & \left\{ \alert{\langle q_0, a, q_f \rangle} \right\};
            \end{array}$
          }
          \vspace{.5mm}
        }\uSinonSi{$\Input = u \mid v$}{
          \vspace{.5mm}
          \uncover<4-|handout>{
            $\langle \structure{\Sigma}, \structure{Q_u}, \{\structure{u_0}\}, \{\structure{u_f}\}, \structure{\tau_{u}} \rangle \gets \structure{\Thompson(u, \Sigma)}$\;
            $\langle \example{\Sigma}, \example{Q_v}, \{\example{v_0}\}, \{\example{v_f}\}, \example{\tau_{v}}\, \rangle \gets \example{\Thompson(v, \Sigma)}$\;
            $\begin{array}{@{}l@{\,\gets\,}c@{\,\cup\,}c@{\,\cup\,}l@{}}
              Q    & \structure{Q_u}    & \example{Q_v}    & \{\alert{q_0}, \alert{q_f}\};\\
              \tau & \structure{\tau_u} & \example{\tau_v} & \left\{
              \begin{array}{@{\alert{\langle}}c@{\alert{, \varepsilon,}\,}c@{\alert{\rangle}, \alert{\langle}}c@{\alert{, \varepsilon,}\,}c@{\alert{\rangle}}l@{}}
                \alert{q_0}&\alert{u_0}&\alert{q_0}&\alert{v_0}&,\\
                \alert{u_f}&\alert{q_f}&\alert{v_f}&\alert{q_f}&
              \end{array}\right\};
            \end{array}$
          }
          \vspace{.5mm}
        }\uSinonSi{$\Input = u \cdot v$}{
          \vspace{.5mm}
          \uncover<5-|handout>{
            $\langle \structure{\Sigma}, \structure{Q_u}, \{\structure{u_0}\}, \{\structure{u_f}\}, \structure{\tau_{u}} \rangle \gets \structure{\Thompson(u, \Sigma)}$\;
            $\langle \example{\Sigma}, \example{Q_v}, \{\example{v_0}\}, \{\example{v_f}\}, \example{\tau_{v}}\, \rangle \gets \example{\Thompson(v, \Sigma)}$\;
            $\begin{array}{@{}l@{\,\gets\,}l}
              Q & \structure{Q_u} \cup \example{Q_v} \setminus \{v_0\};\\
              q_0 & \structure{u_0};\,q_f \gets \example{v_f};\\
              \tau & \structure{\tau_u} \cup \example{\tau_v} \cup \left\{\alert{\langle u_f, \varepsilon, v_0\rangle}\right\}
            \end{array}$
          }
          \vspace{.5mm}
        }\SinonSi{$\Input = u^\star$}{
          \vspace{.5mm}
          \uncover<6-|handout>{
            $\langle \structure{\Sigma}, \structure{Q_u}, \{\structure{u_0}\}, \{\structure{u_f}\}, \structure{\tau_{u}} \rangle \gets \structure{\Thompson(u, \Sigma)}$\;
            $\begin{array}{@{}l@{\,\gets\,}c@{\,\cup\,}l@{}}
              Q & \structure{Q_u} & \{\alert{q_0}, \alert{q_f}\};\\
              \tau & \structure{\tau_u} & \left\{\alert{\langle q_0, \varepsilon, u_0\rangle}, \alert{\langle u_f, \varepsilon, u_0\rangle}, \alert{\langle u_f, \varepsilon, q_f\rangle}\right\}
            \end{array}$
          }
        }
        \Retourner $\langle \Sigma, Q, \{q_0\}, \{q_f\}, \tau \rangle$\;
      }
    \end{algorithm}
  }

  \tf<2-|handout>[x=-1cm, y=26mm]{\footnotesize
    \begin{tikzpicture}[smAutomaton]
      \smState[\smInitial \smAlert]   (a1) at (0.0,0) {$q_0$}; 
      \smState[\smAccepting \smAlert] (a2) at (1.5,0) {$q_f$}; 
    \end{tikzpicture}
  }

  \tf<3-|handout>[x=-1cm, y=15mm]{\footnotesize
    \begin{tikzpicture}[smAutomaton]
      \smState[\smInitial \smAlert]   (b1) at (0.0,0) {$q_0$}; 
      \smState[\smAccepting \smAlert] (b2) at (1.5,0) {$q_f$}; 

      \smPath[\smAlert] (b1) edge node{$a$} (b2); 
    \end{tikzpicture}
  }

  \tf<4-|handout>[y=22mm, x=5mm, anchor=west]{\footnotesize
    \begin{tikzpicture}[smAutomaton]
      \node[smCloud, fill=structure!12]  (c3) at (2.75,1.0) {};
      \node[smCloud, fill=example!12]    (c6) at (2.75,0.0) {};

      \smState [\smInitial\smAlert]   (c1) at (0.75,0.5) {$q_0$};
      \smState [\smStructure]         (c2) at (1.75,1.0) {$u_0$};
      \smState [\smStructure]         (c4) at (3.25,1.0) {$u_f$};
      \smState [\smExample  ]         (c5) at (1.75,0.0) {$v_0$};
      \smState [\smExample  ]         (c7) at (3.25,0.0) {$v_f$};
      \smState [\smAccepting\smAlert] (c8) at (4.25,0.5) {$q_f$};
      
      \smPath[\smAlert] (c1) edge node {$\varepsilon$} (c2);
      \smPath[\smAlert] (c1) edge node {$\varepsilon$} (c5);
      \smPath[\smAlert] (c4) edge node {$\varepsilon$} (c8);
      \smPath[\smAlert] (c7) edge node {$\varepsilon$} (c8);
      
      \path [-latex, dashed, structure!65] (c2) edge[bend left=5mm]  (c4);
      \path [-latex, dashed, structure!65] (c2) edge                 (c4);
      \path [-latex, dashed, structure!65] (c2) edge[bend right=5mm] (c4);
      \path [-latex, dashed, example!65]   (c5) edge[bend left=5mm]  (c7);
      \path [-latex, dashed, example!65]   (c5) edge                 (c7);
      \path [-latex, dashed, example!65]   (c5) edge[bend right=5mm] (c7);
    \end{tikzpicture}
  }
  
  \tf<5-|handout>[y=-6mm, x=5mm, anchor=west]{\footnotesize
    \begin{tikzpicture}[smAutomaton]
      \node[smCloud, fill=structure!12]  (d3) at (1.5,2.5) {};
      \node[smCloud, fill=example!12]    (d6) at (4.0,2.5) {};

      \smState [\smInitial \smStructure] (d2) at (0.5,2.5) {$u_0$};
      \smState [           \smStructure] (d4) at (2.0,2.5) {$u_f$};
      \smState [             \smExample] (d5) at (3.0,2.5) {$v_0$};
      \smState [\smAccepting \smExample] (d7) at (4.5,2.5) {$v_f$};
      
      \smPath[\smAlert] (d4) edge node {$\varepsilon$} (d5);
      
      \path [-latex, dashed, structure!65] (d2) edge[bend left=5mm]  (d4);
      \path [-latex, dashed, structure!65] (d2) edge                 (d4);
      \path [-latex, dashed, structure!65] (d2) edge[bend right=5mm] (d4);
      \path [-latex, dashed, example!65]   (d5) edge[bend left=5mm]  (d7);
      \path [-latex, dashed, example!65]   (d5) edge                 (d7);
      \path [-latex, dashed, example!65]   (d5) edge[bend right=5mm] (d7);

      \node  at (2.5,1.75) {OU};

      \node[smCloud, fill=structure!12]  (d3) at (1.5,1.0) {};
      \node[smCloud, fill=example!12]    (d6) at (4.0,1.0) {};

      \smState [\smInitial \smStructure] (d2) at (0.5,1.0) {$u_0$};
      \node [state, fill=alert!20, rectangle, rounded corners=3mm, text width=13mm, anchor=mid, align=center] (d4) at (2.75,1.0) {$u_f = v_0$};
      \smState [\smAccepting \smExample] (d7) at (4.5,1.0) {$v_f$};
      
      \path [-latex, dashed, structure!65] (d2) edge[bend left=5mm]  (d4);
      \path [-latex, dashed, structure!65] (d2) edge                 (d4);
      \path [-latex, dashed, structure!65] (d2) edge[bend right=5mm] (d4);
      \path [-latex, dashed, example!65]   (d4) edge[bend left=5mm]  (d7);
      \path [-latex, dashed, example!65]   (d4) edge                 (d7);
      \path [-latex, dashed, example!65]   (d4) edge[bend right=5mm] (d7);
    \end{tikzpicture}
  }

  \tf<6-|handout>[y=-32mm, x=5mm, anchor=west]{\footnotesize
    \begin{tikzpicture}[smAutomaton]
      \node[smCloud, fill=structure!12]  (e3) at (2.0,1.0) {};
      \smState[\smInitial   \smAlert]    (e1) at (0.0,1.5) {$q_0$};
      \smState[\smStructure         ]    (e2) at (1.0,1.0) {$u_0$};
      \smState[\smStructure         ]    (e4) at (2.5,1.0) {$u_f$};
      \smState[\smAccepting \smAlert]    (e5) at (0.0,0.5) {$q_f$};
      
      \smPath[\smAlert] (e1) edge                 node {$\varepsilon$} (e2);
      \smPath[\smAlert] (e2) edge                 node {$\varepsilon$} (e5);
      \smPath[\smAlert] (e4) edge[bend left=15mm] node {$\varepsilon$} (e2);

      \path[-latex, dashed, structure!65] (e2) edge[bend left=5mm]  (e4);
      \path[-latex, dashed, structure!65] (e2) edge                 (e4);
      \path[-latex, dashed, structure!65] (e2) edge[bend right=5mm] (e4);
    \end{tikzpicture}
  }
  
\end{frame}

\endgroup

%% SPDX-License-Identifier: CC-BY-SA-4.0
% Author: Matthieu Perrin
% Part: 
% Section: 
% Sub-section: 
% Frame: 

\begingroup

\begin{frame}{Exemple d'utilisation du lemme de pompage}
  
  \tfBlock[top=-5mm]{Montrer que $L \eqdef \{a^nb^nc^n \mid n\in \mathbb{N}\}$ n'est pas algébrique}{
    Soit $\Sigma \eqdef \{a, b, c\}$.%
    \only<2-|handout>{
      Si $L$ est algébrique, $L$ vérifie le lemme de pompage :

      \vspace{-4mm}
      $$
      \begin{array}{c}
        \structure{\exists N\in \mathbb{N}}, \alert{\forall u\in L, |u| \ge N} \Rightarrow (\structure{\exists v, w, x, y, z\in \Sigma^\star}, \\
        u = v \cdot w \cdot x\cdot y \cdot z \land w \cdot y\neq \varepsilon \land |w\cdot x\cdot y| \le N \land \alert{\forall i \in \mathbb{N}}, v\cdot w^i\cdot x\cdot y^i \cdot z \in L)
      \end{array}
      $$
      \vspace{-2mm}
      
      \structure{Soit $N$ donné par le lemme pompage}.
    }
    
    \only<3-|handout>{%
      \alert{Posons $u = a^N b^N c^N$. On a bien $u\in L$ et $|u| = 3N \ge N$}.\\
      \structure{Soit $v \cdot w \cdot x\cdot y \cdot z$ la décomposition de $u$ donnée par le lemme de pompage}. 
    }
  }
  
  \tf<4-|handout>[y=-10mm]{
    \begin{smArray}[width=4mm, height=3mm, name={$u=$}]
      \smCell[\smNone]{\alert{$a$}}      \smCoord{(a1)}
      \smCell[\smNone]{\alert{$\cdots$}}               
      \smCell[\smNone]{\alert{$a$}}      \smCoord{(an)}
      \smCell[\smNone]{\alert{$\cdot$}}               
      \smCell[\smNone]{\alert{$b$}}      \smCoord{(b1)}
      \smCell[\smNone]{\alert{$\cdots$}}               
      \smCell[\smNone]{\alert{$b$}}      \smCoord{(bn)}
      \smCell[\smNone]{\alert{$\cdot$}}               
      \smCell[\smNone]{\alert{$c$}}      \smCoord{(c1)}
      \smCell[\smNone]{\alert{$\cdots$}}               
      \smCell[\smNone]{\alert{$c$}}      \smCoord{(cn)}
      
      \draw [decorate, decoration={brace, amplitude=5pt}] ([xshift=1mm]a1.north west) -- ([xshift=-1mm]an.north east) node[midway,yshift=4mm]{$N$};
      \draw [decorate, decoration={brace, amplitude=5pt}] ([xshift=1mm]b1.north west) -- ([xshift=-1mm]bn.north east) node[midway,yshift=4mm]{$N$};
      \draw [decorate, decoration={brace, amplitude=5pt}] ([xshift=1mm]c1.north west) -- ([xshift=-1mm]cn.north east) node[midway,yshift=4mm]{$N$};

      \draw [decorate, decoration={brace, amplitude=5pt, mirror}] ([xshift=1mm]an.south west) -- ([xshift=-1mm]b1.south east) node[midway,yshift=-4mm]{$|wxy|\le N$};
      \draw [decorate, decoration={brace, amplitude=5pt, mirror}] ([xshift=1mm]bn.south west) -- ([xshift=-1mm]c1.south east) node[midway,yshift=-4mm]{$|wxy|\le N$};
    \end{smArray}       
  }

  \tf<5-|handout>[text, bottom=-1mm]{
    \begin{itemize}
    \item Comme $wy \neq \varepsilon$, $\alpha = wy[1] \in \Sigma$ est une lettre de $wy$.
    \item Comme $|wxy| \le N$, il existe $\beta \in \Sigma$ tel que $\beta$ n'est pas une lettre de $wy$. 
    \item \alert{Posons $i=2$}. $|v \cdot w^2 \cdot x\cdot  y^2\cdot   z|_\alpha > |v\cdot  w^2\cdot  x\cdot  y^2\cdot   z|_\beta$, donc $v\cdot  w^2\cdot  x\cdot  y^2\cdot   z\notin L$. 
    \end{itemize}
    Absurde ! Donc $L$ n'est pas algébrique. 
  }

  \tfExampleBlock<-4>[y=-25mm]{On sait}{}

  \tf<1>[y=-25mm, anchor=north, text]{
    \begin{itemize}
    \item $\begin{array}[t]{l}
      \alert{\forall L \in \textsc{alg}_\Sigma}, \structure{\exists N\in \mathbb{N}}, \forall u\in L, |u| \ge N \Rightarrow (\exists v, w, x, y, z\in \Sigma^\star, \\
      u = v w x y z \land w y\neq \varepsilon \land |wxy| \le N \land \forall i \in \mathbb{N}, v w^i x y^i z \in L)
    \end{array}$
    \end{itemize}
  }

  \tf<2-4>[y=-25mm, anchor=north, left=.35\textwidth]{
    \begin{itemize}
    \item $L \in \textsc{alg}_\Sigma$
    \item<3-> $v, w, x, y, z\in \Sigma^\star$
    \end{itemize}
  }
  
  \tf<2-4>[y=-25mm, anchor=north, width=.35\textwidth]{
    \begin{itemize}
    \item $N \in \mathbb{N}$
    \item<3-> $wy\neq \varepsilon$
    \end{itemize}
  }
  
  \tf<3-4>[y=-25mm, anchor=north, right=.35\textwidth]{
    \begin{itemize}
    \item $v \cdot w \cdot x \cdot y \cdot z = a^N b^N c^N$
    \item $\alert{|wxy| \le N}$
    \end{itemize}
  }
  
\end{frame}

\endgroup

% 
%\subsection{Déterminisation d'un automate fini}
%% SPDX-License-Identifier: CC-BY-SA-4.0
% Author: Matthieu Perrin
% Part: 
% Section: 
% Sub-section: 
% Frame: 

\begingroup

\begin{frame}{Génération d'analyseur lexical}
  \begin{block}{Problème}
    \vspace{-2mm}
    \begin{description}
    \item[Entrée :] une expression rationnelle $r$
    \item[Sortie :] un \alert{analyseur lexical} pour le langage $\mathcal{S}(r)$
      \begin{itemize}
      \item Programme qui décide si son entrée appartient à $\mathcal{S}(r)$
      \end{itemize}
    \end{description}
  \end{block}
  
  \centering
  \scalebox{.9}{\begin{tikzpicture}

      \draw[white] (-1.8,0) rectangle (9.8,6.2);
      
      \draw[rounded corners, structure,fill=structure!20] (0,5) +(-1.2,-.5) rectangle +(1.2,.5) +(0,0) node{\small \begin{tabular}{c}Expression\\ rationnelle \end{tabular}};
      \draw[rounded corners, structure,fill=structure!20] (8,5) +(-1.2,-.5) rectangle +(1.2,.5) +(0,0) node{\small \begin{tabular}{c}Analyseur\\ lexical \end{tabular}};
      \draw[rounded corners, structure,fill=structure!20] (0,3) +(-1.2,-.5) rectangle +(1.2,.5) +(0,0) node{\small \begin{tabular}{c}Automate fini\\ non-déterministe \end{tabular}};
      \draw[rounded corners, alert,fill=alert!20] (4,3) +(-1.2,-.5) rectangle +(1.2,.5) +(0,0) node{\small \begin{tabular}{c}Automate fini\\ déterministe \end{tabular}};
      \draw[rounded corners, structure,fill=structure!20] (8,3) +(-1.2,-.5) rectangle +(1.2,.5) +(0,0) node{\small \begin{tabular}{c}Automate fini\\ minimal \end{tabular}};
      
      \draw[dashed, -latex] (1.2,5) -- (6.8,5);
      \draw[-latex, structure] (0,4.5) -- (0,3.5);
      \draw[-latex,alert] (1.2,3) -- (2.8,3);
      \draw[-latex] (5.2,3) -- (6.8,3);
      \draw[-latex, structure] (8,3.5) -- (8,4.5);

      \draw[example] (0,5.5) node[above]{$a (b|c)^\star$};
      \draw[example] (8,5.5) node[above]{$abc \rightarrow \cmark$, $bac \rightarrow \xmark$ };

      \draw[structure] (0,4.1) node[right]{\tiny Algorithme de};
      \draw[structure] (0,3.9) node[right]{\tiny Thompson};

      \draw[alert] (2,3.15) node{\tiny Sous-ensembles de};
      \draw[alert] (2,2.85) node{\tiny Rabin \& Scott};

      \draw (6,3.15) node{\tiny Méthode de};
      \draw (6,2.85) node{\tiny Moore};

      \draw[structure] (8,4) node[left]{\tiny Transcription};

      \draw[example] (0,2.5) node[below]{\scalebox{.5}{\begin{tikzpicture}
            \node (a) {};
            \node (b) [above=of a] {};
            \node (c) [right=of b] {};
            \node (d) [below=of c] {};
            \node (e) [above right=of d] {};
            \node (f) [right=of e] {};
            \node (g) [below right=of d] {};
            \node (h) [right=of g] {};
            \node (i) [below right=of f] {};
            \node (j) [below=of d] {};

            \node[fill=example!10,state,initial, initial text=] (a1) at (a) {$1$};
            \node[fill=example!10,state] (b1) at (b) {$2$};
            \node[fill=example!10,state] (e1) at (e) {$3$};
            \node[fill=example!10,state] (f1) at (f) {$4$};
            \node[fill=example!10,state] (g1) at (g) {$5$};
            \node[fill=example!10,state] (h1) at (h) {$6$};
            \node[fill=example!10,state] (d1) at (d) {$7$};
            \node[fill=example!10,state] (i1) at (i) {$8$};
            \node[fill=example!10,state] (c1) at (c) {$9$};
            \node[fill=example!10,state, accepting] (j1) at (j) {$10$};

            \path[->]  (a1) edge node[left] {$a$} (b1);
            \path[->]  (e1) edge node[below] {$b$} (f1);
            \path[->]  (g1) edge node[above] {$c$} (h1);
            \path[->]  (d1) edge node[right] {$\varepsilon$} (e1);
            \path[->]  (d1) edge node[right] {$\varepsilon$} (g1);
            \path[->]  (f1) edge node[left ] {$\varepsilon$} (i1);
            \path[->]  (h1) edge node[left ] {$\varepsilon$} (i1);
            \path[->]  (i1) edge node[above] {$\varepsilon$} (d1);
            \path[->]  (c1) edge node[left] {$\varepsilon$} (d1);
            \path[->]  (d1) edge node[left ] {$\varepsilon$} (j1);
            \path[->]  (b1) edge node[above] {$\varepsilon$} (c1);
      \end{tikzpicture}}};

      \draw[example] (4,2.5) node[below]{\scalebox{.5}{\begin{tikzpicture}
            \node[fill=example!10,state,initial, initial text=] (a) {$1$};
            \node[fill=example!10,state, accepting] (b) [below=of a] {$2$};
            \node[fill=example!10,state, accepting] (c) [above right=of b] {$3$};
            \node[fill=example!10,state, accepting] (d) [below right=of b] {$4$};

            \path[->]  (a) edge node[left] {$a$} (b);
            \path[->]  (b) edge node[above] {$b$} (c);
            \path[->]  (b) edge node[below] {$c$} (d);
            \path[->]  (c) edge[bend right=5mm] node[left] {$c$} (d);
            \path[->]  (d) edge[bend right=5mm] node[right] {$b$} (c);
            \path[->]  (c) edge[loop right, looseness=5] node {$b$} (c);
            \path[->]  (d) edge[loop right, looseness=5] node {$c$} (d);
      \end{tikzpicture}}};
      \draw[example] (8,2.5) node[below]{\scalebox{.5}{\begin{tikzpicture}
            \node[fill=example!10,state,initial, initial text=] (a) {$1$};
            \node[fill=example!10,state, accepting] (b) [below=of a] {$2$};

            \path[->]  (a) edge node[left] {$a$} (b);
            \path[->]  (b) edge[loop left, looseness=5] node {$b$} (b);
            \path[->]  (b) edge[loop right, looseness=5] node {$c$} (b);
      \end{tikzpicture}}};
  \end{tikzpicture}}
\end{frame}

\endgroup

%% SPDX-License-Identifier: CC-BY-SA-4.0
% Author: Matthieu Perrin
% Part: 
% Section: 
% Sub-section: 
% Frame: 

\begingroup

\begin{frame}{Équivalence et isomorphisme entre automates}

  \tf[text, top]{
    Soient $A = \langle \Sigma, Q_A, I_A, F_A, \rightarrow_A \rangle$ et $B = \langle \Sigma, Q_B, I_B, F_B, \rightarrow_B \rangle$ deux automates. 
  }

  \tfBlock[y=17mm]{Définition -- Automates équivalents}{
    $A$ et $B$ sont \structure{équivalents}, noté $\alert{A \equiv B}$, s'ils acceptent le même langage :
    $$\alert{A_1 \equiv A_2  \eqdef \mathcal{L}(A_1) = \mathcal{L}(A_2)}.$$
  }

  \tfBlock[bottom=2mm]{Définition -- Automates isomorphes}{
    $A$ et $B$ sont \structure{isomorphes}, noté $\alert{A \simeq B}$, s'ils ne diffèrent que par le nom de leurs états, \textit{c.-à-d.}
    s'il existe une fonction bijective $f : Q_A \rightarrow Q_B$ telle que :

    \vspace{2mm}
    $\structure{\forall a\in \Sigma, \forall q, q',}$\\
    $\structure{\begin{array}[t]{@{\quad\quad}lll}
        q\in Q_A &\Leftrightarrow& f(q) \in Q_B \\
        q\in I_A &\Leftrightarrow& f(q) \in I_B \\
        q\in F_A &\Leftrightarrow& f(q) \in F_B \\
        q \xrightarrow{a}_A q' &\Leftrightarrow& f(q) \xrightarrow{a}_B f(q')
      \end{array}
    }$
    \vspace{2mm}

    Deux automates isomorphes sont équivalents. 
  }

  \tf[bottom=2mm, x=3cm]{\small
    \begin{tikzpicture}[smAutomaton]
      \draw (1,0.75) node{\normalsize $\simeq$};

      \smState[\smStructure \smInitial]   (a0) at (0.0, 1.5) {$A$};
      \smState[\smStructure \smAccepting] (a1) at (0.0, 0.0) {$B$};
      \smState[\smExample \smInitial]     (b0) at (2.0, 1.5) {$0$};
      \smState[\smExample \smAccepting]   (b1) at (2.0, 0.0) {$1$};

      \smPath[\smStructure] (a0) edge[bend left]  node {$a$} (a1);
      \smPath[\smStructure] (a1) edge[bend left]  node {$a$} (a0);
      \smPath[\smStructure] (a1) edge[loop right] node {$b$} (a1);
      \smPath[\smStructure] (a0) edge[loop right] node {$b$} (a0);
      
      \smPath[\smExample]   (b0) edge[bend left]  node {$a$} (b1);
      \smPath[\smExample]   (b1) edge[bend left]  node {$a$} (b0);
      \smPath[\smExample]   (b1) edge[loop right] node {$b$} (b1);
      \smPath[\smExample]   (b0) edge[loop right] node {$b$} (b0);
    \end{tikzpicture}
  }
\end{frame}

\endgroup

%% SPDX-License-Identifier: CC-BY-SA-4.0
% Author: Matthieu Perrin
% Part: 
% Section: 
% Sub-section: 
% Frame: 

\begingroup

\begin{frame}{Interprétation ubiquitaire du non-déterminisme}

  \onBlock[top=-3mm]{Interprétation du non-déterminisme comme de l'ubiquité}{
    \begin{itemize}
    \item L'automate se trouve dans un sous-ensemble des états
    \item Le mot est reconnu si l'un des états du sous-ensemble est final
    \item Ces sous-ensembles forment un nouvel automate, qui est déterministe
    \end{itemize}
  }

  \onExampleBlock[y=1mm]{Exemple : reconnaissance de $\alert{abc}$ par l'automate suivant}{
    \vspace{-2mm}
    $$
    \alertb<1>{\{1\}}
    \uncover<2->{\xrightarrow{a} \alert<2> {\{2, 3, 5, 7, 9, 10\}}}
    \uncover<3->{\xrightarrow{b} \alertb<3>{\{3, 4, 5, 7, 8, 10\}}}
    \uncover<4->{\xrightarrow{c} \alertb<4>{\{3, 5, 6, 7, 8, 10\}}}
    $$
  }
  
  \on[bottom] {
    \begin{tikzpicture}[automaton]
      \state[initial,   alert ob=<1> ] (1)  at (0,2) {$1$};    
      \state[           alert on=<2> ] (2)  at (1,2) {$2$};    
      \state[           alert on=<2->] (3)  at (4,2) {$3$};    
      \state[           alert ob=<3> ] (4)  at (5,2) {$4$};    
      \state[           alert on=<2->] (5)  at (4,0) {$5$};    
      \state[           alert ob=<4->] (6)  at (5,0) {$6$};    
      \state[           alert on=<2->] (7)  at (3,1) {$7$};    
      \state[           alert ob=<3->] (8)  at (6,1) {$8$};    
      \state[           alert on=<2> ] (9)  at (2,2) {$9$};    
      \state[accepting, alert on=<2->] (10) at (2,0) {$10$};   

      \path (1) edge node {$a$}           (2);
      \path (3) edge node {$b$}           (4);
      \path (5) edge node {$c$}           (6);
      \path (7) edge node {$\varepsilon$} (3);
      \path (7) edge node {$\varepsilon$} (5);
      \path (4) edge node {$\varepsilon$} (8);
      \path (6) edge node {$\varepsilon$} (8);
      \path (9) edge node {$\varepsilon$} (7);
      \path (8) edge node {$\varepsilon$} (7);
      \path (7) edge node {$\varepsilon$} (10);
      \path (2) edge node {$\varepsilon$} (9);
    \end{tikzpicture}
  }

\end{frame}

\endgroup

%% SPDX-License-Identifier: CC-BY-SA-4.0
% Author: Matthieu Perrin
% Part: 
% Section: 
% Sub-section: 
% Frame: 

\begingroup

\SetKwFunction{RabinScott}{rabin\_scott}
\SetKwFunction{Fermeture}{$\varepsilon$-fermeture}

\begin{frame}{Méthode des sous-ensembles de Rabin et Scott}

  \onExampleBlock[right=3.3cm, top=-5mm]{Exemple}{\vspace{-3mm}
    \begin{tikzpicture}[automaton, grid size=10mm]
      \state[initial,                      example on=<2-4>] (0) at (0,1) {$0$};
      \state[accepting,alert ob=<5>                        ] (1) at (1,1) {$1$};
      \state[                                              ] (2) at (2,1) {$2$};
      \state[initial,                      example on=<2-4>] (3) at (0,0) {$3$};
      \state[            structure on=<2>, example ob=<3-4>] (4) at (1,0) {$4$};
      \state[            structure on=<2>, example ob=<3-4>] (5) at (2,0) {$5$};
      
      \path[structure on=<2>] (3) edge             node {$\varepsilon$} (4);
      \path[structure on=<2>] (4) edge             node {$\varepsilon$} (5);
      \path[                ] (2) edge             node {$b$}           (1);
      \path[alert ob=<3>    ] (5) edge node {$a$}           (2);
      \path[                ] (2) edge[bend left ] node {$a$}           (5);
      \path[alert ob=<4>    ] (4) edge[bend left ] node {$b$}           (1);
      \path[                ] (1) edge node {$b$}           (4);
      \path[                ] (1) edge[loop above] node {$a$}           (1);
      \path[                ] (2) edge[loop above] node {$b$}           (2);
      \path[alert ob=<4>    ] (4) edge[loop below] node {$b$}           (4);
      \path[alert ob=<3>    ] (5) edge[loop below] node {$a$}           (5);
    \end{tikzpicture}
  }
 
  \onExampleBlock<2->[right=3.3cm,y=-1mm]{Calcul de $s_0$}{
    $\begin{array}{r@{\,=\,}l}
      s_0 & \Fermeture_A(\{0,3\}) \\
          & \{\example{0}, \example{3}, \structure{4}, \structure{5}\}
    \end{array}$
  }  
  
  \on[top=-2mm]{\small
    \begin{algorithm}[H]
      \Fn{\RabinScott($A = \langle \Sigma, Q, I, F, \rightarrow \rangle$ : AFN) : AFD}{
        $\begin{array}{@{}l@{\,\gets\,}l@{}}
          s_0 & \Alertb<2>{\Fermeture_A(I)};\\
          S & \{s_0\};\\
          \tau & \emptyset;
        \end{array}$\\
        \Tantque{\Alertb<3-4>{$\exists s\in S, \exists a\in \Sigma, \nexists s'\in S,~ \langle s, a, s' \rangle \in \tau$}}{
          $\begin{array}{@{}l@{\,\gets\,}l@{}}
            s' & \Structureb<4>{\Fermeture_A(\{q'\in Q \mid \exists q\in s,~ q \xrightarrow{a} q' \})};\\
            S & S \cup \{s'\};\\
            \tau & \tau \cup \{\langle s, a, s' \rangle\};
          \end{array}$\\
        }
        $S_f \gets \Alertb<5>{\{s \in S \mid s \cap F \neq \emptyset\}}$\;
        \Retourner $\langle \Sigma, S, \{s_0\}, S_f, \tau \rangle$\;
      }
      \Fn{$\Fermeture_A(s \subseteq Q) \subseteq Q$}{
        \lTantque{\Structureb<2>{$\exists q\in s,~ \exists q' \notin s,~ q\xrightarrow{\varepsilon} q'$}}{$s \gets s \cup \{q'\}$}
        \Retourner $s$\;
      }
    \end{algorithm}
  }
 
  \on<2->[bottom=2mm, x=-33mm] {
    \begin{tikzpicture}[automaton, y=8mm, x=4mm]
      \useasboundingbox (-.5,-.5) rectangle (11.5,2.5);
      
      \state    [initial               ] (a) at (0,1) {$s_0$}; 
      \state<4->[                      ] (b) at (2,2) {$s_1$}; 
      \state<5->[accepting,alert ob=<5>] (c) at (2,0) {$s_2$}; 
      \state<6->[accepting             ] (d) at (9,2) {$s_3$}; 
      \state<6->[accepting             ] (e) at (9,0) {$s_4$}; 
      \state<6->[accepting             ] (f) at (7,1) {$s_5$};
      \state<6->[accepting             ] (g) at (11,1) {$s_6$}; 
      \state<6->[                      ] (h) at (4,1) {$s_7$}; 
 
      \onlyb<3,4>{
        \state  [alert] (empty) at (4,1) {$?$};
        \path<3>[alert] (a) edge node {$a$} (empty);
        \path<4>[alert] (a) edge node {$b$} (empty);
      }
 
      \path<4-> (a) edge[bend left ] node[swap] {$a$} (b);
      \path<5-> (a) edge[bend right] node       {$b$} (c);
      \path<6-> (b) edge[loop above] node       {$a$} (b);
      \path<6-> (b) edge             node       {$b$} (d);
      \path<6-> (c) edge             node       {$a$} (e);
      \path<6-> (c) edge[loop below] node       {$b$} (c);
      \path<6-> (d) edge[bend right] node       {$a$} (f);
      \path<6-> (d) edge[bend left ] node[swap] {$b$} (g);
      \path<6-> (e) edge[loop below] node       {$a$} (e);
      \path<6-> (e) edge[bend right] node[swap] {$b$} (g);
      \path<6-> (f) edge[bend right] node       {$a$} (e);
      \path<6-> (f) edge             node[swap] {$b$} (h);
      \path<6-> (g) edge[bend right] node[swap] {$a$} (e);
      \path<6-> (g) edge[loop right] node       {$b$} (g);
      \path<6-> (h) edge[bend right] node       {$a$} (b);
      \path<6-> (h) edge[bend left ] node[swap] {$b$} (c);
    \end{tikzpicture}
  }
  
  \on<2->[bottom, x=.27\textwidth]{\footnotesize
    \begin{tabular}{|l|l|l|}
      \hline
      &
      \multicolumn{2}{c|}{\rule{0pt}{1em}\structure{\textbf{Entrées de $\Sigma$}}}\\
      \hline
      \rule{0pt}{1.1em}\structure{\textbf{\'Etats de $S$}} &
      \multicolumn{1}{c|}{$\structure{a}$} &
      \multicolumn{1}{c|}{$\structure{b}$} \\
      \hline
      \rule{0pt}{1em}$s_0=\{\example<2>{0},\example<2>{3},\structure<2>{4},\structure<2>{5}\}$ \hspace\fill\structure{i}  &
      \onlyb<3>{\alert{?}}\only<4-> {$s_1=\{5, 2\}$} &
      \onlyb<4> {\alert{?}}\only<5-> {$s_2=\{1,4,5\}$} \\
      \uncover<4->{$s_1=\{5, 2\}$} &
      \uncover<6->{$s_1=\{5, 2\}$} &
      \uncover<6->{$s_3=\{1, 2\}$} \\
      \uncover<5->{$s_2=\{\alert<5>{1},4,5\}$ \hspace\fill\structure{f}} &
      \uncover<6->{$s_4=\{1,2,5\}$} &
      \uncover<6->{$s_2=\{1,4,5\}$} \\
      \uncover<6->{$s_3=\{1,2\}$ \hspace\fill\structure{f}} &
      \uncover<6->{$s_5=\{1,5\}$} &
      \uncover<6->{$s_6=\{1,2,4,5\}$} \\
      \uncover<6->{$s_4=\{1,2,5\}$ \hspace\fill\structure{f}} &
      \uncover<6->{$s_4=\{1,2,5\}$} &
      \uncover<6->{$s_6=\{1,2,4,5\}$} \\
      \uncover<6->{$s_5=\{1,5\}$ \hspace\fill\structure{f}} &
      \uncover<6->{$s_4=\{1,2,5\}$} &
      \uncover<6->{$s_7=\{4,5\}$} \\
      \uncover<6->{$s_6=\{1,2,4,5\}$ \hspace\fill\structure{f}} &
      \uncover<6->{$s_4=\{1,2,5\}$} &
      \uncover<6->{$s_6=\{1,2,4,5\}$} \\
      \uncover<6->{$s_7=\{4,5\}$} &
      \uncover<6->{$s_1=\{5,2\}$} &
      \uncover<6->{$s_2=\{1,4,5\}$} \\
      \hline
    \end{tabular}
  }
 
\end{frame}

\endgroup

% 
%\subsection{Minimisation d'un automate fini déterministe}
%% SPDX-License-Identifier: CC-BY-SA-4.0
% Author: Matthieu Perrin
% Part: 
% Section: 
% Sub-section: 
% Frame: 

\begingroup

\begin{frame}{Génération d'analyseur lexical}
  \begin{block}{Problème}
    \vspace{-2mm}
    \begin{description}
    \item[Entrée :] une expression rationnelle $r$
    \item[Sortie :] un \alert{analyseur lexical} pour le langage $\mathcal{S}(r)$
      \begin{itemize}
      \item Programme qui décide si son entrée appartient à $\mathcal{S}(r)$
      \end{itemize}
    \end{description}
  \end{block}
  
  \centering
  \scalebox{.9}{\begin{tikzpicture}

      \draw[white] (-1.8,0) rectangle (9.8,6.2);
      
      \draw[rounded corners, structure,fill=structure!20] (0,5) +(-1.2,-.5) rectangle +(1.2,.5) +(0,0) node{\small \begin{tabular}{c}Expression\\ rationnelle \end{tabular}};
      \draw[rounded corners, structure,fill=structure!20] (8,5) +(-1.2,-.5) rectangle +(1.2,.5) +(0,0) node{\small \begin{tabular}{c}Analyseur\\ lexical \end{tabular}};
      \draw[rounded corners, structure,fill=structure!20] (0,3) +(-1.2,-.5) rectangle +(1.2,.5) +(0,0) node{\small \begin{tabular}{c}Automate fini\\ non-déterministe \end{tabular}};
      \draw[rounded corners, alert,fill=alert!20] (4,3) +(-1.2,-.5) rectangle +(1.2,.5) +(0,0) node{\small \begin{tabular}{c}Automate fini\\ déterministe \end{tabular}};
      \draw[rounded corners, structure,fill=structure!20] (8,3) +(-1.2,-.5) rectangle +(1.2,.5) +(0,0) node{\small \begin{tabular}{c}Automate fini\\ minimal \end{tabular}};
      
      \draw[dashed, -latex] (1.2,5) -- (6.8,5);
      \draw[-latex, structure] (0,4.5) -- (0,3.5);
      \draw[-latex,alert] (1.2,3) -- (2.8,3);
      \draw[-latex] (5.2,3) -- (6.8,3);
      \draw[-latex, structure] (8,3.5) -- (8,4.5);

      \draw[example] (0,5.5) node[above]{$a (b|c)^\star$};
      \draw[example] (8,5.5) node[above]{$abc \rightarrow \cmark$, $bac \rightarrow \xmark$ };

      \draw[structure] (0,4.1) node[right]{\tiny Algorithme de};
      \draw[structure] (0,3.9) node[right]{\tiny Thompson};

      \draw[alert] (2,3.15) node{\tiny Sous-ensembles de};
      \draw[alert] (2,2.85) node{\tiny Rabin \& Scott};

      \draw (6,3.15) node{\tiny Méthode de};
      \draw (6,2.85) node{\tiny Moore};

      \draw[structure] (8,4) node[left]{\tiny Transcription};

      \draw[example] (0,2.5) node[below]{\scalebox{.5}{\begin{tikzpicture}
            \node (a) {};
            \node (b) [above=of a] {};
            \node (c) [right=of b] {};
            \node (d) [below=of c] {};
            \node (e) [above right=of d] {};
            \node (f) [right=of e] {};
            \node (g) [below right=of d] {};
            \node (h) [right=of g] {};
            \node (i) [below right=of f] {};
            \node (j) [below=of d] {};

            \node[fill=example!10,state,initial, initial text=] (a1) at (a) {$1$};
            \node[fill=example!10,state] (b1) at (b) {$2$};
            \node[fill=example!10,state] (e1) at (e) {$3$};
            \node[fill=example!10,state] (f1) at (f) {$4$};
            \node[fill=example!10,state] (g1) at (g) {$5$};
            \node[fill=example!10,state] (h1) at (h) {$6$};
            \node[fill=example!10,state] (d1) at (d) {$7$};
            \node[fill=example!10,state] (i1) at (i) {$8$};
            \node[fill=example!10,state] (c1) at (c) {$9$};
            \node[fill=example!10,state, accepting] (j1) at (j) {$10$};

            \path[->]  (a1) edge node[left] {$a$} (b1);
            \path[->]  (e1) edge node[below] {$b$} (f1);
            \path[->]  (g1) edge node[above] {$c$} (h1);
            \path[->]  (d1) edge node[right] {$\varepsilon$} (e1);
            \path[->]  (d1) edge node[right] {$\varepsilon$} (g1);
            \path[->]  (f1) edge node[left ] {$\varepsilon$} (i1);
            \path[->]  (h1) edge node[left ] {$\varepsilon$} (i1);
            \path[->]  (i1) edge node[above] {$\varepsilon$} (d1);
            \path[->]  (c1) edge node[left] {$\varepsilon$} (d1);
            \path[->]  (d1) edge node[left ] {$\varepsilon$} (j1);
            \path[->]  (b1) edge node[above] {$\varepsilon$} (c1);
      \end{tikzpicture}}};

      \draw[example] (4,2.5) node[below]{\scalebox{.5}{\begin{tikzpicture}
            \node[fill=example!10,state,initial, initial text=] (a) {$1$};
            \node[fill=example!10,state, accepting] (b) [below=of a] {$2$};
            \node[fill=example!10,state, accepting] (c) [above right=of b] {$3$};
            \node[fill=example!10,state, accepting] (d) [below right=of b] {$4$};

            \path[->]  (a) edge node[left] {$a$} (b);
            \path[->]  (b) edge node[above] {$b$} (c);
            \path[->]  (b) edge node[below] {$c$} (d);
            \path[->]  (c) edge[bend right=5mm] node[left] {$c$} (d);
            \path[->]  (d) edge[bend right=5mm] node[right] {$b$} (c);
            \path[->]  (c) edge[loop right, looseness=5] node {$b$} (c);
            \path[->]  (d) edge[loop right, looseness=5] node {$c$} (d);
      \end{tikzpicture}}};
      \draw[example] (8,2.5) node[below]{\scalebox{.5}{\begin{tikzpicture}
            \node[fill=example!10,state,initial, initial text=] (a) {$1$};
            \node[fill=example!10,state, accepting] (b) [below=of a] {$2$};

            \path[->]  (a) edge node[left] {$a$} (b);
            \path[->]  (b) edge[loop left, looseness=5] node {$b$} (b);
            \path[->]  (b) edge[loop right, looseness=5] node {$c$} (b);
      \end{tikzpicture}}};
  \end{tikzpicture}}
\end{frame}

\endgroup

%% SPDX-License-Identifier: CC-BY-SA-4.0
% Author: Matthieu Perrin
% Part: 
% Section: 
% Sub-section: 
% Frame: 

\begingroup

\begin{frame}{Minimalité d'un automate fini déterministe}
  \begin{block}{Définition -- Dimension d'un automate}
    La \structure{dimension} d'un automate fini $A=\langle \Sigma, Q, I, F, \mu \rangle$,
    notée \alert{$|A|$} est \\ le nombre d'états de cet automate :

    $$\alert{|A| = |Q|}.$$
  \end{block}

  \begin{block}{Théorème -- Automate minimal}
    Soit $L$ un langage rationnel sur un alphabet $\Sigma$.
    Il existe \alert{un unique} automate déterministe et complet
    \alert{de dimension minimale} (à isomorphisme près) qui reconnaît $L$. 
    On l'appelle l’\structure{automate minimal} du langage.
  \end{block}
\end{frame}

\endgroup

%% SPDX-License-Identifier: CC-BY-SA-4.0
% Author: Matthieu Perrin
% Part: 
% Section: 
% Sub-section: 
% Frame: 

\begingroup

\begin{frame}{Exemple d'utilisation du lemme de pompage}
  
  \tfBlock[top=-5mm]{Montrer que $L \eqdef \{a^nb^nc^n \mid n\in \mathbb{N}\}$ n'est pas algébrique}{
    Soit $\Sigma \eqdef \{a, b, c\}$.%
    \only<2-|handout>{
      Si $L$ est algébrique, $L$ vérifie le lemme de pompage :

      \vspace{-4mm}
      $$
      \begin{array}{c}
        \structure{\exists N\in \mathbb{N}}, \alert{\forall u\in L, |u| \ge N} \Rightarrow (\structure{\exists v, w, x, y, z\in \Sigma^\star}, \\
        u = v \cdot w \cdot x\cdot y \cdot z \land w \cdot y\neq \varepsilon \land |w\cdot x\cdot y| \le N \land \alert{\forall i \in \mathbb{N}}, v\cdot w^i\cdot x\cdot y^i \cdot z \in L)
      \end{array}
      $$
      \vspace{-2mm}
      
      \structure{Soit $N$ donné par le lemme pompage}.
    }
    
    \only<3-|handout>{%
      \alert{Posons $u = a^N b^N c^N$. On a bien $u\in L$ et $|u| = 3N \ge N$}.\\
      \structure{Soit $v \cdot w \cdot x\cdot y \cdot z$ la décomposition de $u$ donnée par le lemme de pompage}. 
    }
  }
  
  \tf<4-|handout>[y=-10mm]{
    \begin{smArray}[width=4mm, height=3mm, name={$u=$}]
      \smCell[\smNone]{\alert{$a$}}      \smCoord{(a1)}
      \smCell[\smNone]{\alert{$\cdots$}}               
      \smCell[\smNone]{\alert{$a$}}      \smCoord{(an)}
      \smCell[\smNone]{\alert{$\cdot$}}               
      \smCell[\smNone]{\alert{$b$}}      \smCoord{(b1)}
      \smCell[\smNone]{\alert{$\cdots$}}               
      \smCell[\smNone]{\alert{$b$}}      \smCoord{(bn)}
      \smCell[\smNone]{\alert{$\cdot$}}               
      \smCell[\smNone]{\alert{$c$}}      \smCoord{(c1)}
      \smCell[\smNone]{\alert{$\cdots$}}               
      \smCell[\smNone]{\alert{$c$}}      \smCoord{(cn)}
      
      \draw [decorate, decoration={brace, amplitude=5pt}] ([xshift=1mm]a1.north west) -- ([xshift=-1mm]an.north east) node[midway,yshift=4mm]{$N$};
      \draw [decorate, decoration={brace, amplitude=5pt}] ([xshift=1mm]b1.north west) -- ([xshift=-1mm]bn.north east) node[midway,yshift=4mm]{$N$};
      \draw [decorate, decoration={brace, amplitude=5pt}] ([xshift=1mm]c1.north west) -- ([xshift=-1mm]cn.north east) node[midway,yshift=4mm]{$N$};

      \draw [decorate, decoration={brace, amplitude=5pt, mirror}] ([xshift=1mm]an.south west) -- ([xshift=-1mm]b1.south east) node[midway,yshift=-4mm]{$|wxy|\le N$};
      \draw [decorate, decoration={brace, amplitude=5pt, mirror}] ([xshift=1mm]bn.south west) -- ([xshift=-1mm]c1.south east) node[midway,yshift=-4mm]{$|wxy|\le N$};
    \end{smArray}       
  }

  \tf<5-|handout>[text, bottom=-1mm]{
    \begin{itemize}
    \item Comme $wy \neq \varepsilon$, $\alpha = wy[1] \in \Sigma$ est une lettre de $wy$.
    \item Comme $|wxy| \le N$, il existe $\beta \in \Sigma$ tel que $\beta$ n'est pas une lettre de $wy$. 
    \item \alert{Posons $i=2$}. $|v \cdot w^2 \cdot x\cdot  y^2\cdot   z|_\alpha > |v\cdot  w^2\cdot  x\cdot  y^2\cdot   z|_\beta$, donc $v\cdot  w^2\cdot  x\cdot  y^2\cdot   z\notin L$. 
    \end{itemize}
    Absurde ! Donc $L$ n'est pas algébrique. 
  }

  \tfExampleBlock<-4>[y=-25mm]{On sait}{}

  \tf<1>[y=-25mm, anchor=north, text]{
    \begin{itemize}
    \item $\begin{array}[t]{l}
      \alert{\forall L \in \textsc{alg}_\Sigma}, \structure{\exists N\in \mathbb{N}}, \forall u\in L, |u| \ge N \Rightarrow (\exists v, w, x, y, z\in \Sigma^\star, \\
      u = v w x y z \land w y\neq \varepsilon \land |wxy| \le N \land \forall i \in \mathbb{N}, v w^i x y^i z \in L)
    \end{array}$
    \end{itemize}
  }

  \tf<2-4>[y=-25mm, anchor=north, left=.35\textwidth]{
    \begin{itemize}
    \item $L \in \textsc{alg}_\Sigma$
    \item<3-> $v, w, x, y, z\in \Sigma^\star$
    \end{itemize}
  }
  
  \tf<2-4>[y=-25mm, anchor=north, width=.35\textwidth]{
    \begin{itemize}
    \item $N \in \mathbb{N}$
    \item<3-> $wy\neq \varepsilon$
    \end{itemize}
  }
  
  \tf<3-4>[y=-25mm, anchor=north, right=.35\textwidth]{
    \begin{itemize}
    \item $v \cdot w \cdot x \cdot y \cdot z = a^N b^N c^N$
    \item $\alert{|wxy| \le N}$
    \end{itemize}
  }
  
\end{frame}

\endgroup

%% SPDX-License-Identifier: CC-BY-SA-4.0
% Author: Matthieu Perrin
% Part: 
% Section: 
% Sub-section: 
% Frame: 

\begingroup

\begin{frame}{Idées de l'algorithme de Moore}
  \begin{tikzpicture}
    \draw (0,4) node{
      \begin{minipage}{8cm}
        \begin{block}{Un algorithme optimiste}
          \begin{itemize}
          \item  Peut-être que tous les états sont équivalents ? 
          \item<2-> Non : seuls les états finaux reconnaissent $\varepsilon$
            \begin{itemize}
            \item Séparer les états finaux et les autres
            \end{itemize}
          \item<3-> Essayer de placer les transitions
            \begin{itemize}
            \item<4-> Partitionner tant qu'on n'y arrive pas
            \end{itemize}
          \end{itemize}
        \end{block}
      \end{minipage}
    };

    \draw (5.1,3.7) node{
      \scalebox{.7}{\begin{tikzpicture}
          \draw[white] (-1,-.5) rectangle (4,4.5);
          \draw[structure] (2,4) node{$A = \{0,1,2, 3\}$};

          \draw<2->[structure] (1,2) node{$B = \{0,1,2\}$};
          \draw<2->[alert] (3,2) node{$C = \{3\}$};

          \draw<4->[structure] (0,0) node{$D = \{0\}$};
          \draw<4->[example] (2,0) node{$E = \{1,2\}$};

          \draw<2->[-latex] (1.75,3.5) -- (1.25,2.5);
          \draw<2->[-latex] (2.25,3.5) -- (2.75,2.5);
          \draw<2-> (2,3.25) node{$\varepsilon$};

          \draw<4->[-latex] (0.75,1.5) -- (0.25,0.5);
          \draw<4->[-latex] (1.25,1.5) -- (1.75,0.5);
          \draw<4-> (1,1.25) node{$b$};

          
      \end{tikzpicture}}
    };

    \draw (-1.5,0.4) node{
      \scalebox{.9}{\begin{tikzpicture}[shorten >=1pt,node distance=1.5cm,on grid,auto]
          \node[structure, fill=structure!20, state,initial, initial text=] (a) {$0$};
          \node<-3>[structure, fill=structure!20, state] (b) [above right =of a] {$1$};
          \node<-3>[structure, fill=structure!20, state] (c) [below right=of a] {$2$};
          \node<1>[structure, fill=structure!20, state, accepting] (d) [above right=of c] {$3$};

          \node<2->[alert, fill=alert!20, state, accepting] (d) [above right=of c] {$3$};

          \node<4->[example, fill=example!20, state] (b) [above right =of a] {$1$};
          \node<4->[example, fill=example!20, state] (c) [below right=of a] {$2$};
          
          \path[->]  (a) edge[bend left ] node[below right] {$a$} (b);
          \path<-2,4->[->]  (a) edge[bend right] node[above right] {$b$} (c);
          \path<-2,4->[->]  (b) edge[bend left ] node[below left ] {$b$} (d);
          \path<-2,4->[->]  (c) edge[bend right] node[above left ] {$b$} (d);
          \path<3>[->, example]  (a) edge[bend right] node[above right] {$b$} (c);
          \path<3>[->, example]  (b) edge[bend left ] node[below left ] {$b$} (d);
          \path<3>[->, example]  (c) edge[bend right] node[above left ] {$b$} (d);

          \path[->]  (b) edge[loop below, looseness=5] node[below] {$a$} (b);
          \path[->]  (c) edge[loop above, looseness=5] node[above] {$a$} (c);
          \path[->]  (d) edge[loop right, looseness=5] node[right] {$a, b$} (d);
      \end{tikzpicture}}
    };
    \draw (3.8,0.4) node{
      \scalebox{.9}{\begin{tikzpicture}[shorten >=1pt,node distance=1.5cm,on grid,auto]
          \draw[white] (-1.5,-1.5) rectangle (4.5,1.5);
          
          \node (s1) {};
          \node (s2) [right=of s1] {};
          \node (s3) [right=of s2] {};

%          \node[state,initial, initial text=] (a) {$0$};
%          \node[state, example, fill=example!20] (b) [right=of a] {$1, 2$};
%          \node[state, accepting] (d) [right=of b] {$3$};


          \node<1>[state, structure, fill=structure!20] (A) at (s1) {$A$};

          \node<2-3>[state, initial, initial text=, structure, fill=structure!20] (B) at (s1) {$B$};
          \node<2->[state, accepting, alert, fill=alert!20] (C) at (s3) {$C$};

          \node<4->[state, initial, initial text=, structure, fill=structure!20] (D) at (s1) {$D$};
          \node<4->[state, example, fill=example!20] (E) at (s2) {$E$};

          \node<3>[example] at (s2) {?};


          
          \path<4->[->]  (D) edge node[above] {$a, b$} (E);
          \path<4->[->]  (E) edge node[above] {$b$} (C);
          \path<4->[->]  (E) edge[loop below, looseness=5] node[below] {$a$} (E);
          
          \path<3>[example, ->]  (B) edge node[above] {$b$} (s2);
          \path<3>[->]  (B) edge[loop below, looseness=5] node[below] {$a$} (B);
          \path<3->[->]  (C) edge[loop right, looseness=5] node[right] {$a, b$} (C);
      \end{tikzpicture}}
    };
  \end{tikzpicture}
\end{frame}

\endgroup

%% SPDX-License-Identifier: CC-BY-SA-4.0
% Author: Matthieu Perrin
% Part: 
% Section: 
% Sub-section: 
% Frame: 

\begingroup

\begin{frame}[fragile]{Algorithme de Moore}
  \begin{block}{Relation d'équivalence}
      \vspace{-2mm}
    Soit $\langle \Sigma, Q, I, F, \mu \rangle$ un automate, $P$ une partition de $Q$, $a\in \Sigma$ et $q, q' \in Q$. 
    \begin{itemize}
    \item On dit que $q$ et $q'$ sont \structure{équivalents}, noté $\alert{q \sim_S q'}$, si toutes leurs transitions sortantes mènent dans la même partie $E$ de $P$ : 
      \vspace{-2mm}
      $$\alert{q \sim_S q' \;  \eqdef \; \forall a \in \Sigma, \exists E \in P, \mu(q, a) \in E \land \mu (q', a) \in E}$$
    \end{itemize}
  \end{block}
  
  \begingroup
  \renewcommand{\emph}[1]{\structure{#1}} 
  \scalebox{1}{\begin{algorithm}[H]
      \SetKwFunction{Moore}{moore}
      \SetKwData{Input}{motif}
      \SetKwFor{Pour}{pour}{faire}{fin pour}

      \Fn{\Moore( $A = \langle \Sigma, Q, I, F, \mu \rangle$ : automate ) : automate}{
        $P \leftarrow \alert{\{F, Q\setminus F\}}$\;
        \lTantque{$P \neq Q/_{\sim_{P}}$}{
          $P \leftarrow Q/_{\sim_{P}}$%
          \footnote[frame, 1]{
            \structure{Rappels :} $[q]_{\sim_P}$ est la classe d'équivalence de $q$, et $Q/_{\sim_P}$ l'ensemble quotient de $Q$.\\
            ~\hspace\fill
            $[q]_{\sim_P} \eqdef \{q' \in Q \,|\, q \sim_P q' \} \hspace\fill Q/_{\sim_P} \eqdef \{[q]_{\sim_P} | q\in Q\}$
            \hspace\fill~
          }%
        }
        $I'\hspace{1.2mm} \leftarrow \{ E \in P | I \subseteq E\}$\;
        $F' \leftarrow \{ E \in P | E \cap F \neq \emptyset \}$\;
        $\mu' \leftarrow \{ \langle E , a, E' \rangle \in P \times \Sigma \times P | \exists q\in E, \exists q'\in E', \langle q, a, q' \rangle \in \mu\}$\;
        \Retourner $\langle \Sigma, P, I', F', \mu' \rangle$\;
      }
    \end{algorithm}}
    \endgroup
\end{frame}
\endgroup

%% SPDX-License-Identifier: CC-BY-SA-4.0
% Author: Matthieu Perrin
% Part: 
% Section: 
% Sub-section: 
% Frame: 

\begingroup

\begin{frame}{Autre exemple}

  Soit $\Sigma$ un alphabet.
  \begin{block}{Montrer que $\textsc{regex}_\Sigma \notin \textsc{rat}_\Sigma$}

    Supposons (par l'absurde) que $\textsc{regex}_\Sigma$ est rationnel. Par le lemme de l'étoile :

    \vspace{-2mm}
    $$
    \structure{
      \begin{array}{c}
        \exists N\in \mathbb{N}, \forall u\in \textsc{regex}_\Sigma, |u| \ge N \Rightarrow (\exists x, y, z\in \Sigma^\star, \\
        u = x\cdot y \cdot z \land y\neq \varepsilon \land |xy| \le N \land \forall i \in \mathbb{N}, x\cdot y^i \cdot z \in \textsc{regex}_\Sigma)
    \end{array}}
    $$

    Soit $N$ donné par le lemme de l'étoile. \\
    \alert{Posons $u = \text{``$($''}^{N} \cdot \text{``$a$''} \cdot \text{``$|b)$''}^N  = ((...((a | b) |b) ... |b) |b) $}.\\
    On a bien $u\in L$ et $|u| = 4N+1 \ge N$. \\
    Soit $x\cdot y \cdot z$ la décomposition de $u$ donnée par le lemme de l'étoile. 

    On a: \hspace\fill $\structure{x = \text{``$($''}^{|x|}}$, \hspace\fill $\structure{y = \text{``$($''}^{|y|}}$ avec $|y| \neq 0$, \hspace\fill et $\structure{z = \text{``$($''}^{N - |x| - |y|} \cdot \text{``$a$''} \cdot \text{``$|b)$''}^N}$.\hspace\fill~ 

    \alert{Pour $i = 2$}, $x y^2 z  =  \text{``$($''}^{N + |y|} \cdot \text{``$a$''} \cdot \text{``$|b)$''}^N  \notin \textsc{regex}_\Sigma$. Absurde ! 
  \end{block}

  \begin{alertblock}{Question}
    \begin{itemize}
    \item Le langage $\textsc{regex}_\Sigma$ n'est pas rationnel. 
    \item Quel formalisme peut-on utiliser pour décrire $\textsc{regex}_\Sigma$ ? 
    \end{itemize}
  \end{alertblock}
\end{frame}

\endgroup

%% SPDX-License-Identifier: CC-BY-SA-4.0
% Author: Matthieu Perrin
% Part: 
% Section: 
% Sub-section: 
% Frame: 

\begingroup

\begin{frame}{Équivalence et isomorphisme entre automates}

  \tf[text, top]{
    Soient $A = \langle \Sigma, Q_A, I_A, F_A, \rightarrow_A \rangle$ et $B = \langle \Sigma, Q_B, I_B, F_B, \rightarrow_B \rangle$ deux automates. 
  }

  \tfBlock[y=17mm]{Définition -- Automates équivalents}{
    $A$ et $B$ sont \structure{équivalents}, noté $\alert{A \equiv B}$, s'ils acceptent le même langage :
    $$\alert{A_1 \equiv A_2  \eqdef \mathcal{L}(A_1) = \mathcal{L}(A_2)}.$$
  }

  \tfBlock[bottom=2mm]{Définition -- Automates isomorphes}{
    $A$ et $B$ sont \structure{isomorphes}, noté $\alert{A \simeq B}$, s'ils ne diffèrent que par le nom de leurs états, \textit{c.-à-d.}
    s'il existe une fonction bijective $f : Q_A \rightarrow Q_B$ telle que :

    \vspace{2mm}
    $\structure{\forall a\in \Sigma, \forall q, q',}$\\
    $\structure{\begin{array}[t]{@{\quad\quad}lll}
        q\in Q_A &\Leftrightarrow& f(q) \in Q_B \\
        q\in I_A &\Leftrightarrow& f(q) \in I_B \\
        q\in F_A &\Leftrightarrow& f(q) \in F_B \\
        q \xrightarrow{a}_A q' &\Leftrightarrow& f(q) \xrightarrow{a}_B f(q')
      \end{array}
    }$
    \vspace{2mm}

    Deux automates isomorphes sont équivalents. 
  }

  \tf[bottom=2mm, x=3cm]{\small
    \begin{tikzpicture}[smAutomaton]
      \draw (1,0.75) node{\normalsize $\simeq$};

      \smState[\smStructure \smInitial]   (a0) at (0.0, 1.5) {$A$};
      \smState[\smStructure \smAccepting] (a1) at (0.0, 0.0) {$B$};
      \smState[\smExample \smInitial]     (b0) at (2.0, 1.5) {$0$};
      \smState[\smExample \smAccepting]   (b1) at (2.0, 0.0) {$1$};

      \smPath[\smStructure] (a0) edge[bend left]  node {$a$} (a1);
      \smPath[\smStructure] (a1) edge[bend left]  node {$a$} (a0);
      \smPath[\smStructure] (a1) edge[loop right] node {$b$} (a1);
      \smPath[\smStructure] (a0) edge[loop right] node {$b$} (a0);
      
      \smPath[\smExample]   (b0) edge[bend left]  node {$a$} (b1);
      \smPath[\smExample]   (b1) edge[bend left]  node {$a$} (b0);
      \smPath[\smExample]   (b1) edge[loop right] node {$b$} (b1);
      \smPath[\smExample]   (b0) edge[loop right] node {$b$} (b0);
    \end{tikzpicture}
  }
\end{frame}

\endgroup

%% SPDX-License-Identifier: CC-BY-SA-4.0
% Author: Matthieu Perrin
% Part: 
% Section: 
% Sub-section: 
% Frame: 

\begingroup

\begin{frame}{Langages rationnels et langages reconnaissables}

  \onBlock[top, left=.7\textwidth]{Théorème de Kleene}{
    Les langages  reconnaissables \\
    par un automate fini, déterministe ou non,  \\
    sont exactement les langages rationnels
  }

  \onImage[top, x=.35\textwidth]{%
    height=30mm,
    title={Stephen C. Kleene},
    licenselogo={\ccPublicDomain},
    license={Domaine public (Harold N. Hone, 1963 -- \href{https://commons.wikimedia.org/wiki/File:Stephen_Cole_Kleene_circa_1963.tif}{Wikimedia})},
    img={Kleene.jpg}
  }

  
  \on[y=-20mm]{\footnotesize
    \begin{tikzpicture}[y=17mm, x=40mm]
      \small
      \node[smBox, minimum width=2.4cm, minimum height=1cm] (exp) at (0,1) {Expression\\rationnelle};
      \node[smBox, minimum width=2.4cm, minimum height=1cm] (lex) at (2,1) {Analyseur\\lexical};
      \node[smBox, minimum width=2.4cm, minimum height=1cm] (nfa) at (0,0) {Automate fini\\non-déterministe};
      \node[smBox, minimum width=2.4cm, minimum height=1cm] (dfa) at (1,0) {Automate fini\\déterministe};
      \node[smBox, minimum width=2.4cm, minimum height=1cm] (min) at (2,0) {Automate fini\\minimal};

      \node[example, above] at (exp.north) {$a (b|c)^\star$};
      \node[example, above] at (lex.north) {$abc \rightarrow \cmark$, $bac \rightarrow \xmark$ };

      \node[alert] at (1,1) {\large $\textsc{rat}_\Sigma =  \textsc{rec}_\Sigma$};

      
      \tiny
      \path[-latex, structure]        (lex) edge[bend right=5mm]  node[above]             {Lemme d'Arden}                     (exp);
      \path[-latex, structure,dashed] (exp) edge[bend right=5mm]                                                             (lex);
      \path[-latex, structure]        (exp) edge[bend left]  node[right, align=left] {Algorithme de\\Thompson}           (nfa);
      \path[-latex, structure]        (nfa) edge[bend left]  node[left, align=right] {Lemme\\d'Arden}                    (exp);
      \path[-latex, structure]        (nfa) edge             node[align=center]      {Sous-ensembles de\\Rabin \& Scott} (dfa);
      \path[-latex, structure]        (dfa) edge             node[align=center]      {Méthode de\\Moore}                 (min);
      \path[-latex, structure]        (min) edge             node[left, align=right] {Transcription}                     (lex);
    \end{tikzpicture}
  }

\end{frame}

\endgroup

% 
%\section{Expressivité des langages rationnels}
% 
%\subsection{Stabilité du formalisme}
%% SPDX-License-Identifier: CC-BY-SA-4.0
% Author: Matthieu Perrin
% Part: 
% Section: 
% Sub-section: 
% Frame: 

\begingroup

\begin{frame}{Stabilité par complémentaire}
  Soit $\Sigma$ un alphabet.

  \begin{block}{Théorème -- Stabilité par complémentaire}
    Soit $L \in \textsc{rat}_\Sigma$. Alors $\overline{L} = \Sigma^\star \setminus L \in \textsc{rat}_\Sigma$. 
  \end{block}

  \pause
  \begin{block}{Preuve}
    \begin{itemize}
    \item Il existe $A = \langle \Sigma, Q, I, F, \mu \rangle$ déterministe complet qui reconnaît $L$.
    \item<3-> $\overline{A} = \langle \Sigma, Q, I, Q \setminus F, \mu \rangle$ reconnaît $\overline{L}$.
    \end{itemize}
  \end{block}
  
  \begin{exampleblock}{Exemple}

    \centering
    \begin{tabular}[t]{ccc}
      $\{ u\in \{a, b\}^\star \;|\; |u|_a \equiv 1 [2] \}$ &&
      \uncover<3->{$\{ u\in \{a, b\}^\star \;|\; |u|_a \equiv 0 [2] \}$}\\
      \scalebox{.75}{\begin{tikzpicture}
          \node[state,initial, initial text=] (a) {$0$};
          \node[state,accepting] (b) [right=of a] {$1$};
          
          \path[->]  (a) edge[bend right] node[below] {$a$} (b);
          \path[->]  (b) edge[bend right] node[above] {$a$} (a);
          \path[->]  (a) edge[loop above, looseness=5] node {$b$} (a);
          \path[->]  (b) edge[loop above, looseness=5] node {$b$} (b);
      \end{tikzpicture}} &&
      \uncover<3->{\scalebox{.75}{\begin{tikzpicture}
            \node[state,initial, initial text=,accepting] (a) {$0$};
            \node[state] (b) [right=of a] {$1$};
            
            \path[->]  (a) edge[bend right] node[below] {$a$} (b);
            \path[->]  (b) edge[bend right] node[above] {$a$} (a);
            \path[->]  (a) edge[loop above, looseness=5] node {$b$} (a);
            \path[->]  (b) edge[loop above, looseness=5] node {$b$} (b);
      \end{tikzpicture}}}
    \end{tabular}
    
  \end{exampleblock}
\end{frame}
\endgroup

%% SPDX-License-Identifier: CC-BY-SA-4.0
% Author: Matthieu Perrin
% Part: 
% Section: 
% Sub-section: 
% Frame: 

\begingroup

\begin{frame}{Stabilité par intersection}

  \tfBlock[top=-3mm]{Théorème -- Stabilité par intersection}{
    Soient $\Sigma$ un alphabet et $L_1, L_2 \in \textsc{rat}_\Sigma$. Alors $L_1 \cap L_2 \in \textsc{rat}_\Sigma$. 
  }

  \tfBlock<1>[y=23mm,anchor=north]{Preuve directe}{
    \begin{itemize}
    \item $L_1 \cap L_2 = \overline{\overline{L_1} \cup \overline{L_2}}$
    \item Remarque : stabilité par différence car $L_1 \setminus L_2 = L_1 \cap \overline{L_2}$
    \end{itemize}
  }

  \tfBlock<2-|handout>[y=23mm,anchor=north]{Preuve par produit d'automates}{
    \begin{itemize}
    \item Il existe $\structure{A_1 = \langle \Sigma, Q_1, I_1, F_1, \rightarrow_1 \rangle}$ $\varepsilon$-libre qui reconnaît $L_1$.
    \item Il existe $\example{A_2 = \langle \Sigma, Q_2, I_2, F_2, \rightarrow_2 \rangle}$ $\varepsilon$-libre qui reconnaît $L_2$.
    \item<3-|handout>
      $L_1 \cap L_2$ est reconnu par \alert{$A_1 \times A_2 = \langle \Sigma, Q_1 \times Q_2, I_1 \times I_2, F_1 \times F_2, \rightarrow_{\times} \rangle$}, avec :\\
      \alert{$\langle q_1, q_2 \rangle \xrightarrow{a}_{\times} \langle q'_1, q'_2\rangle \quad\Leftrightarrow\quad q_1 \xrightarrow{a}_1 q'_1 \land q_2 \xrightarrow{a}_2 q'_2$}
    \end{itemize}
  }

  \tfExampleBlock<2-|handout>[bottom]{Exemple}{
    \begin{itemize}
    \item $\structure{L_1 = \{u \in (a|b)^\star \;|\; |u|_a \equiv 0 \pmod{2}\}}$
    \item $\example{L_2 = \{u \in (a|b)^\star \;|\; |u|_b \equiv 1 \pmod{2}\}}$
    \item<3-|handout> $\alert{L_1 \cap L_2 = \left\{u \in (a|b)^\star \,\middle\mid\,
      \begin{array}{@{}r@{}}
        |u|_a \equiv 0 \pmod{2}\\
        \land |u|_b \equiv 1 \pmod{2}
      \end{array}\right\}}$
    \end{itemize}
  }

  \tf<2-|handout>[bottom, x=35mm]{\scriptsize
    \begin{tikzpicture}[smAutomaton, node distance=1.2cm]
      \smState[\smStructure<-3,5-|handout> \smAccepting]             (a1) at (1.2,2.4) {$a_1$};
      \smState[\smStructure<-3,4|handout> \smInitialRight]          (a0) at (2.4,2.4) {$a_0$};
      \smState[\smExample<-3,4,5,7|handout>\smAccepting\smInitialAbove] (b0) at (0.0,1.2) {$b_0$};
      \smState[\smExample<-3,6|handout>]                            (b1) at (0.0,0.0) {$b_1$};

      \smPath[\smStructure] (a0) edge[bend right] node[above] {$a$} (a1);
      \smPath[\smStructure] (a1) edge[bend right] node[below] {$a$} (a0);
      \smPath[\smStructure] (a1) edge[loop above] node        {$b$} (a1);
      \smPath[\smStructure] (a0) edge[loop above] node        {$b$} (a0);
      \smPath[\smExample]   (b0) edge[bend right] node[left]  {$b$} (b1);
      \smPath[\smExample]   (b1) edge[bend right] node[right] {$b$} (b0);
      \smPath[\smExample]   (b1) edge[loop left]  node        {$a$} (b1);
      \smPath[\smExample]   (b0) edge[loop left]  node        {$a$} (b0);

      \uncover<3-|handout>{
        \smState[\smAlert<3,5,7|handout>\smAccepting]    (a1b0) at (1.2,1.2) {$a_1, b_0$};
        \smState[\smAlert<3,4|handout>\smInitialRight] (a0b0) at (2.4,1.2) {$a_0, b_0$};
        \smState[\smAlert<3,6|handout>]                (a1b1) at (1.2,0.0) {$a_1, b_1$};
        \smState[\smAlert<3|handout>]                (a0b1) at (2.4,0.0) {$a_0, b_1$};
        
        \smPath[\smAlert]  (a0b0) edge[bend right] node[above] {$a$} (a1b0);
        \smPath[\smAlert]  (a1b0) edge[bend right] node[below] {$a$} (a0b0);
        \smPath[\smAlert]  (a0b1) edge[bend right] node[above] {$a$} (a1b1);
        \smPath[\smAlert]  (a1b1) edge[bend right] node[below] {$a$} (a0b1);
        \smPath[\smAlert]  (a0b0) edge[bend right] node[left]  {$b$} (a0b1);
        \smPath[\smAlert]  (a0b1) edge[bend right] node[right] {$b$} (a0b0);
        \smPath[\smAlert]  (a1b0) edge[bend right] node[left]  {$b$} (a1b1);
        \smPath[\smAlert]  (a1b1) edge[bend right] node[right] {$b$} (a1b0);
      }
    \end{tikzpicture}
  }

  \tf<4->[y=-15mm, x=10mm]{
    \begin{smArray}[size=5mm]
      \smCell{a} \smHead<4>
      \smCell{b} \smHead<5>
      \smCell{a} \smHead<6>
    \end{smArray}
  }
  
\end{frame}

\endgroup

%% SPDX-License-Identifier: CC-BY-SA-4.0
% Author: Matthieu Perrin
% Part: 
% Section: 
% Sub-section: 
% Frame: 

\begingroup

\begin{frame}{Stabilité par miroir}

  \vspace{-1mm}
  Soient $\Sigma$ un alphabet, $u \in \Sigma^\star$ un mot, et $L \subseteq \Sigma^\star$ un langage.

  \begin{block}{Définition -- Miroir}
    \begin{itemize}
    \item \vspace{-1mm}Le \structure{miroir d'un mot $u$} est \alert{$(u_1\cdots u_n)^{\textsc{r}} = u_n \cdots u_1$}
    \item \vspace{-1mm}Le \structure{miroir d'un langage $L$} est \alert{$L^{\textsc{r}} = \left\{u^{\textsc{r}} \,\middle\mid\, u \in L\right\}$}
    \end{itemize}
  \end{block}
  
  \vspace{-2mm}
  \begin{alertblock}{Théorème -- Stabilité par miroir}
    Les langages rationnels sont stables par miroir : $\alert{\forall L \in \textsc{rat}_\Sigma, L^{\textsc{r}} \in \textsc{rat}_\Sigma}.$
  \end{alertblock}

  \begin{minipage}[t]{.5\textwidth}
    \structure{Preuve : Expressions rationnelles}

    \vspace{3mm}
    $m \eqdef \left\{\begin{array}{rcl}
    \textsc{reg}_\Sigma & \rightarrow & \textsc{reg}_\Sigma\\
    a & \mapsto & a\\
    u \mid v & \mapsto & m(u) \mid m(v)\\
    u \cdot v & \mapsto & m(v) \cdot m(u)\\
    u^\star & \mapsto & m(u)^\star\\
    \end{array}\right.$

    $\forall u \in \textsc{reg}_\Sigma, \mathcal{L}(m(u)) = (\mathcal{L}(u))^{\textsc{r}}$
    
    \vspace{3mm}\example{Exemple :}\\ $(ab^\star \mid ba^\star)^{\textsc{r}} = b^\star a \mid a^\star b$
    
  \end{minipage}%
  \begin{minipage}[t]{.5\textwidth}
    \structure{Preuve : Automates}

    \vspace{3mm}
    Soit $\example{A = \langle \Sigma, Q, I, F, \rightarrow \rangle}$ un AFN.\\
    On a $(\mathcal{L}(A))^{\textsc{r}} = \mathcal{L}(\alert{\langle \Sigma, Q, F, I, \rightarrow_1 \rangle})$, où

    \vspace{-2mm}
    $$\alert{q' \xrightarrow{a}_1 q \Leftrightarrow q \xrightarrow{a} q'}$$

    \example{Exemple :}
    
    \begin{tikzpicture}[smAutomaton]\scriptsize
      \example{
        \smState[\smAccepting]    (A) at (0, 0) {$A$}; 
        \smState[\smInitialAbove] (S) at (1, 0) {$S$}; 
        \smState[\smAccepting]    (B) at (2, 0) {$B$}; 
        \smPath (S) edge             node[swap] {$a$} (A);
        \smPath (S) edge             node {$b$} (B);
        \smPath (A) edge[loop below] node {$a$} (A);
        \smPath (B) edge[loop below] node {$b$} (B);
      }
      \alert{
        \smState[\smInitialAbove] (A2) at (3, 0) {$A$}; 
        \smState[\smAccepting]    (S2) at (4, 0) {$S$}; 
        \smState[\smInitialAbove] (B2) at (5, 0) {$B$}; 
        \smPath (A2) edge             node {$a$} (S2);
        \smPath (B2) edge             node[swap] {$b$} (S2);
        \smPath (A2) edge[loop below] node {$a$} (A2);
        \smPath (B2) edge[loop below] node {$b$} (B2);
      }
    \end{tikzpicture}
  \end{minipage}
\end{frame}

\endgroup

% 
%\subsection{Lemme de l'étoile}
%% SPDX-License-Identifier: CC-BY-SA-4.0
% Author: Matthieu Perrin
% Part: 
% Section: 
% Sub-section: 
% Frame: 

\begingroup

\begin{frame}{Lemme de pompage}

  \vspace{-5mm}
  \begin{block}{Question -- Expressivité des grammaires algébriques}
    \vspace{-2mm}
    \begin{itemize}
    \item Comment montrer qu'un langage $L$ n'est pas algébrique ? 
      \begin{enumerate}
      \item Si $L$ était algébrique, il vérifierait le \structure{lemme de pompage} 
      \item Or, $L$ ne vérifie pas le \structure{lemme de pompage} 
      \item Donc, par l'absurde, $L$ ne peut pas être algébrique
      \end{enumerate}
    \end{itemize}
  \end{block}
  
  \vspace{-1mm}
  \begin{block}{Rappel -- Lemme de l'étoile}
    \vspace{-1mm}
    $\forall\Sigma, \forall L\in \textsc{rat}_\Sigma, \exists N\in \mathbb{N}, \forall u\in L, |u| \ge N \Rightarrow \exists \structure{x, y, z}\in \Sigma^\star,$

    \vspace{1mm}
    \begin{minipage}{.4\textwidth}
      \begin{itemize}
      \item $u = \structure{x\cdot y \cdot z}$
      \item $\structure{y}\neq \varepsilon$
      \end{itemize}
    \end{minipage}%
    \begin{minipage}{.5\textwidth}
      \begin{itemize}
      \item $|\structure{x\cdot y}| \le N$
      \item $\forall i \in \mathbb{N}, x\cdot \structure{y^i} \cdot z \in L$
      \end{itemize}
    \end{minipage}
  \end{block}

  \vspace{-1mm}
  \begin{alertblock}{Lemme -- Lemme de pompage (ou d'itération)\footnote[frame, 1]{\scriptsize Y. Bar-Hillel, M. Perles, E. Shamir. \textit{On formal properties of simple phrase structure grammars.} 1961}}
    \vspace{-1mm}
    $\forall\Sigma, \forall L\in \textsc{alg}_\Sigma, \exists N\in \mathbb{N}, \forall u\in L, |u| \ge N \Rightarrow \exists \alert{v, w, x, y, z}\in \Sigma^\star,$
    
    \vspace{1mm}
    \begin{minipage}{.4\textwidth}
      \begin{itemize}
      \item $\alert{u = v\cdot w\cdot x \cdot y \cdot z}$
      \item $\alert{w\cdot y} \neq \varepsilon$
      \end{itemize}
    \end{minipage}%
    \begin{minipage}{.5\textwidth}
      \begin{itemize}
      \item $|\alert{w\cdot x \cdot y}| \le N$
      \item $\forall i \in \mathbb{N}, v\cdot \alert{w^i} \cdot x \cdot \alert{y^i} \cdot z \in L$
      \end{itemize}
    \end{minipage}
  \end{alertblock}

  \begin{description}
  \item[Remarque :] Si $L$ est rationnel, on peut avoir $v = w = \varepsilon$.
  \end{description}

\end{frame}

\endgroup

%% SPDX-License-Identifier: CC-BY-SA-4.0
% Author: Matthieu Perrin
% Part: 
% Section: 
% Sub-section: 
% Frame: 

\begingroup

\begin{frame}{Démonstration du lemme de l'étoile}

  \vspace{-2mm}  
  \onlyb<-3>{\hspace\fill\example{\textbf{Exemple}}}%
  \only<4->{\hspace\fill\alert{\textbf{Lemme de l'étoile}}}

  Soient $\Sigma$ un alphabet, et $L$ un langage rationnel sur $\Sigma$.
  \onlyb<-3>{\hspace\fill\example{$L = a(bc)^\star a$}}
  \only<4->{\hspace\fill\alert{$\forall \Sigma, \forall L\in \textsc{rat}_\Sigma,$}}

  Soit $A$ son automate minimal. Posons $N = |A|$.  
  \onlyb<-3>{\hspace\fill\example{$N=4$}}
  \only<4->{\hspace\fill\alert{$\exists N\in \mathbb{N},$}}
  
  Soit $u \in L$ tel que $|u| \ge N$. 
  \onlyb<-3>{\hspace\fill\example{$u = abca$}}
  \only<4->{\hspace\fill\alert{$\forall u\in L, |u| \ge N \Rightarrow ($}}

  \begin{center}
    \begin{tikzpicture}[automaton, y=5mm]
      \state[initial]   (0) at (0, 2) {$0$};
      \state[alert]     (1) at (1, 1) {$1$};
      \state[example]   (2) at (2, 1) {$2$};
      \state[accepting] (3) at (0, 0) {$3$};

      \path[structure]  (0) edge node {$a$} (1);
      \path[example]    (1) edge[bend left] node {$b$} (2);
      \path[example]    (2) edge[bend left] node {$c$} (1);
      \path[structure]  (1) edge node {$a$} (3);
    \end{tikzpicture}

    \pause
    \structure{Pour reconnaître un mot de 4 lettres ou plus, il faut suivre une boucle}
  \end{center}

  Il existe un état $q$ que l'on visite deux fois. Posons :
  \onlyb<-3>{\hspace\fill\example{$q = 1$}}%
  
  \begin{itemize}
  \item $x$ le préfixe de $u$ avant la première visite de $q$
    \onlyb<-3>{\hspace\fill\example{$x = a$}}
    \only<4-|handout>{\hspace\fill\alert{$\exists x, y, z \in \Sigma^\star,$}}
  \item $y$ le facteur de $u$ entre les deux visites de $q$
    \onlyb<-3>{\hspace\fill\example{$y = bc$}}
    \only<4-|handout>{\hspace\fill\alert{$y \neq \varepsilon$}}
  \item $z$ le suffixe de $u$ après la deuxième visite de $q$
    \onlyb<-3>{\hspace\fill\example{$z = a$}}
    \only<4-|handout>{\hspace\fill\alert{$\land\; u = x\cdot y \cdot z$}}
  \end{itemize}

  \pause

  \begin{center}\structure{Si on suit une boucle une fois, on peut la suivre plusieurs fois}\end{center}

  On a :
  \begin{itemize}
  \item On peut prendre la boucle $0$, $2$, $3$, ... fois
    \onlyb<-3>{\hspace\fill\example{$aa \in L, abcbca\in L$}}
    \only<4-|handout>{\hspace\fill\alert{$\land\; \forall i\in \mathbb{N}, x \cdot y^i \cdot z \in L$}}
  \item $N$ lettres suffisent pour repasser dans un état
    \onlyb<-3>{\hspace\fill\example{$|abc| \le 4$}}
    \only<4-|handout>{\hspace\fill\alert{$\land\; |x\cdot y| \le N)$}}
  \end{itemize}

  \phantom{.}
\end{frame}


\endgroup

%% SPDX-License-Identifier: CC-BY-SA-4.0
% Author: Matthieu Perrin
% Part: 
% Section: 
% Sub-section: 
% Frame: 

\begingroup

\begin{frame}{Exemple d'utilisation du lemme de pompage}
  
  \tfBlock[top=-5mm]{Montrer que $L \eqdef \{a^nb^nc^n \mid n\in \mathbb{N}\}$ n'est pas algébrique}{
    Soit $\Sigma \eqdef \{a, b, c\}$.%
    \only<2-|handout>{
      Si $L$ est algébrique, $L$ vérifie le lemme de pompage :

      \vspace{-4mm}
      $$
      \begin{array}{c}
        \structure{\exists N\in \mathbb{N}}, \alert{\forall u\in L, |u| \ge N} \Rightarrow (\structure{\exists v, w, x, y, z\in \Sigma^\star}, \\
        u = v \cdot w \cdot x\cdot y \cdot z \land w \cdot y\neq \varepsilon \land |w\cdot x\cdot y| \le N \land \alert{\forall i \in \mathbb{N}}, v\cdot w^i\cdot x\cdot y^i \cdot z \in L)
      \end{array}
      $$
      \vspace{-2mm}
      
      \structure{Soit $N$ donné par le lemme pompage}.
    }
    
    \only<3-|handout>{%
      \alert{Posons $u = a^N b^N c^N$. On a bien $u\in L$ et $|u| = 3N \ge N$}.\\
      \structure{Soit $v \cdot w \cdot x\cdot y \cdot z$ la décomposition de $u$ donnée par le lemme de pompage}. 
    }
  }
  
  \tf<4-|handout>[y=-10mm]{
    \begin{smArray}[width=4mm, height=3mm, name={$u=$}]
      \smCell[\smNone]{\alert{$a$}}      \smCoord{(a1)}
      \smCell[\smNone]{\alert{$\cdots$}}               
      \smCell[\smNone]{\alert{$a$}}      \smCoord{(an)}
      \smCell[\smNone]{\alert{$\cdot$}}               
      \smCell[\smNone]{\alert{$b$}}      \smCoord{(b1)}
      \smCell[\smNone]{\alert{$\cdots$}}               
      \smCell[\smNone]{\alert{$b$}}      \smCoord{(bn)}
      \smCell[\smNone]{\alert{$\cdot$}}               
      \smCell[\smNone]{\alert{$c$}}      \smCoord{(c1)}
      \smCell[\smNone]{\alert{$\cdots$}}               
      \smCell[\smNone]{\alert{$c$}}      \smCoord{(cn)}
      
      \draw [decorate, decoration={brace, amplitude=5pt}] ([xshift=1mm]a1.north west) -- ([xshift=-1mm]an.north east) node[midway,yshift=4mm]{$N$};
      \draw [decorate, decoration={brace, amplitude=5pt}] ([xshift=1mm]b1.north west) -- ([xshift=-1mm]bn.north east) node[midway,yshift=4mm]{$N$};
      \draw [decorate, decoration={brace, amplitude=5pt}] ([xshift=1mm]c1.north west) -- ([xshift=-1mm]cn.north east) node[midway,yshift=4mm]{$N$};

      \draw [decorate, decoration={brace, amplitude=5pt, mirror}] ([xshift=1mm]an.south west) -- ([xshift=-1mm]b1.south east) node[midway,yshift=-4mm]{$|wxy|\le N$};
      \draw [decorate, decoration={brace, amplitude=5pt, mirror}] ([xshift=1mm]bn.south west) -- ([xshift=-1mm]c1.south east) node[midway,yshift=-4mm]{$|wxy|\le N$};
    \end{smArray}       
  }

  \tf<5-|handout>[text, bottom=-1mm]{
    \begin{itemize}
    \item Comme $wy \neq \varepsilon$, $\alpha = wy[1] \in \Sigma$ est une lettre de $wy$.
    \item Comme $|wxy| \le N$, il existe $\beta \in \Sigma$ tel que $\beta$ n'est pas une lettre de $wy$. 
    \item \alert{Posons $i=2$}. $|v \cdot w^2 \cdot x\cdot  y^2\cdot   z|_\alpha > |v\cdot  w^2\cdot  x\cdot  y^2\cdot   z|_\beta$, donc $v\cdot  w^2\cdot  x\cdot  y^2\cdot   z\notin L$. 
    \end{itemize}
    Absurde ! Donc $L$ n'est pas algébrique. 
  }

  \tfExampleBlock<-4>[y=-25mm]{On sait}{}

  \tf<1>[y=-25mm, anchor=north, text]{
    \begin{itemize}
    \item $\begin{array}[t]{l}
      \alert{\forall L \in \textsc{alg}_\Sigma}, \structure{\exists N\in \mathbb{N}}, \forall u\in L, |u| \ge N \Rightarrow (\exists v, w, x, y, z\in \Sigma^\star, \\
      u = v w x y z \land w y\neq \varepsilon \land |wxy| \le N \land \forall i \in \mathbb{N}, v w^i x y^i z \in L)
    \end{array}$
    \end{itemize}
  }

  \tf<2-4>[y=-25mm, anchor=north, left=.35\textwidth]{
    \begin{itemize}
    \item $L \in \textsc{alg}_\Sigma$
    \item<3-> $v, w, x, y, z\in \Sigma^\star$
    \end{itemize}
  }
  
  \tf<2-4>[y=-25mm, anchor=north, width=.35\textwidth]{
    \begin{itemize}
    \item $N \in \mathbb{N}$
    \item<3-> $wy\neq \varepsilon$
    \end{itemize}
  }
  
  \tf<3-4>[y=-25mm, anchor=north, right=.35\textwidth]{
    \begin{itemize}
    \item $v \cdot w \cdot x \cdot y \cdot z = a^N b^N c^N$
    \item $\alert{|wxy| \le N}$
    \end{itemize}
  }
  
\end{frame}

\endgroup

%% SPDX-License-Identifier: CC-BY-SA-4.0
% Author: Matthieu Perrin
% Part: 
% Section: 
% Sub-section: 
% Frame: 

\begingroup

\begin{frame}{Autre exemple}

  Soit $\Sigma$ un alphabet.
  \begin{block}{Montrer que $\textsc{regex}_\Sigma \notin \textsc{rat}_\Sigma$}

    Supposons (par l'absurde) que $\textsc{regex}_\Sigma$ est rationnel. Par le lemme de l'étoile :

    \vspace{-2mm}
    $$
    \structure{
      \begin{array}{c}
        \exists N\in \mathbb{N}, \forall u\in \textsc{regex}_\Sigma, |u| \ge N \Rightarrow (\exists x, y, z\in \Sigma^\star, \\
        u = x\cdot y \cdot z \land y\neq \varepsilon \land |xy| \le N \land \forall i \in \mathbb{N}, x\cdot y^i \cdot z \in \textsc{regex}_\Sigma)
    \end{array}}
    $$

    Soit $N$ donné par le lemme de l'étoile. \\
    \alert{Posons $u = \text{``$($''}^{N} \cdot \text{``$a$''} \cdot \text{``$|b)$''}^N  = ((...((a | b) |b) ... |b) |b) $}.\\
    On a bien $u\in L$ et $|u| = 4N+1 \ge N$. \\
    Soit $x\cdot y \cdot z$ la décomposition de $u$ donnée par le lemme de l'étoile. 

    On a: \hspace\fill $\structure{x = \text{``$($''}^{|x|}}$, \hspace\fill $\structure{y = \text{``$($''}^{|y|}}$ avec $|y| \neq 0$, \hspace\fill et $\structure{z = \text{``$($''}^{N - |x| - |y|} \cdot \text{``$a$''} \cdot \text{``$|b)$''}^N}$.\hspace\fill~ 

    \alert{Pour $i = 2$}, $x y^2 z  =  \text{``$($''}^{N + |y|} \cdot \text{``$a$''} \cdot \text{``$|b)$''}^N  \notin \textsc{regex}_\Sigma$. Absurde ! 
  \end{block}

  \begin{alertblock}{Question}
    \begin{itemize}
    \item Le langage $\textsc{regex}_\Sigma$ n'est pas rationnel. 
    \item Quel formalisme peut-on utiliser pour décrire $\textsc{regex}_\Sigma$ ? 
    \end{itemize}
  \end{alertblock}
\end{frame}

\endgroup

% 
% 
%\part{Analyse Syntaxique}
% 
% 
%\section{Langages algébriques}
% 
%\subsection{Introduction}
%% SPDX-License-Identifier: CC-BY-SA-4.0
% Author: Matthieu Perrin
% Part: 
% Section: 
% Sub-section: 
% Frame: 

\begingroup

\begin{frame}{Grammaire française}

  Une grammaire est une collection de règles grammaticales du type : 
  \begin{itemize}
  \item \og Une phrase est formée d'un sujet, et d'un groupe verbal \fg
    $$ \mathit{Phrase} \rightarrow \mathit{Sujet}~\mathit{GV}$$
  \end{itemize}

  \begin{center}
    \begin{tikzpicture}
      \draw (5.0,10) node {$\mathit{Phrase}$};
      \draw (4.0,9 ) node {$\mathit{Sujet}$};
      \draw (6.0,9 ) node {$\mathit{GV}$};
      \draw (3.5,8 ) node {$\mathit{GN}$};
      \draw (5.0,8 ) node {$\mathit{V}$};
      \draw (7.0,8 ) node {$\mathit{GN}$};
      \draw (3.5,7 ) node {Jean};
      \draw (5.0,7 ) node {mange};
      \draw (6.5,7 ) node {$\mathit{Det}$};
      \draw (7.5,7 ) node {$\mathit{NC}$};
      \draw (6.5,6 ) node {une};
      \draw (7.5,6 ) node {pomme};

      \draw[-latex] (4.8,9.8) -- (4.2,9.2);
      \draw[-latex] (5.2,9.8) -- (5.8,9.2);
      \draw[-latex] (3.9,8.8) -- (3.6,8.2);
      \draw[-latex] (5.8,8.8) -- (5.2,8.2);
      \draw[-latex] (6.2,8.8) -- (6.8,8.2);
      \draw[-latex] (3.5,7.8) -- (3.5,7.2);
      \draw[-latex] (5.0,7.8) -- (5.0,7.2);
      \draw[-latex] (6.9,7.8) -- (6.6,7.2);
      \draw[-latex] (7.1,7.8) -- (7.4,7.2);
      \draw[-latex] (6.5,6.8) -- (6.5,6.2);
      \draw[-latex] (7.5,6.8) -- (7.5,6.2);
    \end{tikzpicture}
  \end{center}
  
  Les grammaires décrivent des mots et des langages par \structure{réécriture}.\\
  À la différence des définitions rationnelles, elles autorisent la \structure{récursivité}. 
\end{frame}
\endgroup

%% SPDX-License-Identifier: CC-BY-SA-4.0
% Author: Matthieu Perrin
% Part: 
% Section: 
% Sub-section: 
% Frame: 

\begingroup

\begin{frame}{Construction des expressions rationnelles}
  
  Les expressions rationnelles sur un alphabet $\Sigma$ sont construites par induction
  
  \begin{description}[xxxxxxxxxxx]
  \item[$\emptyset$ :] représente le langage vide \alert{$\emptyset$}
  \item[$\varepsilon$ :] représente le langage neutre \alert{$\{\varepsilon\}$}
  \item[$a$ :] représente le langage \alert{$\{a\}$}, pour tout $a\in \Sigma$
  \item[$\mathit{reg}_1 \cdot \mathit{reg}_2$ :] représente le langage \alert{$\mathcal{L}(\mathit{reg}_1) \cdot \mathcal{L}(\mathit{reg}_2)$}
  \item[$\mathit{reg}_1 \mid \mathit{reg}_2$ :] représente le langage \alert{$\mathcal{L}(\mathit{reg}_1) \cup \mathcal{L}(\mathit{reg}_2)$}
  \item[$\mathit{reg}^\star$ :] représente le langage \alert{$\mathcal{L}(\mathit{reg})^\star$}
  \end{description}

  Où \structure{$\mathcal{L}(\mathit{reg})$} est le langage représenté par l'expression rationnelle $\mathit{reg}$

  \pause
  \begin{block}{Notations}
    \begin{itemize}
    \item L'opérateur \alert{$\cdot$} est le plus souvent \structure{omis}
    \item On utilise les règles de \alert{préséance} : \quad
      \structure{1.} \alert{${}^\star$} \quad
      \structure{2.} \alert{$\cdot$}   \quad
      \structure{3.} \alert{$|$}
      \begin{center}
        Par exemple, \example{$\mathcal{L}(ab^\star\mid cd^\star) = \mathcal{L}(((a \cdot b^\star) \mid (c \cdot d^\star)))$}
      \end{center}
    \item D'autres constructions peuvent être ajoutées \emph{(sucre syntaxique)}
      \begin{center}
        \structure{$\mathit{reg}^+ \eqdef (\mathit{reg} \cdot \mathit{reg}^\star)$} \quad\quad
        \structure{$\mathit{reg}^? \eqdef (\mathit{reg} \mid \varepsilon)$}
      \end{center}
    \item Une expression rationnelle \structure{canonique} n'utilise que les opérateurs \alert{$.$}, \alert{$|$} et \alert{${}^\star$}
    \end{itemize}
  \end{block}
  
\end{frame}

\endgroup

% 
%\subsection{Grammaires algébriques}
%% SPDX-License-Identifier: CC-BY-SA-4.0
% Author: Matthieu Perrin
% Part: 
% Section: 
% Sub-section: 
% Frame: 

\begingroup
\begin{frame}{Modélisation mathématique}
  \begin{block}{Définition -- Grammaire algébrique (ou hors-contexte)}
    \vspace{2mm}
    Une \structure{grammaire algébrique} est un quadruplet \alert{$\langle \Sigma, \Gamma, S, R \rangle$} tel que :
    \begin{description}
    \item[\alert{$\Sigma$}] alphabet : \structure{l'ensemble des terminaux}
    \item[\alert{$\Gamma$}] alphabet tel que $\Sigma \cap \Gamma = \emptyset$ : \structure{l'ensemble des non-terminaux}
    \item[\alert{$S$}] $\in \Gamma$ : \structure{l'axiome} (ou symbole initial)
    \item[\alert{$R$}] $\subseteq \Gamma \times (\Sigma \cup \Gamma)^\star$ : \structure{l'ensemble des règles de production}
    \end{description}

    \vspace{3mm}
    Une \structure{règle de production} est un couple \alert{$\langle A, \alpha \rangle \in R$}, noté $\alert{A \rightarrow \alpha}$ tel que :
    \begin{description}
    \item[\alert{$A$}] $\in \Gamma$ : \structure{membre gauche}
    \item[\alert{$\alpha$}] $\in (\Sigma \cup \Gamma)^\star$ : \structure{membre droit}
    \end{description}
  \end{block}

\end{frame}

\endgroup

%% SPDX-License-Identifier: CC-BY-SA-4.0
% Author: Matthieu Perrin
% Part: 
% Section: 
% Sub-section: 
% Frame: 

\begingroup

\begin{frame}{Exemple d'utilisation du lemme de pompage}
  
  \tfBlock[top=-5mm]{Montrer que $L \eqdef \{a^nb^nc^n \mid n\in \mathbb{N}\}$ n'est pas algébrique}{
    Soit $\Sigma \eqdef \{a, b, c\}$.%
    \only<2-|handout>{
      Si $L$ est algébrique, $L$ vérifie le lemme de pompage :

      \vspace{-4mm}
      $$
      \begin{array}{c}
        \structure{\exists N\in \mathbb{N}}, \alert{\forall u\in L, |u| \ge N} \Rightarrow (\structure{\exists v, w, x, y, z\in \Sigma^\star}, \\
        u = v \cdot w \cdot x\cdot y \cdot z \land w \cdot y\neq \varepsilon \land |w\cdot x\cdot y| \le N \land \alert{\forall i \in \mathbb{N}}, v\cdot w^i\cdot x\cdot y^i \cdot z \in L)
      \end{array}
      $$
      \vspace{-2mm}
      
      \structure{Soit $N$ donné par le lemme pompage}.
    }
    
    \only<3-|handout>{%
      \alert{Posons $u = a^N b^N c^N$. On a bien $u\in L$ et $|u| = 3N \ge N$}.\\
      \structure{Soit $v \cdot w \cdot x\cdot y \cdot z$ la décomposition de $u$ donnée par le lemme de pompage}. 
    }
  }
  
  \tf<4-|handout>[y=-10mm]{
    \begin{smArray}[width=4mm, height=3mm, name={$u=$}]
      \smCell[\smNone]{\alert{$a$}}      \smCoord{(a1)}
      \smCell[\smNone]{\alert{$\cdots$}}               
      \smCell[\smNone]{\alert{$a$}}      \smCoord{(an)}
      \smCell[\smNone]{\alert{$\cdot$}}               
      \smCell[\smNone]{\alert{$b$}}      \smCoord{(b1)}
      \smCell[\smNone]{\alert{$\cdots$}}               
      \smCell[\smNone]{\alert{$b$}}      \smCoord{(bn)}
      \smCell[\smNone]{\alert{$\cdot$}}               
      \smCell[\smNone]{\alert{$c$}}      \smCoord{(c1)}
      \smCell[\smNone]{\alert{$\cdots$}}               
      \smCell[\smNone]{\alert{$c$}}      \smCoord{(cn)}
      
      \draw [decorate, decoration={brace, amplitude=5pt}] ([xshift=1mm]a1.north west) -- ([xshift=-1mm]an.north east) node[midway,yshift=4mm]{$N$};
      \draw [decorate, decoration={brace, amplitude=5pt}] ([xshift=1mm]b1.north west) -- ([xshift=-1mm]bn.north east) node[midway,yshift=4mm]{$N$};
      \draw [decorate, decoration={brace, amplitude=5pt}] ([xshift=1mm]c1.north west) -- ([xshift=-1mm]cn.north east) node[midway,yshift=4mm]{$N$};

      \draw [decorate, decoration={brace, amplitude=5pt, mirror}] ([xshift=1mm]an.south west) -- ([xshift=-1mm]b1.south east) node[midway,yshift=-4mm]{$|wxy|\le N$};
      \draw [decorate, decoration={brace, amplitude=5pt, mirror}] ([xshift=1mm]bn.south west) -- ([xshift=-1mm]c1.south east) node[midway,yshift=-4mm]{$|wxy|\le N$};
    \end{smArray}       
  }

  \tf<5-|handout>[text, bottom=-1mm]{
    \begin{itemize}
    \item Comme $wy \neq \varepsilon$, $\alpha = wy[1] \in \Sigma$ est une lettre de $wy$.
    \item Comme $|wxy| \le N$, il existe $\beta \in \Sigma$ tel que $\beta$ n'est pas une lettre de $wy$. 
    \item \alert{Posons $i=2$}. $|v \cdot w^2 \cdot x\cdot  y^2\cdot   z|_\alpha > |v\cdot  w^2\cdot  x\cdot  y^2\cdot   z|_\beta$, donc $v\cdot  w^2\cdot  x\cdot  y^2\cdot   z\notin L$. 
    \end{itemize}
    Absurde ! Donc $L$ n'est pas algébrique. 
  }

  \tfExampleBlock<-4>[y=-25mm]{On sait}{}

  \tf<1>[y=-25mm, anchor=north, text]{
    \begin{itemize}
    \item $\begin{array}[t]{l}
      \alert{\forall L \in \textsc{alg}_\Sigma}, \structure{\exists N\in \mathbb{N}}, \forall u\in L, |u| \ge N \Rightarrow (\exists v, w, x, y, z\in \Sigma^\star, \\
      u = v w x y z \land w y\neq \varepsilon \land |wxy| \le N \land \forall i \in \mathbb{N}, v w^i x y^i z \in L)
    \end{array}$
    \end{itemize}
  }

  \tf<2-4>[y=-25mm, anchor=north, left=.35\textwidth]{
    \begin{itemize}
    \item $L \in \textsc{alg}_\Sigma$
    \item<3-> $v, w, x, y, z\in \Sigma^\star$
    \end{itemize}
  }
  
  \tf<2-4>[y=-25mm, anchor=north, width=.35\textwidth]{
    \begin{itemize}
    \item $N \in \mathbb{N}$
    \item<3-> $wy\neq \varepsilon$
    \end{itemize}
  }
  
  \tf<3-4>[y=-25mm, anchor=north, right=.35\textwidth]{
    \begin{itemize}
    \item $v \cdot w \cdot x \cdot y \cdot z = a^N b^N c^N$
    \item $\alert{|wxy| \le N}$
    \end{itemize}
  }
  
\end{frame}

\endgroup

%% SPDX-License-Identifier: CC-BY-SA-4.0
% Author: Matthieu Perrin
% Part: 
% Section: 
% Sub-section: 
% Frame: 

\begingroup

\begin{frame}{Dérivations}
  Soit \alert{$\langle \Sigma, \Gamma, S, R \rangle$} une grammaire algébrique. 

  \begin{block}{Définitions -- Dérivation}
    Soient $u, v \in (\Sigma \cup \Gamma)^\star$. 

    \begin{itemize}
    \item On dit que \structure{$u$ se dérive directement en $v$}, noté $\alert{u \vdash v}$, si :

      \vspace{-2mm}
      $$\alert{\exists \structure{x}, \structure{y} \in (\Sigma \cup \Gamma)^\star, \exists \example{A \rightarrow \alpha} \in R,
        u = \structure{x} \cdot \example{A} \cdot \structure{y} \land v = \structure{x} \cdot \example{\alpha} \cdot \structure{y}}$$

    \item On note parfois \structure{$x A y \vdash_{A \rightarrow \alpha} x \alpha y$} pour indiquer la règle utilisée.

    \item On dit que \structure{$u$ se dérive en $v$} si \alert{$u \vdash^\star v$},\\
      où \alert{$\vdash^\star$ est la fermeture transitive et réflexive de $\vdash$}.

    \end{itemize}
  \end{block}
  \begin{exampleblock}{Exemple}
    Soit \example{$G = \left\langle \{a, b, c\}, \{S, B\}, S, \left\{\begin{array}{rcl} S &\rightarrow & aSb \,|\, B \\ B &\rightarrow & cB \,|\, \varepsilon \end{array}\right\} \right\rangle$}. On a :

    \begin{itemize}
    \item $\structure{a}\alert{S}\structure{b} \vdash_{S \rightarrow aSb} \structure{a}\alert{aSb}\structure{b}$
    \item $aSb \vdash aaSbb \vdash aaBbb \vdash aacBbb \vdash aaccBbb$
    \item $aSb \vdash^\star aSb \vdash^\star aaccBbb$
    \end{itemize}
  \end{exampleblock}
\end{frame}

\endgroup

%% SPDX-License-Identifier: CC-BY-SA-4.0
% Author: Matthieu Perrin
% Part: 
% Section: 
% Sub-section: 
% Frame: 

\begingroup

\begin{frame}{Langage reconnu par un AFN}

  \on[text, top]{
    Soit $A=\langle \Sigma, Q, I, F, \rightarrow \rangle$ un AFN.

    \begin{block}{Définition -- langage reconnu (ou accepté)}
      Un mot $u$ est \structure{reconnu} par $A$ s'il mène d'un état initial à un état final
      $$\exists i\in I, \exists f\in F,~  \alert{\langle u, i \rangle \leadsto_A^\star \langle \varepsilon, f\rangle}$$
      Le langage \structure{reconnu} par $A$ est l'ensemble $\alert{\mathcal{L}(A)}$ des mots reconnus par $A$
      $$\alert{\mathcal{L}(A) \eqdef \left\{u \in \Sigma^\star \,\middle\mid\, \exists i\in I, \exists f\in F,  \langle u, i \rangle \leadsto_A^\star \langle \varepsilon, f\rangle\right\}}$$
    \end{block}
  }

  \onExampleBlock[bottom]{Exemple}{
    Pour \example{$i=3$}, \example{$f=2$} et \example{$u=bbc$}, on a : 
    \begin{enumerate}
    \item $\langle bbc, 1 \rangle \leadsto^\star \langle \varepsilon, 2 \rangle$
    \item $1 \in I$
    \item $2 \in F$
    \end{enumerate}
    Donc \example{$bbc \in \mathcal{L}(A)$}
  }

  \on[bottom, x=.25\textwidth]{
    \begin{tikzpicture}[automaton, grid size=15mm]
      \state[initial]       (1) at (0,1) {$1$}; 
      \state[accepting]     (2) at (1,1) {$2$}; 
      \state[initial above] (3) at (0,0) {$3$}; 
      \state                (4) at (1,0) {$4$}; 

      \path (1) edge             node       {$a$}           (2);
      \path (1) edge[loop above] node       {$a$}           (1);
      \path (3) edge             node[swap] {$b$}           (4);
      \path (3) edge[loop left ] node       {$b$}           (3);
      \path (4) edge[loop right] node       {$b$}           (4);
      \path (2) edge[loop above] node       {$c$}           (2);
      \path (4) edge             node[swap] {$\varepsilon$} (2);
    \end{tikzpicture}
  }
  
\end{frame}


\endgroup

%% SPDX-License-Identifier: CC-BY-SA-4.0
% Author: Matthieu Perrin
% Part: 
% Section: 
% Sub-section: 
% Frame: 

\begingroup

\begin{frame}{Exemples de grammaires et de langages algébriques}
  \begin{enumerate}
  \item \structure{Langage des parenthèses imbriquées}\\
    \begin{itemize}
    \item $\mathcal{L}(\langle\{\alert{a}, \alert{b}\},\{S\},S,\{S \rightarrow \alert{a}S\alert{b} \;|\; \varepsilon\}\rangle) = \{\alert{a}^n\alert{b}^n, n\geq 0\}$\\
    \item $\mathcal{L}(\langle\{\alert{(}, \, \alert{)}\},\{S\},S,\{S \rightarrow \alert{(}S\alert{)}\, \;|\;\varepsilon \}\rangle) = \{\alert{(}^n\, \alert{)}^n, n\geq 0\}$
    \end{itemize}
  \item \structure{Langage de Dyck}\\
    chaque parenthèse ouvrante correspond à une parenthèse fermante
    \begin{itemize}
    \item sur un couple de parenthèses, $\Sigma = \{\alert{(}, \alert{)}\}$\\
      $\mathcal{L}(\langle \Sigma ,\{S\},S,\{S \rightarrow S\alert{(}S\alert{)} \;|\; \varepsilon \}\rangle)$
    \item sur $n$ couples de parenthèses, $\Sigma_n = \{\alert{(_1}, \alert{)_1}, \ldots, \alert{(_n}, \alert{)_n}\}$\\
      $\mathcal{L}(\langle \Sigma ,\{S\},S,\{S \rightarrow S\alert{(_1}S\alert{)_1} \;|\; \ldots \;|\;S\alert{(_n}S\alert{)_n} \;|\; \varepsilon \}\rangle)$
    \end{itemize}
  \item \structure{Expressions bien parenthésées}\\
    \begin{itemize}
    \item $\mathcal{L}(\langle \{\alert{(}, \alert{)}, \alert{a}\} ,\{S\},S,\{S \rightarrow \alert{(}S\alert{)} \;|\; SS \;|\; a\}\rangle)$
    \end{itemize}
  \item Syntaxe des langages de programmation : C, C++, Java...
  \item Formats de données : HTML, XML, CSV...
  \item Formules logiques, mathématiques...
  \end{enumerate}
\end{frame}
\endgroup

% 
%\subsection{Structure arborescente des langages algébriques}
%% SPDX-License-Identifier: CC-BY-SA-4.0
% Author: Matthieu Perrin
% Part: 
% Section: 
% Sub-section: 
% Frame: 

\begingroup

\begin{frame}{Arbre de dérivation }

  Soit \alert{$G = \langle \Sigma, \Gamma, S, R \rangle$} une grammaire algébrique.

  \begin{block}{Arbre de dérivation (ou arbre de la syntaxe concrète, CST)}

    Chaque génération de $G$ peut être représentée par un arbre.

    \begin{description}
    \item[Racine :] l'axiome $S$\\
    \item[N\oe ud interne :] Symbole non-terminal de $\Gamma$
    \item[Feuille :] Symbole terminal de $\Sigma$, ou $\varepsilon$ ($\varepsilon$ ne peut pas avoir de frère)
    \item[Filiation :] Si les fils de $A$ sont $a_1$, ..., $a_n$, alors $A \rightarrow a_1...a_n \in R$
    \end{description}
    Le mot généré est alors la concaténation des feuilles de l'arbre.
  \end{block}



  \begin{exampleblock}{Exemple}

    \vspace{-5mm}
    \begin{minipage}{.8\textwidth}
      \example{$G = \left\langle \{a, b, c\}, \{S, B\}, S, \left\{\begin{array}{rcl} S &\rightarrow & aSb \,|\, B \\ B &\rightarrow & cB \,|\, \varepsilon \end{array}\right\} \right\rangle$}.\\

      \vspace{2mm}
      Arbre de la syntaxe concrète de $aacbb$ :
    \end{minipage}%
    \begin{minipage}{.2\textwidth}
      \vspace{-2mm}
      \scalebox{.9}{\begin{tikzpicture}
          \draw(5.00,5.0) node{$S$};
          \draw(4.00,4.3) node{$a$};
          \draw(5.00,4.3) node{$S$};
          \draw(6.00,4.3) node{$b$};
          \draw(4.25,3.6) node{$a$};
          \draw(5.00,3.6) node{$S$};
          \draw(5.75,3.6) node{$b$};
          \draw(5.00,2.9) node{$B$};
          \draw(4.50,2.2) node{$c$};
          \draw(5.50,2.2) node{$B$};
          \draw(5.50,1.5) node{$\varepsilon$};

          \draw[-latex] (4.80,4.8) -- (4.20,4.5);
          \draw[-latex] (5.00,4.8) -- (5.00,4.5);
          \draw[-latex] (5.20,4.8) -- (5.80,4.5);

          \draw[-latex] (4.80,4.1) -- (4.45,3.8);
          \draw[-latex] (5.00,4.1) -- (5.00,3.8);
          \draw[-latex] (5.20,4.1) -- (5.55,3.8);

          \draw[-latex] (5.00,3.4) -- (5.00,3.1);

          \draw[-latex] (4.90,2.7) -- (4.60,2.4);
          \draw[-latex] (5.10,2.7) -- (5.40,2.4);

          \draw[-latex] (5.50,2.0) -- (5.50,1.7);

      \end{tikzpicture}}
    \end{minipage}%

  \end{exampleblock}
\end{frame}

\endgroup

%% SPDX-License-Identifier: CC-BY-SA-4.0
% Author: Matthieu Perrin
% Part: 
% Section: 
% Sub-section: 
% Frame: 

\begingroup

\begin{frame}{Arbre de la syntaxe abstraite}

  \tfBlock[top=-5mm]{Arbre de la syntaxe abstraite (AST : Abstract Syntax Tree)}{\vspace{-1mm}
    \begin{itemize}
    \item Les arbres de dérivation peuvent être simplifiés en un AST
      \begin{description}[Feuilles :]
      \item[N\oe uds :] Constructions du langage (par exemple opérateurs)
      \item[Feuilles :] Constantes ou variables
      \end{description}
    \item La suite de la compilation utilise un AST. Par exemple, en CUP\\
      \texttt{S::=    S:g PLUS S:d    \{: RESULT = new Sum(g, d); :\}}
    \end{itemize}
  }

  \tfExampleBlock[y=-2mm]{Exemple -- Dérivation de $u$ par $G$}{\vspace{-1mm}
    \begin{itemize}
    \item $u=\example{((1+1)\times (1+1))}$
    \item $G = \langle \{\example{1}, \example{+}, \example{\times}, \example{(}, \example{)}\}, \{S\}, S, \{S \rightarrow \example{(}S \example{+} S\example{)} | \example{(}S \example{\times} S\example{)} | \example{1} \} \rangle$
    \end{itemize}
  }

  \tf[bottom=-1mm, x=-.25\textwidth] {
    \begin{tikzpicture}
      \node[example] at (3.25,2.5) {Arbre de la syntaxe concrète};

      \node[structure] (a) at (3.25,2.1) {$S$};
      \node[alert]     (b) at (1.0,1.4)  {$($};      \path[-latex] (a) edge (b);  
      \node[structure] (c) at (2.0,1.4)  {$S$};      \path[-latex] (a) edge (c);  
      \node[alert]     (d) at (3.25,1.4) {$\times$}; \path[-latex] (a) edge (d);  
      \node[structure] (e) at (4.5,1.4)  {$S$};      \path[-latex] (a) edge (e);  
      \node[alert]     (f) at (5.5,1.4)  {$)$};      \path[-latex] (a) edge (f);  
      \node[alert]     (g) at (1.0,0.7)  {$($};      \path[-latex] (c) edge (g);  
      \node[structure] (h) at (1.5,0.7)  {$S$};      \path[-latex] (c) edge (h);  
      \node[alert]     (i) at (2.0,0.7)  {$+$};      \path[-latex] (c) edge (i);  
      \node[structure] (j) at (2.5,0.7)  {$S$};      \path[-latex] (c) edge (j);  
      \node[alert]     (k) at (3.0,0.7)  {$)$};      \path[-latex] (c) edge (k);  
      \node[alert]     (l) at (3.5,0.7)  {$($};      \path[-latex] (e) edge (l);  
      \node[structure] (m) at (4.0,0.7)  {$S$};      \path[-latex] (e) edge (m);  
      \node[alert]     (n) at (4.5,0.7)  {$+$};      \path[-latex] (e) edge (n);  
      \node[structure] (o) at (5.0,0.7)  {$S$};      \path[-latex] (e) edge (o);  
      \node[alert]     (p) at (5.5,0.7)  {$)$};      \path[-latex] (e) edge (p);  
      \node[alert]     (q) at (1.5,0.0)  {$1$};      \path[-latex] (h) edge (q);  
      \node[alert]     (r) at (2.5,0.0)  {$1$};      \path[-latex] (j) edge (r);  
      \node[alert]     (s) at (4.0,0.0)  {$1$};      \path[-latex] (m) edge (s);  
      \node[alert]     (t) at (5.0,0.0)  {$1$};      \path[-latex] (o) edge (t);  
    \end{tikzpicture}
  }
  \tf[bottom=-1mm, x=.25\textwidth] {
    \begin{tikzpicture}
      \node[example] at (1.5,2.5) {Arbre de la syntaxe abstraite};

      \node[structure] (a) at (1.5,2.0) {$\times$};
      \node[structure] (b) at (0.5,1.0) {$+$};            \smPath (a) edge (b);  
      \node[structure] (c) at (2.5,1.0) {$+$};            \smPath (a) edge (c);  
      \node[alert]     (d) at (0.0,0.0) {$1$};            \smPath (b) edge (d);  
      \node[alert]     (e) at (1.0,0.0) {$1$};            \smPath (b) edge (e);  
      \node[alert]     (f) at (2.0,0.0) {$1$};            \smPath (c) edge (f);  
      \node[alert]     (g) at (3.0,0.0) {$1$};            \smPath (c) edge (g);  

    \end{tikzpicture}
  }

\end{frame}

\endgroup

%% SPDX-License-Identifier: CC-BY-SA-4.0
% Author: Matthieu Perrin
% Part: 
% Section: 
% Sub-section: 
% Frame: 

\begingroup

\begin{frame}{Ambiguité, priorité, associativité}
  \small

  \begin{block}{Définition -- ambiguïté}
    Une grammaire est ambiguë si deux arbres de dérivation génèrent le même mot.
    \example{Exemple : }   $\left\{\begin{array}{rcl} A & \rightarrow & A \oplus A | A \otimes A | a \end{array}\right.$
  \end{block}
  
  \uncover<2->{
    \structure{Comment régler les principales causes d'ambiguïté ?}
    \begin{itemize}
    \item $\oplus$ est associatif à gauche quand : \hspace\fill $x \oplus y \oplus z$ signifie $(x \oplus y) \oplus z$\\
      \example{Exemple : } \lstinline{4 - 2 - 1}
      
    \item<3-> $\oplus$ est associatif à droite quand : \hspace\fill $x \oplus y \oplus z$ signifie $x \oplus (y \oplus z)$\\
      \example{Exemple : } \lstinline{x = y = 0;}
      
    \item<4-> $\otimes$ est prioritaire sur $\oplus$ quand : \hspace\fill $w \otimes x \oplus y \otimes z$ signifie $(w \otimes x) \oplus (y \otimes z)$\\
      \example{Exemple : } \lstinline{3 * 5 + 2}
      
    \end{itemize}
  }

  \uncover<2->{
    \begin{minipage}[t]{.33\textwidth}
      \structure{Associativité à gauche :}\\
      
      \centering

      $\left\{\begin{array}{rcl}
      A & \rightarrow & A \oplus B | B 
      \end{array}\right.$
      
      \centering
      \scalebox{.7}{\begin{tikzpicture}
          \draw (5,7.1) node{$A$};
          \draw (4,6.4) node{$A$};
          \draw (5,6.4) node{$\oplus$};
          \draw (6,6.4) node{$B$};
          \draw (3,5.7) node{$A$};
          \draw (4,5.7) node{$\oplus$};
          \draw (5,5.7) node{$B$};
          \draw (3,5.0) node{$B$};

          \draw[-latex] (4.8,6.9) -- (4.2,6.6); 
          \draw[-latex] (5.0,6.9) -- (5.0,6.6); 
          \draw[-latex] (5.2,6.9) -- (5.8,6.6); 
          \draw[-latex] (3.8,6.2) -- (3.2,5.9); 
          \draw[-latex] (4.0,6.2) -- (4.0,5.9); 
          \draw[-latex] (4.2,6.2) -- (4.8,5.9); 
          \draw[-latex] (3.0,5.5) -- (3.0,5.2); 
      \end{tikzpicture}}
  \end{minipage}}%
  \uncover<3->{\begin{minipage}[t]{.33\textwidth}
      \structure{Associativité à droite :}\\

      $\left\{\begin{array}{rcl}
      A & \rightarrow & B \oplus A | B 
      \end{array}\right.$
      
      \centering
      \scalebox{.7}{\begin{tikzpicture}
          \draw (5,7.1) node{$A$};
          \draw (4,6.4) node{$B$};
          \draw (5,6.4) node{$\oplus$};
          \draw (6,6.4) node{$A$};
          \draw (5,5.7) node{$B$};
          \draw (6,5.7) node{$\oplus$};
          \draw (7,5.7) node{$A$};
          \draw (7,5.0) node{$B$};

          \draw[-latex] (4.8,6.9) -- (4.2,6.6); 
          \draw[-latex] (5.0,6.9) -- (5.0,6.6); 
          \draw[-latex] (5.2,6.9) -- (5.8,6.6); 
          \draw[-latex] (5.8,6.2) -- (5.2,5.9); 
          \draw[-latex] (6.0,6.2) -- (6.0,5.9); 
          \draw[-latex] (6.2,6.2) -- (6.8,5.9); 
          \draw[-latex] (7.0,5.5) -- (7.0,5.2); 
      \end{tikzpicture}}
  \end{minipage}}%
  \uncover<4->{\begin{minipage}[t]{.33\textwidth}
      \structure{Priorité :}\\

      $\left\{\begin{array}{rcl}
      A & \rightarrow & B \oplus B | B \\
      B & \rightarrow & C \otimes C | C 
      \end{array}\right.$
      
      \centering
      \scalebox{.7}{\begin{tikzpicture}
          \draw (5,7.1) node{$A$};
          \draw (4,6.4) node{$B$};
          \draw (5,6.4) node{$\oplus$};
          \draw (6,6.4) node{$B$};
          \draw (4,5.7) node{$C$};
          \draw (5,5.7) node{$C$};
          \draw (6,5.7) node{$\oplus$};
          \draw (7,5.7) node{$C$};

          \draw[-latex] (4.8,6.9) -- (4.2,6.6); 
          \draw[-latex] (5.0,6.9) -- (5.0,6.6); 
          \draw[-latex] (5.2,6.9) -- (5.8,6.6); 
          \draw[-latex] (5.8,6.2) -- (5.2,5.9); 
          \draw[-latex] (6.0,6.2) -- (6.0,5.9); 
          \draw[-latex] (6.2,6.2) -- (6.8,5.9); 
          \draw[-latex] (4.0,6.2) -- (4.0,5.9); 
      \end{tikzpicture}}
  \end{minipage}}%
\end{frame}

\endgroup

%% SPDX-License-Identifier: CC-BY-SA-4.0
% Author: Matthieu Perrin
% Part: 
% Section: 
% Sub-section: 
% Frame: 

\begingroup

\begin{frame}{Grammaire des expressions rationnelles}
  \label{slide:grammaireRegex}
  Soit $\Sigma = \{a, b\}$. Le langage des expressions rationnelles sur $\Sigma$, \\$\textsc{reg}_\Sigma$ est engendré par la grammaire non-ambiguë
  $\langle \tilde\Sigma, \Gamma, \structure{\textsc{Union}}, R \rangle$, où :
  \begin{itemize}
  \item $\tilde\Sigma = \{\example{\underline{a}}, \example{\underline{b}}, \example{\underline{\emptyset}}, \example{\underline{\varepsilon}}, \example{\underline{(}}, \example{\underline{)}}, \example{\underline{|}}, \example{\underline{\cdot}}, \example{\underline{{}^\star}} \}$
  \item $\Gamma = \{\structure{\textsc{Union}}, \structure{\textsc{Concat}}, \structure{\textsc{Etoile}}, \structure{\textsc{Par}}, \structure{\textsc{Terme}}\}$
  \item $R =   \left\{\begin{array}{lllll}
    \structure{\textsc{Union}} &\rightarrow& \structure{\textsc{Union}} \, \example{\underline{|}} \, \structure{\textsc{Concat}} & | &  \structure{\textsc{Concat}} \\
    \structure{\textsc{Concat}} &\rightarrow& \structure{\textsc{Concat}} \, \example{\underline{\cdot}} \, \structure{\textsc{Etoile}} & | & \structure{\textsc{Etoile}} \\
    \structure{\textsc{Etoile}} &\rightarrow& \structure{\textsc{Etoile}}\,^{\example{\underline{\star}}} & | & \structure{\textsc{Par}} \\
    \structure{\textsc{Par}} &\rightarrow&  \example{\underline{(}} \structure{\textsc{Union}} \example{\underline{)}} & | &  \structure{\textsc{Terme}} \\
    \structure{\textsc{Terme}} &\rightarrow& \example{\underline{\varepsilon}} \,|\, \example{\underline{\emptyset}} \,|\, \example{\underline{a}} \,|\, \example{\underline{b}}  \\
  \end{array}\right\}$
  \end{itemize}

  \begin{minipage}[t]{.5\textwidth}
    \begin{block}{Arbre de la syntaxe concrète}
      \scalebox{.5}{\begin{tikzpicture}
          \draw (6,5.6) node{$\structure{\textsc{Union}}$};
          \draw (6,4.9) node{$\structure{\textsc{Etoile}}$};
          \draw (5,4.2) node{$\structure{\textsc{Etoile}}$};
          \draw (7,4.2) node{${}^{\example{\underline{\star}}}$};
          \draw (5,3.5) node{$\structure{\textsc{Par}}$};
          \draw (4,2.8) node{$\example{\underline{(}}$};
          \draw (5,2.8) node{$\structure{\textsc{Union}}$};
          \draw (6,2.8) node{$\example{\underline{)}}$};
          \draw (1,2.1) node{$\structure{\textsc{Concat}}$};
          \draw (3,2.1) node{$\structure{\textsc{Union}}$};
          \draw (5,2.1) node{$\example{\underline{|}}$};
          \draw (7,2.1) node{$\structure{\textsc{Concat}}$};
          \draw (0,1.4) node{$\structure{\textsc{Concat}}$};
          \draw (1,1.4) node{$\example{\underline{\cdot}}$};
          \draw (2,1.4) node{$\structure{\textsc{Etoile}}$};
          \draw (6,1.4) node{$\structure{\textsc{Concat}}$};
          \draw (7,1.4) node{$\example{\underline{\cdot}}$};
          \draw (8,1.4) node{$\structure{\textsc{Etoile}}$};
          \draw (0,0.7) node{$\structure{\textsc{Terme}}$};
          \draw (2,0.7) node{$\structure{\textsc{Terme}}$};
          \draw (6,0.7) node{$\structure{\textsc{Terme}}$};
          \draw (8,0.7) node{$\structure{\textsc{Terme}}$};
          \draw (0,0.0) node{$\example{\underline{a}}$};
          \draw (2,0.0) node{$\example{\underline{b}}$};
          \draw (6,0.0) node{$\example{\underline{b}}$};
          \draw (8,0.0) node{$\example{\underline{a}}$};
          
          \draw[-latex] (6.0,5.4) -- (6.0,5.1);
          \draw[-latex] (5.8,4.7) -- (5.2,4.4);
          \draw[-latex] (6.2,4.7) -- (6.8,4.4);
          \draw[-latex] (5.0,4.0) -- (5.0,3.7);
          \draw[-latex] (4.8,3.3) -- (4.2,3.0);
          \draw[-latex] (5.0,3.3) -- (5.0,3.0);
          \draw[-latex] (5.2,3.3) -- (5.8,3.0);
          \draw[-latex] (2.3,2.1) -- (1.7,2.1);
          \draw[-latex] (4.6,2.6) -- (3.4,2.3);
          \draw[-latex] (5.0,2.6) -- (5.0,2.3);
          \draw[-latex] (5.4,2.6) -- (6.6,2.3);
          \draw[-latex] (0.8,1.9) -- (0.2,1.6);
          \draw[-latex] (1.0,1.9) -- (1.0,1.6);
          \draw[-latex] (1.2,1.9) -- (1.8,1.6);
          \draw[-latex] (6.8,1.9) -- (6.2,1.6);
          \draw[-latex] (7.0,1.9) -- (7.0,1.6);
          \draw[-latex] (7.2,1.9) -- (7.8,1.6);
          \draw[-latex] (0.0,1.2) -- (0.0,0.9);
          \draw[-latex] (2.0,1.2) -- (2.0,0.9);
          \draw[-latex] (6.0,1.2) -- (6.0,0.9);
          \draw[-latex] (8.0,1.2) -- (8.0,0.9);
          \draw[-latex] (0.0,0.5) -- (0.0,0.2);
          \draw[-latex] (2.0,0.5) -- (2.0,0.2);
          \draw[-latex] (6.0,0.5) -- (6.0,0.2);
          \draw[-latex] (8.0,0.5) -- (8.0,0.2);
      \end{tikzpicture}}
    \end{block}
  \end{minipage}%
  \begin{minipage}[t]{.5\textwidth}
    \begin{block}{Arbre de la syntaxe abstraite}
      \scalebox{.75}{\begin{tikzpicture}
          \draw (3.0,3.0) node{${}^{\example{\underline{\star}}}$};
          \draw (3.0,2.0) node{$\example{\underline{|}}$};
          \draw (1.0,1.0) node{$\example{\underline{\cdot}}$};
          \draw (5.0,1.0) node{$\example{\underline{\cdot}}$};
          \draw (0.0,0.0) node{$\example{\underline{a}}$};
          \draw (2.0,0.0) node{$\example{\underline{b}}$};
          \draw (4.0,0.0) node{$\example{\underline{b}}$};
          \draw (6.0,0.0) node{$\example{\underline{a}}$};

          \draw[-latex] (3.0,2.8) -- (3.0,2.3); 
          \draw[-latex] (2.6,1.8) -- (1.4,1.2); 
          \draw[-latex] (3.4,1.8) -- (4.6,1.2); 
          \draw[-latex] (0.8,0.8) -- (0.2,0.2); 
          \draw[-latex] (1.2,0.8) -- (1.8,0.2); 
          \draw[-latex] (4.8,0.8) -- (4.2,0.2); 
          \draw[-latex] (5.2,0.8) -- (5.8,0.2); 
      \end{tikzpicture}}
    \end{block}
  \end{minipage}
\end{frame}
\endgroup

% 
%\subsection{Algorithmes de décision}
%% SPDX-License-Identifier: CC-BY-SA-4.0
% Author: Matthieu Perrin
% Part: 
% Section: 
% Sub-section: 
% Frame: 

\begingroup

\begin{frame}{Deux grandes approches pour l'analyse syntaxique}

  \begin{alertblock}{Problème}
    Étant donnés :
    \begin{itemize}
      \item une grammaire algébrique $G$
      \item un mot $u \in \Sigma^\star$
    \end{itemize}
    peut-on décider efficacement si \structure{$u \in \mathcal{L}(G)$} ?
  \end{alertblock}

  \begin{block}{Approche 1 -- Algorithme CYK (Cocke–Younger–Kasami)}
    \begin{itemize}
      \item S'applique à \structure{toute grammaire} mise en \structure{forme normale de Chomsky}%Fonctionne uniquement si $G$ est en \structure{forme normale de Chomsky (FNC)}.
      \item Repose sur une analyse exhaustive par \structure{programmation dynamique}
      \item Complexité en $\mathcal{O}(|R| \times |u|^3)$
    \end{itemize}
  \end{block}

  \begin{block}{Approche 2 -- Analyse déterministe (parsers LL ou LR)}
    \begin{itemize}
      \item S'applique aux grammaires \structure{non-ambiguës} et \structure{déterministes}
      \item Utilise un \structure{automate à pile déterministe}
      \item Complexité en $\mathcal{O}(|R| \times |u|)$
    \end{itemize}
  \end{block}

\end{frame}

\endgroup

%% SPDX-License-Identifier: CC-BY-SA-4.0
% Author: Matthieu Perrin
% Part: 
% Section: 
% Sub-section: 
% Frame: 

\begingroup

\begin{frame}{Forme normale de Chomsky}
  
  \begin{block}{Forme normale de Chomsky}
    Soit $G = \langle \Sigma, \Gamma, S, R \rangle$ une grammaire algébrique.

    $G$ est en \structure{forme normale de Chomsky} si les règles de $R$ sont de la forme :
    \begin{enumerate}
    \item $N \rightarrow XY$ \hspace{5mm} avec $N\in \Gamma$ et $X, Y \in \Gamma$
    \item $N \rightarrow a$  \hspace{8mm} avec $N \in \Gamma$ et $a\in \Sigma$
    \end{enumerate}
    ou bien toutes ses règles sont de la forme :
    \begin{enumerate}
    \item $N \rightarrow XY$ \hspace{5mm} avec $N\in \Gamma$ et $X, Y \in \Gamma \setminus \{S\}$
    \item $N \rightarrow a$  \hspace{8mm} avec $N \in \Gamma$ et $a\in \Sigma$
    \item $S \rightarrow \varepsilon$
    \end{enumerate}
  \end{block}

  \begin{block}{Théorème}
    Toute grammaire algébrique peut être transformée en une
    grammaire en \structure{forme normale de Chomsky} reconnaissant le même langage. 
  \end{block}
\end{frame}

\endgroup

%% SPDX-License-Identifier: CC-BY-SA-4.0
% Author: Matthieu Perrin
% Part: 
% Section: 
% Sub-section: 
% Frame: 

\begingroup

\begin{frame}{Transformation en forme normale de Chomsky}\label{slide:FNC}

  \tfBlock[top=-3mm]{Démonstration du théorème}{
    Appliquer successivement les règles suivantes :
    \begin{enumerate}
    \item Introduire une règle \structure{$X \rightarrow x$} pour chaque $x\in \Sigma$
    \item Eliminer les règles \structure{$X \rightarrow ...S...$} 
    \item Eliminer les règles \structure{$X \rightarrow \varepsilon$} pour $X \not= S$
    \item Eliminer les règles \structure{$X \rightarrow Y$}
    \item Remplacer les règles \structure{$N \rightarrow A_1 A_2 ... A_n$} par \structure{$N \rightarrow A_1 B$} et \structure{$B \rightarrow A_2 ... A_n$}
    \end{enumerate}
  }

  \tfExampleBlock<1,8|handout>[left=.5\textwidth, y=-2mm, anchor=north]{Exemple}{
    $\left\{\begin{array}{rcl}
    S &\rightarrow& a S b \;|\; T \\
    T &\rightarrow& c T \;|\; \varepsilon\\
    \end{array}\right.$
  }
  \tfExampleBlock<2>[left=.5\textwidth, y=-2mm, anchor=north]{Exemple}{
    $\left\{\begin{array}{rcl}
    S &\rightarrow& \alert{a} S \alert{b} \;|\; T \\
    T &\rightarrow& \alert{c} T \;|\; \varepsilon\\
    \end{array}\right.$
  }
  \tfExampleBlock<3>[left=.5\textwidth, y=-2mm, anchor=north]{Exemple}{
    $\left\{\begin{array}{rcl}
    S &\rightarrow& A\alert{S}B \;|\; T \\
    T &\rightarrow& CT \;|\; \varepsilon\\
    A &\rightarrow& a  \\
    B &\rightarrow& b\\
    C &\rightarrow& c \\
    \end{array}\right.$
  }
  \tfExampleBlock<4>[left=.5\textwidth, y=-2mm, anchor=north]{Exemple}{
    $\left\{\begin{array}{rcl}
    S &\rightarrow& D \\
    D &\rightarrow& ADB \;|\; T \\
    T &\rightarrow& CT \;|\; \alert{\varepsilon} \\
    A &\rightarrow& a \\
    B &\rightarrow& b \\
    C &\rightarrow& c \\
    \end{array}\right.$
  }
  \tfExampleBlock<5>[left=.5\textwidth, y=-2mm, anchor=north]{Exemple}{
    $\left\{\begin{array}{rcl}
    S &\rightarrow& D \\
    D &\rightarrow& ADB \;|\; T \;|\; \alert{\varepsilon} \\
    T &\rightarrow& CT \;|\; C \\
    A &\rightarrow& a \\
    B &\rightarrow& b \\
    C &\rightarrow& c \\
    \end{array}\right.$
  }
  \tfExampleBlock<6>[left=.5\textwidth, y=-2mm, anchor=north]{Exemple}{
    $\left\{\begin{array}{rcl}
    S &\rightarrow& \alert{D} \;|\; \varepsilon \\
    D &\rightarrow& ADB  \;|\; AB \;|\; \alert{T} \\
    T &\rightarrow& CT \;|\; \alert{C} \\
    A &\rightarrow& a \\
    B &\rightarrow& b \\
    C &\rightarrow& c \\
    \end{array}\right.$
  }
  \tfExampleBlock<7>[left=.5\textwidth, y=-2mm, anchor=north]{Exemple}{
    $\left\{\begin{array}{rcl}
    S &\rightarrow& \alert{ADB}  \;|\; AB \;|\; CT \;|\; c \;|\; \varepsilon \\
    D &\rightarrow& \alert{ADB}  \;|\; AB \;|\; CT \;|\; c \\
    T &\rightarrow& CT \;|\; c \\
    A &\rightarrow& a \\
    B &\rightarrow& b \\
    C &\rightarrow& c \\
    \end{array}\right.$
  }

  \tfBlock<2>[right=.6\textwidth, y=-2mm, anchor=north]{Application de la règle 1}{
    \begin{enumerate}
    \item Introduire trois nouveaux non-terminaux\\ $A$, $B$ et $C$ 
    \item Introduire les règles
      \\$A \rightarrow a$, $B \rightarrow b$, $C \rightarrow c$
    \item Remplacer $a$ par $A$, $b$ par $B$ et $c$ par $C$
    \end{enumerate}
  }
  \tfBlock<3>[right=.6\textwidth, y=-2mm, anchor=north]{Application de la règle 2}{
    \begin{enumerate}
    \item Introduire un nouveau non-terminal $D$ 
    \item Renommer $S$ en $D$
    \item Introduire une règle $S \rightarrow D$
    \end{enumerate}
  }
  \tfBlock<4>[right=.6\textwidth, y=-2mm, anchor=north]{Application de la règle 3}{
    \begin{enumerate}
    \item Isoler $\varepsilon$ dans $T$ :\\
      $\begin{array}{lll}
      T &\rightarrow& T' \;|\; \varepsilon\\
      T' &\rightarrow& CT\\
    \end{array}$
      
    \item Remplacer $T$ par sa définition $T' \;|\; \varepsilon$
    \item Renommer $T'$ en $T$
    \end{enumerate}
  }
  \tfBlock<5>[right=.6\textwidth, y=-2mm, anchor=north]{Application de la règle 3}{
    \begin{enumerate}
    \item Isoler $\varepsilon$ dans $D$ :\\
      $\begin{array}{lll}
      D &\rightarrow& D' \;|\; \varepsilon\\
      D' &\rightarrow& ABD\;|\; T\\
    \end{array}$
      
    \item Remplacer $D$ par sa définition $D' \;|\; \varepsilon$
    \item Renommer $D'$ en $D$
    \end{enumerate}
  }
  \tfBlock<6>[right=.6\textwidth, y=-2mm, anchor=north]{Application de la règle 4}{
    \begin{enumerate}
    \item Remplacer $T \rightarrow C$ par $T \rightarrow c$
    \item Remplacer $D \rightarrow T$ par $D \rightarrow CT \;|\; c$
    \item Remplacer $S \rightarrow D$ par $S \rightarrow ADB  \;|\; AB \;|\; CT \;|\; c$
    \end{enumerate}
  }
  \tfBlock<7>[right=.5\textwidth, y=-2mm, anchor=north]{Application de la règle 5}{
    \begin{enumerate}
    \item Remplacer $S \rightarrow ADB$ par :\\
      $\begin{array}{lll}
      S &\rightarrow& AE\\
      E &\rightarrow& BD\\
    \end{array}$
    \end{enumerate}
  }
  \tf<8|handout>[right=.5\textwidth, y=-10mm, anchor=north]{
    $\Rightarrow \left\{\begin{array}{rcl}
    S &\rightarrow& AE  \;|\; AB \;|\; CT \;|\; c \;|\; \varepsilon \\
    D &\rightarrow& AE  \;|\; AB \;|\; CT \;|\; c \\
    E &\rightarrow& DB \\
    T &\rightarrow& CT \;|\; c \\
    A &\rightarrow& a \\
    B &\rightarrow& b \\
    C &\rightarrow& c \\
    \end{array}\right.$
  }
  
\end{frame}

\endgroup

%% SPDX-License-Identifier: CC-BY-SA-4.0
% Author: Matthieu Perrin
% Part: 
% Section: 
% Sub-section: 
% Frame: 

\begingroup

          \SetKwFunction{CYK}{CYK}
            \SetKwData{Input}{motif}
  
\begin{frame}[fragile]{Algorithme Cocke-Younger-Kasami (CYK)}
  \tf[top, text]{\small
    \begin{algorithm}[H]
      \Fn{\CYK($\langle \Sigma, \Gamma, S, R \rangle$ : grammaire, $u \in \Sigma^\star$) : booléen}{
        \lSi{$u = \varepsilon$}{\Retourner $S \rightarrow \varepsilon \in R$}
        \Pour{$j$ allant de $1$ à $|u|$}{
          \tfAlert<3>{$T[1, j] \leftarrow \{N\in \Gamma | N \rightarrow u[j] \in R\}$}\;
        }
        \Pour{$i$ allant de $2$ à $|u|$}{
          \Pour{$j$ allant de $1$ à $|u|-i+1$}{
            \tfAlert<4->{$\displaystyle T[i, j] \leftarrow \bigcup_{k=1}^{i-1} \{N\in \Gamma | \exists X \in T[k,j], \exists Y \in T[i-k,j+k], N \rightarrow X Y \in R\}$}\;
          }
        }
        \Retourner \tfAlert<2>{$S\in T[|u|, 1]$}\;
      }
    \end{algorithm}
  }

  \tfExampleBlock[bottom]{Exemple}{
    \vspace{2mm}
    $G=\left\{\begin{array}{lll}
    \alert<5,9>{S} &\alert<5,9>{\rightarrow}&  \alert<9>{AC} \;|\; \alert<5>{AB} \;|\; \varepsilon  \\
    \alert<8>{C} &\alert<8>{\rightarrow}&  \alert<8>{TB}  \\
    \alert<5,9>{T} &\alert<5,9>{\rightarrow}&  \alert<9>{AC} \;|\; \alert<5>{AB}  \\
    \alert<3>{A} &\alert<3>{\rightarrow}& \alert<3>{a}  \\
    \alert<3>{B} &\alert<3>{\rightarrow}& \alert<3>{b} \end{array}\right.$
    
    \vspace{2mm}
    $u = aabb$
  }

  \tf[x=17mm, y=-2cm]{
      \begin{tikzpicture}
        {\scriptsize
          \draw (-.75,2.8) node[above left]{\scriptsize $T[i,j]$};
          \draw (0.0 ,2.8) node[above]     {\scriptsize $j=1$};
          \draw (1.5 ,2.8) node[above]     {\scriptsize $j=2$};
          \draw (3.0 ,2.8) node[above]     {\scriptsize $j=3$};
          \draw (4.5 ,2.8) node[above]     {\scriptsize $j=4$};
          \draw (-.75,2.4) node[left]      {\scriptsize $i=1$};
          \draw (-.75,1.6) node[left]      {\scriptsize $i=2$};
          \draw (-.75,0.8) node[left]      {\scriptsize $i=3$};
          \draw (-.75,0.0) node[left]      {\scriptsize $i=4$};
        }
        
        \node (a1) at (0.0,2.4) {}; 
        \node (a2) at (1.5,2.4) {}; 
        \node (a3) at (3.0,2.4) {}; 
        \node (a4) at (4.5,2.4) {}; 
        \node (b1) at (0.0,1.6) {}; 
        \node (b2) at (1.5,1.6) {}; 
        \node (b3) at (3.0,1.6) {}; 
        \node (c1) at (0.0,0.8) {}; 
        \node (c2) at (1.5,0.8) {}; 
        \node (d1) at (0.0,0.0) {}; 
        

        \draw[fill=structure!15] (a1) +(-.75,-.4) rectangle +(.75,.4);
        \draw[fill=structure!15] (a2) +(-.75,-.4) rectangle +(.75,.4);
        \draw[fill=structure!15] (a3) +(-.75,-.4) rectangle +(.75,.4);
        \draw[fill=structure!15] (a4) +(-.75,-.4) rectangle +(.75,.4);
        \draw[fill=structure!15] (b1) +(-.75,-.4) rectangle +(.75,.4);
        \draw[fill=structure!15] (b2) +(-.75,-.4) rectangle +(.75,.4);
        \draw[fill=structure!15] (b3) +(-.75,-.4) rectangle +(.75,.4);
        \draw[fill=structure!15] (c1) +(-.75,-.4) rectangle +(.75,.4);
        \draw[fill=structure!15] (c2) +(-.75,-.4) rectangle +(.75,.4);
        \draw[fill=structure!15] (d1) +(-.75,-.4) rectangle +(.75,.4);

        {\footnotesize
          \draw[example] (a1) +(-.75,-.4) node[above right] {$a$};
          \draw[example] (a2) +(-.75,-.4) node[above right] {$a$};    
          \draw[example] (a3) +(-.75,-.4) node[above right] {$b$};    
          \draw[example] (a4) +(-.75,-.4) node[above right] {$b$};    
          \draw[example] (b1) +(-.75,-.4) node[above right] {$aa$};   
          \draw[example] (b2) +(-.75,-.4) node[above right] {$ab$};   
          \draw[example] (b3) +(-.75,-.4) node[above right] {$bb$};   
          \draw[example] (c1) +(-.75,-.4) node[above right] {$aab$};  
          \draw[example] (c2) +(-.75,-.4) node[above right] {$abb$};  
          \draw[example] (d1) +(-.75,-.4) node[above right] {$aabb$}; 
        }
 
        \only<1>{
          \draw[-latex, example] (1.25,0.9) -- (1.25,2.5);
          \draw[-latex, example] (1.25,0.9) -- (4.25,2.5);
        }\only<2>{
          \node[structure] at (a1) {$\{...\}$};
          \node[structure] at (a2) {$\{...\}$};
          \node[structure] at (a3) {$\{...\}$};
          \node[structure] at (a4) {$\{...\}$};
          \node[structure] at (b1) {$\{...\}$};
          \node[structure] at (b2) {$\{...\}$};
          \node[structure] at (b3) {$\{...\}$};
          \node[structure] at (c1) {$\{...\}$};
          \node[structure] at (c2) {$\{...\}$};
          \node[structure] at (d1) {$\{...\alert{S?}...\}$};
        }\only<3-|handout>{
          \node[structure] at (a1) {$\{A\}$};
          \node[structure] at (a2) {$\{A\}$};
          \node[structure] at (a3) {$\{B\}$};
          \node[structure] at (a4) {$\{B\}$};
        }\only<4-|handout>{
          \node[structure] at (b1) {$\emptyset$};
        }\only<5-|handout>{
          \node[structure] at (b2) {$\{S, T\}$};
        }\only<6-|handout>{
          \node[structure] at (b3) {$\emptyset$};
        }\only<7-|handout>{
          \node[structure] at (c1) {$\emptyset$};
        }\only<7>{
          \draw[alert] (0.0,2.55)  -- (1.5,1.75);
          \draw[alert] (0.0,1.75)  -- (3.0,2.55);
        }\only<8-|handout>{
          \node<8->[structure] at (c2) {$\{C\}$};
        }\only<8>{
          \draw[alert] (1.5,2.55)  -- (3.0,1.75);
          \draw[alert] (1.5,1.75)  -- (4.5,2.55);
        }\only<9-|handout>{
          \node[structure] at (d1) {$\{\alert{S}, T\}$};
        }\only<9>{
          \draw[alert] (0,2.55)  -- (1.5,0.95)  ;
          \draw[alert] (0,1.75)  -- (3.0,1.75)  ;
          \draw[alert] (0,0.95)  -- (4.5,2.55)  ;
        }
        
      \end{tikzpicture}
  }

  \tf[width=40mm, x=43mm, y=-23mm, anchor=north]{\small%
    \only<-2|handout>{%
      \example{%
        $T[i,j]$ représente le facteur \\
        $u_{i,j} = u[j], ..., u[j+i]$\\\vspace{2mm}
      }%
    }%
    \only<2|handout>{%
      \structure{%
        $T[i,j]$ contient \\
        $\{N \in \Gamma | N \vdash^\star u_{i,j}\}$
      }%
    }%
    \only<4>{%
      On cherche \alert{$N \rightarrow XY \vdash aa$}\\
      $\bullet$ $X = A \in T[1,1]$ \\
      $\bullet$ $Y = A \in T[1,2]$ \\
      Pas de règle \alert{$N \rightarrow AA$}\\
    }%
    \only<5>{%
      On cherche \alert{$N \rightarrow XY \vdash ab$}\\
      $\bullet$ $X = A \in T[1,2]$ \\
      $\bullet$ $Y = B \in T[1,3]$ \\
      Deux règles \alert{$N \rightarrow AB$}\\
    }%
    \only<6>{%
      On cherche \alert{$N \rightarrow XY \vdash bb$}\\
      $\bullet$ $X = B \in T[1,3]$ \\
      $\bullet$ $Y = B \in T[1,4]$ \\
      Pas de règle \alert{$N \rightarrow BB$}\\
    }%
    \only<7>{%
      $aab = a \cdot ab = aa \cdot b$\\
      On cherche \alert{$N \rightarrow XY$}\\
      $\bullet$ $X \vdash^\star a$, $Y \vdash^\star ab$  \\
      $\bullet$ $X \vdash^\star aa$, $Y \vdash^\star b$  \\
      \footnotesize \alert{$XY \in \{A\} \cdot \{S, T\} \cup \emptyset \cdot \{B\}$}\\
      Pas de règle \alert{$N \rightarrow AS$} ou \alert{$N \rightarrow AT$}\\
    }%
    \only<8>{%
      $aab = a \cdot bb = ab \cdot b$\\
      On cherche \alert{$N \rightarrow XY$}\\
      $\bullet$ $X \vdash^\star a$, $Y \vdash^\star bb$  \\
      $\bullet$ $X \vdash^\star ab$, $Y \vdash^\star b$  \\
      \footnotesize \alert{$XY \in \{A\} \cdot \emptyset \cup \{S, T\} \cdot \{B\}$}\\
      Une règle \alert{$N \rightarrow TB$}\\
    }%
    \only<9>{
      $\begin{array}{lll}
        aabb &=& a \cdot bbb\\
        &=& aa \cdot bb\\
        &=& aab \cdot b\\
        \alert{aabb} & \alert{\in} &\alert{\mathcal{L}(G)}
      \end{array}$
    }%
  }
\end{frame}

\endgroup

% 
%\subsection{Grammaires rationnelles}
%% SPDX-License-Identifier: CC-BY-SA-4.0
% Author: Matthieu Perrin
% Part: 
% Section: 
% Sub-section: 
% Frame: 

\begingroup

\begin{frame}{Lien entre langages rationnels et algébriques}
  
  \begin{block}{Théorème -- Lien entre langages rationnels et algébriques}
    Tout langage rationnel sur un alphabet $\Sigma$ est algébrique sur $\Sigma$ :
    $$\alert{\textsc{rat}_\Sigma \subsetneq \textsc{alg}_\Sigma}$$
  \end{block}

  \vspace{-2mm}
  \begin{block}{Démonstration}
    \begin{itemize}
    \item Introduction des grammaires rationnelles
      \begin{itemize}
      \item Preuve de l'équivalence avec les automates finis
      \end{itemize}
    \item L'inclusion est stricte, car $\{a^n b^n \mid  n\in \mathbb{N} \} \in \textsc{alg}_\Sigma \setminus \textsc{rat}_\Sigma$
    \end{itemize}
  \end{block}

  \begin{exampleblock}{Exemple}
    Le langage \example{$a^\star b^\star$} est engendré par
    $$\example{\left\langle \{a, b\}, \{S, A\}, S, \left\{ \begin{array}{@{\,}r@{~\rightarrow~}l@{\,\mid\,}l@{\,}}
        S & aS & A\\
        A & bA & \varepsilon
      \end{array}\right\}\right\rangle}$$
  \end{exampleblock}
  
\end{frame}

\endgroup

%% SPDX-License-Identifier: CC-BY-SA-4.0
% Author: Matthieu Perrin
% Part: 
% Section: 
% Sub-section: 
% Frame: 

\begingroup

\begin{frame}{Grammaires linéaires et rationnelles droites}
  Soit \alert{$G = \langle \Sigma, \Gamma, S, R \rangle$} une grammaire algébrique. 

  \vspace{2mm}
  \begin{block}{Définition -- Grammaire linéaire droite}
    On dit que $G$ est \structure{linéaire droite} si tous les membres droits de ses règles
    contiennent au plus un non-terminal, situé à droite de tous les terminaux :

    {\small $$
      \structure{\forall r\in R, \exists a_1, ..., a_n \in \Sigma, \exists A, B \in \Gamma, r = \alert{A \rightarrow a_1 ... a_n B} \lor r = \alert{A \rightarrow a_1 ... a_n}}.
      $$}
  \end{block}
  
  \vspace{2mm}
  \begin{block}{Définition -- Grammaire rationnelle droite}
    On dit que $G$ est \structure{rationnelle droite} si $G$ est linéaire droite et tous les membres droits de ses règles
    contiennent un unique terminal, ou sont $\varepsilon$ :

    {\small $$
      \structure{\forall r\in R, \exists a \in \Sigma, \exists A, B \in \Gamma, r = \alert{A \rightarrow a B} \lor r = \alert{A \rightarrow a} \lor r = \alert{A \rightarrow \varepsilon}}.
      $$}
  \end{block}
\end{frame}

\endgroup

%% SPDX-License-Identifier: CC-BY-SA-4.0
% Author: Matthieu Perrin
% Part: 
% Section: 
% Sub-section: 
% Frame: 

\begingroup

\begin{frame}{Transformation en grammaire rationnelle}
  \begin{block}{Théorème}
    Toute grammaire \structure{linéaire droite} peut être transformée en une
    grammaire \structure{rationnelle droite} reconnaissant le même langage. 
  \end{block}
  
  \begin{block}{Démonstration}
    Appliquer successivement les règles suivantes :
    \begin{enumerate}
    \item Eliminer les règles \structure{$A \rightarrow B$} : remplacer $B$ par sa définition
    \item Remplacer les règles \structure{$A \rightarrow a_1 a_2 ... a_n B$} par \structure{$A \rightarrow a_1 C$} et \structure{$C \rightarrow a_2 ... a_n B$}
    \end{enumerate}
  \end{block}
  
  \begin{exampleblock}{Exemple}
    \vspace{-5mm}
    $$
    \left\{\begin{array}{lll}
    S &\rightarrow& abA \;|\; A \\
    A &\rightarrow& c\\
    \end{array}\right.
    \hspace{2mm}\Rightarrow\hspace{2mm}
    \left\{\begin{array}{lll}
    S &\rightarrow& abA \;|\; c \\
    A &\rightarrow& c\\
    \end{array}\right.
    \hspace{2mm}\Rightarrow\hspace{2mm}
    \left\{\begin{array}{lll}
    S &\rightarrow& aB \;|\; c \\
    A &\rightarrow& c\\
    B &\rightarrow& bA\\
    \end{array}\right.$$
  \end{exampleblock}
\end{frame}

\endgroup

%% SPDX-License-Identifier: CC-BY-SA-4.0
% Author: Matthieu Perrin
% Part: 
% Section: 
% Sub-section: 
% Frame: 

\begingroup

\begin{frame}{Équivalence et isomorphisme entre automates}

  \tf[text, top]{
    Soient $A = \langle \Sigma, Q_A, I_A, F_A, \rightarrow_A \rangle$ et $B = \langle \Sigma, Q_B, I_B, F_B, \rightarrow_B \rangle$ deux automates. 
  }

  \tfBlock[y=17mm]{Définition -- Automates équivalents}{
    $A$ et $B$ sont \structure{équivalents}, noté $\alert{A \equiv B}$, s'ils acceptent le même langage :
    $$\alert{A_1 \equiv A_2  \eqdef \mathcal{L}(A_1) = \mathcal{L}(A_2)}.$$
  }

  \tfBlock[bottom=2mm]{Définition -- Automates isomorphes}{
    $A$ et $B$ sont \structure{isomorphes}, noté $\alert{A \simeq B}$, s'ils ne diffèrent que par le nom de leurs états, \textit{c.-à-d.}
    s'il existe une fonction bijective $f : Q_A \rightarrow Q_B$ telle que :

    \vspace{2mm}
    $\structure{\forall a\in \Sigma, \forall q, q',}$\\
    $\structure{\begin{array}[t]{@{\quad\quad}lll}
        q\in Q_A &\Leftrightarrow& f(q) \in Q_B \\
        q\in I_A &\Leftrightarrow& f(q) \in I_B \\
        q\in F_A &\Leftrightarrow& f(q) \in F_B \\
        q \xrightarrow{a}_A q' &\Leftrightarrow& f(q) \xrightarrow{a}_B f(q')
      \end{array}
    }$
    \vspace{2mm}

    Deux automates isomorphes sont équivalents. 
  }

  \tf[bottom=2mm, x=3cm]{\small
    \begin{tikzpicture}[smAutomaton]
      \draw (1,0.75) node{\normalsize $\simeq$};

      \smState[\smStructure \smInitial]   (a0) at (0.0, 1.5) {$A$};
      \smState[\smStructure \smAccepting] (a1) at (0.0, 0.0) {$B$};
      \smState[\smExample \smInitial]     (b0) at (2.0, 1.5) {$0$};
      \smState[\smExample \smAccepting]   (b1) at (2.0, 0.0) {$1$};

      \smPath[\smStructure] (a0) edge[bend left]  node {$a$} (a1);
      \smPath[\smStructure] (a1) edge[bend left]  node {$a$} (a0);
      \smPath[\smStructure] (a1) edge[loop right] node {$b$} (a1);
      \smPath[\smStructure] (a0) edge[loop right] node {$b$} (a0);
      
      \smPath[\smExample]   (b0) edge[bend left]  node {$a$} (b1);
      \smPath[\smExample]   (b1) edge[bend left]  node {$a$} (b0);
      \smPath[\smExample]   (b1) edge[loop right] node {$b$} (b1);
      \smPath[\smExample]   (b0) edge[loop right] node {$b$} (b0);
    \end{tikzpicture}
  }
\end{frame}

\endgroup

%% SPDX-License-Identifier: CC-BY-SA-4.0
% Author: Matthieu Perrin
% Part: 
% Section: 
% Sub-section: 
% Frame: 

\begingroup

\begin{frame}{Grammaires linéaires et rationnelles gauches}
  Soit \alert{$G = \langle \Sigma, \Gamma, S, R \rangle$} une grammaire algébrique. 

  \begin{block}{Définition -- Grammaire linéaire gauche}
    On dit que $G$ est \structure{linéaire gauche} si tous les membres droits de ses règles
    contiennent au plus un non-terminal, situé à gauche de tous les terminaux :
    {\small $$
      \structure{\forall r\in R, \exists a_1, ..., a_n \in \Sigma, \exists A, B \in \Gamma, r = \alert{A \rightarrow B a_1 ... a_n} \lor r = \alert{A \rightarrow a_1 ... a_n}}.
      $$}
  \end{block}
  
  \vspace{-2mm}
  \begin{block}{Définition -- Grammaire rationnelle gauche}
    On dit que $G$ est \structure{rationnelle gauche} si $G$ est linéaire gauche et tous les membres droits de ses règles
    contiennent un unique terminal, ou sont $\varepsilon$ :
    {\small $$
      \structure{\forall r\in R, \exists a \in \Sigma, \exists A, B \in \Gamma, r = \alert{A \rightarrow B a} \lor r = \alert{A \rightarrow a} \lor r = \alert{A \rightarrow \varepsilon}}.
      $$}
  \end{block}
  
  \vspace{-2mm}
  \begin{block}{Théorème}
    Les grammaires rationnelles gauche génèrent les langages rationnels.
    
    \structure{preuve :} $\mathcal{L}(\langle \Sigma, \Gamma, S, \{ g \rightarrow d^{\textsc{r}} | g \rightarrow d \in R\} \rangle) = (\mathcal{L}(G))^{\textsc{r}}$.
  \end{block}
\end{frame}

\endgroup

%% SPDX-License-Identifier: CC-BY-SA-4.0
% Author: Matthieu Perrin
% Part: 
% Section: 
% Sub-section: 
% Frame: 

\begingroup

\begin{frame}{Formalismes décrivant les langages rationnels}

  ~
  \hspace{-8mm}\scalebox{.95}{\begin{tikzpicture}
      
      \draw[rounded corners, structure,fill=structure!20] (4,7) +(-1.2,-.5) rectangle +(1.2,.5) +(0,0) node{\small \begin{tabular}{c}Expression\\ rationnelle \end{tabular}};
      \draw[rounded corners, structure,fill=structure!20] (4,5) +(-1.2,-.5) rectangle +(1.2,.5) +(0,0) node{\small \begin{tabular}{c}Automate fini\\ non-déterministe \end{tabular}};
      \draw[rounded corners, structure,fill=structure!20] (8,5) +(-1.2,-.5) rectangle +(1.2,.5) +(0,0) node{\small \begin{tabular}{c}Automate fini\\ déterministe \end{tabular}};
      \draw[rounded corners, structure,fill=structure!20] (6,3) +(-1.2,-.5) rectangle +(1.2,.5) +(0,0) node{\small \begin{tabular}{c}Grammaire\\ rationnelle gauche \end{tabular}};
      \draw[rounded corners, structure,fill=structure!20] (2,3) +(-1.2,-.5) rectangle +(1.2,.5) +(0,0) node{\small \begin{tabular}{c}Grammaire\\ rationnelle droite \end{tabular}};
      \draw[rounded corners, structure,fill=structure!20] (6,1) +(-1.2,-.5) rectangle +(1.2,.5) +(0,0) node{\small \begin{tabular}{c}Grammaire\\ linéaire gauche \end{tabular}};
      \draw[rounded corners, structure,fill=structure!20] (2,1) +(-1.2,-.5) rectangle +(1.2,.5) +(0,0) node{\small \begin{tabular}{c}Grammaire\\ linéaire droite \end{tabular}};
      
      \draw[-latex] (3.9,6.5) -- (3.9,5.5);
      \draw[-latex] (4.1,5.5) -- (4.1,6.5);
      \draw (3.9,6) node[left] {\tiny Thompson};
      \draw (4.1,6) node[right]{\tiny Arden};

      \draw[-latex] (5.2,5.1) -- (6.8,5.1);
      \draw (6,5.3) node{\tiny Rabin \& Scott};
      \draw[-latex] (6.8,4.9) -- (5.2,4.9);
      \draw (6,4.7) node{\tiny inclusion};

      \draw[-latex] (1.9,2.5) -- (1.9,1.5);
      \draw[-latex] (2.1,1.5) -- (2.1,2.5);
      \draw (1.9,2) node[left] {\tiny inclusion};
      \draw (2.1,2) node[right]{\tiny direct};

      \draw[-latex] (5.9,2.5) -- (5.9,1.5);
      \draw[-latex] (6.1,1.5) -- (6.1,2.5);
      \draw (5.9,2) node[left] {\tiny inclusion};
      \draw (6.1,2) node[right]{\tiny direct};


      \draw[latex-latex] (4.2,4.5) -- (5.9,3.5);
      \draw (3,4) node[right]{\tiny direct};

      \draw[latex-latex] (3.8,4.5) -- (2.1,3.5);
      \draw (5,4) node[left]{\tiny miroir};

      \draw[example] (2.3,7) node{$a^+ b$};
      \draw (0,5.35) node{\begin{tikzpicture}[shorten >=1pt,node distance=1.5cm,on grid,auto]
          \node[example,fill=example!10,state,initial, initial text=] (a) {$A$};
          \node[example,fill=example!10,state] (b) [right=of a] {$B$};
          \node[example,fill=example!10,state, accepting] (c) [right=of b] {$C$};
          \path[example,->] (a) edge [loop above, looseness=5] node {$a$} (a);
          \path[example,->] (a) edge node[above] {$a$} (b);
          \path[example,->] (b) edge node[above] {$b$} (c);
      \end{tikzpicture}};
      \draw (7.8,6.5) node{\begin{tikzpicture}[shorten >=1pt,node distance=1.5cm,on grid,auto]
          \node[example,fill=example!10,state,initial, initial text=] (a) {$A$};
          \node[example,fill=example!10,state] (b) [right=of a] {$B$};
          \node[example,fill=example!10,state, accepting] (c) [right=of b] {$C$};
          \path[example,->] (a) edge node[above] {$a$} (b);
          \path[example,->] (b) edge [loop above, looseness=5] node {$a$} (b);
          \path[example,->] (b) edge node[above] {$b$} (c);
      \end{tikzpicture}};

      \draw[example] (0.8,3) node[left]{$\left\{\begin{array}{rcl}
        S & \rightarrow & aS | aB\\
        B & \rightarrow & b\\
        \end{array}\right.$};
      
      \draw[example] (0.8,1) node[left]{$\left\{\begin{array}{rcl}
        S & \rightarrow & aS | ab\\
        \end{array}\right.$};

      \draw[example] (7.2,3) node[right]{$\left\{\begin{array}{rcl}
        S & \rightarrow & Bb\\
        B & \rightarrow & a | Ba\\
        \end{array}\right.$};
      
      \draw[example] (7.2,1) node[right]{$\left\{\begin{array}{rcl}
        S & \rightarrow & ab | Bab\\
        B & \rightarrow & Ba | \varepsilon\\
        \end{array}\right.$};

  \end{tikzpicture}}
\end{frame}
\endgroup

%% SPDX-License-Identifier: CC-BY-SA-4.0
% Author: Matthieu Perrin
% Part: 
% Section: 
% Sub-section: 
% Frame: 

\begingroup

\begin{frame}{Généralisation : degré d'une grammaire}
  
  Soit $G = \langle \Sigma, \Gamma, S, R \rangle$ une grammaire algébrique.\\
  \begin{block}{Définition -- Degré d'une grammaire}
    Le \structure{degré} de $G$, noté $\alert{d(G)}$,
    est le nombre maximal de non-terminaux dans le membre droit d'une règle de $R$. 
  \end{block}

  \begin{exampleblock}{Exemples}
    \begin{enumerate}
    \item Les grammaires de \alert{degré 1} sont appelées \structure{linéaires}
      \begin{description}
      \item[Attention] \alert{$S \rightarrow aSb \mid \varepsilon$} est linéaire, mais ni à droite ni à gauche
      \end{description}
    \item Les grammaires de \alert{degré 2} sont appelées \structure{quadratiques}
      \begin{description}
      \item[Théorème] Toute grammaire algébrique peut être transformée en une grammaire quadratique équivalente 
        %      \item[Idée] Remplacer \structure{$N \rightarrow A B C$} par \structure{$N \rightarrow A X$} et \structure{$X \rightarrow B C$}
      \item[Preuve] La forme normale de Chomsky en est un exemple
      \end{description}
    \end{enumerate}
  \end{exampleblock}
\end{frame}


\endgroup

% 
%\subsection{Limites des langages algébriques}
%% SPDX-License-Identifier: CC-BY-SA-4.0
% Author: Matthieu Perrin
% Part: 
% Section: 
% Sub-section: 
% Frame: 

\begingroup

\begin{frame}{Lemme de pompage}

  \vspace{-5mm}
  \begin{block}{Question -- Expressivité des grammaires algébriques}
    \vspace{-2mm}
    \begin{itemize}
    \item Comment montrer qu'un langage $L$ n'est pas algébrique ? 
      \begin{enumerate}
      \item Si $L$ était algébrique, il vérifierait le \structure{lemme de pompage} 
      \item Or, $L$ ne vérifie pas le \structure{lemme de pompage} 
      \item Donc, par l'absurde, $L$ ne peut pas être algébrique
      \end{enumerate}
    \end{itemize}
  \end{block}
  
  \vspace{-1mm}
  \begin{block}{Rappel -- Lemme de l'étoile}
    \vspace{-1mm}
    $\forall\Sigma, \forall L\in \textsc{rat}_\Sigma, \exists N\in \mathbb{N}, \forall u\in L, |u| \ge N \Rightarrow \exists \structure{x, y, z}\in \Sigma^\star,$

    \vspace{1mm}
    \begin{minipage}{.4\textwidth}
      \begin{itemize}
      \item $u = \structure{x\cdot y \cdot z}$
      \item $\structure{y}\neq \varepsilon$
      \end{itemize}
    \end{minipage}%
    \begin{minipage}{.5\textwidth}
      \begin{itemize}
      \item $|\structure{x\cdot y}| \le N$
      \item $\forall i \in \mathbb{N}, x\cdot \structure{y^i} \cdot z \in L$
      \end{itemize}
    \end{minipage}
  \end{block}

  \vspace{-1mm}
  \begin{alertblock}{Lemme -- Lemme de pompage (ou d'itération)\footnote[frame, 1]{\scriptsize Y. Bar-Hillel, M. Perles, E. Shamir. \textit{On formal properties of simple phrase structure grammars.} 1961}}
    \vspace{-1mm}
    $\forall\Sigma, \forall L\in \textsc{alg}_\Sigma, \exists N\in \mathbb{N}, \forall u\in L, |u| \ge N \Rightarrow \exists \alert{v, w, x, y, z}\in \Sigma^\star,$
    
    \vspace{1mm}
    \begin{minipage}{.4\textwidth}
      \begin{itemize}
      \item $\alert{u = v\cdot w\cdot x \cdot y \cdot z}$
      \item $\alert{w\cdot y} \neq \varepsilon$
      \end{itemize}
    \end{minipage}%
    \begin{minipage}{.5\textwidth}
      \begin{itemize}
      \item $|\alert{w\cdot x \cdot y}| \le N$
      \item $\forall i \in \mathbb{N}, v\cdot \alert{w^i} \cdot x \cdot \alert{y^i} \cdot z \in L$
      \end{itemize}
    \end{minipage}
  \end{alertblock}

  \begin{description}
  \item[Remarque :] Si $L$ est rationnel, on peut avoir $v = w = \varepsilon$.
  \end{description}

\end{frame}

\endgroup

%% SPDX-License-Identifier: CC-BY-SA-4.0
% Author: Matthieu Perrin
% Part: 
% Section: 
% Sub-section: 
% Frame: 

\begingroup

\begin{frame}{Démonstration du lemme de l'étoile}

  \vspace{-2mm}  
  \onlyb<-3>{\hspace\fill\example{\textbf{Exemple}}}%
  \only<4->{\hspace\fill\alert{\textbf{Lemme de l'étoile}}}

  Soient $\Sigma$ un alphabet, et $L$ un langage rationnel sur $\Sigma$.
  \onlyb<-3>{\hspace\fill\example{$L = a(bc)^\star a$}}
  \only<4->{\hspace\fill\alert{$\forall \Sigma, \forall L\in \textsc{rat}_\Sigma,$}}

  Soit $A$ son automate minimal. Posons $N = |A|$.  
  \onlyb<-3>{\hspace\fill\example{$N=4$}}
  \only<4->{\hspace\fill\alert{$\exists N\in \mathbb{N},$}}
  
  Soit $u \in L$ tel que $|u| \ge N$. 
  \onlyb<-3>{\hspace\fill\example{$u = abca$}}
  \only<4->{\hspace\fill\alert{$\forall u\in L, |u| \ge N \Rightarrow ($}}

  \begin{center}
    \begin{tikzpicture}[automaton, y=5mm]
      \state[initial]   (0) at (0, 2) {$0$};
      \state[alert]     (1) at (1, 1) {$1$};
      \state[example]   (2) at (2, 1) {$2$};
      \state[accepting] (3) at (0, 0) {$3$};

      \path[structure]  (0) edge node {$a$} (1);
      \path[example]    (1) edge[bend left] node {$b$} (2);
      \path[example]    (2) edge[bend left] node {$c$} (1);
      \path[structure]  (1) edge node {$a$} (3);
    \end{tikzpicture}

    \pause
    \structure{Pour reconnaître un mot de 4 lettres ou plus, il faut suivre une boucle}
  \end{center}

  Il existe un état $q$ que l'on visite deux fois. Posons :
  \onlyb<-3>{\hspace\fill\example{$q = 1$}}%
  
  \begin{itemize}
  \item $x$ le préfixe de $u$ avant la première visite de $q$
    \onlyb<-3>{\hspace\fill\example{$x = a$}}
    \only<4-|handout>{\hspace\fill\alert{$\exists x, y, z \in \Sigma^\star,$}}
  \item $y$ le facteur de $u$ entre les deux visites de $q$
    \onlyb<-3>{\hspace\fill\example{$y = bc$}}
    \only<4-|handout>{\hspace\fill\alert{$y \neq \varepsilon$}}
  \item $z$ le suffixe de $u$ après la deuxième visite de $q$
    \onlyb<-3>{\hspace\fill\example{$z = a$}}
    \only<4-|handout>{\hspace\fill\alert{$\land\; u = x\cdot y \cdot z$}}
  \end{itemize}

  \pause

  \begin{center}\structure{Si on suit une boucle une fois, on peut la suivre plusieurs fois}\end{center}

  On a :
  \begin{itemize}
  \item On peut prendre la boucle $0$, $2$, $3$, ... fois
    \onlyb<-3>{\hspace\fill\example{$aa \in L, abcbca\in L$}}
    \only<4-|handout>{\hspace\fill\alert{$\land\; \forall i\in \mathbb{N}, x \cdot y^i \cdot z \in L$}}
  \item $N$ lettres suffisent pour repasser dans un état
    \onlyb<-3>{\hspace\fill\example{$|abc| \le 4$}}
    \only<4-|handout>{\hspace\fill\alert{$\land\; |x\cdot y| \le N)$}}
  \end{itemize}

  \phantom{.}
\end{frame}


\endgroup

%% SPDX-License-Identifier: CC-BY-SA-4.0
% Author: Matthieu Perrin
% Part: 
% Section: 
% Sub-section: 
% Frame: 

\begingroup

\begin{frame}{Exemple d'utilisation du lemme de pompage}
  
  \tfBlock[top=-5mm]{Montrer que $L \eqdef \{a^nb^nc^n \mid n\in \mathbb{N}\}$ n'est pas algébrique}{
    Soit $\Sigma \eqdef \{a, b, c\}$.%
    \only<2-|handout>{
      Si $L$ est algébrique, $L$ vérifie le lemme de pompage :

      \vspace{-4mm}
      $$
      \begin{array}{c}
        \structure{\exists N\in \mathbb{N}}, \alert{\forall u\in L, |u| \ge N} \Rightarrow (\structure{\exists v, w, x, y, z\in \Sigma^\star}, \\
        u = v \cdot w \cdot x\cdot y \cdot z \land w \cdot y\neq \varepsilon \land |w\cdot x\cdot y| \le N \land \alert{\forall i \in \mathbb{N}}, v\cdot w^i\cdot x\cdot y^i \cdot z \in L)
      \end{array}
      $$
      \vspace{-2mm}
      
      \structure{Soit $N$ donné par le lemme pompage}.
    }
    
    \only<3-|handout>{%
      \alert{Posons $u = a^N b^N c^N$. On a bien $u\in L$ et $|u| = 3N \ge N$}.\\
      \structure{Soit $v \cdot w \cdot x\cdot y \cdot z$ la décomposition de $u$ donnée par le lemme de pompage}. 
    }
  }
  
  \tf<4-|handout>[y=-10mm]{
    \begin{smArray}[width=4mm, height=3mm, name={$u=$}]
      \smCell[\smNone]{\alert{$a$}}      \smCoord{(a1)}
      \smCell[\smNone]{\alert{$\cdots$}}               
      \smCell[\smNone]{\alert{$a$}}      \smCoord{(an)}
      \smCell[\smNone]{\alert{$\cdot$}}               
      \smCell[\smNone]{\alert{$b$}}      \smCoord{(b1)}
      \smCell[\smNone]{\alert{$\cdots$}}               
      \smCell[\smNone]{\alert{$b$}}      \smCoord{(bn)}
      \smCell[\smNone]{\alert{$\cdot$}}               
      \smCell[\smNone]{\alert{$c$}}      \smCoord{(c1)}
      \smCell[\smNone]{\alert{$\cdots$}}               
      \smCell[\smNone]{\alert{$c$}}      \smCoord{(cn)}
      
      \draw [decorate, decoration={brace, amplitude=5pt}] ([xshift=1mm]a1.north west) -- ([xshift=-1mm]an.north east) node[midway,yshift=4mm]{$N$};
      \draw [decorate, decoration={brace, amplitude=5pt}] ([xshift=1mm]b1.north west) -- ([xshift=-1mm]bn.north east) node[midway,yshift=4mm]{$N$};
      \draw [decorate, decoration={brace, amplitude=5pt}] ([xshift=1mm]c1.north west) -- ([xshift=-1mm]cn.north east) node[midway,yshift=4mm]{$N$};

      \draw [decorate, decoration={brace, amplitude=5pt, mirror}] ([xshift=1mm]an.south west) -- ([xshift=-1mm]b1.south east) node[midway,yshift=-4mm]{$|wxy|\le N$};
      \draw [decorate, decoration={brace, amplitude=5pt, mirror}] ([xshift=1mm]bn.south west) -- ([xshift=-1mm]c1.south east) node[midway,yshift=-4mm]{$|wxy|\le N$};
    \end{smArray}       
  }

  \tf<5-|handout>[text, bottom=-1mm]{
    \begin{itemize}
    \item Comme $wy \neq \varepsilon$, $\alpha = wy[1] \in \Sigma$ est une lettre de $wy$.
    \item Comme $|wxy| \le N$, il existe $\beta \in \Sigma$ tel que $\beta$ n'est pas une lettre de $wy$. 
    \item \alert{Posons $i=2$}. $|v \cdot w^2 \cdot x\cdot  y^2\cdot   z|_\alpha > |v\cdot  w^2\cdot  x\cdot  y^2\cdot   z|_\beta$, donc $v\cdot  w^2\cdot  x\cdot  y^2\cdot   z\notin L$. 
    \end{itemize}
    Absurde ! Donc $L$ n'est pas algébrique. 
  }

  \tfExampleBlock<-4>[y=-25mm]{On sait}{}

  \tf<1>[y=-25mm, anchor=north, text]{
    \begin{itemize}
    \item $\begin{array}[t]{l}
      \alert{\forall L \in \textsc{alg}_\Sigma}, \structure{\exists N\in \mathbb{N}}, \forall u\in L, |u| \ge N \Rightarrow (\exists v, w, x, y, z\in \Sigma^\star, \\
      u = v w x y z \land w y\neq \varepsilon \land |wxy| \le N \land \forall i \in \mathbb{N}, v w^i x y^i z \in L)
    \end{array}$
    \end{itemize}
  }

  \tf<2-4>[y=-25mm, anchor=north, left=.35\textwidth]{
    \begin{itemize}
    \item $L \in \textsc{alg}_\Sigma$
    \item<3-> $v, w, x, y, z\in \Sigma^\star$
    \end{itemize}
  }
  
  \tf<2-4>[y=-25mm, anchor=north, width=.35\textwidth]{
    \begin{itemize}
    \item $N \in \mathbb{N}$
    \item<3-> $wy\neq \varepsilon$
    \end{itemize}
  }
  
  \tf<3-4>[y=-25mm, anchor=north, right=.35\textwidth]{
    \begin{itemize}
    \item $v \cdot w \cdot x \cdot y \cdot z = a^N b^N c^N$
    \item $\alert{|wxy| \le N}$
    \end{itemize}
  }
  
\end{frame}

\endgroup

% 
%\section{Hiérarchie de Chomsky}
% 
%\subsection{Grammaires non-restreintes}
%% SPDX-License-Identifier: CC-BY-SA-4.0
% Author: Matthieu Perrin
% Part: 
% Section: 
% Sub-section: 
% Frame: 

\begingroup

\begin{frame}{Reconnaissance d'un langage quelconque}
  \begin{block}{Algorithme de recherche ascendante par force brute} 

    \begin{description}
    \item [Entrées :]
      \begin{itemize}
      \item Une grammaire $G$, si possible contextuelle
      \item Un mot $u$
      \end{itemize}

    \item [Sortie :] une réponse booléenne sur \structure{$u \in \mathcal{L}(G)$}
      %  Certaines (si réponse oui) / toutes (si réponse non) les listes de symboles terminaux/non-terminaux dérivant $m$

    \item [Teminaison :] garantie si $G$ est contextuelle
    \item [Complexité :] exponentielle par rapport à $|u|$
    \end{description}
  \end{block}
\end{frame}

\endgroup

%% SPDX-License-Identifier: CC-BY-SA-4.0
% Author: Matthieu Perrin
% Part: 
% Section: 
% Sub-section: 
% Frame: 

\begingroup


\begin{frame}{Généralisation du concept de grammaire}
  \begin{block}{Définition -- Grammaire non-restreinte}
    \vspace{2mm}
    Une \structure{grammaire non-restreinte} est un quadruplet \alert{$\langle \Sigma, \Gamma, S, R \rangle$} tel que :
    \begin{description}
    \item[\alert{$\Sigma$}] alphabet : \structure{l'ensemble des terminaux}
    \item[\alert{$\Gamma$}] alphabet tel que $\Sigma \cap \Gamma = \emptyset$ : \structure{l'ensemble des non-terminaux}
    \item[\alert{$S$}] $\in \Gamma$ : \structure{l'axiome} (ou symbole initial)
    \item[\alert{$R$}] $\subseteq ((\Sigma \cup \Gamma)^\star \cdot \Gamma \cdot (\Sigma \cup \Gamma)^\star) \times (\Sigma \cup \Gamma)^\star$ : \\\structure{l'ensemble des règles de production}
    \end{description}

    \vspace{3mm}
    Une \structure{règle de production} est un couple \alert{$\langle \alpha, \beta \rangle \in R$}, noté $\alert{\alpha \rightarrow \beta}$ tel que :
    \begin{description}
    \item[\alert{$\alpha$}] $\in (\Sigma \cup \Gamma)^\star \cdot \Gamma \cdot (\Sigma \cup \Gamma)^\star$ : \structure{membre gauche}
    \item[\alert{$\beta$}] $\in (\Sigma \cup \Gamma)^\star$ : \structure{membre droit}
    \end{description}
  \end{block}
  \begin{block}{Remarque -- Grammaire algébrique}
    Une grammaire $\langle \Sigma, \Gamma, S, R \rangle$ est algébrique si, et seulement si,
    
    \vspace{-3mm}  $$\forall \alpha \rightarrow \beta \in R, \alpha \in \Gamma^1.$$
  \end{block}
\end{frame}

\endgroup

%% SPDX-License-Identifier: CC-BY-SA-4.0
% Author: Matthieu Perrin
% Part: 
% Section: 
% Sub-section: 
% Frame: 

\begingroup

\begin{frame}{Langage reconnu par un AFN}

  \on[text, top]{
    Soit $A=\langle \Sigma, Q, I, F, \rightarrow \rangle$ un AFN.

    \begin{block}{Définition -- langage reconnu (ou accepté)}
      Un mot $u$ est \structure{reconnu} par $A$ s'il mène d'un état initial à un état final
      $$\exists i\in I, \exists f\in F,~  \alert{\langle u, i \rangle \leadsto_A^\star \langle \varepsilon, f\rangle}$$
      Le langage \structure{reconnu} par $A$ est l'ensemble $\alert{\mathcal{L}(A)}$ des mots reconnus par $A$
      $$\alert{\mathcal{L}(A) \eqdef \left\{u \in \Sigma^\star \,\middle\mid\, \exists i\in I, \exists f\in F,  \langle u, i \rangle \leadsto_A^\star \langle \varepsilon, f\rangle\right\}}$$
    \end{block}
  }

  \onExampleBlock[bottom]{Exemple}{
    Pour \example{$i=3$}, \example{$f=2$} et \example{$u=bbc$}, on a : 
    \begin{enumerate}
    \item $\langle bbc, 1 \rangle \leadsto^\star \langle \varepsilon, 2 \rangle$
    \item $1 \in I$
    \item $2 \in F$
    \end{enumerate}
    Donc \example{$bbc \in \mathcal{L}(A)$}
  }

  \on[bottom, x=.25\textwidth]{
    \begin{tikzpicture}[automaton, grid size=15mm]
      \state[initial]       (1) at (0,1) {$1$}; 
      \state[accepting]     (2) at (1,1) {$2$}; 
      \state[initial above] (3) at (0,0) {$3$}; 
      \state                (4) at (1,0) {$4$}; 

      \path (1) edge             node       {$a$}           (2);
      \path (1) edge[loop above] node       {$a$}           (1);
      \path (3) edge             node[swap] {$b$}           (4);
      \path (3) edge[loop left ] node       {$b$}           (3);
      \path (4) edge[loop right] node       {$b$}           (4);
      \path (2) edge[loop above] node       {$c$}           (2);
      \path (4) edge             node[swap] {$\varepsilon$} (2);
    \end{tikzpicture}
  }
  
\end{frame}


\endgroup

%% SPDX-License-Identifier: CC-BY-SA-4.0
% Author: Matthieu Perrin
% Part: 
% Section: 
% Sub-section: 
% Frame: 

\begingroup

\begin{frame}{Graphe de dérivation}

  Soit \alert{$G = \langle \Sigma, \Gamma, S, R \rangle$} une grammaire non-restreinte.

  \begin{block}{Graphe de dérivation}
    En général, une dérivation ne peut plus être représentée par un arbre.

    Dans ce cas, on représente parfois une génération par un graphe. 
  \end{block}

  \begin{exampleblock}{Exemple}

      \vspace{3mm}
      \example{$G = \left\langle \{a, b, c\}, \{S, B\}, S, \left\{\begin{array}{rcl} S &\rightarrow & bS \,|\, bA \\ bA &\rightarrow & aa\end{array}\right\} \right\rangle$}.\\


      \vspace{5mm}
      Génération $S \vdash bS \vdash bbA \vdash baa$ :


      \vspace{-10mm}
      \hspace\fill
      \begin{tikzpicture}
          \draw[rounded corners, example, fill=example!20] (4.7,2.7) rectangle (7.3,3.3);
          
          \draw(5.0,5) node{$S$};
          \draw(4.0,4) node{$b$};
          \draw(6.0,4) node{$S$};
          \draw(5.0,3) node{$b$};
          \draw(7.0,3) node{$A$};
          \draw(5.0,2) node{$a$};
          \draw(7.0,2) node{$a$};

          \draw[-latex] (4.8,4.8) -- (4.2,4.2);
          \draw[-latex] (5.2,4.8) -- (5.8,4.2);

          \draw[-latex] (5.8,3.8) -- (5.2,3.2);
          \draw[-latex] (6.2,3.8) -- (6.8,3.2);

          \draw[-latex] (5.2,2.8) -- (6.8,2.2);
          \draw[-latex] (6.8,2.8) -- (5.2,2.2);

          \draw[-latex] (5.0,2.8) -- (5.0,2.2);
          \draw[-latex] (7.0,2.8) -- (7.0,2.2);
          
      \end{tikzpicture}
  \end{exampleblock}
\end{frame}

\endgroup

% 
%\subsection{Langages contextuels}
%% SPDX-License-Identifier: CC-BY-SA-4.0
% Author: Matthieu Perrin
% Part: 
% Section: 
% Sub-section: 
% Frame: 

\begingroup

\begin{frame}{Définition des langages rationnels}
  Soient $\Sigma$ un alphabet, et $L \in \mathscr{P}(\Sigma^\star)$ un langage sur $\Sigma$.

  \begin{block}{Définition --- Langage rationnel}
    $L$ est \structure{rationnel}, noté $\alert{L\in \textsc{rat}_\Sigma}$ s'il est décrit par une expression rationnelle :
    $$\alert{\textsc{rat}_\Sigma = \{\mathcal{S}(u) \mid u \in \textsc{regex}_\Sigma\}}$$
  \end{block}

  \begin{exampleblock}{Conséquences}
    \vspace{2mm}
    \begin{tabular}{rlr}
      \vspace{1mm}\example{Langage vide}& &\alert{$\emptyset \in \textsc{rat}_\Sigma$}\\
      \vspace{1mm}\example{Mot vide}& &\alert{$\{ \varepsilon \} \in \textsc{rat}_\Sigma$}\\
      \vspace{1mm}\example{Caractères}& $\forall a\in \Sigma$&\alert{$ \{ a \} \in \textsc{rat}_\Sigma$}\\
      \vspace{1mm}\example{Union}& $\forall L_1, L_2 \in \textsc{rat}_\Sigma$&\alert{$ L_1 \cup L_2 \in \textsc{rat}_\Sigma$}\\
      \vspace{1mm}\example{Produit}& $\forall L_1, L_2 \in \textsc{rat}_\Sigma$&\alert{$ L_1 \cdot L_2 \in \textsc{rat}_\Sigma$}\\
      \vspace{1mm}\example{Fermeture}& $\forall L_1 \in \textsc{rat}_\Sigma$&\alert{$ L_1^\star \in \textsc{rat}_\Sigma$}\\
    \end{tabular}
  \end{exampleblock}
\end{frame}

\endgroup

%% SPDX-License-Identifier: CC-BY-SA-4.0
% Author: Matthieu Perrin
% Part: 
% Section: 
% Sub-section: 
% Frame: 

\begingroup

\begin{frame}{Exemple d'utilisation du lemme de pompage}
  
  \tfBlock[top=-5mm]{Montrer que $L \eqdef \{a^nb^nc^n \mid n\in \mathbb{N}\}$ n'est pas algébrique}{
    Soit $\Sigma \eqdef \{a, b, c\}$.%
    \only<2-|handout>{
      Si $L$ est algébrique, $L$ vérifie le lemme de pompage :

      \vspace{-4mm}
      $$
      \begin{array}{c}
        \structure{\exists N\in \mathbb{N}}, \alert{\forall u\in L, |u| \ge N} \Rightarrow (\structure{\exists v, w, x, y, z\in \Sigma^\star}, \\
        u = v \cdot w \cdot x\cdot y \cdot z \land w \cdot y\neq \varepsilon \land |w\cdot x\cdot y| \le N \land \alert{\forall i \in \mathbb{N}}, v\cdot w^i\cdot x\cdot y^i \cdot z \in L)
      \end{array}
      $$
      \vspace{-2mm}
      
      \structure{Soit $N$ donné par le lemme pompage}.
    }
    
    \only<3-|handout>{%
      \alert{Posons $u = a^N b^N c^N$. On a bien $u\in L$ et $|u| = 3N \ge N$}.\\
      \structure{Soit $v \cdot w \cdot x\cdot y \cdot z$ la décomposition de $u$ donnée par le lemme de pompage}. 
    }
  }
  
  \tf<4-|handout>[y=-10mm]{
    \begin{smArray}[width=4mm, height=3mm, name={$u=$}]
      \smCell[\smNone]{\alert{$a$}}      \smCoord{(a1)}
      \smCell[\smNone]{\alert{$\cdots$}}               
      \smCell[\smNone]{\alert{$a$}}      \smCoord{(an)}
      \smCell[\smNone]{\alert{$\cdot$}}               
      \smCell[\smNone]{\alert{$b$}}      \smCoord{(b1)}
      \smCell[\smNone]{\alert{$\cdots$}}               
      \smCell[\smNone]{\alert{$b$}}      \smCoord{(bn)}
      \smCell[\smNone]{\alert{$\cdot$}}               
      \smCell[\smNone]{\alert{$c$}}      \smCoord{(c1)}
      \smCell[\smNone]{\alert{$\cdots$}}               
      \smCell[\smNone]{\alert{$c$}}      \smCoord{(cn)}
      
      \draw [decorate, decoration={brace, amplitude=5pt}] ([xshift=1mm]a1.north west) -- ([xshift=-1mm]an.north east) node[midway,yshift=4mm]{$N$};
      \draw [decorate, decoration={brace, amplitude=5pt}] ([xshift=1mm]b1.north west) -- ([xshift=-1mm]bn.north east) node[midway,yshift=4mm]{$N$};
      \draw [decorate, decoration={brace, amplitude=5pt}] ([xshift=1mm]c1.north west) -- ([xshift=-1mm]cn.north east) node[midway,yshift=4mm]{$N$};

      \draw [decorate, decoration={brace, amplitude=5pt, mirror}] ([xshift=1mm]an.south west) -- ([xshift=-1mm]b1.south east) node[midway,yshift=-4mm]{$|wxy|\le N$};
      \draw [decorate, decoration={brace, amplitude=5pt, mirror}] ([xshift=1mm]bn.south west) -- ([xshift=-1mm]c1.south east) node[midway,yshift=-4mm]{$|wxy|\le N$};
    \end{smArray}       
  }

  \tf<5-|handout>[text, bottom=-1mm]{
    \begin{itemize}
    \item Comme $wy \neq \varepsilon$, $\alpha = wy[1] \in \Sigma$ est une lettre de $wy$.
    \item Comme $|wxy| \le N$, il existe $\beta \in \Sigma$ tel que $\beta$ n'est pas une lettre de $wy$. 
    \item \alert{Posons $i=2$}. $|v \cdot w^2 \cdot x\cdot  y^2\cdot   z|_\alpha > |v\cdot  w^2\cdot  x\cdot  y^2\cdot   z|_\beta$, donc $v\cdot  w^2\cdot  x\cdot  y^2\cdot   z\notin L$. 
    \end{itemize}
    Absurde ! Donc $L$ n'est pas algébrique. 
  }

  \tfExampleBlock<-4>[y=-25mm]{On sait}{}

  \tf<1>[y=-25mm, anchor=north, text]{
    \begin{itemize}
    \item $\begin{array}[t]{l}
      \alert{\forall L \in \textsc{alg}_\Sigma}, \structure{\exists N\in \mathbb{N}}, \forall u\in L, |u| \ge N \Rightarrow (\exists v, w, x, y, z\in \Sigma^\star, \\
      u = v w x y z \land w y\neq \varepsilon \land |wxy| \le N \land \forall i \in \mathbb{N}, v w^i x y^i z \in L)
    \end{array}$
    \end{itemize}
  }

  \tf<2-4>[y=-25mm, anchor=north, left=.35\textwidth]{
    \begin{itemize}
    \item $L \in \textsc{alg}_\Sigma$
    \item<3-> $v, w, x, y, z\in \Sigma^\star$
    \end{itemize}
  }
  
  \tf<2-4>[y=-25mm, anchor=north, width=.35\textwidth]{
    \begin{itemize}
    \item $N \in \mathbb{N}$
    \item<3-> $wy\neq \varepsilon$
    \end{itemize}
  }
  
  \tf<3-4>[y=-25mm, anchor=north, right=.35\textwidth]{
    \begin{itemize}
    \item $v \cdot w \cdot x \cdot y \cdot z = a^N b^N c^N$
    \item $\alert{|wxy| \le N}$
    \end{itemize}
  }
  
\end{frame}

\endgroup

%% SPDX-License-Identifier: CC-BY-SA-4.0
% Author: Matthieu Perrin
% Part: 
% Section: 
% Sub-section: 
% Frame: 

\begingroup


\begin{frame}{Lien entre grammaire algébrique et contextuelle}
  \begin{block}{Remarque}
    \begin{itemize}
    \item Une règle $g \cdot N \cdot d \rightarrow g \cdot \alpha \cdot d$ peut être interprétée comme :\\ \og \structure{$N \rightarrow \alpha$, sous la restriction d'un contexte gauche et/ou droit} \fg
    \item Les grammaires algébriques sont souvent appelées \structure{hors contexte} (\textit{context-free}), car $g = d = \varepsilon$ pour toute règle
    \end{itemize}
  \end{block}

  \begin{block}{Théorème -- non-inclusion des grammaires}
    Il existe des grammaires algébriques qui ne sont pas contextuelles. \\
    \example{Contre-exemple} : $\left\langle \{a\}, \{S, A\}, S, \left\{\begin{array}{rcl} S &\rightarrow & aA \\ A &\rightarrow & b\alert{S} \,|\, \alert{\varepsilon} \end{array}\right\} \right\rangle$.
  \end{block}

  \begin{block}{Théorème -- inclusion des langages}
    Tout langage algébrique est contextuel.\\
    \structure{Preuve} : Toute grammaire en forme normale de Chomsky est contextuelle. 
  \end{block}
\end{frame}

\endgroup

% 
%\subsection{Décision d'un langage contextuel}
%% SPDX-License-Identifier: CC-BY-SA-4.0
% Author: Matthieu Perrin
% Part: 
% Section: 
% Sub-section: 
% Frame: 

\begingroup

\begin{frame}{Reconnaissance d'un langage quelconque}
  \begin{block}{Algorithme de recherche ascendante par force brute} 

    \begin{description}
    \item [Entrées :]
      \begin{itemize}
      \item Une grammaire $G$, si possible contextuelle
      \item Un mot $u$
      \end{itemize}

    \item [Sortie :] une réponse booléenne sur \structure{$u \in \mathcal{L}(G)$}
      %  Certaines (si réponse oui) / toutes (si réponse non) les listes de symboles terminaux/non-terminaux dérivant $m$

    \item [Teminaison :] garantie si $G$ est contextuelle
    \item [Complexité :] exponentielle par rapport à $|u|$
    \end{description}
  \end{block}
\end{frame}

\endgroup

%% SPDX-License-Identifier: CC-BY-SA-4.0
% Author: Matthieu Perrin
% Part: 
% Section: 
% Sub-section: 
% Frame: 

\begingroup


\begin{frame}[fragile]{Recherche ascendante par force brute}
  
  \small
  \begin{tikzpicture}
    
    \fill<1>[rounded corners, alert!20] (-4.9,10.52) rectangle (-3.1,10.92);
    \fill<2,3>[rounded corners, alert!20] (-4.4,9.75) rectangle (2.1,10.15);
    \fill<4>[rounded corners, alert!20] (-1.9,10.15) rectangle (-0.5,10.55);
    \fill<5>[rounded corners, alert!20] (-3.7,10.15) rectangle (-2.2,10.55);
    \fill<4,5>[rounded corners, alert!20] (-3.5,8.85) rectangle (-2,9.25);
    
    \draw (0,10) node{\begin{minipage}{\textwidth}\begin{algorithm}[H]
          \SetKwFunction{Forcebrute}{force\_brute}
          \SetKwData{Old}{old}
          \SetKwData{New}{etape}
          \SetKwData{Vu}{union}
          \Fn{\Forcebrute($G = \langle \Sigma, \Gamma, S, R \rangle$ : grammaire, $u$ : mot) : booléen}{
            $\New \leftarrow \{u\}$;
            $\Vu \leftarrow \New$\;
            \Tantque{$\New \neq \emptyset \land S\notin \New$}{
              $\New \leftarrow \{ v\in \Sigma^\star \,|\, \exists w \in \New, v \vdash w \} \setminus \Vu$\;
              $\Vu \leftarrow \Vu \cup \New$\;
            }
            \Retourner $S \in \New$\;
          }
    \end{algorithm}\end{minipage}};

    \draw<-5> (0,7.8) node{\begin{minipage}{\textwidth}\begin{exampleblock}{$abc \in L$}
          {\footnotesize
            $\begin{array}{lllllll}
              \alert<1>{\{\alert<2>{ab}\structure<2>{c}\}} \uncover<2->{&\leftarrow & \{\alert<2>{aB}\structure<3>{c}, \alert<3>{ab}\structure<2>{C}\}}\uncover<3->{&\leftarrow& \{\alert<3>{aB}\structure<3>{C}\}} \\
              \uncover<4->{& \leftarrow & \{aBX\} & \leftarrow & \{aCX\} & \leftarrow & \{aCB\} \\
                & \leftarrow & \{\alert<4>{S}\} \\}
            \end{array}$
          }
    \end{exampleblock}\end{minipage}};

    \uncover<5>{
      \draw (0,5.1) node{\begin{minipage}{\textwidth}\begin{exampleblock}{$abbc \notin L$}
            {\footnotesize
              $\begin{array}{lclcl}
                \{abbc\} & \leftarrow & \{aBbc, abBc, abbC\} & \leftarrow & \{aBBc, aBbC, abBC\} \\
                & \leftarrow & \{aBBC, abBX\} & \leftarrow & \{aBBX, abCX\} \\
                & \leftarrow & \{aBCX, abCB\} & \leftarrow & \{aBXX, aBCB\} \\
                & \leftarrow & \{aCXX, aBXB\} & \leftarrow & \{aCBX, aCXB\} \\
                & \leftarrow & \{aCCX, aCBB\} & \leftarrow & \{aCCB, aCBB\} \\
                & \leftarrow & \alert{\emptyset} \\
              \end{array}$
            }
      \end{exampleblock}\end{minipage}};
    }
    \draw<-5>[example] (4,7.5) node{\begin{minipage}{3cm}
        $\left\{\begin{array}{rcl}
        S   & \rightarrow & aCB  \\
        C B & \rightarrow & C X  \\
        C X & \rightarrow & B X  \\
        B X & \rightarrow & B C  \\
        \alert<2,3>{a B} & \alert<2,3>{\rightarrow} & \alert<2,3>{a b}  \\
        b B & \rightarrow & b b  \\
        \structure<2,3>{C}   & \structure<2,3>{\rightarrow} & \structure<2,3>{c}
        \end{array}\right.$
    \end{minipage}};
  \end{tikzpicture}
\end{frame}

\endgroup

%% SPDX-License-Identifier: CC-BY-SA-4.0
% Author: Matthieu Perrin
% Part: 
% Section: 
% Sub-section: 
% Frame: 

\begingroup


\begin{frame}[fragile]{Terminaison de l'algorithme}

  \begin{block}{Grammaires contextuelles}
    \begin{itemize}
    \item L'algorithme donne une \structure{réponse} :
      \begin{itemize}
      \item \alert{\True} si le mot appartient au langage
      \item \alert{\False} si le mot n'appartient pas au langage
      \end{itemize}
    \item \structure{Démonstration :}
      \begin{itemize}
      \item Si $v \vdash w$, alors $|v| \le |w|$, car les règles sont $\alpha A \beta \rightarrow \alpha \gamma \beta$, avec $\gamma\neq \varepsilon$
      \item Il y a un nombre fini de mots de $(\Sigma \cup \Gamma)^\star$ de taille au plus $|u|$
      \end{itemize}
      \begin{center}
        Les langages contextuels sont \alert{décidables}.
      \end{center}
    \end{itemize}
  \end{block}
  \pause
  \begin{block}{Grammaires non-restreintes}
    \begin{itemize}
    \item L'algorithme donne une \structure{semi-réponse} :
      \begin{itemize}
      \item \alert{\True} si le mot appartient au langage
      \item \alert{\False} ou \alert{boucle infiniment} si le mot n'appartient pas au langage
      \end{itemize}
    \item \structure{Contre-exemple :} {\footnotesize$G = \langle \{a\}, \{S\}, S, \{S \rightarrow a, SS \rightarrow S\}\rangle$\\
      $\{aa\} \leftarrow \{aS, Sa\} \leftarrow \{SS\} \leftarrow \{SSS\} \leftarrow \{SSSS\} \leftarrow ...$}
      \begin{center}
        Ces langages sont \alert{indécidables}, mais \alert{récursivement énumérables}.
      \end{center}
    \end{itemize}      
  \end{block}

\end{frame}
 

\endgroup

% 
%\subsection{Hiérarchie de Chomsky}
%% SPDX-License-Identifier: CC-BY-SA-4.0
% Author: Matthieu Perrin
% Part: 
% Section: 
% Sub-section: 
% Frame: 

\begingroup

\begin{frame}{Classification des grammaires}
  On dit qu'une grammaire $G = \langle \Sigma, \Gamma, S, R \rangle$ \structure{est de type} $i \in \{0, 1, 2, 3\}$ si :
  \begin{description}
  \item[Type 0 :] $G$ est grammaire \structure{non-containte}\\
    \begin{itemize}
    \item $R \subset (\Sigma\cup \Gamma)^\star \cdot \Gamma \cdot (\Sigma\cup \Gamma)^\star) \times (\Sigma \cup \Gamma)^\star$ 
    \item[\example{Exemple :}] $a B c \rightarrow c B a$
    \end{itemize}
  \item[Type 1 :] $G$ est grammaire \structure{contextuelle}\\
    \begin{itemize}
    \item $\forall r\in R, r =  S \rightarrow \varepsilon \lor \exists g, d, \gamma \in (\Sigma \cup \Gamma \setminus \{S\})^\star,$ \\ $\exists A\in \Gamma, \gamma\neq \varepsilon \land r = g A d \rightarrow g \gamma d$
    \item[\example{Exemple :}] $a B c \rightarrow ac B ac | acac$ 
    \end{itemize}
  \item[Type 2 :] $G$ est grammaire \structure{algébrique}\\
    \begin{itemize}
    \item $R \subset \Gamma \times (\Sigma \cup \Gamma)^\star$ 
    \item[\example{Exemple :}] $S \rightarrow a S b | \varepsilon$ 
    \end{itemize}

  \item[Type 3 :] $G$ est grammaire \structure{rationnelle droite}
    \begin{itemize}
    \item $R \subset \Gamma \times (\Sigma \cdot \Gamma^?)^?$ 
    \item[\example{Exemple :}] $S \rightarrow aS | b | \varepsilon$ 
    \end{itemize}

    ou $G$ est grammaire \structure{rationnelle gauche}
    \begin{itemize}
    \item $R \subset \Gamma \times (\Gamma^? \cdot \Sigma)^?$ 
    \item[\example{Exemple :}] $S \rightarrow Sa | b | \varepsilon$ 
    \end{itemize}
  \end{description}
\end{frame}

\endgroup

%% SPDX-License-Identifier: CC-BY-SA-4.0
% Author: Matthieu Perrin
% Part: 
% Section: 
% Sub-section: 
% Frame: 

\begingroup

\begin{frame}{Classification des langages}

  Le \structure{type d'un langage $L$} est le plus grand $i \in \{0, 1, 2, 3\}$ (si un tel $i$ existe) \\
  tel qu'il existe une grammaire de type $i$ qui engendre $L$. 

    \vspace{-1mm}
  \begin{minipage}{.5\textwidth}
    \begin{block}{Types de grammaires}
      \centering\scalebox{.6}{\begin{tikzpicture}
          \fill[structure!20] (0.00,0.75) ellipse(3.0cm and 2.0cm);
          \fill[alert!10] (0.00,1.00) ellipse(2.0cm and 1.0cm);
          \fill[structure!10] (0.00,0.00) ellipse(2.0cm and 1.0cm);
          \fill[structure!20] (0.00,0.25) ellipse(1.0cm and 0.5cm);

          \begin{scope}
            \clip (0.00,1.00) ellipse(2.0cm and 1.0cm);
            \fill[structure!40!alert!15] (0.00,0.00) ellipse(2.0cm and 1.0cm);
            \fill[structure!50!alert!25] (0.00,0.25) ellipse(1.0cm and 0.5cm);
          \end{scope}
          
          \draw[structure] (0.00,0.75) ellipse(3.0cm and 2.0cm);
          \draw[alert] (0.00,1.00) ellipse(2.0cm and 1.0cm);
          \draw[structure] (0.00,0.00) ellipse(2.0cm and 1.0cm);
          \draw[structure] (0.00,0.25) ellipse(1.0cm and 0.5cm);

          
          \draw (0,0.20) node{3 : rationnelle};
          \draw (0,-0.5) node{2 : algébrique};
          \draw (0,1.50) node{1 : contextuelle};
          \draw (0,2.25) node{0 : non-restreinte};
      \end{tikzpicture}}
    \end{block}
  \end{minipage}%
  \begin{minipage}{.5\textwidth}
    \begin{block}{Types de langages}
      \centering\scalebox{.6}{\begin{tikzpicture}
          \draw[structure, fill=structure!10] (1.5,0.75) ellipse(4.0cm and 2.0cm);
          \draw[structure, fill=structure!20] (1.0,0.50) ellipse(3.0cm and 1.5cm);
          \draw[structure, fill=structure!10] (0.5,0.25) ellipse(2.0cm and 1.0cm);
          \draw[structure, fill=structure!20] (0.0,0.00) ellipse(1.0cm and 0.5cm);

          \draw (0.0,0.0) node{3 : rationnel};
          \draw (0.5,0.75) node{2 : algébrique};
          \draw (1.0,1.5) node{1 : contextuel};
          \draw (1.5,2.25) node{0 : récursivement énumérable};
      \end{tikzpicture}}
    \end{block}
  \end{minipage}

    \vspace{-1mm}
  \begin{exampleblock}{Exemple}
    \vspace{-2mm}
    \begin{enumerate}
    \item $G_1 = \langle \{a,b\},\{S,A\},S,R=\{ S \rightarrow  bA  \;|\;  a, bA \rightarrow a\}\rangle$\\
      \begin{itemize}
      \item  \example{pas type 3 ni 2} car $bA \rightarrow a\in R$ et $bA\not\in \Gamma$
      \item  \example{pas type 1} car $bA \rightarrow a\in R$
      \item  \example{type 0}
      \end{itemize}
    \item $G_2 = (\{a,b\},\{S\},S,\{S \rightarrow a\})$
      \begin{itemize}
      \item \example{type 3} car $S\in \Gamma$ et $a\in \Sigma^+$
      \end{itemize}
    \end{enumerate}
    $\mathcal{L}(G_1) = \mathcal{L}(G_2) = \{a\}$ est de \example{type 3}.
  \end{exampleblock}
\end{frame}

\endgroup

%% SPDX-License-Identifier: CC-BY-SA-4.0
% Author: Matthieu Perrin
% Part: 
% Section: 
% Sub-section: 
% Frame: 

\begingroup

\begin{frame}{Décidabilité de la reconnaissance}

  Soit $G$ une grammaire et $u$ un mot, peut-on savoir si \structure{$u\in \mathcal{L}(G)$} ?\\
  La difficulté du problème dépends du type de $G$ :
 
  \begin{description}
  \item[Type 3 :] Langages rationnels
    \begin{itemize}
    \item Le problème est \structure{décidable}
    \item Complexité \structure{$\mathcal{O}\left(|u|\right)$} avec un AFD
    \end{itemize}
  \item[Type 2 :] Langages algébriques
    \begin{itemize}
    \item Le problème est \structure{décidable}
    \item Complexité \structure{$\mathcal{O}\left(|u|^3\right)$} avec l'algorithme CYK
    \end{itemize}
  \item[Type 1 :] Langages contextuels
    \begin{itemize}
    \item Le problème est \structure{décidable}
    \item Complexité \structure{$\mathcal{O}\left(e^{|u|}\right)$} par force brute
    \end{itemize}
  \item[Type 0 :] Langages récursivement énumérables
    \begin{itemize}
    \item Le problème est \structure{indécidable}
    \end{itemize}
  \end{description}
 
\end{frame}

\endgroup

%% SPDX-License-Identifier: CC-BY-SA-4.0
% Author: Matthieu Perrin
% Part: 
% Section: 
% Sub-section: 
% Frame: 

\begingroup


\begin{frame}{Hiérarchie des langages}

  \tfBlock[top]{Formalismes de description d'un langage}{
    \begin{itemize}
    \item Comment représenter un langage \structure{infini} de façon \structure{finie} ?
    \item Il doit avoir une \alert{structure} interne
    \item La ``\structure{complexité}'' de sa structure entraine la ``\structure{difficulté}'' à le décider
    \end{itemize}
  }

  \tf[y=-17mm]{\footnotesize
    \begin{tikzpicture}
      \draw[structure, fill=structure!20] (12mm,12mm) ellipse(40mm and 21mm); \draw (12mm,30mm) node{définissable};
      \draw[structure, fill=structure!10] (10mm,10mm) ellipse(35mm and 18mm); \draw (10mm,25mm) node{récursivement énumérable};
      \draw[structure, fill=structure!20] (08mm,08mm) ellipse(30mm and 15mm); \draw (08mm,20mm) node{décidable};
      \draw[structure, fill=structure!10] (06mm,06mm) ellipse(25mm and 12mm); \draw (06mm,15mm) node{contextuel};
      \draw[structure, fill=structure!20] (04mm,04mm) ellipse(20mm and 09mm); \draw (04mm,10mm) node{algébrique};
      \draw[structure, fill=structure!10] (02mm,02mm) ellipse(15mm and 06mm); \draw (02mm,05mm) node{rationnel};
      \draw[structure, fill=structure!20] (00mm,00mm) ellipse(10mm and 03mm); \draw (00mm,00mm) node{fini};
    \end{tikzpicture}
  }
  
\end{frame}


\endgroup


\endgroup
\endinput
