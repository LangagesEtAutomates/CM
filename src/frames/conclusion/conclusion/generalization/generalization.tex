% SPDX-License-Identifier: CC-BY-SA-4.0
% Author: Matthieu Perrin
% Part: 
% Section: 
% Sub-section: 
% Frame: 

\begingroup

\begin{frame}{Au-delà des langages algébriques}

  \begin{block}{Langages contextuels}
    \begin{itemize}
      \item Définis par des grammaires où le contexte est pris en compte
        \begin{itemize}
        \item Règles de la forme : $\structure{u \alert{A} v \rightarrow u \alert{w} v}$
        \end{itemize}
      \item Reconnaissables par des \structure{automates linéairement bornés}
      \item Exemple : $\example{\{ a^n b^n c^n \mid n \geq 0 \}}$
      \item Utilisation : modélisation des langues naturelles
    \end{itemize}
  \end{block}

  \begin{block}{Langages généraux}
    \begin{itemize}
      \item Définis par des grammaires sans aucune restriction
        \begin{itemize}
        \item Règles de la forme : $\structure{x \rightarrow y}$
        \end{itemize}
      \item Reconnaissables par des \structure{machines de Turing}
      \item Exemple : $\example{\left\{ (p)_{10} \mid p\in \mathbb{N} \land p \text{ premier} \right\}}$
      \item Utilisation : formalisation générale de la notion d’algorithme
      \item \structure{Classe étudiée en L3 dans le cours de} \alert{Calculabilité et Complexité}
    \end{itemize}
  \end{block}

\end{frame}

\endgroup
