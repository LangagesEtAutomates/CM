% SPDX-License-Identifier: CC-BY-SA-4.0
% Author: Matthieu Perrin
% Part: 
% Section: 
% Sub-section: 
% Frame: 

\begingroup

\begin{frame}{Définitions rationnelles}
  Soit $\Sigma$ un alphabet.
  
  \begin{block}{Définition -- Définition rationnelle}
    Une \structure{définition rationnelle} est \alert{une suite de $n$ définitions} ($n \in \mathbb{N}$) de la forme \\
    \hspace*{1cm} \alert{$d_i \rightarrow r_i$} \hspace{5mm} pour $i \in \llbracket n \rrbracket$, où :\\
    \begin{itemize}
    \item chaque \alert{$d_i$ est un nom distinct}, n'appartenant pas à $\tilde{\Sigma}^\star$
    \item chaque \alert{$r_i$ est une expression rationnelle} sur $\Sigma \cup \{d_1, \ldots, d_{i-1}\}$
    \end{itemize}
  \end{block}

  \textbf{Attention :} les definitions ne sont pas récursives

  \begin{exampleblock}{Exemple de définition rationnelle}
    \vspace{-1mm}
    Le langage dont les mots sont les nombres décimaux

    \vspace{3mm}
    \begin{tabular}{rl}
        \vspace{1mm}\example{Alphabet :} & $\{0,1,2,3,4,5,6,7,8,9,$`$,$'$\}$\\
        \example{Définitions rationnelles :} &
          $\left\{\begin{array}{lll}
          DNZ&\rightarrow& 1|2|3|4|5|6|7|8|9\\
          D  &\rightarrow& 0|DNZ\\
          E  &\rightarrow& 0|DNZ\cdot D^\star\\
          N  &\rightarrow& E|E,D^+
          \end{array}\right.$
      \end{tabular}
  \end{exampleblock}

\end{frame}



\endgroup
