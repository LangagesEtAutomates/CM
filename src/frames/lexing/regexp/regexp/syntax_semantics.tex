% SPDX-License-Identifier: CC-BY-SA-4.0
% Author: Matthieu Perrin
% Part: 
% Section: 
% Sub-section: 
% Frame: 

\begingroup

\begin{frame}{Syntaxe et sémantique des expressions rationnelles}
  
  \vspace{-3mm}
  \begin{block}{Syntaxe du langage des expressions rationnelles}
    Soit \structure{$\Sigma$} un alphabet. Posons \alert{$\tilde{\Sigma} = \Sigma \cup \{
    \text{`\example{$\emptyset$}'},
    \text{`\example{$\varepsilon$}'},
    \text{`\example{$($}'},
    \text{`\example{$)$}'},
    \text{`\example{$|$}'},
    \text{`\example{$\cdot$}'},
    \text{`\example{${}^\star$}'}
    \}$}

    \begin{itemize}
    \item Les \structure{expressions rationnelles} sur $\Sigma$ sont des mots sur $\tilde{\Sigma}$
    \item Le \structure{langage des expressions rationnelles} sur $\Sigma$ est noté \alert{$\textsc{regex}_\Sigma \subseteq \tilde{\Sigma}^\star$}%
      \footnote{On donnera une définition plus précise de $\textsc{regex}_\Sigma$ dans la partie dédiée à l'analyse syntaxique.} 
    \end{itemize}
  \end{block}
 
  \begin{block}{Sémantique d'une expression rationnelle}
    \begin{itemize}
      \item\vspace{-1mm} Toute expression rationnelle \alert{$\mathit{reg}$} sur $\Sigma$ \alert{décrit} un langage \alert{$\mathcal{L}(\mathit{reg}) \subseteq \Sigma^\star$}
      \item $\mathcal{L}$ est appelée \structure{la fonction sémantique} des expressions rationnelles
      \item $\alert{\mathcal{L} : \textsc{regex}_\Sigma \rightarrow \mathscr{P}(\Sigma^\star)}$ est définie récursivement :
    \end{itemize}
    $$\structure{
      \begin{array}{r@{~=~}l@{\quad}r@{~=~}l@{\quad}r@{~=~}l}
        \mathcal{L}(\varepsilon) & \{\varepsilon\} &
        \mathcal{L}(\emptyset) & \emptyset &
        \mathcal{L}(\mathit{reg}_1 \mid \mathit{reg}_2) & \mathcal{L}(\mathit{reg}_1) \cup \mathcal{L}(\mathit{reg}_2) \\
        \mathcal{L}(a) & \{a\} &
        \mathcal{L}(\mathit{reg}_1^\star) & \mathcal{L}(\mathit{reg}_1)^\star &
        \mathcal{L}(\mathit{reg}_1 \cdot \mathit{reg}_2) & \mathcal{L}(\mathit{reg}_1) \cdot \mathcal{L}(\mathit{reg}_2)
    \end{array}}
    $$
  \end{block}
 
  \vspace{-2mm}
  \begin{exampleblock}{Exemple}
  \begin{itemize}
    \item\vspace{-1mm} $\mathcal{L}(\example{a (a \mid b) b}) = \mathcal{L}(\example{a}) \alert{\cdot} \mathcal{L}(\example{a \mid b}) \alert{\cdot} \mathcal{L}(\example{b}) = \alert{\{a\} \cdot \{a, b\} \cdot \{b\}}  = \alert{\{aab, abb\}}$
    \item $\example{a (a \mid b) b \in \textsc{regex}_\Sigma}$, mais $\alert{\{aab, abb\} \in \mathscr{P}(\Sigma^\star)}$
    \end{itemize}
  \end{exampleblock}
 
\end{frame}

%\begin{frame}{Langage des expressions rationnelles}
%  
%  Soit $\Sigma$ un alphabet. \\
%  Les \structure{expressions rationnelles} sur $\Sigma$
%  sont des \textit{expressions}, donc des mots, sur l'alphabet
%  $\tilde{\Sigma} = \Sigma \sqcup \{
%  \text{`\example{\underline{$\emptyset$}}'},
%  \text{`\example{\underline{$\varepsilon$}}'},
%  \text{`\example{\underline{$($}}'},
%  \text{`\example{\underline{$)$}}'},
%  \text{`\example{\underline{$|$}}'},
%  \text{`\example{\underline{$\cdot$}}'},
%  \text{`\example{\underline{${}^\star$}}'}
%  \}$ \footnote[frame, 1]{$\sqcup$ désigne l'union disjointe. Parfois, on note $0$, $1$, $+$ et $\times$ au lieu de $\emptyset$, $\varepsilon$, $|$ et $\cdot$}.
% 
%  L'ensemble des expressions rationnelles sur $\Sigma$ forme le langage $\alert{\textsc{regex}_\Sigma}$ définie
%  récursivement de la façon suivante.
% 
%  \begin{block}{Définition -- Expression rationnelle}
%    \alert{$\textsc{regex}_\Sigma$} est le \alert{plus petit sous-ensemble}\footnote[frame,2]{Au sens de l'inclusion} de $\tilde{\Sigma}^\star$ tel que
% 
%    \vspace{1mm}
%    \begin{tabular}{rlrl}
%      \vspace{.5mm}\structure{\myRec}& &$\text{``\example{\underline{$\emptyset$}}''} \alert{\in \textsc{regex}_\Sigma}$\\
%      \vspace{.5mm}\structure{\myRec} & &$\text{``\example{\underline{$\varepsilon$}}''} \alert{\in \textsc{regex}_\Sigma}$\\
%      \vspace{.5mm}\structure{\myRec} & $\forall \structure{a}\in \Sigma$&\alert{$\structure{a} \in \textsc{regex}_\Sigma$}\\
%      \vspace{.5mm}\structure{\myRec}& $\forall \structure{u}, \structure{v} \in \textsc{regex}_\Sigma$&
%      $\text{``\example{\underline{$($}}''} \cdot \structure{u} \cdot \text{``\example{\underline{$|$}}''} \cdot \structure{v} \cdot \text{``\example{\underline{$)$}}''}
%      \alert{\in \textsc{regex}_\Sigma}$
%      & \example{par ex. $(a|b)$}\\
%      \vspace{.5mm}\structure{\myRec}& $\forall \structure{u}, \structure{v} \in \textsc{regex}_\Sigma$&
%      $\text{``\example{\underline{$($}}''} \cdot \structure{u} \cdot \text{``\example{\underline{$\cdot$}}''} \cdot \structure{v} \cdot \text{``\example{\underline{$)$}}''}
%      \alert{\in \textsc{regex}_\Sigma}$
%      & \example{par ex. $(a\cdot (b | a))$}\\
%      \vspace{.5mm}\structure{\myRec}& $\forall \structure{u} \in \textsc{regex}_\Sigma$&
%      $\structure{u} \cdot \text{``\example{\underline{${}^\star$}}''} \alert{\in \textsc{regex}_\Sigma}$
%      & \example{par ex. $(a|b)^\star$}\\
%    \end{tabular}
%  \end{block}
% 
%  On donnera une définition plus propre dans le chapitre suivant. 
%\end{frame}

%\begin{frame}{Sémantique des expressions rationnelles}
%  Soit $\Sigma$ un alphabet.\\
%  Toute expression rationnelle \alert{$r$} sur $\Sigma$ \alert{décrit} un langage \alert{$\mathcal{L}(r) \in \mathscr{P}(\Sigma^\star)$}.
% 
%  \begin{block}{Définition -- Sémantique d'une expression rationnelle}
%    
%    La fonction \alert{$\mathcal{L} : \textsc{regex}_\Sigma \rightarrow \mathscr{P}(\Sigma^\star)$}, appelée \structure{la fonction sémantique} des expressions rationnelle,
%    est définie inductivement de la façon suivante.
% 
%    $$\structure{
%      \mathcal{L} \eqdef \left\{\begin{array}{rcll}
%      \textsc{regex}_\Sigma &\rightarrow& \mathscr{P}(\Sigma^\star)\\
%      \example{\emptyset} &\mapsto& \alert{\emptyset}\\
%      \example{\varepsilon} &\mapsto& \alert{\{\varepsilon\}}\\
%      \example{a} &\mapsto& \alert{\{a\}} & \text{si } a \in \Sigma \\
%      \example{(u | v)} &\mapsto& \alert{\mathcal{L}(u) \cup \mathcal{L}(v)}  & \text{si } u, v \in \textsc{regex}_\Sigma\\
%      \example{(u \cdot v)} &\mapsto& \alert{\mathcal{L}(u) \cdot \mathcal{L}(v)}  & \text{si } u, v \in \textsc{regex}_\Sigma \\
%      \example{u^\star} &\mapsto& \alert{\mathcal{L}(u)^\star}  & \text{si } u \in \textsc{regex}_\Sigma\\
%      \end{array}\right.}
%    $$
%  \end{block}
% 
%  
%  \vspace{-3mm}
%  \begin{exampleblock}{Exemples}
%  \vspace{-2mm}
%  \footnotesize
%  \begin{itemize}
%    \item $\mathcal{L}(\mathstring{\example{a | b}}) = \mathcal{L}(\mathstring{\example{a}}) \alert{\cup} \mathcal{L}(\mathstring{\example{b}}) = \alert{\{a\} \cup \{b\}}  = \alert{\{a, b\}}$
%    \item\vspace{-1mm} $\mathcal{L}(\mathstring{\example{a \cdot (a | b) \cdot b}}) = \mathcal{L}(\example{\mathstring{a}}) \alert{\cdot} \mathcal{L}(\example{\mathstring{a | b}}) \alert{\cdot} \mathcal{L}(\mathstring{\example{b}}) = \alert{\{a\} \cdot \{a, b\} \cdot \{b\}}  = \alert{\{aab, abb\}}$
%    \end{itemize}
%  \end{exampleblock}
%\end{frame}

\endgroup
