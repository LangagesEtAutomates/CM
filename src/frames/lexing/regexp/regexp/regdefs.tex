% SPDX-License-Identifier: CC-BY-SA-4.0
% Author: Matthieu Perrin
% Part: 
% Section: 
% Sub-section: 
% Frame: 

\begingroup

\begin{frame}{Définitions rationnelles}

  \tf[text,top]{
    Une \structure{définition rationnelle} sur un alphabet $\Sigma$ est une liste finie de règles
    $$\alert{
      d_1 \to r_1,\quad
      d_2 \to r_2,\;\dots,\;
      d_n \to r_n
    }$$
    telles que pour chaque $i=1,\dots,n$ :
    \begin{description}[xxxx]
    \item[$d_i$] est un \alert{identificateur unique}, n'appartenant pas à $\tilde{\Sigma}^\star$
    \item[$r_i$] est une \alert{expression rationnelle} sur $\Sigma \cup \{d_1, \ldots, d_{i-1}\}$
    \end{description}
    \structure{Attention :} \alert{$d_i \notin r_j$ si $i\ge j$} (les définitions ne sont pas récursives)
  }
  
  \tfExampleBlock[y=-11mm]{Exemple -- Reconnaissance des adresses email}{
    \example{Par une expression rationnelle :}
    $$(a\mid b\mid ...\mid z\mid \text{-})^+ (. (a\mid b\mid ...\mid z\mid \text{-})^+)^\star \text{@} (a\mid b\mid ...\mid z\mid \text{-})^+ (. (a\mid b\mid ...\mid z\mid \text{-})^+)^+$$
    
    \example{Par une définition rationnelle :}
  }

  \tf[bottom, x=25mm]{
    $\left\{\begin{array}{r@{~~\rightarrow~~}l}
    \example{\mathit{symbole}}     & a\mid b\mid ...\mid z\mid \text{-}\\
    \example{\mathit{label}}       & \example{\mathit{symbole}}^+ \\
    \example{\mathit{utilisateur}} & \example{\mathit{label}} (.\,\example{\mathit{label}}) ^\star \\
    \example{\mathit{domaine}}     & \example{\mathit{label}} (.\,\example{\mathit{label}}) ^+ \\
    \example{\mathit{email}}       & \example{\mathit{utilisateur}} \,\text{@}\, \example{\mathit{domaine}} \\
    \end{array}\right.$
  }
  
\end{frame}

\endgroup
