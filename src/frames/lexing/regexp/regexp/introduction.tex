% SPDX-License-Identifier: CC-BY-SA-4.0
% Author: Matthieu Perrin
% Part: 
% Section: 
% Sub-section: 
% Frame: 

\begingroup

\begin{frame}{Introduction aux langages rationnels}
  
  \begin{block}{Définition}
    Les \structure{langages rationnels} (ou \emph{réguliers}) sont les langages définis par :
    \begin{itemize}
      \item des expressions rationnelles
      \item des automates finis
      \item des grammaires linéaires
      \item des équations de type $X = aX \mid b$, etc
    \end{itemize}
    Ils forment une classe notée $\alert{\textsc{rat}_\Sigma \subsetneq \mathscr{P}(\Sigma^\star)}$.
  \end{block}

  \begin{block}{Pourquoi s’y intéresser ?}
    \begin{itemize}
      \item Ils modélisent des motifs simples et décidables
      \item Ils sont utilisés dans les outils de traitement de texte et de code :
        \begin{itemize}
          \item remplacement automatique (éditeurs de texte, compilateurs)
          \item recherche dans un fichier (\texttt{grep})
          \item découpage lexical (\texttt{JFlex}, analyse lexicale)
          \item vérification de chaînes (\texttt{java.util.regex.Pattern})
        \end{itemize}
    \end{itemize}
  \end{block}

\end{frame}

\endgroup
