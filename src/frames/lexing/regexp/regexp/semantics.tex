% SPDX-License-Identifier: CC-BY-SA-4.0
% Author: Matthieu Perrin
% Part: 
% Section: 
% Sub-section: 
% Frame: 

\begingroup


\begin{frame}{Sémantique des expressions rationnelles}
  Soit $\Sigma$ un alphabet.\\
  Toute expression rationnelle \alert{$r$} sur $\Sigma$ \alert{décrit} un langage \alert{$\mathcal{S}(r) \in \mathscr{P}(\Sigma^\star)$}.

  \begin{block}{Définition -- Sémantique d'une expression rationnelle}
    
    La fonction \alert{$\mathcal{S} : \textsc{regex}_\Sigma \rightarrow \mathscr{P}(\Sigma^\star)$}, appelée \structure{la fonction sémantique} des expressions rationnelle,
    est définie inductivement de la façon suivante.

    $$\structure{
      \mathcal{S} \eqdef \left\{\begin{array}{rcll}
      \textsc{regex}_\Sigma &\rightarrow& \mathscr{P}(\Sigma^\star)\\
      \example{\emptyset} &\mapsto& \alert{\emptyset}\\
      \example{\varepsilon} &\mapsto& \alert{\{\varepsilon\}}\\
      \example{a} &\mapsto& \alert{\{a\}} & \text{si } a \in \Sigma \\
      \example{(u | v)} &\mapsto& \alert{\mathcal{S}(u) \cup \mathcal{S}(v)}  & \text{si } u, v \in \textsc{regex}_\Sigma\\
      \example{(u \cdot v)} &\mapsto& \alert{\mathcal{S}(u) \cdot \mathcal{S}(v)}  & \text{si } u, v \in \textsc{regex}_\Sigma \\
      \example{u^\star} &\mapsto& \alert{\mathcal{S}(u)^\star}  & \text{si } u \in \textsc{regex}_\Sigma\\
      \end{array}\right.}
    $$
  \end{block}

  
  \vspace{-3mm}
  \begin{exampleblock}{Exemples}
  \vspace{-2mm}
  \footnotesize
  \begin{itemize}
    \item $\mathcal{S}(\mathstring{\example{a | b}}) = \mathcal{S}(\mathstring{\example{a}}) \alert{\cup} \mathcal{S}(\mathstring{\example{b}}) = \alert{\{a\} \cup \{b\}}  = \alert{\{a, b\}}$
    \item\vspace{-1mm} $\mathcal{S}(\mathstring{\example{a \cdot (a | b) \cdot b}}) = \mathcal{S}(\example{\mathstring{a}}) \alert{\cdot} \mathcal{S}(\example{\mathstring{a | b}}) \alert{\cdot} \mathcal{S}(\mathstring{\example{b}}) = \alert{\{a\} \cdot \{a, b\} \cdot \{b\}}  = \alert{\{aab, abb\}}$
    \end{itemize}
  \end{exampleblock}
\end{frame}

\endgroup
