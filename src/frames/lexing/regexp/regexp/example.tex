% SPDX-License-Identifier: CC-BY-SA-4.0
% Author: Matthieu Perrin
% Part: 
% Section: 
% Sub-section: 
% Frame: 

\begingroup

\begin{frame}{Exemple introductif}

  \begin{exampleblock}{Quelques adresses email valides}
    \begin{itemize}
    \item matthieu.perrin@univ-nantes.fr
    \item e12345@etu.univ-nantes.fr
    \item mmemichu@gmail.com
    \end{itemize}
  \end{exampleblock}

  \pause
  \begin{block}{Spécifier les adresses email valides}
    \begin{itemize}
    \item En général : un nom d'utilisateur, une arobase, un nom de domaine
    \item Les \structure{noms d'utilisateur} sont des suites de labels, séparés par des points
    \item Les \structure{noms de domaines} également, mais contiennent au moins un point
    \item Un \structure{label} est une suite non-vide de \structure{symboles} (lettres et tirets)
    \end{itemize}

    $$\alert{\begin{array}{r@{~~=~~}l}
    \mathit{symbole}     & a\mid b\mid ...\mid z\mid \text{-}\\
    \mathit{label}       & \structure{\mathit{symbole}}^+ \\
    \mathit{utilisateur} & \structure{\mathit{label}} (.\,\structure{\mathit{label}}) ^\star \\
    \mathit{domaine}     & \structure{\mathit{label}} (.\,\structure{\mathit{label}}) ^+ \\
    \mathit{email}       & \structure{\mathit{utilisateur}} \,\text{@}\, \structure{\mathit{domaine}} \\
    \end{array}}$$
  \end{block}
  
\end{frame}

\endgroup
