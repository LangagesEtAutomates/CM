% SPDX-License-Identifier: CC-BY-SA-4.0
% Author: Matthieu Perrin
% Part: 
% Section: 
% Sub-section: 
% Frame: 

\begingroup

\begin{frame}{Abus de notation}
  \begin{block}{Simplifications}
    \begin{itemize}
    \item L'opérateur \alert{$\cdot$} est le plus souvent \structure{omis}
    \item On utilise les règles de \alert{préséance} pour éviter les parenthèses inutiles
      \begin{center}
        \structure{1.} \alert{${}^\star$} \hspace{1cm}
        \structure{2.} \alert{$\cdot$} \hspace{1cm}
        \structure{3.} \alert{$|$}
      \end{center}
      \item Par exemple, \example{$ab^\star|cd^\star$} représente \example{$((a \cdot b^\star) | (c \cdot d^\star))$}
    \end{itemize}
  \end{block}

  \begin{block}{Sucre syntaxique}
    \begin{itemize}
    \item Ajout de  l'opérateur $\alert{+}$ : $u^+ \eqdef (u \cdot u^\star)$
    \item Ajout de  l'opérateur $\alert{?}$ : $u^? \eqdef ( u | \varepsilon)$
    \item Beaucoup d'autres, selon les outils
    \end{itemize}
  \end{block}

  \begin{block}{Définition -- Expression rationnelle standard}
    Une expression rationnelle est dite \structure{standard}
    si, et seulement si, les seuls opérateurs utilisés sont les opérateurs $.$, $|$ et ${}^\star$
  \end{block}
\end{frame}


\endgroup
