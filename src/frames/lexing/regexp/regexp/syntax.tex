% SPDX-License-Identifier: CC-BY-SA-4.0
% Author: Matthieu Perrin
% Part: 
% Section: 
% Sub-section: 
% Frame: 

\begingroup

\begin{frame}{Langage des expressions rationnelles}
  
  Soit $\Sigma$ un alphabet. \\
  Les \structure{expressions rationnelles} sur $\Sigma$
  sont des \textit{expressions}, donc des mots, sur l'alphabet
  $\tilde{\Sigma} = \Sigma \sqcup \{
  \text{`\example{\underline{$\emptyset$}}'},
  \text{`\example{\underline{$\varepsilon$}}'},
  \text{`\example{\underline{$($}}'},
  \text{`\example{\underline{$)$}}'},
  \text{`\example{\underline{$|$}}'},
  \text{`\example{\underline{$\cdot$}}'},
  \text{`\example{\underline{${}^\star$}}'}
  \}$ \footnote[frame, 1]{$\sqcup$ désigne l'union disjointe. Parfois, on note $0$, $1$, $+$ et $\times$ au lieu de $\emptyset$, $\varepsilon$, $|$ et $\cdot$}.

  L'ensemble des expressions rationnelles sur $\Sigma$ forme le langage $\alert{\textsc{regex}_\Sigma}$ définie
  récursivement de la façon suivante.

  \begin{block}{Définition --- Expression rationnelle}
    \alert{$\textsc{regex}_\Sigma$} est le \alert{plus petit sous-ensemble}\footnote[frame,2]{Au sens de l'inclusion} de $\tilde{\Sigma}^\star$ tel que

    \vspace{1mm}
    \begin{tabular}{rlrl}
      \vspace{.5mm}\structure{\myRec}& &$\text{``\example{\underline{$\emptyset$}}''} \alert{\in \textsc{regex}_\Sigma}$\\
      \vspace{.5mm}\structure{\myRec} & &$\text{``\example{\underline{$\varepsilon$}}''} \alert{\in \textsc{regex}_\Sigma}$\\
      \vspace{.5mm}\structure{\myRec} & $\forall \structure{a}\in \Sigma$&\alert{$\structure{a} \in \textsc{regex}_\Sigma$}\\
      \vspace{.5mm}\structure{\myRec}& $\forall \structure{u}, \structure{v} \in \textsc{regex}_\Sigma$&
      $\text{``\example{\underline{$($}}''} \cdot \structure{u} \cdot \text{``\example{\underline{$|$}}''} \cdot \structure{v} \cdot \text{``\example{\underline{$)$}}''}
      \alert{\in \textsc{regex}_\Sigma}$
      & \example{par ex. $(a|b)$}\\
      \vspace{.5mm}\structure{\myRec}& $\forall \structure{u}, \structure{v} \in \textsc{regex}_\Sigma$&
      $\text{``\example{\underline{$($}}''} \cdot \structure{u} \cdot \text{``\example{\underline{$\cdot$}}''} \cdot \structure{v} \cdot \text{``\example{\underline{$)$}}''}
      \alert{\in \textsc{regex}_\Sigma}$
      & \example{par ex. $(a\cdot (b | a))$}\\
      \vspace{.5mm}\structure{\myRec}& $\forall \structure{u} \in \textsc{regex}_\Sigma$&
      $\structure{u} \cdot \text{``\example{\underline{${}^\star$}}''} \alert{\in \textsc{regex}_\Sigma}$
      & \example{par ex. $(a|b)^\star$}\\
    \end{tabular}
  \end{block}

  On donnera une définition plus propre dans le chapitre suivant (page \ref{slide:grammaireRegex}). 
\end{frame}


\endgroup
