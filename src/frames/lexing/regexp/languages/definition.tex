% SPDX-License-Identifier: CC-BY-SA-4.0
% Author: Matthieu Perrin
% Part: 
% Section: 
% Sub-section: 
% Frame: 

\begingroup

\begin{frame}{Définition des langages rationnels}

  Soient $\Sigma$ un alphabet, et $L \in \mathscr{P}(\Sigma^\star)$ un langage sur $\Sigma$.

  \begin{block}{Définition -- Langage rationnel}
    $L$ est \structure{rationnel}, noté $\alert{L\in \textsc{rat}_\Sigma}$ s'il est décrit par une expression rationnelle :
    $$\alert{\textsc{rat}_\Sigma = \{\mathcal{L}(u) \mid u \in \textsc{regex}_\Sigma\}}$$
  \end{block}

  \begin{exampleblock}{Conséquences}
    \vspace{2mm}
    \begin{tabular}{rlr}
      \vspace{1mm}\example{Langage vide}& &\alert{$\emptyset \in \textsc{rat}_\Sigma$}\\
      \vspace{1mm}\example{Mot vide}& &\alert{$\{ \varepsilon \} \in \textsc{rat}_\Sigma$}\\
      \vspace{1mm}\example{Caractères}& $\forall a\in \Sigma$&\alert{$ \{ a \} \in \textsc{rat}_\Sigma$}\\
      \vspace{1mm}\example{Union}& $\forall L_1, L_2 \in \textsc{rat}_\Sigma$&\alert{$ L_1 \cup L_2 \in \textsc{rat}_\Sigma$}\\
      \vspace{1mm}\example{Produit}& $\forall L_1, L_2 \in \textsc{rat}_\Sigma$&\alert{$ L_1 \cdot L_2 \in \textsc{rat}_\Sigma$}\\
      \vspace{1mm}\example{Fermeture}& $\forall L_1 \in \textsc{rat}_\Sigma$&\alert{$ L_1^\star \in \textsc{rat}_\Sigma$}\\
    \end{tabular}
  \end{exampleblock}

\end{frame}

\endgroup
