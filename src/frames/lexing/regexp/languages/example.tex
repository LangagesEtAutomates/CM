% SPDX-License-Identifier: CC-BY-SA-4.0
% Author: Matthieu Perrin
% Part: 
% Section: 
% Sub-section: 
% Frame: 

\begingroup


\begin{frame}{Exemples de langages rationnels}
  \begin{block}{Langages finis}
    \vspace{-2mm}
    \begin{description}
    \item[Théorème :]
      Tout \alert{langage fini} sur un alphabet $\Sigma$ est rationnel.
    \item[Preuve :]
      Soit $L = \{u_1, ..., u_n\}$ un langage fini sur $\Sigma$.
      $L = \mathcal{S} \left( (u_1[1] \cdot ... \cdot u_1[|u_1|])  \mid ... \mid   (u_n[1] \cdot ... \cdot u_n[|u_n|]) \right) $
    \item[\example{Exemple :}] le langage \example{$\{0,1\}^8$} décrivant les symboles de la table ASCII en binaire sur 8 bits.
    \end{description}
  \end{block}

  \begin{block}{Langages maximaux}
    \vspace{-2mm}
    \begin{description}
    \item[Théorème :]
      Pour tout alphabet $\Sigma$, le langage \alert{$\Sigma^\star$} est rationnel.
    \item[Preuve :]
      Soit $\Sigma = \{a_1, ..., a_n\}$ un alphabet.
      $\Sigma^\star = \mathcal{S}((a_1 \mid ... \mid a_n)^\star)$.
    \item[\example{Exemple :}] le langage décrivant les \example{chaînes de caractères} en C.
    \end{description}
  \end{block}

    \begin{block}{Autre exemples}
      \begin{itemize}
      \item $\mathcal{S}(\example{((a|b)(a|b))^\star})$ : mots sur $\{a, b\}$ de longueur paire.
      \item $\mathcal{S}(\example{a^\star b^\star})$ : mots sur $\{a, b\}$ dont tous les $a$ précèdent tous les $b$.
      \end{itemize}
  \end{block}
  
\end{frame}

\endgroup
