% SPDX-License-Identifier: CC-BY-SA-4.0
% Author: Matthieu Perrin
% Part: 
% Section: 
% Sub-section: 
% Frame: 

\begingroup

\begin{frame}{Équivalence entre expressions rationnelles}

  \begin{block}{Définition -- Équivalence entre expressions rationnelles}
    Deux expressions rationnelles \alert{$u$} et \alert{$v$} sont dites \structure{équivalentes}
    (ou par abus de langage ``égales''), noté \alert{$u \equiv v$},
    si elles décrivent le même langage.
    $$\alert{u \equiv v ~\eqdef~ \mathcal{L}(u) = \mathcal{L}(v)}$$
  \end{block}

  \vspace{-2mm}
  \begin{exampleblock}{Exemples}
    \begin{itemize}
    \item \example{$a^\star \mid b^\star \not\equiv (a \mid b)^\star$} : $ab \in \mathcal{L}((a \mid b)^\star)$ mais $ab \not\in \mathcal{L}(a^\star \mid  b^\star)$
    \item \example{$(a^\star \mid b^\star)^\star \equiv (a \mid b)^\star$} : $\mathcal{L}((a^\star \mid b^\star)^\star) = \mathcal{L}((a \mid b)^\star) = \{a, b\}^\star$
    \end{itemize}
  \end{exampleblock}

  \begin{block}{Rappel -- Relation d'équivalence}
    L'équivalence est bien une relation d'équivalence, car elle vérifie :
    \begin{description}
    \item[Réflexivité :] $\forall u\in\textsc{regex}_\Sigma, u\equiv u$
    \item[Symétrie :]    $\forall u, v\in\textsc{regex}_\Sigma, u\equiv v \Rightarrow v\equiv u$
    \item[Transitivité :]$\forall u, v, w\in\textsc{regex}_\Sigma, (u\equiv v \land v \equiv w) \Rightarrow u\equiv w$
    \end{description}
  \end{block}
  
\end{frame}

\endgroup
