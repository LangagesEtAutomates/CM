% SPDX-License-Identifier: CC-BY-SA-4.0
% Author: Matthieu Perrin
% Part: 
% Section: 
% Sub-section: 
% Frame: 

\begingroup

\begin{frame}{Introduction}

  \begin{block}{\structure{Langages rationnels} (appelés aussi langages réguliers)}
    Parmi tous les langages constitués par les parties d'un
    monoïde libre $\Sigma^\star$ engendrés par un alphabet $\Sigma$,
    on distingue une classe particulière : \\
    la classe $\alert{\textsc{rat}_\Sigma}$ des langages rationnels sur $\Sigma$
  \end{block}
  
  \begin{block}{Intérêts} 
    \begin{itemize}
    \item Caractérisation précise du langage
    \item Une méthode simple permet de décider pour chacun de ces langages si un mot lui appartient ou non : les automates finis
    \item Présent dans de très nombreux langages de programmation et utilitaires pour :
      \begin{itemize}
      \item Tester l'appartenance d'une sous-chaîne à un langage rationnel
      \item Effectuer une recherche étendue dans un fichier ou un texte
      \item Effectuer des remplacements de sous-chaînes par d'autres 
      \item Diviser une chaîne de caractères en sous-chaîne de caractères / \alert{tokens}
      \end{itemize}
    \end{itemize}
  \end{block}

\end{frame}

\endgroup
