% SPDX-License-Identifier: CC-BY-SA-4.0
% Author: Matthieu Perrin
% Part: 
% Section: 
% Sub-section: 
% Frame: 

\begingroup

\begin{frame}{Exemple introductif}
  \begin{exampleblock}{Quelques adresses email valides}
    \begin{itemize}
    \item matthieu.perrin@univ-nantes.fr
    \item e12345@etu.univ-nantes.fr
    \item mmemichu@gmail.com
    \end{itemize}
  \end{exampleblock}

  \pause
  \begin{block}{Spécifier les adresses email valides}
    \begin{itemize}
    \item En général : un nom d'utilisateur, une arobase, un nom de domaine
      \vspace{-3mm}\uncover<5->{\alert{$$(a\mid b\mid ...\mid z\mid \text{-}\mid .)^+ \text{@} (a\mid b\mid ...\mid z\mid \text{-})^+ (. (a\mid b\mid ...\mid z\mid \text{-})^+)^+ $$}}
    \item\vspace{-6mm} Le nom de domaine est une suite de labels, séparés par des points
      \vspace{-3mm}\uncover<4->{\alert{$$(a\mid b\mid ...\mid z\mid \text{-})^+ (. (a\mid b\mid ...\mid z\mid \text{-})^+)^+ $$}}
    \item\vspace{-6mm} Un label est une suite non-vide de symboles (lettres et tirets)
      \vspace{-3mm}\uncover<3->{\alert{$$\mathit{symbole} = a\mid b\mid ...\mid z\mid \text{-} \hspace{10mm}  \mathit{label} = (a\mid b\mid ...\mid z\mid \text{-})^+ $$}}
    \end{itemize}
  \end{block}
\end{frame}


\endgroup
