% SPDX-License-Identifier: CC-BY-SA-4.0
% Author: Matthieu Perrin
% Part: 
% Section: 
% Sub-section: 
% Frame: 

\begingroup


\begin{frame}{Construction des expressions rationnelles}

  Les expressions rationnelles sont construites par induction en utilisant :
  \begin{description}
  \item[$\emptyset$]
    \begin{itemize}
    \item Représente le langage vide $\emptyset$
    \end{itemize}
  \item[$\varepsilon$]
    \begin{itemize}
    \item Représente le langage neutre $\{\varepsilon\}$
    \end{itemize}
  \item[$a$] pour $a\in \Sigma$
    \begin{itemize}
    \item Représente le langage $\{a\}$
    \end{itemize}
  \item[$reg_1 \cdot reg_2$] pour $reg_1, reg_2$ des expressions rationnelles
    \begin{itemize}
    \item Représente le langage $L_1 \cdot L_2$, tel que $reg_1$ représente $L_1$ et $reg_2$ représente $L_2$ 
    \end{itemize}
  \item[$reg_1 | reg_2$] pour $reg_1, reg_2$ des expressions rationnelles
    \begin{itemize}
    \item Représente le langage $L_1 \cup L_2$, tel que $reg_1$ représente $L_1$ et $reg_2$ représente $L_2$ 
    \end{itemize}
  \item[$reg_1^\star$] pour $reg_1$ une expression rationnelle
    \begin{itemize}
    \item Représente le langage $L_1^\star$, tel que $reg_1$ représente $L_1$
    \end{itemize}
  \end{description}
    \begin{itemize}
    \item D'autres constructions peuvent être ajoutées ($reg_1^?$, $reg_1^+$...)
    \end{itemize}

\end{frame}


\endgroup
