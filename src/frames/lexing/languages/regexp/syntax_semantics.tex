% SPDX-License-Identifier: CC-BY-SA-4.0
% Author: Matthieu Perrin
% Part: 
% Section: 
% Sub-section: 
% Frame: 

\begingroup

\begin{frame}{Syntaxe et sémantique des expressions rationnelles}
  
  \vspace{-3mm}
  \begin{block}{Syntaxe du langage des expressions rationnelles}
    Soit \structure{$\Sigma$} un alphabet. Posons \alert{$\tilde{\Sigma} = \Sigma \cup \{
    \text{`\example{$\emptyset$}'},
    \text{`\example{$\varepsilon$}'},
    \text{`\example{$($}'},
    \text{`\example{$)$}'},
    \text{`\example{$|$}'},
    \text{`\example{$\cdot$}'},
    \text{`\example{${}^\star$}'}
    \}$}

    \begin{itemize}
    \item Les \structure{expressions rationnelles} sur $\Sigma$ sont des mots sur $\tilde{\Sigma}$
    \item Le \structure{langage des expressions rationnelles} sur $\Sigma$ est noté \alert{$\textsc{regex}_\Sigma \subseteq \tilde{\Sigma}^\star$}%
      \footnote{On donnera une définition plus précise de $\textsc{regex}_\Sigma$ dans la partie dédiée à l'analyse syntaxique.} 
    \end{itemize}
  \end{block}
 
  \begin{block}{Sémantique d'une expression rationnelle}
    \begin{itemize}
      \item\vspace{-1mm} Toute expression rationnelle \alert{$\mathit{reg}$} sur $\Sigma$ \alert{décrit} un langage \alert{$\mathcal{L}(\mathit{reg}) \subseteq \Sigma^\star$}
      \item $\mathcal{L}$ est appelée \structure{la fonction sémantique} des expressions rationnelles
      \item $\alert{\mathcal{L} : \textsc{regex}_\Sigma \rightarrow \mathscr{P}(\Sigma^\star)}$ est définie récursivement :
    \end{itemize}
    $$\structure{
      \begin{array}{r@{~=~}l@{\quad}r@{~=~}l@{\quad}r@{~=~}l}
        \mathcal{L}(\varepsilon) & \{\varepsilon\} &
        \mathcal{L}(\emptyset) & \emptyset &
        \mathcal{L}(\mathit{reg}_1 \mid \mathit{reg}_2) & \mathcal{L}(\mathit{reg}_1) \cup \mathcal{L}(\mathit{reg}_2) \\
        \mathcal{L}(a) & \{a\} &
        \mathcal{L}(\mathit{reg}_1^\star) & \mathcal{L}(\mathit{reg}_1)^\star &
        \mathcal{L}(\mathit{reg}_1 \cdot \mathit{reg}_2) & \mathcal{L}(\mathit{reg}_1) \cdot \mathcal{L}(\mathit{reg}_2)
    \end{array}}
    $$
  \end{block}
 
  \vspace{-2mm}
  \begin{exampleblock}{Exemple}
  \begin{itemize}
    \item\vspace{-1mm} $\mathcal{L}(\example{a (a \mid b) b}) = \mathcal{L}(\example{a}) \alert{\cdot} \mathcal{L}(\example{a \mid b}) \alert{\cdot} \mathcal{L}(\example{b}) = \alert{\{a\} \cdot \{a, b\} \cdot \{b\}}  = \alert{\{aab, abb\}}$
    \item $\example{a (a \mid b) b \in \textsc{regex}_\Sigma}$, mais $\alert{\{aab, abb\} \in \mathscr{P}(\Sigma^\star)}$
    \end{itemize}
  \end{exampleblock}
 
\end{frame}

\endgroup
