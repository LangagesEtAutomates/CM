% SPDX-License-Identifier: CC-BY-SA-4.0
% Author: Matthieu Perrin
% Part: 
% Section: 
% Sub-section: 
% Frame: 

\begingroup

\begin{frame}{Exemples de langages rationnels}

  \begin{block}{Langages finis}
    \begin{description}
    \item[Théorème :]
      Tout \alert{langage fini} sur un alphabet $\Sigma$ est rationnel.
    \item[Preuve :]
      Soit $L = \{u_1, ..., u_n\}$ un langage fini sur $\Sigma$ :
      \structure{$L = \mathcal{L} \left( u_1 \mid ... \mid u_n \right)$}
    \item[\example{Exemple :}] le langage \example{$\{0,1\}^8$} des symboles ASCII en binaire sur 8 bits.
    \end{description}
  \end{block}

  \begin{block}{Langages maximaux}
    \begin{description}
    \item[Théorème :]
      Pour tout alphabet $\Sigma$, le langage \alert{$\Sigma^\star$} est rationnel.
    \item[Preuve :]
      Soit $\Sigma = \{a_1, ..., a_n\}$ un alphabet : \structure{$\Sigma^\star = \mathcal{L}((a_1 \mid ... \mid a_n)^\star)$}.
    \item[\example{Exemple :}] le langage décrivant les \example{chaînes de caractères} en C.
    \end{description}
  \end{block}

  \begin{block}{Autre exemples}
    \begin{itemize}
    \item $\mathcal{L}(\example{((a\mid b)(a\mid b))^\star})$ : mots sur $\{a, b\}$ de longueur paire.
    \item $\mathcal{L}(\example{a^\star b^\star})$ : mots sur $\{a, b\}$ dont tous les $a$ précèdent tous les $b$.
    \end{itemize}
  \end{block}
  
\end{frame}

\endgroup
