% SPDX-License-Identifier: CC-BY-SA-4.0
% Author: Matthieu Perrin
% Part: 
% Section: 
% Sub-section: 
% Frame: 

\begingroup

\begin{frame}{Démonstration du lemme de l'étoile}
  \small
  \vspace{-2mm}  
  \only<-3|handout:1>{\hspace\fill\example{\textbf{Exemple}}}%
  \only<4-|handout:2>{\hspace\fill\alert{\textbf{Lemme de l'étoile}}}

  Soient $\Sigma$ un alphabet, et $L$ un langage rationnel sur $\Sigma$.
  \only<-3|handout:1>{\hspace\fill\example{$L = a(bc)^\star a$}}
  \only<4-|handout:2>{\hspace\fill\alert{$\forall \Sigma, \forall L\in \textsc{rat}_\Sigma,$}}

  Soit $A$ son automate minimal. Posons $N = |A|$.  
  \only<-3|handout:1>{\hspace\fill\example{$N=4$}}
  \only<4-|handout:2>{\hspace\fill\alert{$\exists N\in \mathbb{N},$}}
  
  Soit $u \in L$ tel que $|u| \ge N$. 
  \only<-3|handout:1>{\hspace\fill\example{$u = abca$}}
  \only<4-|handout:2>{\hspace\fill\alert{$\forall u\in L, |u| \ge N \Rightarrow ($}}

  \begin{center}
    \scalebox{.75}{\begin{tikzpicture}[shorten >=1pt,node distance=1.5cm,on grid,auto]
        \node [state,initial, initial text=] (0) {$0$};
        \node [state, alert, fill=alert!20] (1) [right=of 0]  {$1$};
        \node [state, example, fill=example!20] (2) [right=of 1]  {$2$};
        \node [state,accepting] (3) [right=of 2]  {$3$};

        \path [->, structure]  (0) edge node[above] {$a$} (1);
        \path [->, example]  (1) edge node[below] {$b$} (2);
        \path [->, example]  (2) edge[bend right] node[above] {$c$} (1);
        \path [->, structure]  (1) edge[bend right] node[below] {$a$} (3);
    \end{tikzpicture}}

    \pause
    \structure{Pour reconnaître un mot de 4 lettres ou plus, il faut suivre une boucle}
  \end{center}

  Il existe un état $q$ que l'on visite deux fois. Posons :
  \only<-3|handout:1>{\hspace\fill\example{$q = 1$}}%
  
  \begin{itemize}
  \item $x$ le préfixe de $u$ avant la première visite de $q$
    \only<-3|handout:1>{\hspace\fill\example{$x = a$}}
    \only<4-|handout:2>{\hspace\fill\alert{$\exists x, y, z \in \Sigma^\star,$}}
  \item $y$ le facteur de $u$ entre les deux visites de $q$
    \only<-3|handout:1>{\hspace\fill\example{$y = bc$}}
    \only<4-|handout:2>{\hspace\fill\alert{$y \neq \varepsilon$}}
  \item $z$ le suffixe de $u$ après la deuxième visite de $q$
    \only<-3|handout:1>{\hspace\fill\example{$z = a$}}
    \only<4-|handout:2>{\hspace\fill\alert{$\land\; u = x\cdot y \cdot z$}}
  \end{itemize}

  \pause

  \begin{center}\structure{Si on suit une boucle une fois, on peut la suivre plusieurs fois}\end{center}

  On a :
  \begin{itemize}
  \item On peut prendre la boucle $0$, $2$, $3$, ... fois
    \only<-3|handout:1>{\hspace\fill\example{$aa \in L, abcbca\in L$}}
    \only<4-|handout:2>{\hspace\fill\alert{$\land\; \forall i\in \mathbb{N}, x \cdot y^i \cdot z \in L$}}
  \item $N$ lettres suffisent pour repasser dans un état
    \only<-3|handout:1>{\hspace\fill\example{$|abc| \le 4$}}
    \only<4-|handout:2>{\hspace\fill\alert{$\land\; |x\cdot y| \le N)$}}
  \end{itemize}

  \phantom{.}
\end{frame}


\endgroup
