% SPDX-License-Identifier: CC-BY-SA-4.0
% Author: Matthieu Perrin
% Part: 
% Section: 
% Sub-section: 
% Frame: 

\begingroup

\begin{frame}{Interprétation ubiquitaire du non-déterminisme}

  \tfBlock[top=-3mm]{Interprétation du non-déterminisme comme de l'ubiquité}{
    \begin{itemize}
    \item L'automate se trouve dans un sous-ensemble des états
    \item Le mot est reconnu si l'un des états du sous-ensemble est final
    \item Ces sous-ensembles forment un nouvel automate, qui est déterministe
    \end{itemize}
  }

  \tfExampleBlock[y=1mm]{Exemple : reconnaissance de $\alert{abc}$ par l'automate suivant}{
    \vspace{-2mm}
    $$
    \{1\} \uncover<2->{\xrightarrow{a} \{ 2, 3, 5, 7, 9, 10\}} \uncover<3->{\xrightarrow{b} \{3, 4, 5, 7, 8, 10\}} \uncover<4->{\xrightarrow{c} \{3, 5, 6, 7, 8, 10\}}
    $$
  }
  
  \tf[bottom] {
    \begin{tikzpicture}[smAutomaton]
      \smState[\smInitial   \smAlert<1> ] (1)  at (0.0, 2.0) {$1$};    
      \smState[             \smAlert<2> ] (2)  at (1.5, 2.0) {$2$};    
      \smState[             \smAlert<2->] (3)  at (5.0, 2.0) {$3$};    
      \smState[             \smAlert<3> ] (4)  at (6.5, 2.0) {$4$};    
      \smState[             \smAlert<2->] (5)  at (5.0, 0.0) {$5$};    
      \smState[             \smAlert<4->] (6)  at (6.5, 0.0) {$6$};    
      \smState[             \smAlert<2->] (7)  at (4.0, 1.0) {$7$};    
      \smState[             \smAlert<3->] (8)  at (7.5, 1.0) {$8$};    
      \smState[             \smAlert<2> ] (9)  at (3.0, 2.0) {$9$};    
      \smState[\smAccepting \smAlert<2->] (10) at (3.0, 0.0) {$10$};   


      \smPath (1) edge node {$a$}           (2);
      \smPath (3) edge node {$b$}           (4);
      \smPath (5) edge node {$c$}           (6);
      \smPath (7) edge node {$\varepsilon$} (3);
      \smPath (7) edge node {$\varepsilon$} (5);
      \smPath (4) edge node {$\varepsilon$} (8);
      \smPath (6) edge node {$\varepsilon$} (8);
      \smPath (9) edge node {$\varepsilon$} (7);
      \smPath (8) edge node {$\varepsilon$} (7);
      \smPath (7) edge node {$\varepsilon$} (10);
      \smPath (2) edge node {$\varepsilon$} (9);

    \end{tikzpicture}
  }

\end{frame}

\endgroup
