% SPDX-License-Identifier: CC-BY-SA-4.0
% Author: Matthieu Perrin
% Part: 
% Section: 
% Sub-section: 
% Frame: 

\begingroup

\begin{frame}{Interprétation ubiquitaire du non-déterminisme}
  \begin{block}{Interprétations du non-déterminisme comme de l'ubiquité}
    \begin{itemize}
    \item L'automate se trouve dans un sous-ensemble des états
    \item Le mot est reconnu si l'un des états du sous-ensemble est final
    \item Ces sous-ensembles forment un nouvel automate, qui est déterministe
    \end{itemize}
  \end{block}
  
  \begin{exampleblock}{Exemple : reconnaissance de $\alert{abc}$ par l'automate suivant}
    \begin{center}
      \scalebox{.75}{\begin{tikzpicture}
          \node (a) {};
          \node (b) [right=of a] {};
          \node (c) [right=of b] {};
          \node (d) [right=of c] {};
          \node (e) [above right=of d] {};
          \node (f) [right=of e] {};
          \node (g) [below right=of d] {};
          \node (h) [right=of g] {};
          \node (i) [below right=of f] {};
          \node (j) [below=of d] {};

          \node<2->[state,initial, initial text=] (a1) at (a) {$1$};
          \node<1>[state,initial, initial text=, alert, fill=alert!20] (a1) at (a) {$1$};

          \node<1,3->  [state] (b1) at (b) {$2$};
          \node<2>  [state,alert, fill=alert!20] (b1) at (b) {$2$};

          \node<1>  [state] (e1) at (e) {$3$};
          \node<2->  [state,alert, fill=alert!20] (e1) at (e) {$3$};

          \node<-2,4->  [state] (f1) at (f) {$4$};
          \node<3>  [state, alert, fill=alert!20] (f1) at (f) {$4$};

          \node<1>  [state] (g1) at (g) {$5$};
          \node<2->  [state,alert, fill=alert!20] (g1) at (g) {$5$};

          \node<-3>  [state] (h1) at (h) {$6$};
          \node<4->  [state,alert, fill=alert!20] (h1) at (h) {$6$};


          \node<1>  [state] (d1) at (d) {$7$};
          \node<2->  [state,alert, fill=alert!20] (d1) at (d) {$7$};

          \node<-2>  [state] (i1) at (i) {$8$};
          \node<3->  [state,alert, fill=alert!20] (i1) at (i) {$8$};

          \node<1,3-> [state] (c1) at (c) {$9$};
          \node<2>  [state,alert, fill=alert!20] (c1) at (c) {$9$};

          \node<1>  [state, accepting] (j1) at (j) {$10$};
          \node<2->  [state,accepting, alert, fill=alert!20]  (j1) at (j) {$10$};


          \path [->](a1) edge node[above] {$a$} (b1);
          \path [->](e1) edge node[below] {$b$} (f1);
          \path [->](g1) edge node[above] {$c$} (h1);
          
          \path [->](d1) edge node[right] {$\varepsilon$} (e1);
          \path [->](d1) edge node[right] {$\varepsilon$} (g1);
          
          \path [->](f1) edge node[left ] {$\varepsilon$} (i1);
          \path [->](h1) edge node[left ] {$\varepsilon$} (i1);
          
          \path [->](i1) edge node[above] {$\varepsilon$} (d1);
          \path [->](c1) edge node[above] {$\varepsilon$} (d1);
          \path [->](d1) edge node[left ] {$\varepsilon$} (j1);

          \path [->](b1) edge node[above] {$\varepsilon$} (c1);
      \end{tikzpicture}}
    \end{center}
    $$
    \{1\} \uncover<2->{\xrightarrow{a} \{ 2, 3, 5, 7, 9, 10\}} \uncover<3->{\xrightarrow{b} \{3, 4, 5, 7, 8, 10\}} \uncover<4->{\xrightarrow{c} \{3, 5, 6, 7, 8, 10\}}
    $$
  \end{exampleblock}
\end{frame}

\endgroup
