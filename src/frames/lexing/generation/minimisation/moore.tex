% SPDX-License-Identifier: CC-BY-SA-4.0
% Author: Matthieu Perrin
% Part: 
% Section: 
% Sub-section: 
% Frame: 

\begingroup

\SetKwFunction{Moore}{moore}
\SetKwData{Input}{motif}

\begin{frame}{L'algorithme de Moore}

  \tf[text,top=-2mm]{\small
    \begin{algorithm}[H]
      \Fn{\Moore( $A = \langle \Sigma, Q, I, F, \rightarrow \rangle$ : AFD ) : AFD}{
        \tfAlert<2>{$S \leftarrow \{F, Q\setminus F\}$}\;
        \Repeter{
          $\tau \gets \left\{ \left\langle s, a, s' \right\rangle \in S \times \Sigma \times S \,\middle\mid\, \exists q\in s, \exists q'\in s',~ q\xrightarrow{a} q'\right\} $\;
          \eSi{\tfAlert<3>{$\exists \langle s, a, s_1 \rangle, \langle s, a, s_2 \rangle \in \tau,~s_1 \neq s_2$}}{
            $\begin{array}{@{}l@{\,\gets\,}l@{}}
              s'_1 & \left\{q \in s \,\middle\mid\, \exists q' \in s_1, q \xrightarrow{a}q'\right\};\\
              s'_2 & \left\{q \in s \,\middle\mid\, \exists q' \in s_2, q \xrightarrow{a}q'\right\};\\
              S    & S \setminus \left\{s\right\} \cup \left\{s'_1, s'_2\right\};
            \end{array}$
          }{
            $\begin{array}{@{}l@{\,\gets\,}l@{}}
              S_0 & \left\{ s \in S \,\middle\mid\, I \subseteq s\right\};\\
              S_f & \left\{ s \in S \,\middle\mid\, s \cap F \neq \emptyset\right\};
            \end{array}$\\
            \Retourner $\langle \Sigma, S, S_0, S_f, \tau \rangle$\;
          }
        }
      }
    \end{algorithm}
  }

  \tfBlock[y=-10mm, anchor=north]{Un algorithme optimiste}{
    \vspace{-2mm}
    \begin{itemize}
    \item  Peut-être que tous les états sont équivalents ? 
    \item<2-|handout> Non : seuls les états finaux reconnaissent $\varepsilon$
      \begin{itemize}
      \item Séparer les états finaux et les autres
      \end{itemize}
    \item<3-|handout> Essayer de placer les transitions
      \begin{itemize}
      \item<4-|handout> Partitionner tant qu'on n'y arrive pas
      \end{itemize}
    \end{itemize}
  }

  \tfExampleBlock[top=-5mm, right=.29\textwidth]{Exemple}{\scriptsize
    \begin{tikzpicture}[smAutomaton]
      \smState[\smInitial  \smExample\smStructure<2,3>       ] (0) at (0,0.8) {$0$};
      \smState[            \smExample\smStructure<2-|handout>] (1) at (1,1.6) {$1$};
      \smState[            \smExample\smStructure<2-|handout>] (2) at (1,0.0) {$2$};
      \smState[\smAccepting\smExample\smAlert<2-|handout>    ] (3) at (2,0.8) {$3$};
      
      \smPath                  (0) edge[bend left ] node       {$a$} (1);
      \smPath[\smStructure<3>] (0) edge[bend right] node[swap] {$b$} (2);
      \smPath[\smAlert<3>]     (1) edge[bend left ] node       {$b$} (3);
      \smPath[\smAlert<3>]     (2) edge[bend right] node[swap] {$b$} (3);
      \smPath                  (1) edge[loop below] node       {$a$} (1);
      \smPath                  (2) edge[loop above] node       {$a$} (2);
      \smPath                  (3) edge[loop right] node       {$a, b$} (3);
    \end{tikzpicture}
  }

  \tfExampleBlock[y=-3mm, right=.29\textwidth]{Automate minimal}{\scriptsize
    \begin{tikzpicture}[smAutomaton]
      \smState[\smInitial\smExample\smStructure<2,3>] (D) at (0.0,0.0) {\alt<4->{$D$}{\alt<1>{$A$}{$B$}}};
      \uncover<2-|handout>{
        \smState[\smAccepting\smAlert] (C) at (2.0,0.0) {$C$};
      }
      \uncover<3-|handout>{
        \smPath                (C) edge[loop right] node[right] {$a, b$} (C);
      }
      \uncover<3>{
        \smPath[\smStructure]  (D) edge[loop right] node[above] {$b$?}    (D);
        \smPath[\smAlert]      (D) edge             node[above] {$b$?}    (C);
        \smPath                (D) edge[loop below] node[below] {$a$}     (D);
      }
      \uncover<4-|handout>{
        \smState[\smStructure] (E) at (1,0.0) {$E$};
        \smPath                (D) edge             node[above] {$a$} node[below] {$b$} (E);
        \smPath                (E) edge             node[above] {$b$}                   (C);
        \smPath                (E) edge[loop below] node[below] {$a$}                   (E);
      }
    \end{tikzpicture}
  }

  \tfExampleBlock[y=-10mm, anchor=north, right=.29\textwidth]{Partitionnement}{\footnotesize
    \begin{tikzpicture}
      \node[example]     (a) at (1.50,1.6) {$A = \{0,1,2, 3\}$};
      \uncover<2-|handout>{
        \node[structure] (b) at (0.75,0.8) {$B = \{0,1,2\}$};    \smPath (a) edge (b);
        \node[alert]     (c) at (2.25,0.8) {$C = \{3\}$};        \smPath (a) edge (c);
        \node at([yshift=-1mm]a) {$\varepsilon$};
      }
      \uncover<4-|handout>{
        \node[example]   (d) at (0.00,0.00) {$D = \{0\}$};       \smPath (b) edge (d);
        \node[structure] (e) at (1.50,0.00) {$E = \{1,2\}$};     \smPath (b) edge (e);
        \node at ([yshift=-1mm]b) {$b$};
      }
    \end{tikzpicture}
  }

\end{frame}

\endgroup
