% SPDX-License-Identifier: CC-BY-SA-4.0
% Author: Matthieu Perrin
% Part: 
% Section: 
% Sub-section: 
% Frame: 

\begingroup

\begin{frame}{Minimalité d'un automate fini déterministe}
  \begin{block}{Définition -- Dimension d'un automate}
    La \structure{dimension} d'un automate fini $A=\langle \Sigma, Q, I, F, \mu \rangle$,
    notée \alert{$|A|$} est \\ le nombre d'états de cet automate :

    $$\alert{|A| = |Q|}.$$
  \end{block}

  \begin{block}{Théorème -- Automate minimal}
    Soit $L$ un langage rationnel sur un alphabet $\Sigma$.
    Il existe \alert{un unique} automate déterministe et complet
    \alert{de dimension minimale} (à isomorphisme près) qui reconnaît $L$. 
    On l'appelle l’\structure{automate minimal} du langage.
  \end{block}
\end{frame}

\endgroup
