% SPDX-License-Identifier: CC-BY-SA-4.0
% Author: Matthieu Perrin
% Part: 
% Section: 
% Sub-section: 
% Frame: 

\begingroup

\begin{frame}{Idées de l'algorithme de Moore}
  \begin{tikzpicture}
    \draw (0,4) node{
      \begin{minipage}{8cm}
        \begin{block}{Un algorithme optimiste}
          \begin{itemize}
          \item  Peut-être que tous les états sont équivalents ? 
          \item<2-> Non : seuls les états finaux reconnaissent $\varepsilon$
            \begin{itemize}
            \item Séparer les états finaux et les autres
            \end{itemize}
          \item<3-> Essayer de placer les transitions
            \begin{itemize}
            \item<4-> Partitionner tant qu'on n'y arrive pas
            \end{itemize}
          \end{itemize}
        \end{block}
      \end{minipage}
    };

    \draw (5.1,3.7) node{
      \scalebox{.7}{\begin{tikzpicture}
          \draw[white] (-1,-.5) rectangle (4,4.5);
          \draw[structure] (2,4) node{$A = \{0,1,2, 3\}$};

          \draw<2->[structure] (1,2) node{$B = \{0,1,2\}$};
          \draw<2->[alert] (3,2) node{$C = \{3\}$};

          \draw<4->[structure] (0,0) node{$D = \{0\}$};
          \draw<4->[example] (2,0) node{$E = \{1,2\}$};

          \draw<2->[-latex] (1.75,3.5) -- (1.25,2.5);
          \draw<2->[-latex] (2.25,3.5) -- (2.75,2.5);
          \draw<2-> (2,3.25) node{$\varepsilon$};

          \draw<4->[-latex] (0.75,1.5) -- (0.25,0.5);
          \draw<4->[-latex] (1.25,1.5) -- (1.75,0.5);
          \draw<4-> (1,1.25) node{$b$};

          
      \end{tikzpicture}}
    };

    \draw (-1.5,0.4) node{
      \scalebox{.9}{\begin{tikzpicture}[shorten >=1pt,node distance=1.5cm,on grid,auto]
          \node[structure, fill=structure!20, state,initial, initial text=] (a) {$0$};
          \node<-3>[structure, fill=structure!20, state] (b) [above right =of a] {$1$};
          \node<-3>[structure, fill=structure!20, state] (c) [below right=of a] {$2$};
          \node<1>[structure, fill=structure!20, state, accepting] (d) [above right=of c] {$3$};

          \node<2->[alert, fill=alert!20, state, accepting] (d) [above right=of c] {$3$};

          \node<4->[example, fill=example!20, state] (b) [above right =of a] {$1$};
          \node<4->[example, fill=example!20, state] (c) [below right=of a] {$2$};
          
          \path[->]  (a) edge[bend left ] node[below right] {$a$} (b);
          \path<-2,4->[->]  (a) edge[bend right] node[above right] {$b$} (c);
          \path<-2,4->[->]  (b) edge[bend left ] node[below left ] {$b$} (d);
          \path<-2,4->[->]  (c) edge[bend right] node[above left ] {$b$} (d);
          \path<3>[->, example]  (a) edge[bend right] node[above right] {$b$} (c);
          \path<3>[->, example]  (b) edge[bend left ] node[below left ] {$b$} (d);
          \path<3>[->, example]  (c) edge[bend right] node[above left ] {$b$} (d);

          \path[->]  (b) edge[loop below, looseness=5] node[below] {$a$} (b);
          \path[->]  (c) edge[loop above, looseness=5] node[above] {$a$} (c);
          \path[->]  (d) edge[loop right, looseness=5] node[right] {$a, b$} (d);
      \end{tikzpicture}}
    };
    \draw (3.8,0.4) node{
      \scalebox{.9}{\begin{tikzpicture}[shorten >=1pt,node distance=1.5cm,on grid,auto]
          \draw[white] (-1.5,-1.5) rectangle (4.5,1.5);
          
          \node (s1) {};
          \node (s2) [right=of s1] {};
          \node (s3) [right=of s2] {};

%          \node[state,initial, initial text=] (a) {$0$};
%          \node[state, example, fill=example!20] (b) [right=of a] {$1, 2$};
%          \node[state, accepting] (d) [right=of b] {$3$};


          \node<1>[state, structure, fill=structure!20] (A) at (s1) {$A$};

          \node<2-3>[state, initial, initial text=, structure, fill=structure!20] (B) at (s1) {$B$};
          \node<2->[state, accepting, alert, fill=alert!20] (C) at (s3) {$C$};

          \node<4->[state, initial, initial text=, structure, fill=structure!20] (D) at (s1) {$D$};
          \node<4->[state, example, fill=example!20] (E) at (s2) {$E$};

          \node<3>[example] at (s2) {?};


          
          \path<4->[->]  (D) edge node[above] {$a, b$} (E);
          \path<4->[->]  (E) edge node[above] {$b$} (C);
          \path<4->[->]  (E) edge[loop below, looseness=5] node[below] {$a$} (E);
          
          \path<3>[example, ->]  (B) edge node[above] {$b$} (s2);
          \path<3>[->]  (B) edge[loop below, looseness=5] node[below] {$a$} (B);
          \path<3->[->]  (C) edge[loop right, looseness=5] node[right] {$a, b$} (C);
      \end{tikzpicture}}
    };
  \end{tikzpicture}
\end{frame}

\endgroup
