% SPDX-License-Identifier: CC-BY-SA-4.0
% Author: Matthieu Perrin
% Part: 
% Section: 
% Sub-section: 
% Frame: 

\begingroup

\begin{frame}{Idées de l'algorithme de Moore}

  \tfBlock[top]{Un algorithme optimiste}{
    \begin{itemize}
    \item  Peut-être que tous les états sont équivalents ? 
    \item<2-> Non : seuls les états finaux reconnaissent $\varepsilon$
      \begin{itemize}
      \item Séparer les états finaux et les autres
      \end{itemize}
    \item<3-> Essayer de placer les transitions
      \begin{itemize}
      \item<4-> Partitionner tant qu'on n'y arrive pas
      \end{itemize}
    \end{itemize}
  }

  \tf[top, x=.33\textwidth]{
    \begin{tikzpicture}
      \node[structure]   (a) at (1.5,3) {$A = \{0,1,2, 3\}$};
      \uncover<2-|handout>{
        \node[structure] (b) at (.75,1.5) {$B = \{0,1,2\}$};
        \node[alert]     (c) at (2.25,1.5) {$C = \{3\}$};
        \node at([yshift=-5mm]a) {$\varepsilon$};
        \smPath (a) edge (b);
        \smPath (a) edge (c);
      }
      \uncover<4-|handout>{
        \node[structure] (d) at (0,0) {$D = \{0\}$};
        \node[example]   (e) at (1.5,0) {$E = \{1,2\}$};
        \node at([yshift=-5mm]b) {$b$};
        \smPath (b) edge (d);
        \smPath (b) edge (e);
      }
    \end{tikzpicture}
  }
  
    \tf[bottom, x=-.3\textwidth]{
      \begin{tikzpicture}[smAutomaton]
        \smState[\smInitial  \smStructure              ] (0) at (0.0,1.5) {$0$};
        \smState[            \smStructure\smExample<4->] (1) at (1.5,3.0) {$1$};
        \smState[            \smStructure\smExample<4->] (2) at (1.5,0.0) {$2$};
        \smState[\smAccepting\smStructure\smAlert<2->  ] (3) at (3.0,1.5) {$3$};

        \smPath                (0) edge[bend left ] node[below right] {$a$} (1);
        \smPath[\smExample<3>] (0) edge[bend right] node[above right] {$b$} (2);
        \smPath[\smAlert<3>]   (1) edge[bend left ] node[below left ] {$b$} (3);
        \smPath[\smAlert<3>]   (2) edge[bend right] node[above left ] {$b$} (3);
        \smPath                (1) edge[loop below] node[below]       {$a$} (1);
        \smPath                (2) edge[loop above] node[above]       {$a$} (2);
        \smPath                (3) edge[loop right] node[right]       {$a, b$} (3);
      \end{tikzpicture}
    }

    \tf[y=-25mm, x=.3\textwidth]{
      \begin{tikzpicture}[smAutomaton]
        \smState[\smInitial\smStructure] (D) at (0.0,0.0) {\alt<1>{$A$}{\alt<-3>{$B$}{$D$}}};
        \uncover<2->{
          \smState[\smAccepting\smAlert] (C) at (3.0,0.0) {$C$};
        }
        \uncover<3->{
          \smPath             (C) edge[loop right] node[right] {$a, b$} (C);
        }
        \uncover<3>{
          \node      (X) at (1.5,0.0) {?};
          \smPath[\smExample]  (D) edge             node[above] {$b$}    (X);
          \smPath              (D) edge[loop below] node[below] {$a$}    (D);
        }
        \uncover<4->{
          \smState[\smExample] (E) at (1.5,0.0) {$E$};
          \smPath             (D) edge             node[above] {$a, b$} (E);
          \smPath             (E) edge             node[above] {$b$}    (C);
          \smPath             (E) edge[loop below] node[below] {$a$}    (E);
        }
      \end{tikzpicture}
    }
\end{frame}

\endgroup
