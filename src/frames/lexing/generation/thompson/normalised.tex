% SPDX-License-Identifier: CC-BY-SA-4.0
% Author: Matthieu Perrin
% Part: 
% Section: 
% Sub-section: 
% Frame: 

\begingroup

\begin{frame}{Automate normalisé}

  Soit $A = \langle \Sigma, Q, I, F, \rightarrow\rangle$ un automate fini. 

  \begin{block}{Définition -- Automate unitaire}
    $A$ est dit \structure{unitaire} s'il possède un unique état initial \hspace\fill
    $\alert{\exists q_0 \in Q,~ I = \{q_0\}}$
  \end{block}

  \begin{block}{Définition -- Automate standard}
    $A$ est dit \structure{standard} si :
    \begin{enumerate}
    \item il est \alert{unitaire}
    \item aucune transition n'arrive sur l'état initial \hspace\fill
    $\alert{\nexists q\in Q,~ \nexists a\in \Sigma,~ q \xrightarrow{a} q_0}$
    \end{enumerate}
  \end{block}

  \begin{block}{Définition -- Automate normalisé}
    $A$ est dit \structure{normalisé} si :
    \begin{enumerate}
    \item il est \alert{standard}
    \item il possède un unique état final \hspace\fill $\alert{\exists q_f \in Q,~ F = \{q_f\}}$
    \item aucune transition ne sort de l'état final \hspace\fill
    $\alert{\nexists q\in Q,~ \nexists a\in \Sigma,~ q_f \xrightarrow{a} q}$
    \end{enumerate}
  \end{block}
  
\end{frame}

\endgroup
