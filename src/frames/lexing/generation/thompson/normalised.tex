% SPDX-License-Identifier: CC-BY-SA-4.0
% Author: Matthieu Perrin
% Part: 
% Section: 
% Sub-section: 
% Frame: 

\begingroup

\begin{frame}{Automate normalisé}

  Soit $A = \langle \Sigma, Q, I, F, \mu\rangle$ un automate fini. 

  \begin{block}{Définition -- Automate unitaire}
    $A$ est dit \structure{unitaire} s'il ne possède qu'un seul état initial :
    $\alert{\exists i \in Q, I = \{i\}}.$
  \end{block}

  \begin{block}{Définition -- Automate standard}
    $A$ est dit \structure{standard} si :
    \begin{enumerate}
    \item il est \alert{unitaire}
    \item aucune transition n'arrive sur l'état initial :
    $\alert{\forall q\in Q, \forall a\in \Sigma, \langle q, a, i\rangle \notin \mu}.$
    \end{enumerate}
  \end{block}

  \begin{block}{Définition -- Automate normalisé}
    $A$ est dit \structure{normalisé} si :
    \begin{enumerate}
    \item il est \alert{standard}
    \item il ne possède qu'un seul état final : $\alert{\exists f \in Q, F = \{f\}}.$
    \item aucune transition n'a pour origine l'état final :
    $\alert{\forall q\in Q, \forall a\in \Sigma, \langle f, a, q\rangle \notin \mu}.$
    \end{enumerate}
  \end{block}
\end{frame}
\endgroup
