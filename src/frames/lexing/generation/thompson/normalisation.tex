% SPDX-License-Identifier: CC-BY-SA-4.0
% Author: Matthieu Perrin
% Part: 
% Section: 
% Sub-section: 
% Frame: 

\begingroup

\begin{frame}{Normalisation d'un automate fini}
  \begin{block}{Théorème -- Normalisation}
    Soit $\Sigma$ un alphabet et $L \in \textsc{rec}_\Sigma$.

    Il existe un automate normalisé qui reconnaît $L$.
  \end{block}
  \begin{block}{Démonstration}
    Soit $T = \langle \Sigma, Q, I, F, \mu \rangle$ un automate qui reconnaît $L$.
    \begin{itemize}
    \item Soient $i, f \notin Q$
    \item Soit $\mu' = \mu \cup \structure{\{\langle i, \varepsilon, q\rangle | q \in I\}} \cup \alert{\{\langle q, \varepsilon, f\rangle | q \in F\}}$
    \end{itemize}
    $\langle \Sigma, Q \cup \{i, f\}, \structure{\{i\}}, \alert{\{f\}}, \mu' \rangle$ est un automate normalisé qui reconnaît $L$.
  \end{block}

  \begin{exampleblock}{Exemple}
  \scalebox{.8}{\begin{tikzpicture}[shorten >=1pt,node distance=1.5cm,on grid,auto]
      \node (i)   {};
      \node [state,initial, initial text=] (q0) [above right=of i]   {$q_0$};
      \node [state,initial, initial text=] (q1) [below right=of i]  {$q_1$};
      \node [state,accepting] (q2) [right=of q0]  {$q_2$};
      \node [state,accepting] (q3) [right=of q1]  {$q_3$};
      \node (f) [below right=of q2]  {};

      \path [->]  (q0) edge node[above] {$a$} (q2);
      \path [->]  (q0) edge node[above] {$a$} (q3);
      \path [->]  (q1) edge node[below] {$b$} (q2);
      \path [->]  (q1) edge node[below] {$b$} (q3);
  \end{tikzpicture}} \hspace\fill%\hspace{1cm}$\Rightarrow$\hspace{1cm}
  \scalebox{.8}{\begin{tikzpicture}[shorten >=1pt,node distance=1.5cm,on grid,auto]
      \node [state,initial, initial text=, structure, fill=structure!20] (i)   {$i$};

      \node [state] (q0) [above right=of i]   {$q_0$};
      \node [state] (q1) [below right=of i]  {$q_1$};
      \node [state] (q2) [right=of q0]  {$q_2$};
      \node [state] (q3) [right=of q1]  {$q_3$};

      \node [state,accepting, alert, fill=alert!20] (f) [below right=of q2]  {$f$};

      \path [->]  (q0) edge node[above] {$a$} (q2);
      \path [->]  (q0) edge node[above] {$a$} (q3);
      \path [->]  (q1) edge node[below] {$b$} (q2);
      \path [->]  (q1) edge node[below] {$b$} (q3);

      \path [->, structure]  (i)  edge node[above] {$\varepsilon$} (q0);
      \path [->, structure]  (i)  edge node[below] {$\varepsilon$} (q1);
      \path [->, alert]  (q2) edge node[above] {$\varepsilon$} (f);
      \path [->, alert]  (q3) edge node[below] {$\varepsilon$} (f);

      
  \end{tikzpicture}}
  \end{exampleblock}
\end{frame}


\endgroup
