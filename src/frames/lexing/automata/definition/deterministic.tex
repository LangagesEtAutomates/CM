% SPDX-License-Identifier: CC-BY-SA-4.0
% Author: Matthieu Perrin
% Part: 
% Section: 
% Sub-section: 
% Frame: 

\begingroup

\begin{frame}{Automates déterministes}

  Soit $A = \langle \Sigma, Q, I, F, \mu \rangle$ un AFN.
  
  \begin{block}{Définition -- Automate fini déterministe (AFD)}
    On dit que $A$ est \structure{déterministe} si toutes les conditions sont vérifiées
    \begin{description}[fonction partielle :]
    \item[$\varepsilon$-libre :] $A$ ne possède pas d'$\varepsilon$-transition

      $$\alert{\mu \subseteq Q \times \Sigma \times Q}$$

    \item[unitaire :] $A$ ne possède pas qu'un seul état initial

      $$\alert{\exists q_0\in Q, I = \{q_0\}}$$

    \item[fonction partielle :] pour chaque état $q$ et chaque symbole $a$, il existe au plus une transition sortant de $q$ étiquetée $a$

      $$\alert{\forall q, q_1, q_2\in Q,  \forall a\in \Sigma, \left(\structure{q\xrightarrow{a} q_1} \in \mu \land \structure{q\xrightarrow{a} q_2} \in \mu\right) \structure{\Rightarrow q_1 = q_2}}$$
    \end{description}
  \end{block}

\end{frame}

\endgroup
