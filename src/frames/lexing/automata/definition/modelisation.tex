% SPDX-License-Identifier: CC-BY-SA-4.0
% Author: Matthieu Perrin
% Part: 
% Section: 
% Sub-section: 
% Frame: 

\begingroup

\begin{frame}{Modélisation mathématique}
  \vspace{-4mm}
  \begin{block}{Définition -- Automate fini non-déterministe (AFN)}
    \vspace{3mm}
    Un \structure{automate fini} est un quintuplet \alert{$\langle \Sigma, Q, I, F, \mu \rangle$} tel que :
    \begin{description}
    \item[\alert{$\Sigma$}] ensemble fini non vide : \structure{l'alphabet}
    \item[\alert{$Q$}] ensemble fini non vide : \structure{les états possibles}
    \item[\alert{$I$}] $\subseteq Q$ : \structure{les états initiaux}
    \item[\alert{$F$}] $\subseteq Q$ : \structure{les états finaux (ou accepteurs)}
    \item[\alert{$\mu$}] $\subseteq  Q \times (\Sigma \cup \{\varepsilon\}) \times Q$ : \structure{la relation de transition}
    \end{description}

    \vspace{3mm}
    Une \structure{transition} est un triplet \alert{$\langle q, a, q' \rangle \in \mu$} tel que :
    \begin{description}
    \item[\alert{$q$}] $\in Q$ : \structure{l'état de départ}
    \item[\alert{$a$}] $\in \Sigma \cup \{\varepsilon\}$ : \structure{l'étiquette}
    \item[\alert{$q'$}] $\in Q$ : \structure{l'état d'arrivée}
    \end{description}
  \end{block}

  \vspace{-1.3cm}\hspace{7cm}  \scalebox{.9}{\begin{tikzpicture}[shorten >=1pt,node distance=1.5cm,on grid,auto]
      \node [state] (q) {$q$}; 
      \node [state] (q1) [right=of q]  {$q'$}; 
      \path [->]    (q) edge node {$a$} (q1);
  \end{tikzpicture}}
\end{frame}

\endgroup
