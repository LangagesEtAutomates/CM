% SPDX-License-Identifier: CC-BY-SA-4.0
% Author: Matthieu Perrin
% Part: 
% Section: 
% Sub-section: 
% Frame: 

\begingroup

\begin{frame}{Modélisation mathématique}

  \begin{block}{Définition -- Automate fini non-déterministe (AFN)}
    \vspace{3mm}
    Un \structure{automate fini} est un quintuplet \alert{$\langle \Sigma, Q, I, F, \rightarrow \rangle$} tel que :
    \begin{description}[xxxxx]
    \item[\alert{$\Sigma$}] ensemble fini non vide : \structure{l'alphabet}
    \item[\alert{$Q$}] ensemble fini non vide : \structure{les états}
    \item[\alert{$I$}] $\subseteq Q$ : \structure{les états initiaux}
    \item[\alert{$F$}] $\subseteq Q$ : \structure{les états finaux (ou accepteurs)}
    \item[\alert{$\rightarrow$}] $\subseteq  Q \times (\Sigma \cup \{\varepsilon\}) \times Q$ : \structure{la relation de transition}
    \end{description}

    \vspace{3mm}
    Une \structure{transition} est un triplet \alert{$\langle q, a, q' \rangle \in \rightarrow$}, que l'on note \alert{$q\xrightarrow{a} q'$}, tel que :
    \begin{description}[xxxxx]
    \item[\alert{$q$}] $\in Q$ : \structure{l'état de départ}
    \item[\alert{$a$}] $\in \Sigma \cup \{\varepsilon\}$ : \structure{l'étiquette}
    \item[\alert{$q'$}] $\in Q$ : \structure{l'état d'arrivée}
    \end{description}
  \end{block}

  \on[y=-25mm, x=30mm]{
    \begin{tikzpicture}[automaton]
      \state (q)  at (0,0) {$q$}; 
      \state (q1) at (1,0) {$q'$}; 
      \path  (q) edge node {$a$} (q1);
    \end{tikzpicture}
  }

\end{frame}

\endgroup
