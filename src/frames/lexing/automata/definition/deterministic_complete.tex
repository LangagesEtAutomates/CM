% SPDX-License-Identifier: CC-BY-SA-4.0
% Author: Matthieu Perrin
% Part: 
% Section: 
% Sub-section: 
% Frame: 

\begingroup

\begin{frame}{Automates déterministes et complets}

  \vspace{-1mm}
  \small 
  Soit $A = \langle \Sigma, Q, I, F, \mu \rangle$ un AFN.

  \vspace{-1mm}
  \begin{block}{Définition -- AFN complet}
    On dit que $A$ est \structure{complet} si pour chaque état $q$ et chaque symbole $a$, il existe au moins une transition sortant de $q$ étiquetée $a$ :

    \vspace{-3mm}
    $$\alert{\forall q\in Q,  \forall a\in \Sigma, \exists q' \in Q : \langle q,a, q' \rangle \in \mu}$$
  \end{block}
  
  \vspace{-3mm}
  \begin{block}{Définition -- Automate fini déterministe (AFD)}
    On dit que $A$ est \structure{déterministe} si toutes les conditions sont vérifiées : 
    \begin{description}
    \item[$\varepsilon$-libre] $A$ ne possède pas d'$\varepsilon$-transition : \alert{$\mu \subseteq Q \times \Sigma \times Q$}
    \item[unitaire] $A$ ne possède pas qu'un seul état initial : \alert{$|I| = 1$}
    \item[fonction partielle]  pour chaque état $q$ et chaque symbole $a$, il existe au plus une transition sortant de $q$ étiquetée $a$ : 
    \end{description}
    $$\alert{\forall q\in Q,  \forall a\in \Sigma, \unique q' \in Q : \langle q, a, q' \rangle \in \mu}$$
  \end{block}


  \vspace{-2mm}
  \begin{alertblock}{Remarque}
    \begin{itemize}
    \item \vspace{-3mm} Si $A$ est déterministe et complet, $\mu$ est une fonction.
    \item \vspace{-1mm} Dans ce cas, on note $q' = \mu(q, a)$ pour $\langle q, a, q' \rangle \in \mu$
    \end{itemize}
  \end{alertblock}
\end{frame}

\endgroup
