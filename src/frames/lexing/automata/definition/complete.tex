% SPDX-License-Identifier: CC-BY-SA-4.0
% Author: Matthieu Perrin
% Part: 
% Section: 
% Sub-section: 
% Frame: 

\begingroup

\begin{frame}{Automates complets}

  Soit $A = \langle \Sigma, Q, I, F, \mu \rangle$ un AFN.

  \begin{block}{Définition -- AFN complet}
    On dit que $A$ est \structure{complet} si pour chaque état $q$ et chaque symbole $a$, il existe au moins une transition sortant de $q$ étiquetée $a$ :

    $$\alert{\forall q\in Q,  \forall a\in \Sigma, \exists q' \in Q, q \xrightarrow{a} q' \in \mu}$$
  \end{block}
  
\end{frame}

\endgroup
