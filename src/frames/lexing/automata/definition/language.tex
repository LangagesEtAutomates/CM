% SPDX-License-Identifier: CC-BY-SA-4.0
% Author: Matthieu Perrin
% Part: 
% Section: 
% Sub-section: 
% Frame: 

\begingroup

\begin{frame}{Langage reconnu par un AFN}

  \tf[text, top]{
    Soit $A=\langle \Sigma, Q, I, F, \rightarrow \rangle$ un AFN.

    \begin{block}{Définition -- langage reconnu (ou accepté)}
      Un mot $u$ est \structure{reconnu} par $A$ s'il mène d'un état initial à un état final
      $$\exists i\in I, \exists f\in F,~  \alert{\langle u, i \rangle \leadsto_A^\star \langle \varepsilon, f\rangle}$$
      Le langage \structure{reconnu} par $A$ est l'ensemble $\alert{\mathcal{L}(A)}$ des mots reconnus par $A$
      $$\alert{\mathcal{L}(A) \eqdef \left\{u \in \Sigma^\star \,\middle\mid\, \exists i\in I, \exists f\in F,  \langle u, i \rangle \leadsto_A^\star \langle \varepsilon, f\rangle\right\}}$$
    \end{block}
  }

  \tfExampleBlock[bottom]{Exemple}{
    Pour \example{$i=3$}, \example{$f=2$} et \example{$u=bbc$}, on a : 
    \begin{enumerate}
    \item $\langle bbc, 1 \rangle \leadsto^\star \langle \varepsilon, 2 \rangle$
    \item $1 \in I$
    \item $2 \in F$
    \end{enumerate}
    Donc \example{$bbc \in \mathcal{L}(A)$}
  }

  \tf[bottom, x=.25\textwidth]{
    \begin{tikzpicture}[smAutomaton]\small
      \smState[\smInitial]      (1) at (0.0,1.5) {$1$}; 
      \smState[\smAccepting]    (2) at (1.5,1.5) {$2$}; 
      \smState[\smInitialAbove] (3) at (0.0,0,0) {$3$}; 
      \smState                  (4) at (1.5,0,0) {$4$}; 

      \smPath (1) edge             node       {$a$}           (2);
      \smPath (1) edge[loop above] node       {$a$}           (1);
      \smPath (3) edge             node[swap] {$b$}           (4);
      \smPath (3) edge[loop left ] node       {$b$}           (3);
      \smPath (4) edge[loop right] node       {$b$}           (4);
      \smPath (2) edge[loop above] node       {$c$}           (2);
      \smPath (4) edge             node[swap] {$\varepsilon$} (2);
    \end{tikzpicture}
  }
  
\end{frame}


\endgroup
