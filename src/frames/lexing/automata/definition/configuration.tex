% SPDX-License-Identifier: CC-BY-SA-4.0
% Author: Matthieu Perrin
% Part: 
% Section: 
% Sub-section: 
% Frame: 

\begingroup

\begin{frame}{Configurations d'un AFN}
  
  \onBlock[top=-3mm]{Définition -- Configuration d'un AFN}{
    Soit $A=\langle \Sigma, Q, I, F, \rightarrow \rangle$ un AFN. \\
    Une \structure{configuration} de $A$ est un couple \alert{$\langle u, q \rangle$} tel que :
    \begin{description}[xxxx]
    \item[\alert{$u$}] $\in \Sigma^\star$ : \structure{mot restant à reconnaître}
    \item[\alert{$q$}] $\in Q$ \,: \structure{l'état courant dans la simulation}
    \end{description}
    Une configuration $\langle u, q \rangle$ est dite \structure{initiale} si \alert{$q\in I$} et \structure{acceptante} si \alert{$u=\varepsilon \land q \in F$}
  }

  \onExampleBlock[y=-1mm]{Exemples}{}

  \on[bottom, left=.33\textwidth]{
    \centering
    \example{\large $\langle bbbcc, 3 \rangle$}
    \\\vspace{1mm}
    \begin{tikzpicture}[word, size=5mm]
      \cell[example]{b} \smhead[example]
      \cell[example]{b}
      \cell[example]{b}
      \cell[example]{c}
      \cell[example]{c}
    \end{tikzpicture}
    \\\vspace{1mm}
    \begin{tikzpicture}[automaton]
      \state[initial]                (1) at (0,1) {$1$}; 
      \state[accepting]              (2) at (1,1) {$2$}; 
      \state[example, initial above] (3) at (0,0) {$3$}; 
      \state[]                       (4) at (1,0) {$4$}; 

      \path (1) edge             node       {$a$}           (2);
      \path (1) edge[loop above] node       {$a$}           (1);
      \path (3) edge             node[swap] {$b$}           (4);
      \path (3) edge[loop left ] node       {$b$}           (3);
      \path (4) edge[loop right] node       {$b$}           (4);
      \path (2) edge[loop above] node       {$c$}           (2);
      \path (4) edge             node[swap] {$\varepsilon$} (2);
    \end{tikzpicture}
  }

  \on[bottom, width=.33\textwidth]{
    \centering
    \example{\large $\langle bcc, 4 \rangle$}
    \\\vspace{1mm}
    \begin{tikzpicture}[word, size=5mm]
      \cell{b}
      \cell{b}
      \cell[example]{b} \smhead[example]
      \cell[example]{c}
      \cell[example]{c}
    \end{tikzpicture}
    \\\vspace{1mm}
    \begin{tikzpicture}[automaton]
      \state[initial]       (1) at (0,1) {$1$}; 
      \state[accepting]     (2) at (1,1) {$2$}; 
      \state[initial above] (3) at (0,0) {$3$}; 
      \state[example]       (4) at (1,0) {$4$}; 

      \path (1) edge             node[swap] {$a$}           (2);
      \path (1) edge[loop above] node       {$a$}           (1);
      \path (3) edge             node       {$b$}           (4);
      \path (3) edge[loop left ] node       {$b$}           (3);
      \path (4) edge[loop right] node       {$b$}           (4);
      \path (2) edge[loop above] node       {$c$}           (2);
      \path (4) edge             node       {$\varepsilon$} (2);
    \end{tikzpicture}
  }

  \on[bottom, right=.33\textwidth]{
    \centering
    \example{\large $\langle \varepsilon, 2 \rangle$}
    \\\vspace{1mm}
    \begin{tikzpicture}[word, size=5mm]
      \cell[open]{}
      \cell{b}
      \cell{b}
      \cell{b} 
      \cell{c}
      \cell{c}
      \cell[open]{}\smhead[example]
    \end{tikzpicture}
    \\\vspace{1mm}
    \begin{tikzpicture}[automaton]
      \state[initial]            (1) at (0,1) {$1$}; 
      \state[accepting, example] (2) at (1,1) {$2$}; 
      \state[initial above]      (3) at (0,0) {$3$}; 
      \state[]                   (4) at (1,0) {$4$}; 

      \path (1) edge             node[swap] {$a$}           (2);
      \path (1) edge[loop above] node       {$a$}           (1);
      \path (3) edge             node       {$b$}           (4);
      \path (3) edge[loop left ] node       {$b$}           (3);
      \path (4) edge[loop right] node       {$b$}           (4);
      \path (2) edge[loop above] node       {$c$}           (2);
      \path (4) edge             node       {$\varepsilon$} (2);
    \end{tikzpicture}
  }
  
\end{frame}

\endgroup
