% SPDX-License-Identifier: CC-BY-SA-4.0
% Author: Matthieu Perrin
% Part: 
% Section: 
% Sub-section: 
% Frame: 

\begingroup

\begin{frame}{Configurations d'un AFN}
  
  \tfBlock[top=-3mm]{Définition -- Configuration d'un AFN}{
    Soit $A=\langle \Sigma, Q, I, F, \rightarrow \rangle$ un AFN. \\
    Une \structure{configuration} de $A$ est un couple \alert{$\langle u, q \rangle$} tel que :
    \begin{description}[xxxx]
    \item[\alert{$u$}] $\in \Sigma^\star$ : \structure{mot restant à reconnaître}
    \item[\alert{$q$}] $\in Q$ \,: \structure{l'état courant dans la simulation}
    \end{description}
    Une configuration $\langle u, q \rangle$ est dite \structure{initiale} si \alert{$q\in I$} et \structure{acceptante} si \alert{$u=\varepsilon \land q \in F$}
  }

  \tfExampleBlock[y=-1mm]{Exemples}{}

  \tf[bottom, left=.33\textwidth]{
    \centering
    \example{\large $\langle bbbcc, 3 \rangle$}
    \\\vspace{1mm}
    \begin{smArray}[size=5mm]
      \smCell[\smExample]{b} \smHead[example]
      \smCell[\smExample]{b}
      \smCell[\smExample]{b}
      \smCell[\smExample]{c}
      \smCell[\smExample]{c}
    \end{smArray}
    \\\vspace{1mm}
    \begin{tikzpicture}[smAutomaton]\footnotesize
      \smState[\smInitial]                (1) at (0.0,1.2) {$1$}; 
      \smState[\smAccepting]              (2) at (1.5,1.2) {$2$}; 
      \smState[\smExample\smInitialAbove] (3) at (0.0,0.0) {$3$}; 
      \smState[]                          (4) at (1.5,0.0) {$4$}; 

      \smPath (1) edge             node[swap] {$a$}           (2);
      \smPath (1) edge[loop above] node       {$a$}           (1);
      \smPath (3) edge             node       {$b$}           (4);
      \smPath (3) edge[loop left ] node       {$b$}           (3);
      \smPath (4) edge[loop right] node       {$b$}           (4);
      \smPath (2) edge[loop above] node       {$c$}           (2);
      \smPath (4) edge             node       {$\varepsilon$} (2);
    \end{tikzpicture}
  }

  \tf[bottom, width=.33\textwidth]{
    \centering
    \example{\large $\langle bcc, 4 \rangle$}
    \\\vspace{1mm}
    \begin{smArray}[size=5mm]
      \smCell{b}
      \smCell{b}
      \smCell[\smExample]{b} \smHead[example]
      \smCell[\smExample]{c}
      \smCell[\smExample]{c}
    \end{smArray}
    \\\vspace{1mm}
    \begin{tikzpicture}[smAutomaton]\footnotesize
      \smState[\smInitial]      (1) at (0.0,1.2) {$1$}; 
      \smState[\smAccepting]    (2) at (1.5,1.2) {$2$}; 
      \smState[\smInitialAbove] (3) at (0.0,0.0) {$3$}; 
      \smState[\smExample]      (4) at (1.5,0.0) {$4$}; 

      \smPath (1) edge             node[swap] {$a$}           (2);
      \smPath (1) edge[loop above] node       {$a$}           (1);
      \smPath (3) edge             node       {$b$}           (4);
      \smPath (3) edge[loop left ] node       {$b$}           (3);
      \smPath (4) edge[loop right] node       {$b$}           (4);
      \smPath (2) edge[loop above] node       {$c$}           (2);
      \smPath (4) edge             node       {$\varepsilon$} (2);
    \end{tikzpicture}
  }

  \tf[bottom, right=.33\textwidth]{
    \centering
    \example{\large $\langle \varepsilon, 2 \rangle$}
    \\\vspace{1mm}
    \begin{smArray}[size=5mm]
      \smCell[\smNone]{}
      \smCell{b}
      \smCell{b}
      \smCell{b} 
      \smCell{c}
      \smCell{c}
      \smCell[\smNone]{}\smHead[example]
    \end{smArray}
    \\\vspace{1mm}
    \begin{tikzpicture}[smAutomaton]\footnotesize
      \smState[\smInitial]             (1) at (0.0,1.2) {$1$}; 
      \smState[\smAccepting\smExample] (2) at (1.5,1.2) {$2$}; 
      \smState[\smInitialAbove]        (3) at (0.0,0.0) {$3$}; 
      \smState[]                       (4) at (1.5,0.0) {$4$}; 

      \smPath (1) edge             node[swap] {$a$}           (2);
      \smPath (1) edge[loop above] node       {$a$}           (1);
      \smPath (3) edge             node       {$b$}           (4);
      \smPath (3) edge[loop left ] node       {$b$}           (3);
      \smPath (4) edge[loop right] node       {$b$}           (4);
      \smPath (2) edge[loop above] node       {$c$}           (2);
      \smPath (4) edge             node       {$\varepsilon$} (2);
    \end{tikzpicture}
  }
  
\end{frame}

\endgroup
