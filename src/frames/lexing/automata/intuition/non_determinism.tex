% SPDX-License-Identifier: CC-BY-SA-4.0
% Author: Matthieu Perrin
% Part: 
% Section: 
% Sub-section: 
% Frame: 

\begingroup

\begin{frame}{Généralisation : automates finis non-déterministes}

  Pour un mot donné, il peut exister plusieurs chemins dans l'automate
  \begin{enumerate}
  \item \structure{Plusieurs états initiaux} : l'automate \textit{décide} dans lequel il commence.
  \item \alert{Plusieurs transitions avec même origine et étiquette} : l'automate \textit{décide} laquelle il tire.
  \item {\color{example} $\varepsilon$-transition} : transition qui n'utilise aucune entrée.
    C'est une transition \textit{spontanée} : l'automate \textit{décide} de changer d'état sans lire de symbole.
  \end{enumerate}
  
  \begin{exampleblock}{Exemple}
    \centering  
    \begin{tikzpicture}[automaton, x=20mm, y=10mm]
      \state[initial, structure] (s1) at (0,2)  {$s_1$}; 
      \state                     (s2) at (1,2)  {$s_2$}; 
      \state[accepting]          (s3) at (2,1)  {$s_3$}; 
      \state[initial, structure] (s4) at (0,0)  {$s_4$}; 
      \state                     (s5) at (1,0)  {$s_5$}; 

      \path[alert]   (s1) edge             node[swap]  {$a$}           (s2);
      \path[alert]   (s1) edge[loop above] node        {$a$}           (s1);
      \path          (s2) edge[loop above] node        {$a$}           (s2);
      \path[alert]   (s4) edge             node[swap]  {$b$}           (s5);
      \path[alert]   (s4) edge[loop below] node        {$b$}           (s4);
      \path          (s5) edge[loop below] node        {$b$}           (s5);
      \path          (s3) edge[loop right] node        {$c$}           (s3);
      \path[example] (s2) edge             node        {$\varepsilon$} (s3);
      \path[example] (s5) edge             node[swap]  {$\varepsilon$} (s3);
    \end{tikzpicture}
  \end{exampleblock}

\end{frame}

\endgroup
