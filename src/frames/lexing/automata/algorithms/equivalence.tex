% SPDX-License-Identifier: CC-BY-SA-4.0
% Author: Matthieu Perrin
% Part: 
% Section: 
% Sub-section: 
% Frame: 

\begingroup

\begin{frame}{Équivalence et isomorphisme entre automates}

  \on[text, top]{
    Soient $A = \langle \Sigma, Q_A, I_A, F_A, \rightarrow_A \rangle$ et $B = \langle \Sigma, Q_B, I_B, F_B, \rightarrow_B \rangle$ deux automates. 
  }

  \onBlock[y=17mm]{Définition -- Automates équivalents}{
    $A$ et $B$ sont \structure{équivalents}, noté $\alert{A \equiv B}$, s'ils acceptent le même langage :
    $$\alert{A_1 \equiv A_2  \eqdef \mathcal{L}(A_1) = \mathcal{L}(A_2)}.$$
  }

  \onBlock[bottom=2mm]{Définition -- Automates isomorphes}{
    $A$ et $B$ sont \structure{isomorphes}, noté $\alert{A \simeq B}$, s'ils ne diffèrent que par le nom de leurs états, \textit{c.-à-d.}
    s'il existe une fonction bijective $f : Q_A \rightarrow Q_B$ telle que :

    \vspace{2mm}
    $\structure{\forall a\in \Sigma, \forall q, q',}$\\
    $\structure{\begin{array}[t]{@{\quad\quad}lll}
        q\in Q_A &\Leftrightarrow& f(q) \in Q_B \\
        q\in I_A &\Leftrightarrow& f(q) \in I_B \\
        q\in F_A &\Leftrightarrow& f(q) \in F_B \\
        q \xrightarrow{a}_A q' &\Leftrightarrow& f(q) \xrightarrow{a}_B f(q')
      \end{array}
    }$
    \vspace{2mm}

    Deux automates isomorphes sont équivalents. 
  }

  \on[bottom=2mm, x=3cm]{\small
    \begin{tikzpicture}[automaton, x=10mm, y=15mm]
      \draw (1,.5) node{\normalsize $\simeq$};

      \state[structure, initial]   (a0) at (0, 1) {$A$};
      \state[structure, accepting] (a1) at (0, 0) {$B$};
      \state[example, initial]     (b0) at (2, 1) {$0$};
      \state[example, accepting]   (b1) at (2, 0) {$1$};

      \path[structure] (a0) edge[bend left]  node {$a$} (a1);
      \path[structure] (a1) edge[bend left]  node {$a$} (a0);
      \path[structure] (a1) edge[loop right] node {$b$} (a1);
      \path[structure] (a0) edge[loop right] node {$b$} (a0);
      
      \path[example]   (b0) edge[bend left]  node {$a$} (b1);
      \path[example]   (b1) edge[bend left]  node {$a$} (b0);
      \path[example]   (b1) edge[loop right] node {$b$} (b1);
      \path[example]   (b0) edge[loop right] node {$b$} (b0);
    \end{tikzpicture}
  }
\end{frame}

\endgroup
