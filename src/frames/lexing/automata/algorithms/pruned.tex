% SPDX-License-Identifier: CC-BY-SA-4.0
% Author: Matthieu Perrin
% Part: 
% Section: 
% Sub-section: 
% Frame: 

\begingroup

\begin{frame}{Automate émondé}

  \vspace{-2mm}
  \begin{block}{Définition -- Propriétés des états}
    Soit $A=\langle \Sigma, Q, I, F, \rightarrow \rangle$ un AFN. Un état $q \in Q$ est dit
    \begin{description}[co-accessible]
    \item[accessible] s'il peut être atteint à partir d'un état initial

      \vspace{-3mm}
      $$\alert{\exists i\in I, \exists u\in \Sigma^\star, \langle u, i\rangle \leadsto^\star \langle \varepsilon, q\rangle}$$
      \vspace{-6mm}

    \item[inaccessible] s'il n'est pas accessible
    \item[co-accessible] si on peut atteindre un état final à partir de $q$

      \vspace{-3mm}
      $$\alert{\exists f\in F, \exists u\in \Sigma^\star, \langle u, q\rangle \leadsto^\star \langle \varepsilon, f\rangle}$$
      \vspace{-6mm}

    \item[stérile] s'il n'est pas co-accessible
    \item[utile] s'il est accessible et co-accessible
    \item[inutile] s'il n'est pas utile
    \end{description}
  \end{block}
  
  \begin{block}{Définition -- Automate émondé}
    Un AFN est dit \structure{émondé} si tous ses états sont utiles. 
  \end{block}

  \begin{block}{Théorème -- Émondage}
    Tout AFN est équivalent à un automate émondé.
  \end{block}

\end{frame}


\endgroup
