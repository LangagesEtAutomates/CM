% SPDX-License-Identifier: CC-BY-SA-4.0
% Author: Matthieu Perrin
% Part: 
% Section: 
% Sub-section: 
% Frame: 

\begingroup

\begin{frame}{Application du lemme d'Arden}

  Soient $A$ et $B$ deux expressions rationnelles. \\On a les deux identités remarquables : 
  \begin{enumerate}
  \item\vspace{3mm} \structure{$(AB)^\star \equiv \varepsilon \mid  A(BA)^\star B$}
    
    \begin{center}
      \alert{Idée : montrer que $(AB)^\star$ et $\varepsilon \mid  A(BA)^\star B$ sont solutions de $X = AB X \mid \varepsilon$}
    \end{center}
    
    \begin{itemize}
    \item Posons $C = A \setminus \{\varepsilon\}$ et $D = B \setminus \{\varepsilon\}$.\\
      \begin{itemize}
      \item On a $(A B)^\star \equiv (C D)^\star$  \hspace\fill car $\forall L, L^\star = (L \mid  \varepsilon)^\star$
      \item et $\varepsilon  \mid  A (B A)^\star B \equiv \varepsilon \mid  C (D C)^\star D$
      \end{itemize}
    \item Montrons que $\varepsilon \mid  C (D C)^\star D$ est solution de $X = CD X \mid  \varepsilon$.
      \begin{itemize}
      \item $\begin{array}[t]{rclrcl}
        \structure{CD (\example{\varepsilon \mid  C (D C)^\star D}) \mid  \varepsilon}
        & = & CD \mid  CD C (D C)^\star D \mid  \varepsilon\\
        & = & CD \mid  C (D C)^+ D \mid  \varepsilon\\
        & = & C (D C)^\star D \mid  \varepsilon\\
        & = & \example{\varepsilon \mid  C (D C)^\star D}\\
      \end{array}
        $
      \end{itemize}
    \item Par le lemme d'Arden, $(CD)^\star$ est \alert{l'unique} solution de $X = CD X \mid  \varepsilon$
    \item Donc $(CD)^\star \equiv \varepsilon \mid  C (D C)^\star D$, donc $(AB)^\star \equiv \varepsilon \mid  A (BA)^\star B$
    \end{itemize}
  \item\vspace{3mm} \structure{$(A \mid  B)^\star \equiv A^\star (B A^\star)^\star  \equiv (A^\star B)^\star A^\star$}
    \begin{itemize}
    \item Idem, avec l'équation $X = (C\mid D) X \mid  \varepsilon$
    \end{itemize}
  \end{enumerate}
\end{frame}

\endgroup
