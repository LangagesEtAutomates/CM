% SPDX-License-Identifier: CC-BY-SA-4.0
% Author: Matthieu Perrin
% Part: 
% Section: 
% Sub-section: 
% Frame: 

\begingroup

\begin{frame}{Langages reconnaissables par un AFN}
  Soit $\Sigma$ un alphabet.
  \begin{block}{Définition -- Langage reconnaissable}
    Un langage $L$ sur $\Sigma$ est \structure{reconnaissable} s'il est reconnu par un AFN. 

    L'ensemble des langages reconnaissables sur $\Sigma$ est noté \alert{$\textsc{rec}_\Sigma$} :
%    $$\alert{\textsc{rec}_\Sigma \eqdef \{ L \in \mathscr{P}(\Sigma^\star) \mid \exists A, L = \mathcal{L}(A)\}} $$
    $$\alert{\textsc{rec}_\Sigma \eqdef \left\{ L \subseteq \Sigma^\star \;\middle|\; \exists A = \langle \Sigma, Q, I, F, \rightarrow \rangle,\; L = \mathcal{L}(A) \right\}}$$

  \end{block}
  
  \vspace{-3mm}
  \begin{alertblock}{Théorème}
    Tout langage reconnaissable est rationnel : $\alert{\textsc{rec}_\Sigma \subseteq \textsc{rat}_\Sigma}$.
  \end{alertblock}

  \begin{center}
    \structure{Étant donné un AFN $A$, comment déterminer une expression rationnelle décrivant $\mathcal{L}(A)$ ?}
  \end{center}
  
  \begin{block}{Étapes}
    \begin{itemize}
    \item Langage droit d'un état
    \item Système d'équations rationnelles
    \item Lemme d'Arden
    \end{itemize}
  \end{block}
\end{frame}

\endgroup
