% SPDX-License-Identifier: CC-BY-SA-4.0
% Author: Matthieu Perrin
% Part: 
% Section: 
% Sub-section: 
% Frame: 

\begingroup

\begin{frame}{Résolution de système d'équations rationnelles}
    \tfBlock[top]{Système d'équations rationnelles}{
    $\left\{\begin{array}{rcl}
    L_0 &=& bL_0 \mid a L_1 \\
    L_1 &=& b L_1 \mid b L_2 \\
    L_2 &=& L_3 \mid \varepsilon \\
    L_3 &=& a L_3 \mid b L_1 \\
    \end{array}\right.$
  }

  \tf[top,x=.25\textwidth]{\small
    \begin{tikzpicture}[smAutomaton, node distance=1.3cm]
      \smState[\smInitial]   (q0)   {$q_0$};
      \smState               (q1) [right=of q0]  {$q_1$};
      \smState[\smAccepting] (q2) [below=of q1]  {$q_2$}; 
      \smState               (q3) [right=of q1]  {$q_3$}; 

      \smPath[\smStructure] (q0) edge             node {$a$} (q1);
      \smPath               (q1) edge             node[swap] {$b$} (q2);
      \smPath               (q2) edge             node[swap] {$\varepsilon$} (q3);
      \smPath               (q3) edge             node[swap] {$b$} (q1);
      \smPath[\smExample]   (q0) edge[loop above] node       {$b$} (q0);
      \smPath               (q1) edge[loop above] node       {$b$} (q1);
      \smPath               (q3) edge[loop above] node       {$a$} (q3);
    \end{tikzpicture}
  }

  \tf[text,bottom]{
    \begin{enumerate}
    \item \structure{$L_1 = b L_1 \mid b L_2$}. $\varepsilon \notin \{b\}$, donc d'après Arden, $L_1 = b^\star b L_2 = b^+ L_2$. 
    \item \structure{$L_0 = b L_0 \mid a L_1$}. $\varepsilon \notin \{b\}$, donc d'après Arden, $L_0 = b^\star a L_1 = b^\star a b^+ L_2$. 
    \item \structure{$L_3 = a L_3 \mid b L_1$}. $\varepsilon \notin \{a\}$, donc d'après Arden, $L_3 = a^\star b L_1 = a^\star b b^+ L_2$. 
    \item $\structure{L_2 = L_3 \mid \varepsilon} = a^\star b b^+ L_2 \mid \varepsilon$. $\varepsilon \notin a^\star b b^+$, donc $L_2 = (a^\star b.b^+)^\star$. 
    \end{enumerate}

    $$
    \left\{
    \begin{array}{rclclcl}
      L_0 &=& b^\star a b^+ (a^\star bb^+)^\star\\
      L_1 &=& b^+ (a^\star bb^+)^\star \\
      L_2 &=& (a^\star bb^+)^\star \\
      L_3 &=& (a^\star bb^+)^+\\
    \end{array}
    \right.
    $$

    \alert{L'automate reconnaît le langage $\mathcal{L}(A) = b^\star a b^+ (a^\star bb^+)^\star$.}
  }
\end{frame}
\endgroup
