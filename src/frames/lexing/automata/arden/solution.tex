% SPDX-License-Identifier: CC-BY-SA-4.0
% Author: Matthieu Perrin
% Part: 
% Section: 
% Sub-section: 
% Frame: 

\begingroup

\begin{frame}{Résolution de système d'équations rationnelles}
  
  \onBlock[top]{Système d'équations rationnelles}{
    $\left\{\begin{array}{@{~}c@{~=~}r@{\;\mid\;}l@{}}
    L_0 & bL_0 & a L_1 \\
    L_1 & b L_1 & b L_2 \\
    L_2 & L_3 & \varepsilon \\
    L_3 & a L_3 & b L_1 \\
    \end{array}\right.$
  }

  \on[top,x=.25\textwidth]{
    \begin{tikzpicture}[automaton]
      \state[initial]   (q0) at (0,1)  {$q_0$};
      \state            (q1) at (1,1)  {$q_1$};
      \state[accepting] (q2) at (2,1)  {$q_2$}; 
      \state            (q3) at (1,0)  {$q_3$}; 

      \path[structure] (q0) edge             node {$a$} (q1);
      \path            (q1) edge             node {$b$} (q2);
      \path            (q2) edge             node {$\varepsilon$} (q3);
      \path            (q3) edge             node {$b$} (q1);
      \path[example]   (q0) edge[loop above] node {$b$} (q0);
      \path            (q1) edge[loop above] node {$b$} (q1);
      \path            (q3) edge[loop left ] node {$a$} (q3);
    \end{tikzpicture}
  }

  \on[text,bottom]{
    \begin{enumerate}
    \item \structure{$L_1 = b L_1 \mid b L_2$}. $\varepsilon \notin \{b\}$, donc d'après Arden, $L_1 = b^\star b L_2 = b^+ L_2$. 
    \item \structure{$L_0 = b L_0 \mid a L_1$}. $\varepsilon \notin \{b\}$, donc d'après Arden, $L_0 = b^\star a L_1 = b^\star a b^+ L_2$. 
    \item \structure{$L_3 = a L_3 \mid b L_1$}. $\varepsilon \notin \{a\}$, donc d'après Arden, $L_3 = a^\star b L_1 = a^\star b b^+ L_2$. 
    \item $\structure{L_2 = \phantom{a}L_3 \mid \varepsilon} = a^\star b b^+ L_2 \mid \varepsilon$. $\varepsilon \notin a^\star b b^+$, donc $L_2 = (a^\star b.b^+)^\star$. 
    \end{enumerate}

    $$
    \left\{
    \begin{array}{@{~}r@{~=~}l}
      L_0 & b^\star a b^+ (a^\star bb^+)^\star\\
      L_1 & b^+ (a^\star bb^+)^\star \\
      L_2 & (a^\star bb^+)^\star \\
      L_3 & (a^\star bb^+)^+\\
    \end{array}
    \right.
    $$
    \alert{L'automate reconnaît le langage $\mathcal{L}(A) = b^\star a b^+ (a^\star bb^+)^\star$.}
  }
  
\end{frame}

\endgroup
