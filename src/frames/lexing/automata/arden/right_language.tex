% SPDX-License-Identifier: CC-BY-SA-4.0
% Author: Matthieu Perrin
% Part: 
% Section: 
% Sub-section: 
% Frame: 

\begingroup

\begin{frame}{Langage reconnu depuis un état}

  \onExampleBlock[y=-8mm]{Exemple}{}

  \onBlock[top=-2mm]{Définition -- Langage droit}{
    Soit $A=\langle \Sigma, Q, I, F, \rightarrow \rangle$ un AFN.\\
    Le \structure{langage droit} d'un état $q \in Q$, noté \alert{$\mathcal{L}_A(q)$},
    est l'ensemble des mots reconnus par un chemin d'actions menant de $q$ à un état final de $A$. 
    $$\alert{\mathcal{L}_A(q) \eqdef \{u \in \Sigma^\star \mid \exists f\in F,  \langle u, q \rangle \leadsto_A^\star \langle \varepsilon, f\rangle\}}$$
 
    \begin{description}
    \item[Remarques :] $\displaystyle\mathcal{L}(A) = \bigcup_{i\in I} \mathcal{L}_A(i)$ \hspace\fill $\displaystyle\mathcal{L}_A(q) = \mathcal{L}\left(\langle \Sigma, Q, \{q\}, F, \rightarrow \rangle\right)$
    \end{description}
  }
 
 
  \on[y=-25mm, x=-.3\textwidth]{
    \begin{tikzpicture}[automaton]
      \state[initial]   (s1) at (0,1) {$q_1$}; 
      \state[accepting] (s2) at (1,1) {$q_2$}; 
      \state[initial]   (s4) at (0,0) {$q_3$}; 
      \state[example]   (s5) at (1,0) {$q_4$}; 
 
      \path (s1) edge             node[above] {$a$} (s2);
      \path (s1) edge[loop above] node        {$a$} (s1);
      \path (s4) edge             node[above] {$b$} (s5);
      \path (s4) edge[loop below] node        {$b$} (s4);
      \path (s5) edge[loop below] node        {$b$} (s5);
      \path (s2) edge[loop right] node        {$c$} (s2);
      \path (s5) edge             node        {$\varepsilon$} (s2);
    \end{tikzpicture}
  }
 
  \on[y=-25mm, x=.2\textwidth]{
    \example{$\mathcal{L}_A(q_4) = \mathcal{L}\left(
      \begin{tikzpicture}[automaton, baseline=(base)]
        \coordinate (base) at (0,-3mm);
 
        \state[initial, example] (q4) at (0,0) {$q_4$}; 
        \state[accepting]        (q2) at (1,0) {$q_2$}; 
 
        \path (q4) edge[loop below] node {$b$}           (q4);
        \path (q2) edge[loop right] node {$c$}           (q2);
        \path (q4) edge             node {$\varepsilon$} (q2);
      \end{tikzpicture}
      \right) = b^\star c^\star $}
  }
  
\end{frame}


\endgroup
