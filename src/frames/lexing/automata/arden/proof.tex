% SPDX-License-Identifier: CC-BY-SA-4.0
% Author: Matthieu Perrin
% Part: 
% Section: 
% Sub-section: 
% Frame: 

\begingroup

\begin{frame}{Démonstration du lemme d'Arden}

  \tf<1->[text, top]{
    Soient $\Sigma$ un alphabet, et $A, B, \mathcal{X} \in \mathscr{P}(\Sigma^\star)$ tels que $\mathcal{X} = A\mathcal{X} \cup B$.
    \begin{enumerate}
    \item \structure{$A^\star B$ est solution de $X = AX \mid  B$} \hspace\fill \alert{$A^\star B = A (A^\star B) \cup B$}
    \item<1,3-> \structure{$A^\star B$ est la plus petite solution de $X = AX \mid  B$} \hspace\fill \alert{$A^\star B \subseteq \mathcal{X}$}
    \item<1,7-> \structure{Si $\varepsilon \notin A$, $A^\star B$ est la seule solution de $X = AX \mid  B$} \hspace\fill \alert{$\varepsilon \notin A \Rightarrow A^\star B = \mathcal{X}$}
    \end{enumerate}
  }

  \tf<2>[text, y=15mm]{
    \begin{itemize}
    \item $\begin{array}[t]{rcll}
      \structure{A} (\example{A^\star B}) \structure{\cup B}
      & = & (A A^\star) B \cup B & \text{Par associativité}\\
      & = & (A A^\star \cup \varepsilon) B  & \text{Par factorisation de $B$}\\
      & = & \example{A^\star B}  & \text{Par définition de $A^\star$}\\
    \end{array}$
    \end{itemize}
  }

  \tf<3-6>[text, y=-5mm]{
    \begin{itemize}
    \item Montrons que \alert{$\forall n\in \mathbb{N}, A^n B \subseteq \mathcal{X}$}\only<4->{ par récurrence sur $n$}.\\
      \uncover<4->{
        \structure{Initialisation :} Montrons \alert{$A^0 B \subseteq \mathcal{X}$}.
      }
      \uncover<5->{
        \begin{itemize}
        \item $\begin{array}[t]{rcll}
          A^0 B\hspace{1.5mm}~
          & = & B & \text{par définition de $A^0$}\\
          & \subseteq & A\mathcal{X} \cup B & \text{par croissance de l'union}\\
          & \subseteq & \mathcal{X} & \text{par définition de $\mathcal{X}$}\\
        \end{array}$
        \end{itemize}
      }
      \uncover<4->{
        \structure{Hérédité :} Supposons \example{$A^n B \subseteq \mathcal{X}$} pour un $n \in \mathbb{N}$, montrons \alert{$A^{n+1} B \subseteq \mathcal{X}$}.
      }
      \uncover<6->{
        \begin{itemize}
        \item \hspace{-2mm}$\begin{array}[t]{rcll}
          A^{n+1} B
          & = & A (A^{n} B) & \text{par associativité}\\
          & \subseteq & A\mathcal{X} & \text{par l'hypothèse de récurrence}\\
          & \subseteq & A\mathcal{X} \cup B & \text{par croissance de l'union}\\
          & \subseteq & \mathcal{X} & \text{par définition de $\mathcal{X}$}\\
        \end{array}$
        \end{itemize}
      }
    \item Finalement, $\displaystyle A^\star B = \bigcup_{n\in \mathbb{N}} A^n B \subseteq \bigcup_{n\in \mathbb{N}} \mathcal{X} = \mathcal{X}$
    \end{itemize}
  }

  \tf<7->[text, bottom=3mm]{
    \begin{itemize}
    \item Supposons que \example{$\varepsilon \notin A$}. 
    \item<8-> \vspace{-1mm}Montrons que $\alert{\forall n\in \mathbb{N}, \mathcal{X} \cap \Sigma^n \subseteq A^\star B}$\only<9->{ par récurrence forte sur $n$}.\\
      \uncover<9->{
        Soit $n \in \mathbb{N}$. Supposons $\example{\forall m < n, \mathcal{X} \cap \Sigma^m \subseteq A^\star B}$. Montrons \alert{$\mathcal{X} \cap \Sigma^n \subseteq A^\star B$}.%
        \begin{itemize}
        \item Soit \example{$u \in \mathcal{X} \cap \Sigma^n$}. Montrons \alert{$u\in A^\star B$}.
        \item<10-> $u \in \mathcal{X} = A\mathcal{X} \mid  B$. Deux cas possibles \\
          \structure{Cas $u \in A\mathcal{X}$ :}
          \uncover<11->{
            \begin{itemize}
            \item Il existe une décomposition $u=ax$, avec $a \in A$ et $x \in \mathcal{X}$.
            \item Comme $\varepsilon \notin A$, $|x| = |u| - |a| < |u| = n$.
            \item Par l'hypothèse de récurrence forte, $\mathcal{X} \cap \Sigma^{|x|} \subseteq A^\star B$, donc $x \in A^\star B$.
            \item $u = ax \in A\cdot A^\star B \subseteq A^\star B$.
            \end{itemize}
          }
          \structure{Cas $u \in B$ :} 
          \uncover<12->{
            \begin{itemize}
            \item $u \in A^0 B \subseteq A^\star B$
            \end{itemize}
          }
        \end{itemize}
      }
    \item Montrons que \alert{$\mathcal{X} \subseteq A^\star B$}. \only<8->{Soit $u \in \mathcal{X}$. On a $u\in \mathcal{X} \cap \Sigma^{|u|} \subseteq A^\star B$.}
    \item \vspace{-1mm}Par double inclusion, $\mathcal{X} = A^\star B$
    \end{itemize}
  }

  
  \tf<handout>[text, top]{
    Soient $\Sigma$ un alphabet, et $A, B, \mathcal{X} \in \mathscr{P}(\Sigma^\star)$ tels que $\mathcal{X} = A\mathcal{X} \cup B$.
    \begin{enumerate}
    \item \structure{$A^\star B$ est solution de $X = AX \mid  B$} \hspace\fill \alert{$A^\star B = A (A^\star B) \cup B$}
      \begin{itemize}
      \item $\structure{A} (\example{A^\star B}) \structure{\cup B}
        =  (A A^\star) B \cup B
        = (A A^\star \cup \varepsilon) B 
        = \example{A^\star B}$
      \end{itemize}
      
    \item \structure{$A^\star B$ est la plus petite solution de $X = AX \mid  B$} \hspace\fill \alert{$A^\star B \subseteq \mathcal{X}$}
      
      \begin{itemize}
      \item Montrons que \alert{$\forall n\in \mathbb{N}, A^n B \subseteq \mathcal{X}$} par récurrence sur $n$.\\
        \structure{Initialisation :}  $
        A^0 B
        = B
        \subseteq A\mathcal{X} \cup B
        \subseteq \mathcal{X}$
        
        \structure{Hérédité :} Supposons \example{$A^n B \subseteq \mathcal{X}$} pour un $n \in \mathbb{N}$, montrons \alert{$A^{n+1} B \subseteq \mathcal{X}$}.
        \begin{itemize}
        \item $
          A^{n+1} B
          = A (A^{n} B) 
          \subseteq A\mathcal{X} 
          \subseteq A\mathcal{X} \cup B 
          \subseteq \mathcal{X} 
          $
        \end{itemize}
      \item Finalement, $\displaystyle A^\star B = \bigcup_{n\in \mathbb{N}} A^n B \subseteq \bigcup_{n\in \mathbb{N}} \mathcal{X} = \mathcal{X}$
      \end{itemize}
      
    \item \structure{Si $\varepsilon \notin A$, $A^\star B$ est la seule solution de $X = AX \mid  B$} \hspace\fill \alert{$\varepsilon \notin A \Rightarrow A^\star B = \mathcal{X}$}
      \begin{itemize}
      \item Montrons que $\alert{\forall n\in \mathbb{N}, \mathcal{X} \cap \Sigma^n \subseteq A^\star B}$ par récurrence forte sur $n$.\\
        Soit $n \in \mathbb{N}$. Supposons $\example{\forall m < n, \mathcal{X} \cap \Sigma^m \subseteq A^\star B}$. Montrons \alert{$\mathcal{X} \cap \Sigma^n \subseteq A^\star B$}.\\
        Soit \example{$u \in \mathcal{X} \cap \Sigma^n$}. On a $u \in \mathcal{X} = A\mathcal{X} \mid  B$. Deux cas possibles :\\
        \begin{description}[xxxx]
        \item[Cas $u \in A\mathcal{X}$ :]
          Il existe une décomposition $u=ax$, avec $a \in A$ et $x \in \mathcal{X}$.\\
          Comme $\varepsilon \notin A$, $|x| = |u| - |a| < |u| = n$.\\
          Par récurrence forte, $\mathcal{X} \cap \Sigma^{|x|} \subseteq A^\star B$, donc $x \in A^\star B$.
          $u = ax \in A\cdot A^\star B \subseteq A^\star B$.
        \item[Cas $u \in B$ :]
          $u \in A^0 B \subseteq A^\star B$
        \end{description}

      \item Montrons que \alert{$\mathcal{X} \subseteq A^\star B$}. Soit $u \in \mathcal{X}$. On a $u\in \mathcal{X} \cap \Sigma^{|u|} \subseteq A^\star B$.
      \item Par double inclusion, $\mathcal{X} = A^\star B$
      \end{itemize}
    \end{enumerate}
  }
  
\end{frame}

\endgroup
