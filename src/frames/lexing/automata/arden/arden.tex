% SPDX-License-Identifier: CC-BY-SA-4.0
% Author: Matthieu Perrin
% Part: 
% Section: 
% Sub-section: 
% Frame: 

\begingroup

\begin{frame}{Lemme d'Arden}
  Soient $\Sigma$ un alphabet, et $A, B \in \mathscr{P}(\Sigma^\star)$ deux langages.
  \begin{block}{Lemme -- Lemme d'Arden}
    \begin{itemize}
    \item \alert{$A^\star B$} est la \structure{plus petite solution} de l'équation \structure{$X = AX \mid B$}
    \item si \structure{$\varepsilon \notin A$}, alors \alert{$A^\star B$} est l'\alert{unique} solution de l'équation \structure{$X = AX \mid B$}
    \end{itemize}
  \end{block}
  \begin{exampleblock}{Exemple}
    \noindent\begin{minipage}{.4\textwidth}
    \begin{tikzpicture}[automaton]
      \smState[\smInitial]   (s1)                {$0$}; 
      \smState[\smAccepting] (s2) [right=of s1]  {$1$};
      \smPath (s1) edge[loop above] node {$a$} (s1);
      \smPath (s1) edge             node {$b$} (s2);
    \end{tikzpicture}
    \end{minipage}%
    \begin{minipage}{.6\textwidth}
      $ L = a L \mid b = a^\star b$
    \end{minipage}
  \end{exampleblock}
  \begin{block}{Remarques}
    \begin{itemize}
    \item {\small L'équation $X = AX \mid B$ est appelée \structure{équation rationnelle linéaire à droite}}
    \item {\small Pour l'équation $X = XA \mid B$ linéaire à gauche, mêmes résultats avec $B A^\star$}
    \item {\small Si \structure{$\varepsilon \in A$}, $\Sigma^\star$ est également solution des deux équations}
    \end{itemize}
  \end{block}
\end{frame}

\endgroup
