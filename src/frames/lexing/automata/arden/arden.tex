% SPDX-License-Identifier: CC-BY-SA-4.0
% Author: Matthieu Perrin
% Part: 
% Section: 
% Sub-section: 
% Frame: 

\begingroup

\begin{frame}{Lemme d'Arden}

  \onBlock[top]{Lemme -- Lemme d'Arden}{
    Soient $\Sigma$ un alphabet, et $A, B \in \mathscr{P}(\Sigma^\star)$ deux langages.
    \begin{itemize}
    \item \alert{$A^\star B$} est la \structure{plus petite solution} de l'équation \structure{$X = AX \mid B$}
    \item si \structure{$\varepsilon \notin A$}, alors \alert{$A^\star B$} est l'\alert{unique} solution de l'équation \structure{$X = AX \mid B$}
    \end{itemize}
  }

  \onExampleBlock[y=5mm]{Exemple}{}

  \on[y=-5mm, x=-.20\textwidth]{
    \begin{tikzpicture}[automaton]
      \state[initial]   (0) at (0,0)  {$0$}; 
      \state[accepting] (1) at (1,0)  {$1$};
      \path (0) edge[loop above] node {$a$} (0);
      \path (0) edge             node {$b$} (1);
    \end{tikzpicture}
  }

  \on[y=-8mm, x=.20\textwidth]{
    $L = a L \mid b = a^\star b$
  }
  
  \onBlock[bottom]{Remarques}{
    \begin{itemize}
    \item {\small L'équation $X = AX \mid B$ est appelée \structure{équation rationnelle linéaire à droite}}
    \item {\small Pour l'équation $X = XA \mid B$ linéaire à gauche, mêmes résultats avec $B A^\star$}
    \item {\small Si \structure{$\varepsilon \in A$}, $\Sigma^\star$ est également solution des deux équations}
    \end{itemize}
  }

\end{frame}

\endgroup
