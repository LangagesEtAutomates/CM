% SPDX-License-Identifier: CC-BY-SA-4.0
% Author: Matthieu Perrin
% Part: 
% Section: 
% Sub-section: 
% Frame: 

\begingroup

\begin{frame}[fragile]{Exemple d'algorithme d'analyse lexicale}
  
  \vspace{-3mm}
  
  ~\hspace{-5mm}
  \begin{tikzpicture}
    
    \draw[white] (0,-.6) rectangle (11.7,7.3);
    
    \draw (7,6.5) node[right]{\begin{minipage}{4cm}
        \begin{block}{Expression rationnelle}
          \vspace{-5mm}
          $$ a^+ b^+ $$
        \end{block}
    \end{minipage}};

    \mode<beamer>{
    \fill<2>[alert!20, rounded corners]    (0.4,6.25)  rectangle (1.0,6.6);
    \fill<2>[example!20, rounded corners]  (2.15,6.25) rectangle (2.75,6.6);
    \fill<3>[alert!20, rounded corners]    (0.4,6.25)  rectangle (4,6.6);

    \fill<4-5>[structure!20, rounded corners]    (0.9,5.5)  rectangle (4.2,5.85);
    \fill<4>[alert!20,  rounded corners]    (1.3,5.2)  rectangle (3.0,5.5);

    \fill<6-8>[structure!20, rounded corners]    (0.9,4.8)  rectangle (4.8,5.15);
    \fill<6>[alert!20,  rounded corners]    (1.3,4.45)  rectangle (3.0,4.8);

    \fill<9>[alert!20,  rounded corners]    (1.7,3.6)  rectangle (3.2,3.95);
    }    
    \draw (5.4,5.2) node{\begin{minipage}{\textwidth}\scalebox{.85}{\begin{algorithm}[H]
            \SetKwFunction{CheckEmail}{decider}
            \SetKwData{Mot}{u}
            \SetKwData{VuA}{vuA}
            \SetKwData{VuB}{vuB}
            \SetKw{False}{false}
            \SetKw{True}{true}
            
            \Fn{\CheckEmail(\Mot : mot) : booléen}{
              $\VuA \leftarrow \False$, $\VuB \leftarrow \False$\;
              \Pour{$i$ de $1$ à $|\Mot|$}{
                \uSi{$\Mot[i] = \text{`a'} \land \lnot \VuB $}{
                  $\VuA \leftarrow \True$\;
                }\uSinonSi{$\Mot[i]=\text{`b'} \land \VuA$}{
                  $\VuB \leftarrow \True$\;
                }\lSinon{\Retourner \False}
              }
              \Retourner $\VuA \land \VuB$\; 
            }
    \end{algorithm}}\end{minipage}};
    
    \draw<2-> (3,1.5) node{\begin{minipage}{6cm}
        \begin{block}{Représentation graphique}
          \begin{tikzpicture}[shorten >=1pt,node distance=1.5cm,on grid,auto]
            \draw[white] (-.1,-.5) rectangle (5,2);
            
            \node<2->[state] (s) at (1.1,1.8)   {$q$}; 
            \draw<2-> (1.5,1.8) node[right]{état de l'algorithme}; 
            
            \node<3->[state,initial, initial text=] (is) at (1.1,0.9)   {$i$}; 
            \draw<3-> (1.5,0.9) node[right]{initial}; 
            
            \node<9->[state,accepting] (fs) at (3.6,0.9)   {$f$}; 
            \draw<9-> (4,0.9) node[right]{final}; 
            
            \node<4->[state] (s1) at (1.1,0)   {$q_1$}; 
            \node<4->[state] (s2) at (2.7,0)   {$q_2$}; 
            \path<4->[->] (s1) edge node {a} (s2);
            \draw<4-> (3.1,0) node[right]{transition}; 
          \end{tikzpicture}
        \end{block}
    \end{minipage}};
    
    \draw<2-> (10,1.5) node{\begin{minipage}{6cm}
        \begin{tikzpicture}[shorten >=1pt,node distance=1.5cm,on grid,auto]
          \draw[white] (5,-.5) -- (11.5,3.5);
          
          \node (s1) at (6.2,0.7)  {}; 
          \node (s0) [above=of s1] {}; 
          \node (s2) [right=of s0] {}; 
          \node (s3) [below=of s2] {}; 
          
          
          \node<-2>  [state]                                                      (q_0) at (s0) {$\alert<2>{0}{\color<2>{example}0}$}; 
          \node[state] (q_1) at (s1) {$\alert<2>{0}{\color<2>{example}1}$}; 
          \node      [state] (q_2) at (s2) {$\alert<2>{1}{\color<2>{example}0}$}; 
          \node[state] (q_3) at (s3) {$\alert<2>{1}{\color<2>{example}1}$}; 
          \path<5->     [->]             (q_0) edge                          node       {$a$} (q_2);
          \path<6->     [->]             (q_2) edge[loop above, looseness=5] node       {$a$} (q_2);
          \path<7->     [->]             (q_2) edge                          node       {$b$} (q_3);
          \path<9->     [->]             (q_3) edge[loop right, looseness=5] node       {$b$} (q_3);

          \mode<beamer>{
          \node<3>   [state,initial, initial text=,alert,fill=alert!20] (q_0) at (s0)  {$00$};
          \node<4>   [state,initial, initial text=,structure,fill=structure!20]   (q_0) at (s0)  {$00$};
          \node<5->  [state,initial, initial text=]                               (q_0) at (s0)  {$00$};
          \node<4-5> [state,alert,fill=alert!20]                        (q_2) at (s2) {$10$}; 
          \node<6>   [state,structure,fill=structure!20]                          (q_2) at (s2) {$10$}; 
          \node<7->  [state]                                                      (q_2) at (s2) {$10$}; 
          \node<6-8> [state,alert,fill=alert!20] (q_3) at (s3) {$11$}; 
          \node<9>   [state, accepting,alert,fill=alert!20] (q_3) at (s3) {$11$}; 
          
          \path<4>      [->, structure]  (q_0) edge                          node       {$a$} (q_2);
          \path<5>      [->, structure]  (q_2) edge[loop above, looseness=5] node       {$a$} (q_2);
 
          \path<6>      [->, structure]  (q_2) edge                          node       {$b$} (q_3);
          \path<7-8>    [->, structure]  (q_3) edge[loop right, looseness=5] node       {$b$} (q_3);
}
        \end{tikzpicture}
    \end{minipage}};

    \draw<2> (10.1,4.7) node{\begin{minipage}{6cm}
        \begin{block}{Variables}
            \SetKwData{VuA}{vuA}
            \SetKwData{VuB}{vuB}
            \begin{itemize}
          \item \alert{\VuA} : a-t-on vu un $a$ ?
          \item \example{\VuB} : a-t-on vu un $b$ ?
          \end{itemize}
        \end{block}
    \end{minipage}};

    \mode<beamer>{
    \draw<3-> (10.1,4.5) node{\begin{minipage}{6cm}
        \begin{block}{Exemple d'exécution}
          \begin{tikzpicture}
            
            \draw[white] (1,0.9) rectangle (4,2);
            \draw (2,1.7) node{aabbb};
            
            \draw<9>[alert] (2.6,1.2) node{Mot reconnu};
            
            \draw<3>[alert,-latex] (1.46,1.4) -- (1.46,1.55);
            \draw<4>[alert,-latex] (1.64,1.4) -- (1.64,1.55);
            \draw<5>[alert,-latex] (1.82,1.4) -- (1.82,1.55);
            \draw<6>[alert,-latex] (2.0,1.4) -- (2.0,1.55);
            \draw<7>[alert,-latex](2.18,1.4) -- (2.18,1.55);
            \draw<8>[alert,-latex](2.36,1.4) -- (2.36,1.55);
            \draw<9>[alert,-latex](2.54,1.4) -- (2.54,1.55);
          \end{tikzpicture}
        \end{block}
    \end{minipage}};
    }
  \end{tikzpicture}
\end{frame}

\endgroup
