% SPDX-License-Identifier: CC-BY-SA-4.0
% Author: Matthieu Perrin
% Part: 
% Section: 
% Sub-section: 
% Frame: 

\begingroup

\begin{frame}{Langage droit d'un état}

  \tfExampleBlock[y=-12mm]{Exemple}{}

  \tfBlock[top=-2mm]{Définition -- Langage droit}{
    Soit $A=\langle \Sigma, Q, I, F, \mu \rangle$ un AFN.\\
    Le \structure{langage droit} d'un état $q \in Q$, noté \alert{$\mathcal{LD}_A(q)$},
    est l'ensemble des mots reconnus par un chemin d'actions menant de $q$ à un état final de $A$. 
    
    $$\alert{\mathcal{LD}_A(q) \eqdef \bigcup_{f\in F} \mathcal{L}_A(q, f)}$$

    \begin{description}
    \item[Remarques :] $\displaystyle\mathcal{L}(A) = \bigcup_{i\in I} \mathcal{LD}_A(i)$ \hspace\fill $\displaystyle\mathcal{LD}_A(q) = \mathcal{L}\left(\langle \Sigma, Q, \{q\}, F, \mu \rangle\right)$
    \end{description}
  }


  \tf[y=-25mm, x=-.2\textwidth]{\small
        \begin{tikzpicture}[smAutomaton]
        \smState[\smInitial]   (s1) at (0.0,1.2) {$q_1$}; 
        \smState[\smAccepting] (s2) at (1.2,1.2) {$q_2$}; 
        \smState[\smInitial]   (s4) at (0.0,0.0) {$q_3$}; 
        \smState[\smExample]   (s5) at (1.2,0.0) {$q_4$}; 

        \smPath (s1) edge             node[above] {$a$} (s2);
        \smPath (s1) edge[loop above] node        {$a$} (s1);
        \smPath (s4) edge             node[above] {$b$} (s5);
        \smPath (s4) edge[loop below] node        {$b$} (s4);
        \smPath (s5) edge[loop below] node        {$b$} (s5);
        \smPath (s2) edge[loop right] node        {$c$} (s2);
        \smPath (s5) edge node                    {$\varepsilon$} (s2);
    \end{tikzpicture}
  }

  \tf[y=-25mm, x=.2\textwidth]{
      \example{
        $\begin{array}{rcl}%
          \mathcal{LD}_A(q_4) &=& \mathcal{L}_A\left(q_4, q_3\right)\\
          &=& b^\star c^\star \\
        \end{array}$
      }
  }
  
\end{frame}


\endgroup
