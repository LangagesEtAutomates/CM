% SPDX-License-Identifier: CC-BY-SA-4.0
% Author: Matthieu Perrin
% Part: 
% Section: 
% Sub-section: 
% Frame: 

\begingroup

\begin{frame}{Automate complet}

  \tfBlock[top=-2mm]{Définition -- AFN complet}{
    Un AFN $A = \langle \Sigma, Q, I, F, \rightarrow \rangle$ est dit \structure{complet} si pour chaque état $q$ et chaque symbole $a$, il existe au moins une transition sortant de $q$ étiquetée $a$ :
    $$\forall q\in Q,  \forall a\in \Sigma, \alert{\exists q' \in Q,~q \xrightarrow{a} q'}$$
  }

  \tfBlock{Théorème -- Complétion d'un automate}{
    Tout AFN est équivalent à un AFN complet.
  }

  \tfBlock[bottom=3mm]{Démonstration}{
    $\mathcal{L}(A) = \mathcal{L}\left(\left\langle \Sigma, Q \cup \{\example{q_\bot}\}, I, F, \rightarrow_B \right\rangle\right)$, avec \\[2mm]
    $\begin{array}{@{}r@{~}c@{~\big\{\,\big\langle}c@{,\,}c@{,\,}c@{\big\rangle\, \big\mid\; }l@{}}
      \rightarrow_B~= & & q     & a & q'    & q \xrightarrow{a} q' \,\big\}\\
      &            \cup & \alert{q}     & \alert{a} & \alert{q_\bot} & \alert{\nexists q'\in Q,~ q \xrightarrow{a} q'} \,\big\}\\
      &            \cup & \example{q_\bot} & \example{a} & \example{q_\bot} & \example{a\in \Sigma} \,\big\}\\
    \end{array}$
  }
  
  \tf[bottom, x=.33\textwidth]{
    \begin{tikzpicture}[smAutomaton]\small
      \smState[\smInitial]   (1) at (0.0,2.4) {$1$}; 
      \smState[\smAccepting] (2) at (2.4,2.4) {$2$}; 
      \smState[\smInitial]   (3) at (0.0,0.0) {$3$}; 
      \smState               (4) at (2.4,0.0) {$4$}; 
      \smState[\smExample]   (5) at (1.2,1.2) {$q_\bot$}; 

      \smPath             (1) edge             node         {$a$}           (2);
      \smPath             (1) edge[loop above] node         {$a$}           (1);
      \smPath             (3) edge             node[swap]   {$b$}           (4);
      \smPath             (3) edge[loop below] node         {$b$}           (3);
      \smPath             (4) edge[loop below] node         {$b$}           (4);
      \smPath             (2) edge[loop above] node         {$c$}           (2);
      \smPath             (4) edge             node[swap]   {$\varepsilon$} (2);
      \smPath[\smAlert]   (1) edge             node[sloped] {$b, c$}        (5);
      \smPath[\smAlert]   (2) edge             node[sloped] {$a, b$}        (5);
      \smPath[\smAlert]   (3) edge             node[sloped] {$a, c$}        (5);
      \smPath[\smAlert]   (4) edge             node[sloped] {$a, c$}        (5);
      \smPath[\smExample] (5) edge[loop left]  node         {$a,b,c$}       (5);
    \end{tikzpicture}
  }

\end{frame}

\endgroup
