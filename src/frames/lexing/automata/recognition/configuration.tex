% SPDX-License-Identifier: CC-BY-SA-4.0
% Author: Matthieu Perrin
% Part: 
% Section: 
% Sub-section: 
% Frame: 

\begingroup


\begin{frame}{Configurations et actions d'un AFN}
  Soit $A=\langle \Sigma, Q, I, F, \mu \rangle$ un AFN. 
  
  \begin{block}{Définition -- configuration d'un AFN}
    Une \structure{configuration} de $A$ est un couple \alert{$\langle u, q \rangle$} tel que :
    \begin{description}
    \item[\alert{$u$}] $\in \Sigma^\star$ : \structure{mot restant à reconnaître}
    \item[\alert{$q$}] $\in Q$ : \structure{l'état courant dans la simulation}
    \end{description}
  \end{block}

  \begin{block}{Définition -- action (ou \structure{calcul}, ou \structure{déplacement})}
    Les \structure{actions}  de $A$ sont définies par une
    relation binaire \alert{$\leadsto_A$} sur les configurations.
    On a \alert{$\langle u, q\rangle \leadsto_A \langle v, q'\rangle$} ssi :
    \begin{enumerate}
    \item il existe \alert{$a \in \Sigma\cup\{\varepsilon\}$} tel que
    \item \alert{$u=a v$} et
    \item \alert{$\langle q, a, q' \rangle \in \mu$}.
    \end{enumerate}
    On note \alert{$\leadsto_A^\star$} \structure{la fermeture transitive et réflexive de $\leadsto_A$}.
    
    \vspace{1mm}On notera \alert{$C \leadsto C'$} et \alert{$C \leadsto^\star C'$} si $A$ est clair d'après le contexte.
  \end{block}

%\begin{itemize}
%  \item État : situation statique dans l’automate (programme)  
%  \item Configuration : situation dynamique dans l’exécution (processus)
%\end{itemize}
  
\end{frame}

\endgroup
