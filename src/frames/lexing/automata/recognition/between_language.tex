% SPDX-License-Identifier: CC-BY-SA-4.0
% Author: Matthieu Perrin
% Part: 
% Section: 
% Sub-section: 
% Frame: 

\begingroup

\begin{frame}{Langage entre états}

  \tfExampleBlock[y=-10mm]{Exemple}{}

  \tf[text, top]{
    Soit $A=\langle \Sigma, Q, I, F, \mu \rangle$ un AFN.
    \begin{block}{Définition -- Langage entre états}
      Le \structure{langage entre deux états} $q$ et $q'$ de $Q$, noté \alert{$\mathcal{L}_A(q, q')$},
      est l'ensemble des mots $u$ tels qu'il existe un chemin d'actions menant de $\langle u, q \rangle$ à $\langle \varepsilon, q'\rangle$.

      \vspace{-2mm}
      $$\alert{\mathcal{L}_A \eqdef \left\{\begin{array}{ccc}
        Q \times Q &\rightarrow& \mathcal{P}\left(\Sigma^\star\right)\\
        q, q' & \mapsto & \{u \in \Sigma^\star | \langle u,q\rangle \leadsto_A^\star \langle\varepsilon,q'\rangle\}
        \end{array}\right.}$$

      \begin{description}
      \item[Remarque :] $\displaystyle\mathcal{L}_A(q, q') = \{u \in \Sigma^\star | \alert{\forall v \in \Sigma^\star}, \langle u\alert{v},q\rangle \leadsto_A^\star \langle \alert{v},q'\rangle\}$
      \end{description}
    \end{block}
  }


  \tf[bottom, x=.4\textwidth]{
    \begin{tikzpicture}[smAutomaton]
      \smState[\smInitial]   (s1) at (0.0,1.2) {$s_1$}; 
      \smState[\smAccepting] (s2) at (1.5,1.2) {$s_2$}; 
      \smState[\smInitial]   (s3) at (0.0,0.0) {$s_3$}; 
      \smState               (s4) at (1.5,0.0) {$s_4$}; 

      \smPath (s1) edge             node[above] {$a$}           (s2);
      \smPath (s1) edge[loop above] node        {$a$}           (s1);
      \smPath (s3) edge             node[swap]  {$b$}           (s4);
      \smPath (s3) edge[loop below] node        {$b$}           (s3);
      \smPath (s4) edge[loop below] node        {$b$}           (s4);
      \smPath (s2) edge[loop right] node        {$c$}           (s2);
      \smPath (s4) edge             node        {$\varepsilon$} (s2);
    \end{tikzpicture}
  }

  \tf[bottom, text=.4\textwidth, x=2mm]{
    $$\begin{array}{rcl}
      \langle aac, s_1 \rangle &\leadsto & \langle ac, s_1 \rangle\\
      \langle ac, s_1 \rangle &\leadsto & \langle c, s_2 \rangle\\
      \langle c, s_2 \rangle &\leadsto & \langle \varepsilon, s_2 \rangle\\
      \\
      \example{\langle aac, s_1 \rangle} & \example{\leadsto^\star} & \example{\langle \varepsilon, s_2 \rangle}
    \end{array}$$
    $$\example{aac \in \mathcal{L}_A(s_1, s_2)}$$
  }

  \tf[bottom, left=.3\textwidth]{
    $$\begin{array}{rcl}
      \langle bbb, s_3 \rangle &\leadsto & \langle bb, s_3 \rangle\\
      \langle bb, s_3 \rangle &\leadsto & \langle b, s_4 \rangle\\
      \langle b, s_4 \rangle &\leadsto & \langle \varepsilon, s_4 \rangle\\
      \\
      \example{\langle bbb, s_3 \rangle} & \example{\leadsto^\star} & \example{\langle \varepsilon, s_4 \rangle}
    \end{array}$$
    $$\example{bbb \in \mathcal{L}_A(s_3, s_4)}$$
  }

\end{frame}


\endgroup
