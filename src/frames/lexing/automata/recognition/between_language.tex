% SPDX-License-Identifier: CC-BY-SA-4.0
% Author: Matthieu Perrin
% Part: 
% Section: 
% Sub-section: 
% Frame: 

\begingroup

\begin{frame}{Langage entre états}

  
  Soit $A=\langle \Sigma, Q, I, F, \mu \rangle$ un AFN.
  \begin{block}{Définition -- Langage entre états}
    Le \structure{langage entre deux états} $q$ et $q'$ de $Q$, noté \alert{$\mathcal{L}_A(q, q')$},
    est l'ensemble des mots $u$ tels qu'il existe un chemin d'actions menant de $\langle u, q \rangle$ à $\langle \varepsilon, q'\rangle$.

    \vspace{-2mm}
    $$\alert{\mathcal{L}_A \eqdef \left\{\begin{array}{ccc}
      Q \times Q &\rightarrow& \mathcal{P}\left(\Sigma^\star\right)\\
      q, q' & \mapsto & \{u \in \Sigma^\star | \langle u,q\rangle \leadsto_A^\star \langle\varepsilon,q'\rangle\}
      \end{array}\right.}$$

    \begin{description}
    \item[Remarque :] $\displaystyle\mathcal{L}_A(q, q') = \{u \in \Sigma^\star | \alert{\forall v \in \Sigma^\star}, \langle u\alert{v},q\rangle \leadsto_A^\star \langle \alert{v},q'\rangle\}$
    \end{description}
  \end{block}
  \vspace{-3mm}

    \begin{exampleblock}{Exemple}
    \noindent\begin{minipage}{.3\textwidth}%
    \scalebox{.8}{\begin{tikzpicture}[shorten >=1pt,node distance=1.5cm,on grid,auto]
        \node [state,initial, initial text=] (s1)   {$s_1$}; 
        \node [state,accepting] (s2) [right=of s1]  {$s_2$}; 

        \node [state,initial, initial text=] (s4) [below=of s1]  {$s_3$}; 
        \node [state=] (s5) [right=of s4]  {$s_4$}; 

        \path [->]    (s1) edge node[above] {a} (s2);
        \path [->]    (s1) edge[loop above, looseness=5] node {a} (s1);

        \path [->]    (s4) edge node[above] {b} (s5);
        \path [->]    (s4) edge[loop below, looseness=5] node {b} (s4);
        \path [->]    (s5) edge[loop below, looseness=5] node {b} (s5);

        \path [->]    (s2) edge[loop right, looseness=5] node {c} (s2);

        \path [->]    (s5) edge node {$\varepsilon$} (s2);
    \end{tikzpicture}}\end{minipage}%
    \begin{minipage}{.4\textwidth}
      $$\begin{array}{rcl}
        \langle aac, s_1 \rangle &\leadsto & \langle ac, s_1 \rangle\\
        \langle ac, s_1 \rangle &\leadsto & \langle c, s_2 \rangle\\
        \langle c, s_2 \rangle &\leadsto & \langle \varepsilon, s_2 \rangle\\
        \\
        \example{\langle aac, s_1 \rangle} & \example{\leadsto^\star} & \example{\langle \varepsilon, s_2 \rangle}
      \end{array}$$
      $$\example{aac \in \mathcal{L}_A(s_1, s_2)}$$
    \end{minipage}%
    \begin{minipage}{.3\textwidth}
      $$\begin{array}{rcl}
        \langle bbb, s_3 \rangle &\leadsto & \langle bb, s_3 \rangle\\
        \langle bb, s_3 \rangle &\leadsto & \langle b, s_4 \rangle\\
        \langle b, s_4 \rangle &\leadsto & \langle \varepsilon, s_4 \rangle\\
        \\
        \example{\langle bbb, s_3 \rangle} & \example{\leadsto^\star} & \example{\langle \varepsilon, s_4 \rangle}
      \end{array}$$
      $$\example{bbb \in \mathcal{L}_A(s_3, s_4)}$$
    \end{minipage}
  \end{exampleblock}
\end{frame}


\endgroup
