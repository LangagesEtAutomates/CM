% SPDX-License-Identifier: CC-BY-SA-4.0
% Author: Matthieu Perrin
% Part: 
% Section: 
% Sub-section: 
% Frame: 

\begingroup

\begin{frame}{Langage reconnu par un AFN}

  \tf[text, top]{
    Soit $A=\langle \Sigma, Q, I, F, \mu \rangle$ un AFN.

    \begin{block}{Définition -- langage reconnu (ou accepté)}

      Un mot d'un langage entre un état initial et un état final est dit \structure{reconnu} par $A$.

      Le langage \structure{reconnu} par $A$ est l'ensemble $\alert{\mathcal{L}(A)}$ des mots reconnus par $A$.

      \alert{$$\begin{array}{lll}
          \mathcal{L}(A) &\eqdef& \displaystyle \bigcup_{i\in I} \bigcup_{f\in F} \mathcal{L}_A(i,f)\\
          &=& \{u \in \Sigma^\star \mid \exists i\in I, \exists f\in F,  \langle u, i \rangle \leadsto_A^\star \langle \varepsilon, f\rangle\}
        \end{array}$$}
    \end{block}
  }

  \tfExampleBlock[bottom]{Exemple}{
    Pour \example{$i=s_1$}, \example{$f=s_2$} et \example{$u=aac$}, on a : 
    \begin{enumerate}
    \item $s_1 \in I$
    \item $s_2 \in F$
    \item $aac \in \mathcal{L}_A(s_1, s_2)$
    \end{enumerate}
    Donc \example{$aac \in \mathcal{L}(A)$}
  }

  \tf[bottom, x=.25\textwidth]{
    \begin{tikzpicture}[smAutomaton]
      \smState[\smInitial]   (s1) at (0.0,1.2) {$s_1$}; 
      \smState[\smAccepting] (s2) at (1.5,1.2) {$s_2$}; 
      \smState[\smInitial]   (s3) at (0.0,0,0) {$s_3$}; 
      \smState               (s4) at (1.5,0,0) {$s_4$}; 

      \smPath (s1) edge             node[above] {$a$}           (s2);
      \smPath (s1) edge[loop above] node        {$a$}           (s1);
      \smPath (s3) edge             node[swap]  {$b$}           (s4);
      \smPath (s3) edge[loop below] node        {$b$}           (s3);
      \smPath (s4) edge[loop below] node        {$b$}           (s4);
      \smPath (s2) edge[loop right] node        {$c$}           (s2);
      \smPath (s4) edge             node        {$\varepsilon$} (s2);
    \end{tikzpicture}
  }
  
\end{frame}


\endgroup
