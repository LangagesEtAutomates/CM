% SPDX-License-Identifier: CC-BY-SA-4.0
% Author: Matthieu Perrin
% Part: 
% Section: 
% Sub-section: 
% Frame: 

\begingroup

\begin{frame}{Automate émondé}
  \begin{block}{Définition -- Propriétés des états}
    Soit $A=\langle \Sigma, Q, I, F, \mu \rangle$ un AFN. Un état $q \in Q$ est dit
    \begin{description}
    \item[accessible] s'il peut être atteint à partir d'un état initial : \alert{$\mathcal{LG}_A(q) \neq \emptyset$}
    \item[inaccessible] s'il n'est pas accessible
    \item[co-accessible] si on peut atteindre un état final à partir de $q$ : \alert{$\mathcal{LD}_A(q) \neq \emptyset$}
    \item[stérile] s'il n'est pas co-accessible
    \item[utile] s'il est accessible et co-accessible
    \item[inutile] s'il n'est pas utile
    \end{description}
  \end{block}
  \begin{block}{Définition -- Automate émondé}
    Un AFN est dit \structure{émondé} si tous ses états sont utiles. 
  \end{block}
\end{frame}


\endgroup
