% SPDX-License-Identifier: CC-BY-SA-4.0
% Author: Matthieu Perrin
% Part: 
% Section: 
% Sub-section: 
% Frame: 

\begingroup

\SetKwFunction{RabinScott}{rabin\_scott}

\begin{frame}{Théorème d'équivalence entre AFN et AFD}

  \onBlock[top, left=48mm]{Théorème}{
    Tout langage reconnaissable par un automate fini  non-déterministe 
    est reconnaissable par un automate fini déterministe
  }

  \onBlock[y=-15mm]{Démonstration}{
    Méthode des sous-ensembles de Rabin et Scott\footnote{Prix Turing 1976 pour leurs travaux sur le non-déterminisme}
    \begin{description}
    \item [Entrée :] automate fini non-déterministe \structure{$A$}
    \item [Sortie :] automate fini déterministe \structure{$\RabinScott(A)$} tel que
      \vspace{-2mm}
      $$\alert{\mathcal{L}(A) = \mathcal{L}(\RabinScott(A))}$$ 
    \end{description}
    \vspace{-3mm}
    \begin{itemize}
    \item Les états de $\RabinScott(A)$ sont des ensembles d'états de $A$ 
    \item $\RabinScott(A)$ est également complet
    \end{itemize}
  }

  \onImage[x=12mm,y=15mm]{%
    height=32mm,
    title={Michael O. Rabin},
    license={{\href{https://creativecommons.org/licenses/by-sa/4.0/}{CC BY-SA}} (\ccbysa{} — \href{https://commons.wikimedia.org/wiki/File:M_O_Rabin.jpg}{Wikimedia})},
    img={Rabin.jpg}
  }

  \onImage[x=42mm,y=15mm]{%
    height=32mm,
    title={Dana S. Scott},
    license={{\href{https://creativecommons.org/licenses/by-sa/4.0/}{CC BY-SA}} (\ccbysa{} — 2003, \href{https://commons.wikimedia.org/wiki/File:Scott_Dana_small.jpg}{Wikimedia})},
    img={Scott.jpg}
  }
  
\end{frame}

\endgroup
