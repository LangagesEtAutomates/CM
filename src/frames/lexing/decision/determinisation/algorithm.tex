% SPDX-License-Identifier: CC-BY-SA-4.0
% Author: Matthieu Perrin
% Part: 
% Section: 
% Sub-section: 
% Frame: 

\begingroup

\SetKwFunction{RabinScott}{rabin\_scott}
\SetKwFunction{Fermeture}{$\varepsilon$-fermeture}

\begin{frame}{Méthode des sous-ensembles de Rabin et Scott}

  \onExampleBlock[right=3.3cm, top=-5mm]{Exemple}{\vspace{-3mm}
    \begin{tikzpicture}[automaton, grid size=10mm]
      \state[initial,                      example on=<2-4>] (0) at (0,1) {$0$};
      \state[accepting,alert ob=<5>                        ] (1) at (1,1) {$1$};
      \state[                                              ] (2) at (2,1) {$2$};
      \state[initial,                      example on=<2-4>] (3) at (0,0) {$3$};
      \state[            structure on=<2>, example ob=<3-4>] (4) at (1,0) {$4$};
      \state[            structure on=<2>, example ob=<3-4>] (5) at (2,0) {$5$};
      
      \path[structure on=<2>] (3) edge             node {$\varepsilon$} (4);
      \path[structure on=<2>] (4) edge             node {$\varepsilon$} (5);
      \path[                ] (2) edge             node {$b$}           (1);
      \path[alert ob=<3>    ] (5) edge node {$a$}           (2);
      \path[                ] (2) edge[bend left ] node {$a$}           (5);
      \path[alert ob=<4>    ] (4) edge[bend left ] node {$b$}           (1);
      \path[                ] (1) edge node {$b$}           (4);
      \path[                ] (1) edge[loop above] node {$a$}           (1);
      \path[                ] (2) edge[loop above] node {$b$}           (2);
      \path[alert ob=<4>    ] (4) edge[loop below] node {$b$}           (4);
      \path[alert ob=<3>    ] (5) edge[loop below] node {$a$}           (5);
    \end{tikzpicture}
  }
 
  \onExampleBlock<2->[right=3.3cm,y=-1mm]{Calcul de $s_0$}{
    $\begin{array}{r@{\,=\,}l}
      s_0 & \Fermeture_A(\{0,3\}) \\
          & \{\example{0}, \example{3}, \structure{4}, \structure{5}\}
    \end{array}$
  }  
  
  \on[top=-2mm]{\small
    \begin{algorithm}[H]
      \Fn{\RabinScott($A = \langle \Sigma, Q, I, F, \rightarrow \rangle$ : AFN) : AFD}{
        $\begin{array}{@{}l@{\,\gets\,}l@{}}
          s_0 & \Alertb<2>{\Fermeture_A(I)};\\
          S & \{s_0\};\\
          \tau & \emptyset;
        \end{array}$\\
        \Tantque{\Alertb<3-4>{$\exists s\in S, \exists a\in \Sigma, \nexists s'\in S,~ \langle s, a, s' \rangle \in \tau$}}{
          $\begin{array}{@{}l@{\,\gets\,}l@{}}
            s' & \Structureb<4>{\Fermeture_A(\{q'\in Q \mid \exists q\in s,~ q \xrightarrow{a} q' \})};\\
            S & S \cup \{s'\};\\
            \tau & \tau \cup \{\langle s, a, s' \rangle\};
          \end{array}$\\
        }
        $S_f \gets \Alertb<5>{\{s \in S \mid s \cap F \neq \emptyset\}}$\;
        \Retourner $\langle \Sigma, S, \{s_0\}, S_f, \tau \rangle$\;
      }
      \Fn{$\Fermeture_A(s \subseteq Q) \subseteq Q$}{
        \lTantque{\Structureb<2>{$\exists q\in s,~ \exists q' \notin s,~ q\xrightarrow{\varepsilon} q'$}}{$s \gets s \cup \{q'\}$}
        \Retourner $s$\;
      }
    \end{algorithm}
  }
 
  \on<2->[bottom=2mm, x=-33mm] {
    \begin{tikzpicture}[automaton, y=8mm, x=4mm]
      \useasboundingbox (-.5,-.5) rectangle (11.5,2.5);
      
      \state    [initial               ] (a) at (0,1) {$s_0$}; 
      \state<4->[                      ] (b) at (2,2) {$s_1$}; 
      \state<5->[accepting,alert ob=<5>] (c) at (2,0) {$s_2$}; 
      \state<6->[accepting             ] (d) at (9,2) {$s_3$}; 
      \state<6->[accepting             ] (e) at (9,0) {$s_4$}; 
      \state<6->[accepting             ] (f) at (7,1) {$s_5$};
      \state<6->[accepting             ] (g) at (11,1) {$s_6$}; 
      \state<6->[                      ] (h) at (4,1) {$s_7$}; 
 
      \onlyb<3,4>{
        \state  [alert] (empty) at (4,1) {$?$};
        \path<3>[alert] (a) edge node {$a$} (empty);
        \path<4>[alert] (a) edge node {$b$} (empty);
      }
 
      \path<4-> (a) edge[bend left ] node[swap] {$a$} (b);
      \path<5-> (a) edge[bend right] node       {$b$} (c);
      \path<6-> (b) edge[loop above] node       {$a$} (b);
      \path<6-> (b) edge             node       {$b$} (d);
      \path<6-> (c) edge             node       {$a$} (e);
      \path<6-> (c) edge[loop below] node       {$b$} (c);
      \path<6-> (d) edge[bend right] node       {$a$} (f);
      \path<6-> (d) edge[bend left ] node[swap] {$b$} (g);
      \path<6-> (e) edge[loop below] node       {$a$} (e);
      \path<6-> (e) edge[bend right] node[swap] {$b$} (g);
      \path<6-> (f) edge[bend right] node       {$a$} (e);
      \path<6-> (f) edge             node[swap] {$b$} (h);
      \path<6-> (g) edge[bend right] node[swap] {$a$} (e);
      \path<6-> (g) edge[loop right] node       {$b$} (g);
      \path<6-> (h) edge[bend right] node       {$a$} (b);
      \path<6-> (h) edge[bend left ] node[swap] {$b$} (c);
    \end{tikzpicture}
  }
  
  \on<2->[bottom, x=.27\textwidth]{\footnotesize
    \begin{tabular}{|l|l|l|}
      \hline
      &
      \multicolumn{2}{c|}{\rule{0pt}{1em}\structure{\textbf{Entrées de $\Sigma$}}}\\
      \hline
      \rule{0pt}{1.1em}\structure{\textbf{\'Etats de $S$}} &
      \multicolumn{1}{c|}{$\structure{a}$} &
      \multicolumn{1}{c|}{$\structure{b}$} \\
      \hline
      \rule{0pt}{1em}$s_0=\{\example<2>{0},\example<2>{3},\structure<2>{4},\structure<2>{5}\}$ \hspace\fill\structure{i}  &
      \onlyb<3>{\alert{?}}\only<4-> {$s_1=\{5, 2\}$} &
      \onlyb<4> {\alert{?}}\only<5-> {$s_2=\{1,4,5\}$} \\
      \uncover<4->{$s_1=\{5, 2\}$} &
      \uncover<6->{$s_1=\{5, 2\}$} &
      \uncover<6->{$s_3=\{1, 2\}$} \\
      \uncover<5->{$s_2=\{\alert<5>{1},4,5\}$ \hspace\fill\structure{f}} &
      \uncover<6->{$s_4=\{1,2,5\}$} &
      \uncover<6->{$s_2=\{1,4,5\}$} \\
      \uncover<6->{$s_3=\{1,2\}$ \hspace\fill\structure{f}} &
      \uncover<6->{$s_5=\{1,5\}$} &
      \uncover<6->{$s_6=\{1,2,4,5\}$} \\
      \uncover<6->{$s_4=\{1,2,5\}$ \hspace\fill\structure{f}} &
      \uncover<6->{$s_4=\{1,2,5\}$} &
      \uncover<6->{$s_6=\{1,2,4,5\}$} \\
      \uncover<6->{$s_5=\{1,5\}$ \hspace\fill\structure{f}} &
      \uncover<6->{$s_4=\{1,2,5\}$} &
      \uncover<6->{$s_7=\{4,5\}$} \\
      \uncover<6->{$s_6=\{1,2,4,5\}$ \hspace\fill\structure{f}} &
      \uncover<6->{$s_4=\{1,2,5\}$} &
      \uncover<6->{$s_6=\{1,2,4,5\}$} \\
      \uncover<6->{$s_7=\{4,5\}$} &
      \uncover<6->{$s_1=\{5,2\}$} &
      \uncover<6->{$s_2=\{1,4,5\}$} \\
      \hline
    \end{tabular}
  }
 
\end{frame}

\endgroup
