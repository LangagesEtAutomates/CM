% SPDX-License-Identifier: CC-BY-SA-4.0
% Author: Matthieu Perrin
% Part: 
% Section: 
% Sub-section: 
% Frame: 

\begingroup


\begin{frame}{Exemple}

  \on[y=22mm, x=-.25\textwidth]{
    \begin{tikzpicture}[automaton, grid size=10mm]
        \state[initial]   (a) at (0,1) {$0$};
        \state[example]   (b) at (1,2) {$1$};
        \state[example]   (c) at (1,0) {$2$};
        \state[accepting] (d) at (2,1) {$3$};
        
        \path  (a) edge[bend left ] node[swap] {$a$} (b);
        \path  (a) edge[bend right] node {$b$} (c);
        \path  (b) edge[bend left ] node[swap] {$b$} (d);
        \path  (c) edge[bend right] node {$b$} (d);
        \path  (b) edge[loop below] node {$a$} (b);
        \path  (c) edge[loop above] node {$a$} (c);
        \path  (d) edge[loop right] node {$a, b$} (d);
    \end{tikzpicture}
  }

  \on[y=22mm, right=.5\textwidth]{
    $\left\{
    \begin{array}{lcl}
      \mathcal{L}_A(0) & = & (a|b) a^\star b (a|b)^\star\\
      \example{\mathcal{L}_A(1)} & = & \example{a^\star b (a|b)^\star}\\
      \example{\mathcal{L}_A(2)} & = & \example{a^\star b (a|b)^\star}\\
      \mathcal{L}_A(3) & = & (a|b)^\star\\
    \end{array}
    \right.$
  }
  
  \on[text,y=7mm]{
    \begin{itemize}
    \item Les états $1$ et $2$ ont le même langage droit, on les fusionne en $\{1,2\}$
    \end{itemize}
  }

  \on[y=-8mm, x=-.25\textwidth]{
    \begin{tikzpicture}[automaton]
        \state[initial]   (a) at (0.0,0.0) {$0$};
        \state[example]   (b) at (1.2,0.0) {$1, 2$};
        \state[accepting] (d) at (2.4,0.0) {$3$};
        
        \path (a) edge             node {$a, b$} (b);
        \path (b) edge             node {$b$} (d);
        \path (b) edge[loop below] node {$a$} (b);
        \path (d) edge[loop right] node {$a, b$} (d);
    \end{tikzpicture}
  }

  \on[y=-8mm, right=.5\textwidth]{
    $\left\{
    \begin{array}{lcl}
      \mathcal{L}_A(\{0\}) & = & (a|b) a^\star b (a|b)^\star\\
      \example{\mathcal{L}_A(\{1, 2\})} & \example{=} & \example{a^\star b (a|b)^\star}\\
      \mathcal{L}_A(\{3\}) & = & (a|b)^\star\\
    \end{array}
    \right.$
  }

  \onAlertBlock[bottom]{Algorithme de minimisation}{
    \begin{itemize}
    \item Deux états sont \structure{équivalents} s'ils ont le même langage droit
    \item Les états de l'automate minimal sont les classes d'équivalence
    \end{itemize}
    \begin{description}
    \item [Question :] comment calculer les classes d'équivalence ?
    \end{description}
  }  
  
\end{frame}

\endgroup
