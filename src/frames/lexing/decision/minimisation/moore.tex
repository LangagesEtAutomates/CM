% SPDX-License-Identifier: CC-BY-SA-4.0
% Author: Matthieu Perrin
% Part: 
% Section: 
% Sub-section: 
% Frame: 

\begingroup

\SetKwFunction{Moore}{moore}

\begin{frame}{L'algorithme de Moore}

  \on[text,top=-2mm]{\small
    \begin{algorithm}[H]
      \Fn{\Moore( $A = \langle \Sigma, Q, I, F, \rightarrow \rangle$ : AFD ) : AFD}{
        \Alertb<2>{$S \leftarrow \{F, Q\setminus F\}$}\;
        \Repeter{
          $\tau \gets \left\{ \left\langle s, a, s' \right\rangle \in S \times \Sigma \times S \,\middle\mid\, \exists q\in s, \exists q'\in s',~ q\xrightarrow{a} q'\right\} $\;
          \eSi{\Alertb<3>{$\exists \langle s, a, s_1 \rangle, \langle s, a, s_2 \rangle \in \tau,~s_1 \neq s_2$}}{
            $\begin{array}{@{}l@{\,\gets\,}l@{}}
              s'_1 & \left\{q \in s \,\middle\mid\, \exists q' \in s_1, q \xrightarrow{a}q'\right\};\\
              s'_2 & \left\{q \in s \,\middle\mid\, \exists q' \in s_2, q \xrightarrow{a}q'\right\};\\
              S    & S \setminus \left\{s\right\} \cup \left\{s'_1, s'_2\right\};
            \end{array}$
          }{
            $\begin{array}{@{}l@{\,\gets\,}l@{}}
              S_0 & \left\{ s \in S \,\middle\mid\, I \subseteq s\right\};\\
              S_f & \left\{ s \in S \,\middle\mid\, s \cap F \neq \emptyset\right\};
            \end{array}$\\
            \Retourner $\langle \Sigma, S, S_0, S_f, \tau \rangle$\;
          }
        }
      }
    \end{algorithm}
  }

  \onBlock[y=-10mm, anchor=north]{Un algorithme optimiste}{
    \vspace{-2mm}
    \begin{itemize}
    \item  Peut-être que tous les états sont équivalents ? 
    \item<2-> Non : seuls les états finaux reconnaissent $\varepsilon$
      \begin{itemize}
      \item Séparer les états finaux et les autres
      \end{itemize}
    \item<3-> Essayer de placer les transitions
      \begin{itemize}
      \item<4-> Partitionner tant qu'on n'y arrive pas
      \end{itemize}
    \end{itemize}
  }
  
  \onExampleBlock[top=-5mm, right=.29\textwidth]{Exemple}{
    \begin{tikzpicture}[automaton, grid size=8mm]
      \state[initial,  example, structure ob=<2-3>] (0) at (0,1) {$0$};
      \state[          example, structure on=<2>  ] (1) at (1,2) {$1$};
      \state[          example, structure on=<2>  ] (2) at (1,0) {$2$};
      \state[accepting,example, alert on=<2->     ] (3) at (2,1) {$3$};
      
      \path                   (0) edge[bend left ] node       {$a$} (1);
      \path[structure ob=<3>] (0) edge[bend right] node[swap] {$b$} (2);
      \path[alert ob=<3>]     (1) edge[bend left ] node       {$b$} (3);
      \path[alert ob=<3>]     (2) edge[bend right] node[swap] {$b$} (3);
      \path                   (1) edge[loop below] node       {$a$} (1);
      \path                   (2) edge[loop above] node       {$a$} (2);
      \path                   (3) edge[loop right] node       {$a, b$} (3);
    \end{tikzpicture}
  }
  
  \onExampleBlock[y=-3mm, right=.29\textwidth]{Automate minimal}{\scriptsize
    \begin{tikzpicture}[automaton, grid size=10mm]
      \state[initial, example, structure ob=<2-3>] (D) at (0.0,0.0) {\alt<4->{$D$}{\alt<1>{$A$}{$B$}}};
      \uncover<2->{
        \state[accepting, alert] (C) at (2.0,0.0) {$C$};
      }
      \uncover<3->{
        \path                (C) edge[loop right] node[right] {$a, b$} (C);
      }
      \uncoverb<3>{
        \path[structure]  (D) edge[loop right] node[above] {$b$?}    (D);
        \path[alert]      (D) edge             node[above] {$b$?}    (C);
        \path             (D) edge[loop below] node[below] {$a$}     (D);
      }
      \uncover<4->{
        \state[structure] (E) at (1,0.0) {$E$};
        \path                (D) edge             node[above] {$a$} node[below] {$b$} (E);
        \path                (E) edge             node[above] {$b$}                   (C);
        \path                (E) edge[loop below] node[below] {$a$}                   (E);
      }
    \end{tikzpicture}
  }
  
  \onExampleBlock[y=-10mm, anchor=north, right=.29\textwidth]{Partitionnement}{
    \begin{tikzpicture}[tree,x=15mm,y=8mm]
      \tree    [node=example,   name=a,               ]{$A = \{0,1,2, 3\}$}{
        \tree  [node=structure, name=b,        on=<2->]{$B = \{0,1,2\}$}{
          \tree[node=example,                  on=<4->]{$D = \{0\}$}{}
          \tree[node=structure,                on=<4->]{$E = \{1,2\}$}{}
        }
        \tree  [node=alert,     xshift=-.5,               on=<2->]{$C = \{3\}$}{}
      }
      \node[on=<2->] at([yshift=-1mm]a) {$\varepsilon$};
      \node[on=<4->] at([yshift=-1mm]b) {$b$};
    \end{tikzpicture}
  }

\end{frame}

\endgroup
