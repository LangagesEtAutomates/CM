% SPDX-License-Identifier: CC-BY-SA-4.0
% Author: Matthieu Perrin
% Part: 
% Section: 
% Sub-section: 
% Frame: 

\begingroup

\SetKwFunction{Thompson}{thompson}
\SetKwData{Motif}{motif}

\begin{frame}{Génération d'un automate}

  \onBlock[top, left=80mm]{Théorème}{
    Tout langage rationnel
    est reconnaissable par un automate fini non-déterministe
    $$ \alert{\textsc{rat}_\Sigma \subseteq  \textsc{rec}_\Sigma} $$
  }
    
  \onBlock[bottom=7mm]{Démonstration}{
    Algorithme de Thompson\footnote{Prix Turing 1983 avec Dennis Ritchie pour le développement d'Unix}
    \begin{description}
    \item [Entrée :] expression rationnelle \structure{$\Motif \in \textsc{rat}_\Sigma$}
    \item [Sortie :] automate fini non-déterministe \structure{$\Thompson(\Motif)$} tel que
      $$\alert{\mathcal{L}(\Motif)} = \mathcal{L}(\Thompson(\Motif))$$
      \vspace{-5mm}
      \begin{itemize}
      \item par récurrence sur la structure de $\Motif$
      \item construit des automates \alert{normalisés}
      \end{itemize}
    \end{description}
  }

  \onImage[x=40mm,y=15mm]{%
    width=3cm,
    title={Ken Thompson},
    license={{\href{https://creativecommons.org/licenses/by-sa/4.0/}{CC BY-SA}} (\ccbysa{} —  A.C.Diller, 2019, \href{https://commons.wikimedia.org/wiki/File:Brian_Kernighan_and_Ken_Thompson.jpg}{Wikimedia})},
    img={Thompson.jpg}
  }
\end{frame}

\endgroup
