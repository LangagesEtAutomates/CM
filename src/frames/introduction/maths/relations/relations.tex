% SPDX-License-Identifier: CC-BY-SA-4.0
% Author: Matthieu Perrin
% Part: 
% Section: 
% Sub-section: 
% Frame: 

\begingroup

\begin{frame}{Relation}
  Soient $A$ et $B$ deux ensembles. 
  \begin{block}{Définition}
    \begin{itemize}
    \item Une \structure{relation (binaire)} est un \alert{sous-ensemble de $A \times B$}. 
    \item Une relation relation est \structure{homogène} si $A = B$. 
    \item Soit $\bowtie$ une relation binaire, on note parfois \structure{$a \,\bowtie\, b$} si \alert{$\langle a, b \rangle \in \bowtie$}. 
    \end{itemize}
  \end{block}
  \begin{block}{Remarque : définition par compréhension}
    Une relation binaire peut être vue comme un prédicat sur deux arguments
  \end{block}
  
  \begin{exampleblock}{Exemple}
    Soit $E$ un ensemble. L'inclusion est une relation homogène sur $\mathscr{P}(E)$ : 
    $$\subseteq \,\eqdef \{\langle A, B \rangle \in \mathscr{P}(E) \times \mathscr{P}(E) \,|\, A \subseteq B \}$$ 
  \end{exampleblock}

  \vspace{-2mm}
  \begin{itemize}
  \item Généralisation : Une \structure{relation $n$-aire} est un \alert{ensemble de $n$-uplets}.
  \end{itemize}
\end{frame}


\endgroup
