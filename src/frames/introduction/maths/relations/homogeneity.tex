% SPDX-License-Identifier: CC-BY-SA-4.0
% Author: Matthieu Perrin
% Part: 
% Section: 
% Sub-section: 
% Frame: 

\begingroup


\begin{frame}{Propriétés des relations homogènes}
  Soit $\bowtie$ une relation binaire homogène sur un ensemble $E$.
  \begin{block}{Définitions -- Propriétés usuelles}
    \vspace{-1mm}
    \begin{description}
    \item[Reflexivité]  $\forall x\in E, \alert{x\bowtie x}$
    \item[Symétrie]     $\forall x\in E, \forall y\in E, \alert{x \bowtie y \Rightarrow y \bowtie x}$
    \item[Asymétrie]    $\forall x\in E, \forall y\in E, \alert{x \bowtie y \Rightarrow \lnot(y \bowtie x)}$
    \item[Antisymétrie] $\forall x\in E, \forall y\in E, \alert{x \bowtie y \land y \bowtie x \Rightarrow x=y}$
    \item[Transitivité] $\forall x\in E, \forall y\in E, \forall z\in E, \alert{x \bowtie y \land y \bowtie z \Rightarrow x \bowtie z}$
    \item[Totalité]     $\forall x\in E, \forall y\in E, \alert{x \bowtie y \lor y \bowtie x}$
    \end{description}
  \end{block}

  \vspace{-1mm}
  \begin{block}{Définition -- Fermeture réflexive et transitive}
    La \structure{fermeture réflexive et transitive} de $\bowtie$, notée $\bowtie^\star$, est la plus petite relation réflexive et transitive contenant $\bowtie$. 

    \vspace{-2mm}
    $$\alert{\bowtie^\star = \bigcap \{ \mathrel{R} \in\mathscr{P}(E\times E) | \bowtie \subseteq \mathrel{R} \land \mathrel{R} \text{ réflexive } \land \mathrel{R} \text{ transitive}\}}$$

    Informellement, \example{$x  \bowtie^\star y$} si \example{$x = y$} ou \example{$x = a_1 \bowtie a_2 \bowtie ... \bowtie a_n = y$}.
  \end{block}
\end{frame}

\endgroup
