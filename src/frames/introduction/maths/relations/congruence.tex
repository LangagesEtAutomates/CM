% SPDX-License-Identifier: CC-BY-SA-4.0
% Author: Matthieu Perrin
% Part: 
% Section: 
% Sub-section: 
% Frame: 

\begingroup

\begin{frame}{Exemple : congruence sur les entiers}

  Soit $k\in \mathbb{Z}$. Définissons la relation $\sim_k$ sur $\mathbb{Z}$ comme suit:

  $$\alert{\sim_k \eqdef \{\langle a, b\rangle \in \mathbb{Z}\times \mathbb{Z} | \exists i\in \mathbb{Z}, a = b + i k \}}.$$
  
  Montrons que $\sim_k$ est une relation d'équivalence sur $\mathbb{Z}$.
  \begin{description}
  \item[Réflexivité]  Montrons que \alert{$\forall a \in \mathbb{Z}, a \sim_k a$}
    \begin{itemize}
    \item Soit \structure{$a\in \mathbb{Z}$}.
    \item On a $a=a + 0k$, donc $\structure{a\sim_k a}$.
    \end{itemize}
  \item[Transitivité] Montrons que \alert{$\forall a, b, c \in \mathbb{Z}, a \sim_k b \land b \sim_k c \Rightarrow a \sim_k c$}
    \begin{itemize}
    \item Soient $\structure{a, b, c \in \mathbb{Z}}$, tels que $\structure{a \sim_k b}$ et $\structure{b \sim_k c}$. 
    \item Il existe $i, j \in \mathbb{Z}$ tels que $a = b + i k$ et $b = c + j k$.
    \item Donc $a = (c + j k) + i k = c + (i+j) k$, donc $\structure{a \sim_k c}$.
    \end{itemize}
  \item[Symétrie] Montrons que \alert{$\forall a, b \in \mathbb{Z}, a \sim_k b \Rightarrow b \sim_k a$}
    \begin{itemize}
    \item Soient $\structure{a, b \in \mathbb{Z}}$, tels que $\structure{a \sim_k b}$. 
    \item Il existe $i \in \mathbb{Z}$ tels que $a = b + i k$.
    \item Donc $b = a + (-i) k$, donc $\structure{b \sim_k a}$.
    \end{itemize}
  \end{description}
\end{frame}

\endgroup
