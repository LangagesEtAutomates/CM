% SPDX-License-Identifier: CC-BY-SA-4.0
% Author: Matthieu Perrin
% Part: 
% Section: 
% Sub-section: 
% Frame: 

\begingroup

\begin{frame}{Ensembles}
  \begin{block}{Définition -- ensemble}
    \begin{itemize}
    \item Un \structure{ensemble} est une collection non-ordonnée d'objets uniques.
    \item Chaque objet d'un ensemble est appelé un \structure{élément} de cet ensemble.
    \item Soit $E$ un ensemble, un élément $a$ de $E$ \structure{appartient à} $E$, noté \structure{$a \in E$}
    \end{itemize}
  \end{block}
  \begin{exampleblock}{Exemples}
    \begin{itemize}
    \item \example{$\mathbb{N} = \{0, 1, 2, 3, \dots\}$} : l'ensemble des entiers naturels
    \item \example{$3\in \mathbb{N}$} est vrai, $\pi \in \mathbb{N}$ est faux, noté \example{$\pi \notin \mathbb{N}$}
    \item \example{$\forall x, x\notin\emptyset$} : $\emptyset$ est appelé l'\structure{ensemble vide}
    \end{itemize}
  \end{exampleblock}
  \begin{alertblock}{Attention}
    Il y a des règles sur quelles collections sont ou non des ensembles !
    \begin{itemize}
    \item Ces règles sont appelées \alert{axiomes} (par exemple, ZFC)
    \item On admet l'existence des ensembles usuels ($\mathbb{N}$, $\mathbb{R}$, fonctions, etc)
    \item L'étude des axiomes n'est ni l'objet de ce cours, ni un prérequis
    \end{itemize}
  \end{alertblock}
\end{frame}
\endgroup
