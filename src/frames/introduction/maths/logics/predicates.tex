% SPDX-License-Identifier: CC-BY-SA-4.0
% Author: Matthieu Perrin
% Part: 
% Section: 
% Sub-section: 
% Frame: 

\begingroup

\begin{frame}{Calcul des prédicats}
  \begin{block}{Définition -- prédicat}
    Un \structure{prédicat} est une \alert{formule logique paramétrée par un/des paramètre(s)} 
    \begin{itemize}
    \item En français, groupe verbal à l'infinitif 
    \item Les paramètres sont des variables 
    \item Instanciation possible des paramètres par des \alert{arguments}
    \item La \alert{valeur de vérité} dépend des arguments
    \end{itemize}
  \end{block}
  \begin{exampleblock}{Exemples}
    \begin{description}
    \item [Être bleu] $ \begin{array}[t]{lclcl}
      \mathit{Bleu}(x) &\eqdef& x \text{ est bleu}&&\\
      \mathit{Bleu}(\text{le ciel}) &\equiv& \text{le ciel est bleu} &\equiv& \text{vrai} \\
      \mathit{Bleu}(\text{le tableau}) &\equiv& \text{le tableau est bleu} &\equiv& \text{faux}
    \end{array}$
    \item [Être inférieur à] $\begin{array}[t]{lclcl}
      \mathit{Inf}(x, y)  &\eqdef& x < y &&\\
      \mathit{Inf}(3, 7)  &\equiv& 3 < 7  &\equiv& \text{vrai} \\
      \mathit{Inf}(12, 7) &\equiv& 12 < 7 &\equiv& \text{faux}
    \end{array}$
    \end{description}
  \end{exampleblock}  
  
\end{frame}


\endgroup
