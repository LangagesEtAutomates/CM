% SPDX-License-Identifier: CC-BY-SA-4.0
% Author: Matthieu Perrin
% Part: 
% Section: 
% Sub-section: 
% Frame: 

\begingroup

\begin{frame}{Bibliographie}
  
  \begin{block}{Livres principaux}\footnotesize
    \begin{itemize}
    \item K. D. Cooper \& L. Torczon. \textit{\structure{Engineering a compiler} -- $2^\text{nd}$ Ed.}, Morgan Kaufmann 2013
    \item P. Wolper. \textit{\structure{Introduction à la calculabilité} -- $3^\text{nd}$ Ed.}, Dunod 2006
    \item O. Carton. \textit{Langages Formels -- Calculabilité et complexité}, Vuibert 2008
    \end{itemize}
  \end{block}

  \begin{block}{Autres livres}\footnotesize
    \begin{itemize}
    \item J.-M. Autebert, \textit{Langages algébriques}, Masson 1989
    \item A. Aho, R. Sethi \& J. Ullman. \textit{Compilateurs : principes, techniques et outils}, InterEditions, 1991. Voir aussi la seconde édition en anglais
    \item J.-M. Autebert, \textit{Théorie des langages et des automates}, Masson 1994
    \item P. Linz, \textit{Formal Languages and Automata}, Jones and Barnett Publishers, 2006
    \item P. Séébold, \textit{Fondamentaux de la théorie des automates}, Ellipses, 2020
    \end{itemize}
  \end{block}

  \begin{block}{Outil pédagogique}\footnotesize
    \begin{itemize}
    \item S. H. Rodger. \textit{JFLAP}. \url{http://www.jflap.org/} 
    \end{itemize}
  \end{block}

\end{frame}

\endgroup
