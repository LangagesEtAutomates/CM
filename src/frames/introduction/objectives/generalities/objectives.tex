% SPDX-License-Identifier: CC-BY-SA-4.0
% Author: Matthieu Perrin
% Part: 
% Section: 
% Sub-section: 
% Frame: 

\begingroup

\begin{frame}{Généralités}
  
  \vspace{-1mm}
  \begin{block}{Prérequis}
    \begin{description}[Mathématiques :]
    \item[Mathématiques :]\vspace{-1mm} logique, théorie des ensembles
    \item[Informatique :] algorithmique, programmation impérative 
    \end{description}
  \end{block}

  \vspace{-1mm}
  \begin{block}{Objectifs théoriques}
    \begin{itemize}
    \item Comprendre les fondements mathématiques de l'informatique théorique
      \begin{itemize}
      \item Structures de mots et de langages
      \item Notion de non-déterminisme
      \item Frontières entre types de langages
      \end{itemize}
    \item Comprendre le fonctionnement de la partie avant des compilateurs
      \begin{description}[Analyse syntaxique :]
      \item[Analyse lexicale :] théorie des automates finis
      \item[Analyse syntaxique :] théorie des automates à pile 
      \end{description}
    \end{itemize}
  \end{block}

  \vspace{-1mm}
  \begin{block}{Objectifs pratiques}
    \begin{itemize}
    \item Créer un interpréteur en Java pour le langage d'algorithmique 
      \begin{itemize}
      \item JFlex et expressions rationnelles
      \item CUP et grammaires algébriques
      \end{itemize}
    \end{itemize}
  \end{block}

\end{frame}

\endgroup
