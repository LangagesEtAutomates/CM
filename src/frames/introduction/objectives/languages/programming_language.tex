% SPDX-License-Identifier: CC-BY-SA-4.0
% Author: Matthieu Perrin
% Part: 
% Section: 
% Sub-section: 
% Frame: 

\begingroup

\begin{frame}{Exemple : un langage de programmation}

  C++ a également un lexique et une syntaxe.
  \begin{block}{Lexique du C++}
    \begin{description}
    \item[Mots-clés] \lstinline{if}, \lstinline{while}, \lstinline{class}, \lstinline{typedef}, ...
    \item[Identifiants :] \lstinline{x}, \lstinline{age}, \lstinline{number\_of\_participants}
    \item[Constantes :] \lstinline{5}, \lstinline{-3.14}, \lstinline{0x3A88C6}, \lstinline{'\\t'}, \lstinline{"Hello world"}, ...
    \item[Symboles :] \lstinline{(}, \lstinline{)}, \lstinline{\;}, \lstinline{+}, ...
    \end{description}
    \alert{Remarque :} le lexique du C++ est infini... mais ``simple'' $\rightarrow$ \structure{langage rationnel}
  \end{block}
  
  \begin{block}{Syntaxe du C++}
    \begin{description}
    \item[Expressions :] \lstinline{(-b + std::sqrt(b * b - 4 * a * c)) / (2 * a)}
    \item[Structures de contrôle :] \lstinline{if(condition()) \{p\_true()\;\} else \{p\_false()\;\}}
    \item[Structures de données :] \lstinline{struct pair \{int x\; int y;\}\;}
    \end{description}
    \alert{Remarque :} syntaxe définie récursivement $\rightarrow$ \structure{langage algébrique}
  \end{block}

\end{frame}

\endgroup
