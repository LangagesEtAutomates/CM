% SPDX-License-Identifier: CC-BY-SA-4.0
% Author: Matthieu Perrin
% Part: 
% Section: 
% Sub-section: 
% Frame: 

\begingroup

\begin{frame}{Exemple : un langage naturel}

  \begin{block}{Lexique français}
    \begin{description}
    \item[Langage :] L'ensemble des mots du \structure{dictionnaire}
    \item[Alphabet :] $\Sigma_{LF} \eqdef \{\text{`a'}, ...,  \text{`z'}, \text{`à'}, ..., \text{`ü'}, \text{`ç'}, \text{`-'}, \text{`\textquotesingle'}\}$
    \item[Mots :] $\{\text{``a''}, \text{``à''}, \text{``abaissement''}, ..., \text{``zut''}, \text{``zygomatique''}, \text{``zygote''}\}$
    \end{description}
  \end{block}
  
  \pause
  \begin{block}{Syntaxe française}
    \begin{description}
    \item[Langage :] L'ensemble des phrases/textes \structure{grammaticalement} corrects
    \item[Alphabet :] $\Sigma_{SF} \eqdef \Sigma_{LF} \cup \{\text{`A'}, ..., \text{`Z'}, \text{`~'}, \text{`.'}, \text{`?'}, ...\}$
    \item[Mots :] $\{\text{``A beau mentir qui vient de loin''}, \text{``À bon chat, bon rat''}, ...\}$
    \end{description}
  \end{block}
  
  \pause
  \begin{alertblock}{Sémantique française}
    \begin{itemize}
    \item \alert{Attention :} `a' (lettre) $\neq$ ``a'' (mot) $\neq$ a (conjugaison de avoir)
    \item On s'intéressera peu à la sémantique dans ce cours
    \item On n'écrira généralement pas les guillemets
    \end{itemize}
  \end{alertblock}

\end{frame}

\endgroup
