% SPDX-License-Identifier: CC-BY-SA-4.0
% Author: Matthieu Perrin
% Part: 
% Section: 
% Sub-section: 
% Frame: 

\begingroup

\begin{frame}{Et les automates ?}

  \on[text, top=-2mm]{
    \begin{description}
    \item[Langage :] abstraction d'un problème binaire
    \item[Automate :] abstraction d'une solution au problème binaire
      \begin{itemize}
      \item Modèle mathématique d'une structure algorithmique
      \item Différents types selon l'organisation de la mémoire
      \end{itemize}
    \end{description}
  }
 
  \onBlock<2->[y=10mm]{La hiérarchie de Chomsky}{}
 
  \on<2->[x=-15mm, y=-15mm]{\small
    \begin{tikzpicture}[2Darray]
      \arrayColumn[header=Langage,width=28mm]{
        \arrayLine[anchor=center,]{Récursivement\\ énumérable}
        \arrayLine[anchor=center,]{Contextuel}
        \arrayLine[structure]{Algébrique}
        \arrayLine[structure]{Rationnel}
      }
      \arrayColumn[header=Automate,width=28mm]{
        \arrayLine{Machine de Turing}
        \arrayLine{Automate\\linéairement borné}
        \arrayLine[structure]{Automate à pile\\non déterministe}
        \arrayLine[structure]{Automate fini}
      }
      \arrayColumn[header=Exemples,width=28mm]{
        \arrayLine{Primalité, \\Sudoku...}
        \arrayLine{Langues\\naturelles}
        \arrayLine[structure]{Langages de\\programmation}
        \arrayLine[structure]{Identifiants,\\ nombres...}
      }
    \end{tikzpicture}
  }
  
  \onImage<2->[x=45mm,y=-17mm]{%
    width=2.2cm,
    title={Noam Chomsky},
    license={{\href{https://creativecommons.org/licenses/by-sa/4.0/}{CC BY-SA}} (\ccbysa{} — 2017, \href{https://commons.wikimedia.org/wiki/File:Noam_Chomsky_portrait_2017_retouched.png}{Wikimedia})},
    img={Chomsky.jpg}
  }
  
\end{frame}

\endgroup
