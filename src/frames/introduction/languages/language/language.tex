% SPDX-License-Identifier: CC-BY-SA-4.0
% Author: Matthieu Perrin
% Part: 
% Section: 
% Sub-section: 
% Frame: 

\begingroup

\begin{frame}{Langage}

  Soit $\Sigma$ un alphabet.

  \begin{block}{Langage}
    \begin{itemize}
    \item Un \structure{langage} sur $\Sigma$ est un sous-ensemble de $\Sigma^\star$.
    \item Ainsi, \structure{l'ensemble des langages} sur $\Sigma$ est \alert{$\mathscr{P}(\Sigma^\star)$}.
    \item Par compréhension, un langage représente un prédicat.
    \end{itemize}
  \end{block}

  \begin{exampleblock}{Exemples de langages}
    \begin{itemize}
    \item \example{Langage vide :} $\emptyset \in \mathscr{P}(\Sigma^\star)$
    \item \example{Langage neutre :} $\{\varepsilon\} \in \mathscr{P}(\Sigma^\star)$
    \item Un \example{langage fini} est un ensemble fini de mots (\example{ex :} lexique français)
    \item Un \example{langage infini} est un langage qui n'est pas fini
    \item $\forall n\in \mathbb{N}, \Sigma^n \in \mathscr{P}(\Sigma^\star)$
    \item $\{ u \in \Sigma^\star \mid u_1 = a\} \in \mathscr{P}(\Sigma^\star)$, ...
    \end{itemize}

  \end{exampleblock}

\end{frame}

\endgroup
