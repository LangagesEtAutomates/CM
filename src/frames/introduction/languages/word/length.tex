% SPDX-License-Identifier: CC-BY-SA-4.0
% Author: Matthieu Perrin
% Part: 
% Section: 
% Sub-section: 
% Frame: 

\begingroup

\begin{frame}{Longueur et occurrences}
  
  Soient $\Sigma$ un alphabet, $a\in\Sigma$, $u\in \Sigma^\star$ et $i\in \llbracket |u|\rrbracket$.
  \begin{itemize}
  \item La \structure{longueur} de $u$, notée $\alert{|u|}$, est l'unique $n$ tel que $u\in \Sigma^n$
  \item On note \alert{$u_i = u(i)$} le $i^\text{ème}$ caractère de $u$
  \item Une \structure{occurrence de $a$ dans $u$} est un indice $i\in \llbracket |u|\rrbracket$ tel que $u_i = a$
  \item Le nombre d'occurrences de $a$ dans $u$ est noté $\alert{|u|_a}$
  \end{itemize}

  \begin{exampleblock}{Exemples}
    \begin{itemize}
    \item $|\text{``banane''}| = 6$
    \item $\forall x\in \Sigma^\star, |x| = 0 \Leftrightarrow x = \varepsilon$ car $\Sigma^0 = \{\varepsilon\}$
    \item Les occurrences de `a' dans ``banane'' sont les indices 2 et 4
    \item $|\text{``banane''}|_{\text{`a'}} = 2$
    \end{itemize}
  \end{exampleblock}

\end{frame}

\endgroup
