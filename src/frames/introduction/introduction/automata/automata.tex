% SPDX-License-Identifier: CC-BY-SA-4.0
% Author: Matthieu Perrin
% Part: 
% Section: 
% Sub-section: 
% Frame: 

\begingroup

\begin{frame}{Et les automates ?}

  \tf[text, top=-2mm]{
    \begin{description}
    \item[Langage :] abstraction d'un problème binaire
    \item[Automate :] abstraction d'une solution au problème binaire
      \begin{itemize}
      \item Modèle mathématique d'une structure algorithmique
      \item Différents types selon l'organisation de la mémoire
      \end{itemize}
    \end{description}
  }

  \tfBlock<2-|handout>[y=10mm]{La hiérarchie de Chomsky}{}

  \tf<2-|handout>[x=-15mm, y=-15mm]{
    \begin{sm2D}[width=28mm, height=9mm]
      \smColumn[header=Langage]{
        \smLine{Récursivement\\ énumérable}
        \smLine{Contextuel}
        \smLine[fill=structure!20]{Algébrique}
        \smLine[fill=structure!20]{Rationnel}
      }
      \smColumn[header=Automate]{
        \smLine{Machine de Turing}
        \smLine{Automate\\linéairement borné}
        \smLine[fill=structure!20]{Automate à pile\\non déterministe}
        \smLine[fill=structure!20]{Automate fini}
      }
      \smColumn[header=Exemples]{
        \smLine{Primalité, \\Sudoku...}
        \smLine{Langues\\naturelles}
        \smLine[fill=structure!20]{Langages de\\programmation}
        \smLine[fill=structure!20]{Identifiants,\\ nombres...}
      }
    \end{sm2D}
  }
  
  \tfImage<2-|handout>[x=45mm,y=-17mm]{%
    width=2.2cm,
    title={Noam Chomsky},
    license={{CC-BY-SA} (2017, \href{https://commons.wikimedia.org/wiki/File:Noam_Chomsky_portrait_2017_retouched.png}{Wikimedia})},
    img={Chomsky.jpg}
  }
  
\end{frame}

\endgroup
