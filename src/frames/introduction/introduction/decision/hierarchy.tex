% SPDX-License-Identifier: CC-BY-SA-4.0
% Author: Matthieu Perrin
% Part: 
% Section: 
% Sub-section: 
% Frame: 

\begingroup


\begin{frame}{Hiérarchie des langages}

  \tfBlock[top]{Formalismes de description d'un langage}{
    \begin{itemize}
    \item Comment représenter un langage \structure{infini} de façon \structure{finie} ?
    \item Il doit avoir une \alert{structure} interne
    \item La ``\structure{complexité}'' de sa structure entraine la ``\structure{difficulté}'' à le décider
    \end{itemize}
  }

  \tf[y=-17mm]{\footnotesize
    \begin{tikzpicture}
      \draw[structure, fill=structure!20] (12mm,12mm) ellipse(40mm and 21mm); \draw (12mm,30mm) node{définissable};
      \draw[structure, fill=structure!10] (10mm,10mm) ellipse(35mm and 18mm); \draw (10mm,25mm) node{récursivement énumérable};
      \draw[structure, fill=structure!20] (08mm,08mm) ellipse(30mm and 15mm); \draw (08mm,20mm) node{décidable};
      \draw[structure, fill=structure!10] (06mm,06mm) ellipse(25mm and 12mm); \draw (06mm,15mm) node{contextuel};
      \draw[structure, fill=structure!20] (04mm,04mm) ellipse(20mm and 09mm); \draw (04mm,10mm) node{algébrique};
      \draw[structure, fill=structure!10] (02mm,02mm) ellipse(15mm and 06mm); \draw (02mm,05mm) node{rationnel};
      \draw[structure, fill=structure!20] (00mm,00mm) ellipse(10mm and 03mm); \draw (00mm,00mm) node{fini};
    \end{tikzpicture}
  }
  
\end{frame}


\endgroup
