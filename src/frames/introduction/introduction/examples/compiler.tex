% SPDX-License-Identifier: CC-BY-SA-4.0
% Author: Matthieu Perrin
% Part: 
% Section: 
% Sub-section: 
% Frame: 

\begingroup

\begin{frame}{Architecture d'un compilateur}
  
  \tf[top]{
    \begin{tikzpicture}[anchor=mid]
      \draw[rounded corners, fill=example!15] (0.7,0.4) rectangle (3.1,4.0);
      \draw[rounded corners, fill=example!15] (3.6,0.4) rectangle (5.4,4.0);
      \draw[rounded corners, fill=example!15] (5.9,0.4) rectangle (8.3,4.0);
      \draw (1.9,3.75) node{Front end};
      \draw (4.5,3.75) node{Optimiseur};
      \draw (7.1,3.75) node{Back end};

      \footnotesize
      \draw[rounded corners, fill=structure!20] (1.0,2) +(-.2,-1.5) rectangle +(.2,1.5) +(0,0) node[rotate=90]{\footnotesize Analyse lexicale};
      \draw[rounded corners, fill=structure!20] (1.6,2) +(-.2,-1.5) rectangle +(.2,1.5) +(0,0) node[rotate=90]{\footnotesize Analyse syntaxique};
      \draw[rounded corners, fill=structure!00] (2.2,2) +(-.2,-1.5) rectangle +(.2,1.5) +(0,0) node[rotate=90]{\footnotesize Analyse contextuelle};
      \draw[rounded corners, fill=structure!00] (2.8,2) +(-.2,-1.5) rectangle +(.2,1.5) +(0,0) node[rotate=90]{\footnotesize Élaboration};
      \draw[rounded corners, fill=structure!00] (3.9,2) +(-.2,-1.5) rectangle +(.2,1.5) +(0,0) node[rotate=90]{\footnotesize Optimisation $1$};
      \draw[rounded corners, fill=structure!00] (5.1,2) +(-.2,-1.5) rectangle +(.2,1.5) +(0,0) node[rotate=90]{\footnotesize Optimisation $n$};
      \draw[rounded corners, fill=structure!00] (6.2,2) +(-.2,-1.5) rectangle +(.2,1.5) +(0,0) node[rotate=90]{\footnotesize Sélection d'instructions};
      \draw[rounded corners, fill=structure!00] (6.8,2) +(-.2,-1.5) rectangle +(.2,1.5) +(0,0) node[rotate=90]{\footnotesize Ordonnancement};
      \draw[rounded corners, fill=structure!00] (7.4,2) +(-.2,-1.5) rectangle +(.2,1.5) +(0,0) node[rotate=90]{\footnotesize Allocation de registres};
      \draw[rounded corners, fill=structure!00] (8.0,2) +(-.2,-1.5) rectangle +(.2,1.5) +(0,0) node[rotate=90]{\footnotesize Génération};
      \draw[rounded corners, fill=structure!00] (4.5,2) +(0,.1) node{...};
      
      \draw[-latex] (-.2,2) -- (.8,2);
      \draw[-latex] (1.2,2) -- (1.4,2);
      \draw[-latex] (1.8,2) -- (2.0,2);
      \draw[-latex] (2.4,2) -- (2.6,2);
      \draw[-latex] (3.0,2) -- (3.7,2);
      \draw[-latex] (4.1,2) -- (4.9,2);
      \draw[-latex] (5.3,2) -- (6.0,2);
      \draw[-latex] (6.4,2) -- (6.6,2);
      \draw[-latex] (6.9,2) -- (7.2,2);
      \draw[-latex] (7.6,2) -- (7.8,2);
      \draw[-latex] (8.2,2) -- (9.2,2);
      
      \draw (.2,2.15) node {\scriptsize Langage};
      \draw (.2,1.85) node {\scriptsize source};
      
      \draw (8.8,2.15) node {\scriptsize Langage};
      \draw (8.8,1.85) node {\scriptsize cible};
    \end{tikzpicture}
  }

  \tfBlock[bottom]{Objectifs du cours}{
    \begin{itemize}
    \item \vspace{-2mm} Comprendre comment un compilateur ``lit'' le code source
    \item Concevoir un interpréteur pour le langage algorithmique
    \end{itemize}
  }

  \tf[y=-17mm]{
    \begin{tikzpicture}
      \node                                                                (start)   at (0,   3.0) {};
      \node[smBox, minimum width=2cm, minimum height=1.1cm]                (lexer)   at (2.5, 3.0) {Analyseur\\lexical\vspace{2mm}\\\textit{Lexer}};
      \node[smBox, minimum width=2cm, minimum height=1.1cm]                (parser)  at (7,   3.0) {Analyseur\\syntaxique\vspace{2mm}\\\textit{Parser}};
      \node                                                                (end)     at (10,  3.0) {};

      \scriptsize
      \path[-latex] (start) edge
      node[above,align=center]{Langage\\ source}
      node[below, example]{$15+2 * x$}
      (lexer);

      \path[-latex] (lexer) edge
      node[above,align=center]{Chaîne de tokens}
      node[below, example]{$\textsc{int}~\textsc{plus}~\textsc{int}~\textsc{times}~\textsc{var}$}
      (parser);

      \path[-latex] (parser) edge
      node[above,align=center]{Arbre de la syntaxe\\abstraite (AST)}
      node[below, example]{\begin{tikzpicture}[example]
          \node (a) at (2.0,1.8) {$+$};
          \node (b) at (1.5,1.4) {$15$}; \path (a) edge (b); 
          \node (c) at (2.5,1.4) {$*$};  \path (a) edge (c); 
          \node (d) at (2.0,1.0) {$2$};  \path (c) edge (d); 
          \node (e) at (3.0,1.0) {$x$};  \path (c) edge (e); 
      \end{tikzpicture}}
      (end);
    \end{tikzpicture}
  }
    
\end{frame}

\endgroup
