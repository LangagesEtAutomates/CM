% SPDX-License-Identifier: CC-BY-SA-4.0
% Author: Matthieu Perrin
% Part: 
% Section: 
% Sub-section: 
% Frame: 

\begingroup

\begin{frame}{Objet d'étude}

  \begin{block}{Théorie des langages formels}
    Se place à l'intersection de plusieurs disciplines
      \begin{description}[Mathématiques :]
      \item[Linguistique :]  Modélisation abstraite des \alert{langues naturelles}
      \item[Mathématiques :] S'ancre dans la \alert{théorie des ensembles}
      \item[Logique :]       Définit un cadre d'étude des \alert{prédicats logiques}
      \item[Informatique :]  Produit des \alert{algorithmes sur les chaînes de caractères}
      \end{description}
  \end{block}

  \vspace{2mm}
  \begin{block}{Définitions clés}
    \begin{description}
    \item[Alphabet :] \alert{Ensemble fini non vide} de \structure{symboles}\hspace\fill        \example{Exemple : $\{a, b, ..., z\}$}
    \item[Mot :] \alert{Suite finie de symboles} de l'alphabet\hspace\fill        \example{Exemple : ``langage''}
    \item[Langage :] \alert{Ensemble de mots} \hspace\fill \example{Exemple : $\{\text{``langage''}, \text{``langue''}\}$}
    \end{description}
  \end{block}
  
\end{frame}

\endgroup
