% SPDX-License-Identifier: CC-BY-SA-4.0
% Author: Matthieu Perrin
% Part: 
% Section: 
% Sub-section: 
% Frame: 

\begingroup

\begin{frame}{Structure de monoïde}

  Soit $\Sigma$ un alphabet. 
  
  \begin{block}{Propriétés de la concaténation}
    \begin{description}[Non-commutativité :]
    \item[Fermeture :] $\forall u, v \in \Sigma^\star, \alert{u\cdot v \in \Sigma^\star}$
    \item[Associativité :] $\forall u, v, w \in \Sigma^\star, \alert{(u\cdot v)\cdot w = u\cdot (v\cdot w)}$
      \begin{itemize}
      \item On notera $\structure{u\cdot v\cdot w}$
      \end{itemize}
    \item[Neutralité d'$\varepsilon$ :] $\forall u \in \Sigma^\star, \alert{\varepsilon \cdot u = u = u \cdot \varepsilon}$
    \item[Simplifiabilité :] $\forall u, v, w \in \Sigma^\star,$
      \begin{description}
      \item[À gauche :] $\alert{u \cdot v = u \cdot w \Rightarrow v=w}$
      \item[À droite :] $\alert{v \cdot u = w \cdot u \Rightarrow v=w}$
      \end{description}
    \item[Non-commutativité :] $\exists u,v \in \Sigma^\star, \alert{u\cdot v \neq v\cdot u}$
    \item[Décomposabilité :] $\forall u\in \Sigma^\star, \exists! v_1,..., v_{|u|} \in \Sigma^1, u = v_1 \cdots v_{|u|}$ 
    \end{description}
  \end{block}

  \begin{block}{Remarque}
    Un \structure{monoïde} est un ensemble muni d'une loi de composition interne associative, et d'un élément neutre. \\
    Le monoïde \structure{$\langle \Sigma^\star, \cdot , \varepsilon \rangle$} est appelé \alert{monoïde libre engendré par $\Sigma$}.
  \end{block}

\end{frame}

\endgroup
