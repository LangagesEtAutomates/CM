% SPDX-License-Identifier: CC-BY-SA-4.0
% Author: Matthieu Perrin
% Part: 
% Section: 
% Sub-section: 
% Frame: 

\begingroup

\begin{frame}{Relations d'ordre sur les mots}

  Soient $\Sigma$ un alphabet, muni d'un ordre total \structure{$\le$}, et $u, v \in \Sigma^\star$.

  \begin{block}{Définitions}
    \begin{description}[Ordre lexicographique]
    \item[Ordre préfixiel]: noté \alert{$\le_p$}, c'est un \example{ordre partiel} défini par : \\
      \alert{$u \le_p v$} si \alert{$u$ est un préfixe de $v$}.
 
    \item[Ordre lexicographique]: noté \alert{$\le_l$}, c'est un \example{ordre total} défini par :\\
       \alert{$u \le_l v$} si l'une de ces deux conditions est vérifiée :
        \begin{itemize}
        \item \alert{$u \le_p v$}
        \item $\exists x, y, z \in \Sigma^\star, a, b \in \Sigma,~ \alert{u = x\cdot a\cdot y \land v = x\cdot b\cdot z \land a \le b}$
        \end{itemize}

    \item[Ordre hiérarchique]: noté \alert{$\le_h$}, c'est un \example{ordre total} défini par :\\
      \alert{$u \le_h v$} si \alert{$|u| < |v| \lor (|u| = |v| \land u \le_l v)$}.
    \end{description}
  \end{block}

  \begin{exampleblock}{Exemples}
    \begin{itemize}
      \item $\text{barbare} \le_p \text{barbares}$. 
      \item $\text{barbare} \le_l \text{barbes}$ car $a \le e$. 
      \item $\text{barbes} \le_h \text{barbare}$ car $|\text{barbes}| = 6 < 7 = |\text{barbare}|$
    \end{itemize}
  \end{exampleblock}
  
\end{frame}

\endgroup
