% SPDX-License-Identifier: CC-BY-SA-4.0
% Author: Matthieu Perrin
% Part: 
% Section: 
% Sub-section: 
% Frame: 

\begingroup

\begin{frame}{Alphabet}
  
  Soit $n\in \mathbb{N}$. On note $\alert{\llbracket n\rrbracket \eqdef \{x\in \mathbb{N} \,|\, 1 \le x \le n\} = \{1, ..., n\}}$.
  
  \begin{block}{Définition -- alphabet}
    Soit $E$ un ensemble.
    \begin{itemize}
    \item On dit que $E$ est \structure{fini} s'il existe\footnote{Si un tel $n$ existe, il est unique.} $n\in \mathbb{N}$ et une bijection de $\llbracket n\rrbracket$ dans $E$. 
    \item $n$ est appelé le \structure{cardinal} de $E$, noté \alert{$|E|$}.
    \item Un \structure{alphabet} est un \alert{ensemble fini non-vide}. 
    \end{itemize}
  \end{block}

  \begin{exampleblock}{Exemple}
    $\{a, b, c\}$ est un alphabet de cardinal 3, d'après la bijection suivante :

    $$
    \left\{\begin{array}{ccl}
    \llbracket 3\rrbracket &\rightarrow& \{a, b, c\}\\
    1 &\mapsto& a\\ 
    2 &\mapsto& b\\ 
    3 &\mapsto& c\\ 
    \end{array}\right.
    $$
  \end{exampleblock}

\end{frame}

\endgroup
