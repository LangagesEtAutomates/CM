% SPDX-License-Identifier: CC-BY-SA-4.0
% Author: Matthieu Perrin
% Part: 
% Section: 
% Sub-section: 
% Frame: 

\begingroup

\begin{frame}{Transformation en automate à pile}

  \on[text, top=-3mm]{
    \begin{description}[Acceptation :]
    \item[États :] Les états sont les mêmes que dans l'automate sous-jacent.
    \item[Pile :] La pile contient un chemin menant à l'état courant
      \begin{itemize}
      \item symboles = états, fond de pile = état initial.
      \end{itemize}
    \item<2-|handout>[\alert{Décalage :}] Pour tout $a\in Sigma$ et transition $q \xrightarrow{a} q'$ :
      \begin{tikzpicture}[automaton, x=20mm, baseline=(q.base)]
        \state (q) at (0,0) {$q$};
        \state (p) at (1,0) {$q'$};
        \path[structure, densely dotted]  (q) edge[bend left =5mm] node       {$a$} (p);
        \path[alert]                      (q) edge[bend right=5mm] node[swap] {\smPAtrans{a}{\varepsilon}{q'}} (p);
      \end{tikzpicture}
    \item<3-|handout>[\example{Réduction :}] Pour tout $A\in \Gamma$ et $A \rightarrow \alpha_1 \ldots \alpha_n \in R$, et transition $q \xrightarrow{A} q'$ :
      \begin{tikzpicture}[automaton, y=5mm]
        \state[structure] (0) at (0,0) {$q_0$};
        \state[alert]     (1) at (1,1) {$q_1$};
        \state[alert]     (m) at (2,1) {\tiny$q_{n-1}$};
        \state[example]   (n) at (3,1) {$q_n$};
        \state[example]   (p) at (4,0) {$q'$};
        
        \path[alert, densely dotted]      (0) edge node       {$\alpha_1$}                                      (1);
        \path[alert, densely dotted]      (1) edge node       {$\ldots$}                                        (m);
        \path[alert, densely dotted]      (m) edge node       {$\alpha_n$}                                      (n);
        \path[structure, densely dotted]  (0) edge node[swap] {$A$}                                             (p);
        \path[example]                    (n) edge node       {\smPAtrans{\varepsilon}{q_0 \ldots q_n}{q_0 q'}} (p);
      \end{tikzpicture}
    \end{description}
  }

  \on[bottom, x=-4cm]{\footnotesize
    \begin{tikzpicture}[automaton, x=10mm, y=10mm]
      \tikzset{
        lrpath/.style={},
      }
      \state[structure on=<3>, initial] (0) at (0.0,0.0) {$0$};
      \state[example on=<3>           ] (1) at (1.2,0.0) {$1$};
      \state[                         ] (2) at (1.2,1.0) {$2$};
      \state[alert on=<3>             ] (3) at (0.0,2.0) {$3$}; 
      \state[alert on=<3>             ] (4) at (1.2,2.0) {$4$};  
      \state[example on=<3>           ] (5) at (2.4,2.0) {$5$};
      \state[accepting                ] (6) at (2.4,0.0) {$6$};
      
      \path[structure ob=<2>, alert on=<3>] (0) edge            node[swap] {$a$}  (3);
      \path[structure ob=<2>              ] (0) edge            node       {$c$}  (2);
      \path[structure on=<3>              ] (0) edge            node[swap] {$A$}  (1);
      \path[structure ob=<2>              ] (3) edge[loop above] node       {$a$}  (3);
      \path[structure ob=<2>              ] (3) edge            node[swap] {$c$}  (2);
      \path[structure ob=<2>, alert on=<3>] (4) edge            node       {$b$}  (5);
      \path[alert on=<3>                  ] (3) edge            node       {$A$}  (4);
      \path[structure ob=<2>              ] (1) edge            node[swap] {$\$$} (6);
    \end{tikzpicture}
  }

  \on[bottom, x=20mm]{\footnotesize
    \begin{tikzpicture}[pushdown, x=10mm, y=10mm]
      \state[initial, initial text=$0$] (0) at (0,0) {$0$};
      \state[example on=<3>           ] (1) at (2,0) {$1$};
      \state[                         ] (2) at (1,1) {$2$};
      \state[                         ] (3) at (0,2) {$3$}; 
      \state[                         ] (4) at (2,2) {$4$};  
      \state[example on=<3>           ] (5) at (4,2) {$5$};
      \state[accepting                ] (6) at (4,0) {$6$};
      
      \uncover<2->{
        \path[alert]  (3) edge[loop left]     node       {\smPAtrans{a} {\varepsilon}{3}} (3);
        \path[alert]  (0) edge                 node       {\smPAtrans{a} {\varepsilon}{3}} (3);
        \path[alert]  (0) edge                 node[sloped,swap] {\smPAtrans{c} {\varepsilon}{2}} (2);
        \path[alert]  (3) edge                 node[sloped]     {\smPAtrans{c} {\varepsilon}{2}} (2);
        \path[alert]  (4) edge[bend right=5mm] node[swap] {\smPAtrans{b} {\varepsilon}{5}} (5);
        \path[alert]  (1) edge                 node[swap] {\smPAtrans{\$}{\varepsilon}{6}} (6);
      }
      \uncover<3->{
        \path[example]      (5) edge                 node[swap,sloped]       {\smPAtrans{\varepsilon}{0345}{01}} (1);
      }
      \uncover<4->{
        \path[example]      (2) edge                 node[swap,sloped]       {\smPAtrans{\varepsilon}{02}{01}} (1);
        \path[example]      (2) edge                 node[sloped]       {\smPAtrans{\varepsilon}{32}{34}} (4);
        \path[example]      (5) edge[bend right=5mm] node[swap] {\smPAtrans{\varepsilon}{3345}{34}} (4);
      }
    \end{tikzpicture}
  }

\end{frame}

\endgroup
