% SPDX-License-Identifier: CC-BY-SA-4.0
% Author: Matthieu Perrin
% Part: 
% Section: 
% Sub-section: 
% Frame: 

\begingroup

\begin{frame}{États d'un parser LR}
  Soit $G=\langle \Sigma, \Gamma, S, R \rangle$ une grammaire algébrique.

  \begin{block}{Définition -- Items et ensembles d'items}
    \begin{itemize}
    \item On suppose l'existence d'une seule règle $S\rightarrow \alpha$ dans $R$ :  \structure{$S \rightarrow A \$$}
      \begin{itemize}
      \item $\$$ signifie \og end of file \fg
      \item On peut ajouter la règle : \hspace\fill
      $\example{
        \left\{ S \rightarrow a S b \mid c \right\}
        \Rightarrow 
        \left\{
        S \rightarrow A\$, 
        A \rightarrow a A b \mid c
        \right\}
      }$
      \end{itemize}
    \item Un \structure{item} est un couple $\langle A \rightarrow \alpha_1 \ldots \alpha_n, k \rangle \in R \times \mathbb{N}$, tel que $0 \le k\le n$, noté :
      $$\alert{A \rightarrow \alpha_1 \ldots \alpha_k  \bullet \alpha_{k+1}\ldots \alpha_n}$$
    \item La \structure{fermeture} d'un ensemble d'items $E$ est le plus petit ensemble $E'$ tq :
      \begin{itemize}
      \item $E \subseteq E'$
      \item Si $E'$ contient un item devant un non-terminal $B$, \\alors $E'$ contient aussi tous les items au début des règles de $B$ :
        $$\alert{\forall \structure{A \rightarrow \alpha \bullet B \beta} \in E', \forall \structure{B \rightarrow \gamma} \in R,  \structure{B \rightarrow \bullet \gamma} \in E'}$$
      \end{itemize}
      $$\example{
        \mathit{fermeture}\left(\left\{\begin{array}{@{\,}r@{\,}c@{\,}l@{\,}}
        S &\rightarrow& a \bullet S b\\
        S &\rightarrow& c
        \end{array}\right\}\right)
        =
        \left\{\begin{array}{@{\,}r@{\,}c@{\,}l@{\,}}
        S &\rightarrow& \bullet a \bullet S b\\
        S &\rightarrow& \bullet c
        \end{array}\right\}
      }$$
    \item Les états de l'analyseur sont des ensembles fermés d'items.
    \end{itemize}
  \end{block}
\end{frame}

\endgroup
