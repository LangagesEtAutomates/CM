% SPDX-License-Identifier: CC-BY-SA-4.0
% Author: Matthieu Perrin
% Part: 
% Section: 
% Sub-section: 
% Frame: 

\begingroup

\begin{frame}{Deux grandes approches pour l'analyse syntaxique}

  \begin{alertblock}{Problème}
    Étant donnés :
    \begin{itemize}
      \item une grammaire algébrique $G$
      \item un mot $u \in \Sigma^\star$
    \end{itemize}
    peut-on décider efficacement si \structure{$u \in \mathcal{L}(G)$} ?
  \end{alertblock}

  \begin{block}{Approche 1 -- Algorithme CYK (Cocke–Younger–Kasami)}
    \begin{itemize}
      \item S'applique à \structure{toute grammaire} mise en \structure{forme normale de Chomsky}%Fonctionne uniquement si $G$ est en \structure{forme normale de Chomsky (FNC)}.
      \item Repose sur une analyse exhaustive par \structure{programmation dynamique}
      \item Complexité en $\mathcal{O}(|R| \times |u|^3)$
    \end{itemize}
  \end{block}

  \begin{block}{Approche 2 -- Analyse déterministe (parsers LL ou LR)}
    \begin{itemize}
      \item S'applique aux grammaires \structure{non-ambiguës} et \structure{déterministes}
      \item Utilise un \structure{automate à pile déterministe}
      \item Complexité en $\mathcal{O}(|R| \times |u|)$
    \end{itemize}
  \end{block}

\end{frame}

\endgroup
