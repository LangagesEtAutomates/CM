% SPDX-License-Identifier: CC-BY-SA-4.0
% Author: Matthieu Perrin
% Part: 
% Section: 
% Sub-section: 
% Frame: 

\begingroup

\begin{frame}{Forme normale de Chomsky}
  
  \begin{block}{Forme normale de Chomsky}
    Une grammaire $G = \langle \Sigma, \Gamma, S, \rightarrow \rangle$ est en \structure{forme normale de Chomsky} si
    \begin{itemize}
    \item\vspace{2mm} \structure{$\varepsilon\notin\mathcal{L}(G)$} et toutes les règles sont de la forme :
      \begin{enumerate}
      \item\vspace{1mm} $N \rightarrow XY$ \hspace{5mm} avec $N\in \Gamma$ et $X, Y \in \Gamma$
      \item $N \rightarrow a$  \hspace{7.5mm} avec $N \in \Gamma$ et $a\in \Sigma$
      \end{enumerate}
    \item\vspace{2mm} ou bien \structure{$\varepsilon\in\mathcal{L}(G)$} et toutes les règles sont de la forme :
      \begin{enumerate}
      \item\vspace{1mm} $N \rightarrow XY$ \hspace{5mm} avec $N\in \Gamma$ et $X, Y \in \Gamma \setminus \{S\}$
      \item $N \rightarrow a$  \hspace{7.5mm} avec $N \in \Gamma$ et $a\in \Sigma$
      \item $S \rightarrow \varepsilon$
      \end{enumerate}
    \end{itemize}
  \end{block}

  \begin{block}{Théorème}
    Toute grammaire algébrique peut être transformée en une
    grammaire en \structure{forme normale de Chomsky} reconnaissant le même langage. 
  \end{block}

\end{frame}

\endgroup
