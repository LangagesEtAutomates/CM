% SPDX-License-Identifier: CC-BY-SA-4.0
% Author: Matthieu Perrin
% Part: 
% Section: 
% Sub-section: 
% Frame: 

\begingroup

\begin{frame}{Forme normale de Chomsky}
  
  \begin{block}{Forme normale de Chomsky}
    Soit $G = \langle \Sigma, \Gamma, S, R \rangle$ une grammaire algébrique.

    $G$ est en \structure{forme normale de Chomsky} si les règles de $R$ sont de la forme :
    \begin{enumerate}
    \item $N \rightarrow XY$ \hspace{5mm} avec $N\in \Gamma$ et $X, Y \in \Gamma$
    \item $N \rightarrow a$  \hspace{8mm} avec $N \in \Gamma$ et $a\in \Sigma$
    \end{enumerate}
    ou bien toutes ses règles sont de la forme :
    \begin{enumerate}
    \item $N \rightarrow XY$ \hspace{5mm} avec $N\in \Gamma$ et $X, Y \in \Gamma \setminus \{S\}$
    \item $N \rightarrow a$  \hspace{8mm} avec $N \in \Gamma$ et $a\in \Sigma$
    \item $S \rightarrow \varepsilon$
    \end{enumerate}
  \end{block}

  \begin{block}{Théorème}
    Toute grammaire algébrique peut être transformée en une
    grammaire en \structure{forme normale de Chomsky} reconnaissant le même langage. 
  \end{block}
\end{frame}

\endgroup
