% SPDX-License-Identifier: CC-BY-SA-4.0
% Author: Matthieu Perrin
% Part: 
% Section: 
% Sub-section: 
% Frame: 

\begingroup

\begin{frame}{Automates à pile déterministes}

  Soit $A = \langle \Sigma, \Pi, Q, I, F, \diamond, \mu \rangle$ un automate à pile.

  \begin{block}{Définition -- Automate à pile déterministe}
    On dit que $A$ est \structure{déterministe}\footnote[frame]{Il existe d'autres définitions, généralement plus restrictives, dans la littérature classique.} si ces deux conditions sont vérifiées : 
    \begin{enumerate}
    \item $A$ ne possède qu'un seul état initial : \alert{$|I| = 1$}
    \item Pour tout état \structure{$q \in Q$} et transitions \structure{$q\xrightarrow{\footnotesize\smPAtrans{a_1}{p_1}{p'_1}} q_1 \in \mu$}
      et \structure{$q\xrightarrow{\footnotesize\smPAtrans{a_2}{p_2}{p'_2}} q_2 \in \mu$} :
      \begin{itemize}
      \item Soit on peut distinguer selon les symboles lus $a_1$ et $a_2$ :
        $$\alert{a_1 \notin \mathit{Prefix}(a_2) \land a_2 \notin \mathit{Prefix}(a_1)}$$
      \item Soit on peut distinguer selon la pile  $p_1$ et $p_2$ :
        $$\alert{p_1 \notin \mathit{Prefix}(p_2) \land p_2 \notin \mathit{Prefix}(p_1)}$$
      \end{itemize}
    \end{enumerate}
  \end{block}

  \begin{alertblock}{Remarque}
    \begin{itemize}
    \item \vspace{-1mm} On note $\mu(q, a, p) \eqdef \langle q', p'\rangle$ lorsque $q \xrightarrow{\footnotesize\smPAtrans{a}{p}{p'}} q' \in \mu$ (si elle est définie).
%      On note $\mu(q, a, p) \eqdef \langle q', p'\rangle$ tel que $q\xrightarrow{\footnotesize\smPAtrans{a}{p}{p'}} q' \in \mu$
    \end{itemize}
  \end{alertblock}
\end{frame}


\endgroup
