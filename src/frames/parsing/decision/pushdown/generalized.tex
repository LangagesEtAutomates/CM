% SPDX-License-Identifier: CC-BY-SA-4.0
% Author: Matthieu Perrin
% Part: 
% Section: 
% Sub-section: 
% Frame: 

\begingroup

\SetKwFunction{Dick}{decide}

\begin{frame}{Aller au-delà des automates finis}

  \tfBlock[top=-5mm]{Automates finis généralisés}{
    Si un automate fini n'est pas suffisant, ajouter une structure de données
    \begin{itemize}
    \item Entier : \structure{automate à compteur}
    \item Pile : \structure{automate à pile}
    \item Tableau de taille $|u|$ : \structure{automate linéairement borné}
    \item Ruban infini : \structure{machine de Turing}
    \end{itemize}
  }

  \tfExampleBlock[bottom=-2mm]{Exemple : le langage $\{a^n b^n \mid n>0\}$}{
    \begin{algorithm}[H]
      \Fn{$\Dick(u : \text{mot}) : \text{booléen}$}{
        $q\leftarrow 0$; $c \leftarrow 0$\;
        \Pour{$i$ allant de $1$ à $|u|$}{
          \uSi{$q = 0 \land u[i] = \text{`$a$'}$}{
            $c \leftarrow c + 1$
          }
          \uSinonSi{$c > 0 \land u[i] = \text{`$b$'}$}{
            $q\leftarrow 1$;
            $c \leftarrow c - 1$\;
          }
          \lSinon{\Retourner \False}
        }
        \Retourner $q=1 \land c=0$\;
      }
    \end{algorithm}
  }

  \tfBlock[right=.4\textwidth,y=-1mm]{Grammaire}{
    $S \rightarrow a S b \mid ab $
  }
  
  \tfBlock[right=.4\textwidth, bottom=3mm]{Automate à compteur}{
    \begin{tikzpicture}[smAutomaton]
      \smState[\smInitial]   (a) at (0,0) {0};
      \smState[\smAccepting] (b) at (2,0) {1};
      \smPath (a) edge[loop above] node {$a / c\text{++}$}   (a);
      \smPath (a) edge             node {$b / c\text{-\,-}$} (b);
      \smPath (b) edge[loop above] node {$b / c\text{-\,-}$} (b);
    \end{tikzpicture}
  }
  
\end{frame}

\endgroup
