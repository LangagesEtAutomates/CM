% SPDX-License-Identifier: CC-BY-SA-4.0
% Author: Matthieu Perrin
% Part: 
% Section: 
% Sub-section: 
% Frame: 

\begingroup

\begin{frame}{Langages reconnus par un automate à pile}
  
  Soit $A=\langle \Sigma, \Pi, Q, I, F, \diamond, \rightarrow \rangle$ un automate à pile.

  \begin{block}{Définitions -- Langages reconnus}
    \begin{description}[xxxxxx]
    \item[$\mathcal{L}_F(A)$] est le langage des mots reconnus \alert{par état final}
      $$\alert{\mathcal{L}_F(A) = \{u \in \Sigma^\star \mid
      \exists i\in I, \exists \structure{f\in F}, \exists \structure{\pi\in\Pi^\star}, \langle u, i, \diamond \rangle \leadsto_A^\star \langle \varepsilon, \structure{f}, \structure{\pi} \rangle\}}$$
    \item[$\mathcal{L}_\varepsilon(A)$] est le langage des mots reconnus \alert{par pile vide}
      $$\alert{\mathcal{L}_\varepsilon(A) = \{u \in \Sigma^\star \mid
      \exists i\in I, \exists \structure{q\in Q}, \langle u, i, \diamond \rangle \leadsto_A^\star \langle \varepsilon, \structure{q}, \structure{\varepsilon} \rangle\}}$$
    \item[$\mathcal{L}(A)$] est le langage des mots reconnus \alert{par état final et pile vide}
      $$\alert{\mathcal{L}(A) = \{u \in \Sigma^\star \mid
      \exists i\in I, \exists \structure{f\in F}, \langle u, i, \diamond \rangle \leadsto_A^\star \langle \varepsilon, \structure{f}, \structure{\varepsilon} \rangle\}}$$
    \end{description}
  \end{block}

  \begin{block}{Théorème -- Équivalence des classes de langages}
    Les trois méthodes de reconnaissance définissent les mêmes langages
    \begin{itemize}
    \item Ajouter des $\varepsilon$-transitions vers un état final et une boucle pour vider la pile
    \end{itemize}
  \end{block}

\end{frame}

\endgroup
