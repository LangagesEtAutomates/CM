% SPDX-License-Identifier: CC-BY-SA-4.0
% Author: Matthieu Perrin
% Part: 
% Section: 
% Sub-section: 
% Frame: 

\begingroup

\begin{frame}{Modélisation mathématique}

  \vspace{-2mm}
  \begin{block}{Définition -- Automate à pile non-déterministe}
    Un \structure{automate à pile} est un septuplet \alert{$\langle \Sigma, \Pi, Q, I, F, \diamond, \rightarrow \rangle$} tel que :
    \begin{description}[xxxx]
    \item[\alert{$\Sigma$}] ensemble fini non vide : \structure{l'alphabet d'entrée}
    \item[\alert{$\Pi$}] ensemble fini non vide : \structure{l'alphabet de pile}
    \item[\alert{$Q$}] ensemble fini non vide : \structure{les états possibles}
    \item[\alert{$I$}] $\subseteq Q$ : \structure{les états initiaux}
    \item[\alert{$F$}] $\subseteq Q$ : \structure{les états finaux (ou accepteurs)}
    \item[\alert{$\diamond$}] $\in \Pi$ : \structure{le symbole de fond de pile}
    \item[\alert{$\rightarrow$}] $\subseteq  (Q \times \Sigma^\star \times \Pi^\star) \times (Q \times \Pi^\star)$ : \structure{la relation de transition}
    \end{description}

    \vspace{-1mm}
    Une \structure{transition} est un tuple \alert{$\langle \langle q, u, p \rangle, \langle q', p' \rangle \rangle \in \rightarrow$}, noté \alert{$q \xrightarrow{\smPAtrans{u}{p}{p'}} q'$}, tel que :
    \begin{description}[xxxx]
    \item[\alert{$q$}] $\in Q$ : \structure{l'état de départ}
    \item[\alert{$u$}] $\in \Sigma^\star$ : \structure{le mot reconnu}
    \item[\alert{$p$}] $\in \Pi^\star$ : \structure{le mot dépilé}
    \item[\alert{$q'$}] $\in Q$ : \structure{l'état d'arrivée}
    \item[\alert{$p'$}] $\in \Pi^\star$ : \structure{le mot empilé}
    \end{description}
  \end{block}

  \tf[x=3cm, y=-3cm]{
    \begin{tikzpicture}[smAutomaton]
      \smState (q)  at (0.0,0) {$q$}; 
      \smState (q1) at (2.5,0) {$q'$};
      \smPath  (q) edge node {\smPAtrans{u}{p}{p'}} (q1);
    \end{tikzpicture}
  }

\end{frame}

\endgroup
