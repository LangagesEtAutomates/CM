% SPDX-License-Identifier: CC-BY-SA-4.0
% Author: Matthieu Perrin
% Part: 
% Section: 
% Sub-section: 
% Frame: 

\begingroup

\begin{frame}{Équivalence entre automates à pile et grammaires}

  \onBlock[top=-5mm]{Théorème -- Équivalence entre formalismes}{
    Tout langage peut être reconnu par un automate à pile ssi il est algébrique. 
  }
  
  \onBlock[y=10mm]{Preuve}{
    \begin{enumerate}
    \item \structure{Des automates à pile aux grammaires algébriques}
      \begin{itemize}
      \item Admis, car un peu technique
      \end{itemize}
    \item \structure{Des grammaires algébriques aux automates à pile}\\
      Soit $G = \langle \Sigma, \Gamma, S, R\rangle$ une grammaire en forme normale de Chomsky.\\
      Alors $\mathcal{L}(G)$ est reconnu sur pile vide par l'automate à pile :
    \end{enumerate}
  }
 
  \on[y=-10mm]{\footnotesize
    \begin{tikzpicture}[pushdown]
      \state[initial, accepting, initial text=$S$] (q) {};
      \path (q) edge [loop right] node {
        $\begin{array}{ll}
          \smPAtrans{a}{A}{\varepsilon}  & \text{ pour } A\rightarrow a \\
          \smPAtrans{\varepsilon}{A}{CB} & \text{ pour } A\rightarrow BC \\
        \end{array}$
      } (q);
    \end{tikzpicture}
  }
  
  \onExampleBlock<2->[y=-18mm]{Exemple : $\{ a^n c b^n \mid n\in \mathbb{N} \}$}{}

  \on<2->[bottom=-1mm,x=-.45\textwidth]{
    \begin{tikzpicture}[stack, x=7mm, y=7mm]
      \cell[alert ob=<2>]{\oneof[$S$]{\on<3->{$T$}\on<5->{$B$}\on<7->{}}}
      \cell              {\oneof[]{\on<3>{$A$}\on<5>{$S$}}}
    \end{tikzpicture}
  }

  \on<2->[bottom=13mm,x=-.3\textwidth]{
    \begin{tikzpicture}[word, x=5mm, y=5mm]
      \cell[alert ob=<4->]{$a$} \smhead[on=<-3>]
      \cell[alert ob=<6->]{$c$} \smhead[ob=<4-5>]
      \cell[alert ob=<7->]{$b$} \smhead[ob=<6>]
    \end{tikzpicture}
  }

  \on<2->[bottom,x=-.15\textwidth]{\small
    \begin{tikzpicture}[pushdown]
      \state[initial above, accepting, initial text={\alert<2>{$S$}}] (q) {};
      \path (q) edge [loop left] node {
        $\begin{array}{r}
          \alert<4>{\smPAtrans{a}{A}{\varepsilon}} \\
          \alert<7>{\smPAtrans{b}{B}{\varepsilon}} \\
          \alert<6>{\smPAtrans{c}{S}{\varepsilon}} \\
        \end{array}$
      } (q);
      \path (q) edge [loop right] node {
        $\begin{array}{l}
          \alert<3>{\smPAtrans{\varepsilon}{S}{TA}} \\
          \alert<5>{\smPAtrans{\varepsilon}{T}{BS}} \\
        \end{array}$
      } (q);
    \end{tikzpicture}
  }

  \on<2->[bottom=5mm, x=.2\textwidth, width=2.5cm]{
    \example{Grammaire FNC}\\\vspace{2mm}
    $\left\{\begin{array}{@{\,}r@{~\rightarrow~}l@{\,}}
    S & \alert<3>{AT} \mid \alert<6>{c}\\
    T & \alert<5>{SB}\\
    A & \alert<4>{a}\\
    B & \alert<7>{b}\\
    \end{array}\right.$
  }

  \on<2->[bottom,x=.4\textwidth]{
    \begin{tikzpicture}[tree, x=7mm,y=7mm, tree node internal/.append style={structure,}, tree node leave/.append style={example,},]
      \tree{\alertb<2>{$S$}}{
        \tree[on=<3->]{\alertb<3>{$A$}}{
          \tree[on=<4->,yshift=-1]{\alertb<4>{$a$}}{}
        }
        \tree[on=<3->]{\alertb<3>{$T$}}{
          \tree[on=<5->]{\alertb<5>{$S$}}{
            \tree[on=<6->]{\alertb<6>{$c$}}{}
          }
          \tree[on=<5->]{\alertb<5>{$B$}}{
            \tree[on=<7->]{\alertb<7>{$b$}}{}
          }
        }
      }
    \end{tikzpicture}
  }
  
\end{frame}

\endgroup
