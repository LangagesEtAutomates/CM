% SPDX-License-Identifier: CC-BY-SA-4.0
% Author: Matthieu Perrin
% Part: 
% Section: 
% Sub-section: 
% Frame: 

\begingroup

\begin{frame}{Équivalence entre automates à pile et grammaires}

  \tfBlock[top=-5mm]{Théorème -- Équivalence entre formalismes}{
    Tout langage peut être reconnu par un automate à pile ssi il est algébrique. 
  }
  
  \tfBlock[y=10mm]{Preuve}{
    \begin{enumerate}
    \item \structure{Des automates à pile aux grammaires algébriques}
      \begin{itemize}
      \item Admis, car un peu technique
      \end{itemize}
    \item \structure{Des grammaires algébriques aux automates à pile}\\
      Soit $G = \langle \Sigma, \Gamma, S, R\rangle$ une grammaire en forme normale de Chomsky.\\
      Alors $\mathcal{L}(G)$ est reconnu sur pile vide par l'automate à pile :
    \end{enumerate}
  }
 
  \tf[y=-10mm]{\footnotesize
    \begin{tikzpicture}[smAutomaton, initial text=$S$]
      \smState[\smInitial\smAccepting] (q) {};
      \smPath (q) edge [loop right] node {
        $\begin{array}{ll}
          \smPAtrans{a}{A}{\varepsilon}  & \text{ pour } A\rightarrow a \\
          \smPAtrans{\varepsilon}{A}{CB} & \text{ pour } A\rightarrow BC \\
        \end{array}$
      } (q);
    \end{tikzpicture}
  }
  
  \tfExampleBlock<2-|handout>[y=-18mm]{Exemple : $\{ a^n c b^n \mid n\in \mathbb{N} \}$}{}

  \tf<2-|handout>[bottom=-5mm,x=-.45\textwidth]{
    \begin{smStack}
      \smCell[\smAlert<2>]{\oneof[$S$]{\on<3->{$T$}\on<5->{$B$}\on<7->{}}}
      \smCell             {\oneof[]{\on<3>{$A$}\on<5>{$S$}}}
    \end{smStack}
  }

  \tf<2-|handout>[bottom=13mm,x=-.3\textwidth]{
    \begin{smArray}[height=4mm]
      \smCell[\smAlert<4->]{$a$} \smHead<-3|handout>
      \smCell[\smAlert<6->]{$c$} \smHead<4-5>
      \smCell[\smAlert<7->]{$b$} \smHead<6>
    \end{smArray}
  }

  \tf<2-|handout>[bottom,x=-.15\textwidth]{\small
    \begin{tikzpicture}[smAutomaton, initial text={\alert<2>{$S$}}]
      \smState[\smInitialAbove\smAccepting] (q) {};
      \smPath (q) edge [loop left] node {
        $\begin{array}{r}
          \alert<4>{\smPAtrans{a}{A}{\varepsilon}} \\
          \alert<7>{\smPAtrans{b}{B}{\varepsilon}} \\
          \alert<6>{\smPAtrans{c}{S}{\varepsilon}} \\
        \end{array}$
      } (q);
      \smPath (q) edge [loop right] node {
        $\begin{array}{l}
          \alert<3>{\smPAtrans{\varepsilon}{S}{TA}} \\
          \alert<5>{\smPAtrans{\varepsilon}{T}{BS}} \\
        \end{array}$
      } (q);
    \end{tikzpicture}
  }

  \tf<2-|handout>[bottom=5mm, x=.2\textwidth, width=2.5cm]{
    \example{Grammaire FNC}\\\vspace{2mm}
    $\left\{\begin{array}{@{\,}r@{~\rightarrow~}l@{\,}}
    S & \alert<3>{AT} \mid \alert<6>{c}\\
    T & \alert<5>{SB}\\
    A & \alert<4>{a}\\
    B & \alert<7>{b}\\
    \end{array}\right.$
  }

  \tf<2-|handout>[bottom,x=.4\textwidth]{
    \begin{tikzpicture}
      \smDraw<2-|handout>{
        \node[structure] (S) at (.375,2.1) {\alert<2>{$S$}};
      }
      \smDraw<3-|handout>{
        \node[structure] (A) at (0.00,1.4) {\alert<3>{$A$}}; \path[-latex] (S) edge (A); 
        \node[structure] (T) at (0.75,1.4) {\alert<3>{$T$}}; \path[-latex] (S) edge (T); 
      }
      \smDraw<4-|handout>{
        \node[example]   (a) at (0.00,0.0) {\alert<4>{$a$}}; \path[-latex] (A) edge (a); 
      }
      \smDraw<5-|handout>{
        \node[structure] (C) at (0.50,0.7) {\alert<5>{$S$}}; \path[-latex] (T) edge (C); 
        \node[structure] (B) at (1.00,0.7) {\alert<5>{$B$}}; \path[-latex] (T) edge (B); 
      }
      \smDraw<6-|handout>{
        \node[example]   (c) at (0.50,0.0) {\alert<6>{$c$}}; \path[-latex] (C) edge (c); 
      }
      \smDraw<7-|handout>{
        \node[example]   (b) at (1.00,0.0) {\alert<7>{$b$}}; \path[-latex] (B) edge (b); 
      }
    \end{tikzpicture}
  }
  
\end{frame}

\endgroup
