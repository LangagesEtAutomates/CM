% SPDX-License-Identifier: CC-BY-SA-4.0
% Author: Matthieu Perrin
% Part: 
% Section: 
% Sub-section: 
% Frame: 

\begingroup

\begin{frame}{Arbre de la syntaxe abstraite}

  \tfBlock[top=-5mm]{Arbre de la syntaxe abstraite (AST : Abstract Syntax Tree)}{\vspace{-1mm}
    \begin{itemize}
    \item Les arbres de dérivation peuvent être simplifiés en un AST
      \begin{description}[Feuilles :]
      \item[N\oe uds :] Constructions du langage (par exemple opérateurs)
      \item[Feuilles :] Constantes ou variables
      \end{description}
    \item La suite de la compilation utilise un AST. Par exemple, en CUP\\
      \texttt{S::=    S:g PLUS S:d    \{: RESULT = new Sum(g, d); :\}}
    \end{itemize}
  }

  \tfExampleBlock[y=-2mm]{Exemple -- Dérivation de $u$ par $G$}{\vspace{-1mm}
    \begin{itemize}
    \item $u=\example{((1+1)\times (1+1))}$
    \item $G = \langle \{\example{1}, \example{+}, \example{\times}, \example{(}, \example{)}\}, \{S\}, S, \{S \rightarrow \example{(}S \example{+} S\example{)} | \example{(}S \example{\times} S\example{)} | \example{1} \} \rangle$
    \end{itemize}
  }

  \tf[bottom=-1mm, x=-.25\textwidth] {
    \begin{tikzpicture}
      \node[example] at (3.25,2.5) {Arbre de la syntaxe concrète};

      \node[structure] (a) at (3.25,2.1) {$S$};
      \node[alert]     (b) at (1.0,1.4)  {$($};      \path[-latex] (a) edge (b);  
      \node[structure] (c) at (2.0,1.4)  {$S$};      \path[-latex] (a) edge (c);  
      \node[alert]     (d) at (3.25,1.4) {$\times$}; \path[-latex] (a) edge (d);  
      \node[structure] (e) at (4.5,1.4)  {$S$};      \path[-latex] (a) edge (e);  
      \node[alert]     (f) at (5.5,1.4)  {$)$};      \path[-latex] (a) edge (f);  
      \node[alert]     (g) at (1.0,0.7)  {$($};      \path[-latex] (c) edge (g);  
      \node[structure] (h) at (1.5,0.7)  {$S$};      \path[-latex] (c) edge (h);  
      \node[alert]     (i) at (2.0,0.7)  {$+$};      \path[-latex] (c) edge (i);  
      \node[structure] (j) at (2.5,0.7)  {$S$};      \path[-latex] (c) edge (j);  
      \node[alert]     (k) at (3.0,0.7)  {$)$};      \path[-latex] (c) edge (k);  
      \node[alert]     (l) at (3.5,0.7)  {$($};      \path[-latex] (e) edge (l);  
      \node[structure] (m) at (4.0,0.7)  {$S$};      \path[-latex] (e) edge (m);  
      \node[alert]     (n) at (4.5,0.7)  {$+$};      \path[-latex] (e) edge (n);  
      \node[structure] (o) at (5.0,0.7)  {$S$};      \path[-latex] (e) edge (o);  
      \node[alert]     (p) at (5.5,0.7)  {$)$};      \path[-latex] (e) edge (p);  
      \node[alert]     (q) at (1.5,0.0)  {$1$};      \path[-latex] (h) edge (q);  
      \node[alert]     (r) at (2.5,0.0)  {$1$};      \path[-latex] (j) edge (r);  
      \node[alert]     (s) at (4.0,0.0)  {$1$};      \path[-latex] (m) edge (s);  
      \node[alert]     (t) at (5.0,0.0)  {$1$};      \path[-latex] (o) edge (t);  
    \end{tikzpicture}
  }
  \tf[bottom=-1mm, x=.25\textwidth] {
    \begin{tikzpicture}
      \node[example] at (1.5,2.5) {Arbre de la syntaxe abstraite};

      \node[structure] (a) at (1.5,2.0) {$\times$};
      \node[structure] (b) at (0.5,1.0) {$+$};            \smPath (a) edge (b);  
      \node[structure] (c) at (2.5,1.0) {$+$};            \smPath (a) edge (c);  
      \node[alert]     (d) at (0.0,0.0) {$1$};            \smPath (b) edge (d);  
      \node[alert]     (e) at (1.0,0.0) {$1$};            \smPath (b) edge (e);  
      \node[alert]     (f) at (2.0,0.0) {$1$};            \smPath (c) edge (f);  
      \node[alert]     (g) at (3.0,0.0) {$1$};            \smPath (c) edge (g);  

    \end{tikzpicture}
  }

  
\end{frame}

\endgroup
