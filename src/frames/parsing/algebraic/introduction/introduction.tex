% SPDX-License-Identifier: CC-BY-SA-4.0
% Author: Matthieu Perrin
% Part: 
% Section: 
% Sub-section: 
% Frame: 

\begingroup

\begin{frame}{Introduction aux langages algébriques}
  \begin{block}{Définition}
    Les \structure{langages algébriques} (ou \emph{langages context-free}) sont définis par :
    \begin{itemize}
      \item des grammaires algébriques
      \item des automates à pile non déterministes
      \item des arbres de dérivation syntaxique
    \end{itemize}
    Ils forment une classe notée $\structure{\textsc{cf}_\Sigma}$, avec $\alert{\textsc{rat}_\Sigma \subsetneq \textsc{cf}_\Sigma \subsetneq \mathcal{P}(\Sigma^\star)}$.
  \end{block}

  \begin{block}{Pourquoi s'y intéresser ?}
    \begin{itemize}
      \item Ils modélisent la structure syntaxique de nombreux langages.
      \item Ils permettent la dérivation d’arbres syntaxiques et d'AST :
        \begin{itemize}
          \item reconnaissance de la structure d’un programme (\texttt{CUP}, \texttt{Bison}, \texttt{ANTLR}, etc.)
          \item base des compilateurs et interpréteurs
          \item vérification de la syntaxe (analyse syntaxique)
        \end{itemize}
    \end{itemize}
  \end{block}
\end{frame}

\endgroup
