% SPDX-License-Identifier: CC-BY-SA-4.0
% Author: Matthieu Perrin
% Part: 
% Section: 
% Sub-section: 
% Frame: 

\begingroup

\begin{frame}{Exemple illustratif : grammaire française}

  \begin{itemize}
  \item Une grammaire est une collection de règles grammaticales du type :\vspace{1mm}
    \begin{center}
      \example{\og Une phrase est formée d'un sujet, et d'un groupe verbal \fg}\vspace{-2mm}
      $$\example{\mathit{Phrase} \rightarrow \mathit{Sujet}~\mathit{GV}}$$

      \begin{tikzpicture}
        \node[anchor=base, structure] (a) at (3.75,4) {$\mathit{Phrase}$};
        \node[anchor=base, structure] (b) at (2.00,3) {$\mathit{Sujet}$};
        \node[anchor=base, structure] (c) at (5.50,3) {$\mathit{Groupe~verbal}$};
        \node[anchor=base, structure] (d) at (2.00,2) {$\mathit{Groupe~nominal}$};
        \node[anchor=base, structure] (e) at (4.00,2) {$\mathit{Verbe}$};
        \node[anchor=base, structure] (f) at (7.00,2) {$\mathit{Groupe~nominal}$};
        \node[anchor=base, structure] (i) at (6.00,1) {$\mathit{Determinant}$};
        \node[anchor=base, structure] (j) at (8.00,1) {$\mathit{Nom~commun}$};
        \node[anchor=base, alert]     (g) at (2.00,0) {Jean};
        \node[anchor=base, alert]     (h) at (4.00,0) {mange};
        \node[anchor=base, alert]     (k) at (6.00,0) {une};
        \node[anchor=base, alert]     (l) at (8.00,0) {pomme};

        \path[-latex] (a) edge (b);
        \path[-latex] (a) edge (c);
        \path[-latex] (b) edge (d);
        \path[-latex] (d) edge (g);
        \path[-latex] (c) edge (e);
        \path[-latex] (c) edge (f);
        \path[-latex] (e) edge (h);
        \path[-latex] (f) edge (i);
        \path[-latex] (f) edge (j);
        \path[-latex] (i) edge (k);
        \path[-latex] (j) edge (l);
      \end{tikzpicture}
      
    \end{center}
    
  \item Les grammaires décrivent des mots et des langages par \structure{réécriture}.
  \item À la différence des définitions rationnelles, elles autorisent la \structure{récursivité}.
  \end{itemize}

\end{frame}

\endgroup
