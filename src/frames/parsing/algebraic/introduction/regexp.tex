% SPDX-License-Identifier: CC-BY-SA-4.0
% Author: Matthieu Perrin
% Part: 
% Section: 
% Sub-section: 
% Frame: 

\begingroup

\begin{frame}{Retour sur les expressions rationnelles}
  \begin{block}{Rappel -- Définition des expressions rationnelles}
    Soit $\Sigma$ un alphabet, et $\tilde{\Sigma} = \Sigma \sqcup \{ \emptyset, \varepsilon, (, ), |, \cdot, {}^\star \}.$

    \alert{$\textsc{reg}_\Sigma$} est le plus petit sous-ensemble de $\tilde{\Sigma}^\star$ tel que, \\

    \vspace{5mm}
    \begin{minipage}{.5\textwidth}
      $\forall a\in \Sigma, \forall u, v \in \textsc{reg}_\Sigma$,

      \vspace{2mm}
      $\begin{array}{lll}
        \vspace{1mm}\structure{\myRec} & \emptyset     & \in \textsc{reg}_\Sigma\\
        \vspace{1mm}\structure{\myRec} & \varepsilon   & \in \textsc{reg}_\Sigma\\
        \vspace{1mm}\structure{\myRec} & a             & \in \textsc{reg}_\Sigma\\
        \vspace{1mm}\structure{\myRec} & ( u | v )     & \in \textsc{reg}_\Sigma\\
        \vspace{1mm}\structure{\myRec} & ( u \cdot v ) & \in \textsc{reg}_\Sigma\\
        \vspace{1mm}\structure{\myRec} & u^\star        & \in \textsc{reg}_\Sigma\\
      \end{array}$
    \end{minipage}%
    \begin{minipage}{.5\textwidth}
      \structure{Présentation comme une grammaire :}
      
      \vspace{2mm}
      $\left\{\begin{array}{llll}
      \vspace{1mm}\text{Reg} &\rightarrow& \emptyset \\
      \vspace{1mm}\text{Reg} &\rightarrow& \varepsilon \\
      \vspace{1mm}\text{Reg} &\rightarrow& a & \forall a\in \Sigma \\
      \vspace{1mm}\text{Reg} &\rightarrow& (\text{Reg} | \text{Reg}) \\
      \vspace{1mm}\text{Reg} &\rightarrow& (\text{Reg} \cdot \text{Reg}) \\
      \vspace{1mm}\text{Reg} &\rightarrow& \text{Reg}^\star \\
      \end{array}\right.$
    \end{minipage}%
  \end{block}
\end{frame}

\endgroup
