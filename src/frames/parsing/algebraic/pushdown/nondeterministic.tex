% SPDX-License-Identifier: CC-BY-SA-4.0
% Author: Matthieu Perrin
% Part: 
% Section: 
% Sub-section: 
% Frame: 

\begingroup

\begin{frame}{Langages algébriques non-déterministes}
  Soit $\Sigma = \{a, b\}$.
  
  \begin{exampleblock}{Exemples et contre-exemples}
    \begin{itemize}
    \item Les langages $\example{\{ a^n b^n | n\in \mathbb{N}\}}$ et $\example{\{ a^n b^{2n} | n\in \mathbb{N}\}}$ sont déterministes.
    \item Leur union $\example{\{ a^n b^n | n\in \mathbb{N}\} \cup \{ a^n b^{2n} | n\in \mathbb{N}\}}$ est non-déterministe.
      \begin{itemize}
      \item Faut-il dépiler à chaque fois ou une fois sur deux ? 
      \end{itemize}
    \item Le langage $\example{\{ w \cdot c \cdot w^{\textsc{r}} | w\in \Sigma^\star\}}$ est déterministe.
      \begin{center}
        \structure{\scalebox{.8}{\begin{tikzpicture}[shorten >=1pt,node distance=2.5cm,on grid,auto]
              \node [state,initial above, initial text=$\diamond$] (b)  {$1$}; 
              \node [state, accepting] (c) [right=of b]  {$2$}; 
              \path [->]    (b) edge node[above] {$c/\varepsilon/\varepsilon$} (c);
              \path [->]    (b) edge[loop left, looseness=5] node {$\begin{array}{c}a/\varepsilon/A\\b/\varepsilon/B\end{array}$} (b);
              \path [->]    (c) edge[loop right, looseness=5] node {$\begin{array}{c}a/A/\varepsilon\\b/B/\varepsilon\end{array}$} (c);
        \end{tikzpicture}}}\\
        (Arrêt sur état final)
      \end{center}
    \item Par contre, le langage des palindromes est non-déterministe :
      $$\example{\{ w \cdot w^{\textsc{r}} | w\in \Sigma^\star\}}$$
    \end{itemize}
  \end{exampleblock}
\end{frame}

\endgroup
