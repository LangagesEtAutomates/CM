% SPDX-License-Identifier: CC-BY-SA-4.0
% Author: Matthieu Perrin
% Part: 
% Section: 
% Sub-section: 
% Frame: 

\begingroup

\begin{frame}{Langages reconnus par un automate à pile}
  
  Soit $A=\langle \Sigma, \Pi, Q, I, F, \diamond, \mu \rangle$ un automate à pile.

  \begin{block}{Définition -- langage reconnu}
    \begin{itemize}
    \item\vspace{-1mm} Le \structure{langage des mots reconnus par état final}, noté $\alert{\mathcal{L}_F(A)}$ contient les mots menant d'un état initial à un état final, quelle que soit la pile à l'arrivée. 
      $$\alert{\mathcal{L}_F(A) = \{u \in \Sigma^\star | \exists i\in I, \exists \structure{f\in F}, \exists \structure{\pi\in\Pi^\star}, \langle u, i, \diamond \rangle \leadsto_A^\star \langle \varepsilon, \structure{f}, \structure{\pi} \rangle\}}$$
    \item\vspace{-1mm} Le \structure{langage des mots reconnus par pile vide}, noté $\alert{\mathcal{L}_\varepsilon(A)}$ contient les mots menant d'un état initial à un état quelconque et une pile vide. 
      $$
      \alert{\mathcal{L}_\varepsilon(A) = \{u \in \Sigma^\star | \exists i\in I, \exists \structure{q\in Q}, \langle u, i, \diamond \rangle \leadsto_A^\star \langle \varepsilon, \structure{q}, \structure{\varepsilon} \rangle\}}
      $$
    \item\vspace{-1mm} Le \structure{langage des mots reconnus par état final et pile vide}, noté $\alert{\mathcal{L}(A)}$ contient les mots menant d'un état initial à un état final et une pile vide. 
      $$
      \alert{\mathcal{L}(A) = \{u \in \Sigma^\star | \exists i\in I, \exists \structure{f\in F}, \langle u, i, \diamond \rangle \leadsto_A^\star \langle \varepsilon, \structure{f}, \structure{\varepsilon} \rangle\}}
      $$
    \end{itemize}
  \end{block}

  \vspace{-2mm}
  \begin{block}{Équivalence entre méthodes de reconnaissance}
    \begin{itemize}
    \item \vspace{-2mm}Preuve en TD
    \end{itemize}
  \end{block}
\end{frame}

\endgroup
