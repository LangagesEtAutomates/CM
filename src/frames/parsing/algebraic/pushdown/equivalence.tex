% SPDX-License-Identifier: CC-BY-SA-4.0
% Author: Matthieu Perrin
% Part: 
% Section: 
% Sub-section: 
% Frame: 

\begingroup

\begin{frame}{Équivalence entre automates à pile et grammaires}
  \begin{block}{Théorème -- Équivalence entre formalismes}
    Tout langage peut être reconnu par un automate à pile ssi il est algébrique. 
  \end{block}
  
  \begin{block}{Preuve}
    \begin{enumerate}
    \item \structure{Des grammaires algébriques aux automates à pile}\\
      
      Soit $G = \langle \Sigma, \Gamma, S, R\rangle$ une grammaire en forme normale de Chomsky.
      
      Alors $\mathcal{L}(G) = \mathcal{L}(\langle \Sigma, \Gamma, \{q\}, \{q\}, \{q\}, S, \mu \rangle)$, avec
      $$\mu = \left\{ q \xrightarrow{\smPAtrans{a}{A}{\varepsilon}} q \,\middle\mid\, A \rightarrow a \in R \right\} \cup
      \left\{q\xrightarrow{\smPAtrans{\varepsilon}{A}{BC}}q \,\middle\mid\,  A \rightarrow BC\in R \right\}$$

      \begin{center}
        \scalebox{1}{\begin{tikzpicture}[smAutomaton, initial text=$S$]
            \smState[\smInitial\smAccepting] (q) {$q$};
            \smPath (q) edge [loop right] node {
              $\begin{array}{ll}
                \smPAtrans{a}{A}{\varepsilon}  & \text{ pour } A\rightarrow a \\
                \smPAtrans{\varepsilon}{A}{BC} & \text{ pour } A\rightarrow BC \\
              \end{array}$
            } (q);
        \end{tikzpicture}}
      \end{center}

      
    \item \structure{Des automates à pile aux grammaires algébriques}
      \begin{itemize}
      \item Admis, car un peu technique
      \end{itemize}
    \end{enumerate}
  \end{block}
\end{frame}

\endgroup
