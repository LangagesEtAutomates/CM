% SPDX-License-Identifier: CC-BY-SA-4.0
% Author: Matthieu Perrin
% Part: 
% Section: 
% Sub-section: 
% Frame: 

\begingroup

\begin{frame}{Automates à pile déterministes}

  Soit $A = \langle \Sigma, \Pi, Q, I, F, \diamond, \mu \rangle$ un automate à pile.

  \vspace{-1mm}
  \begin{block}{Définition -- Automate à pile déterministe}
    On dit que $A$ est \structure{déterministe} si toutes les conditions sont vérifiées : 
    \begin{enumerate}
    \item $A$ ne possède pas qu'un seul état initial : \alert{$|I| = 1$}
    \item Pour tout état $q \in Q$ et tout symbole de pile $p \in \Pi$, on a : 
      \begin{itemize}
      \item soit chaque lettre est consommée par au plus une transition \\
        et il n'existe pas d'$\varepsilon$-transition \\
      \end{itemize}
      $$\alert{\forall a, \unique q', p' : \langle \langle q, a, p\rangle, \langle q', p'\rangle \rangle \in \mu \land \forall q', p' : \langle \langle q, \varepsilon, p\rangle, \langle q', p'\rangle \rangle \notin \mu}$$
      \begin{itemize}
      \item \vspace{-5mm}soit il existe une unique $\varepsilon$-transition\\
        et aucune transition consommant des lettres :
      \end{itemize}
      $$\alert{\existsunique q', p' : \langle \langle q, \varepsilon, p\rangle, \langle q', p'\rangle \rangle \in \mu \land \forall u\in \Sigma^+, q', p' : \langle \langle q, u, p\rangle, \langle q', p'\rangle \rangle \notin \mu}$$
    \end{enumerate}
  \end{block}

  \vspace{-2mm}
  \begin{alertblock}{Remarque}
    \begin{itemize}
    \item \vspace{-1mm} On note $\mu(q, a, p) \eqdef \langle q', p'\rangle$ tel que $\langle \langle q, a, p\rangle, \langle q', p'\rangle \rangle \in \mu$
    \end{itemize}
  \end{alertblock}
\end{frame}

\endgroup
