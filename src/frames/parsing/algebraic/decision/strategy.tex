% SPDX-License-Identifier: CC-BY-SA-4.0
% Author: Matthieu Perrin
% Part: 
% Section: 
% Sub-section: 
% Frame: 

\begingroup

\begin{frame}{Statégie d'analyse syntaxique}
  \begin{alertblock}{Question}
    Étant donnés un mot $u$ et une grammaire $G$, est-ce que $u \in \mathcal{L}(G)$ ?
  \end{alertblock}

  \pause

  \begin{block}{Algorithme Cocke-Younger-Kasami (CYK)}

    \begin{description}
    \item [Entrées]
      \begin{itemize}
      \item Une grammaire algébrique $G = \langle \Sigma, \Gamma, S, R \rangle$ \\ sous \structure{forme normale de Chomsky}
      \item Un mot $u$
      \end{itemize}

    \item [Sortie :] Une réponse booléenne sur \structure{$u \in \mathcal{L}(G)$}

    \item [Complexité :] $\mathcal{O}\left(|R| \times |u|^3 \right)$
    \end{description}
  \end{block}
\end{frame}
\endgroup
