% SPDX-License-Identifier: CC-BY-SA-4.0
% Author: Matthieu Perrin
% Part: 
% Section: 
% Sub-section: 
% Frame: 

\begingroup


\begin{frame}[fragile]{Terminaison de l'algorithme}

  \begin{block}{Grammaires contextuelles}
    \begin{itemize}
    \item L'algorithme donne une \structure{réponse} :
      \begin{itemize}
      \item \alert{\True} si le mot appartient au langage
      \item \alert{\False} si le mot n'appartient pas au langage
      \end{itemize}
    \item \structure{Démonstration :}
      \begin{itemize}
      \item Si $v \vdash w$, alors $|v| \le |w|$, car les règles sont $\alpha A \beta \rightarrow \alpha \gamma \beta$, avec $\gamma\neq \varepsilon$
      \item Il y a un nombre fini de mots de $(\Sigma \cup \Gamma)^\star$ de taille au plus $|u|$
      \end{itemize}
      \begin{center}
        Les langages contextuels sont \alert{décidables}.
      \end{center}
    \end{itemize}
  \end{block}
  \pause
  \begin{block}{Grammaires non-restreintes}
    \begin{itemize}
    \item L'algorithme donne une \structure{semi-réponse} :
      \begin{itemize}
      \item \alert{\True} si le mot appartient au langage
      \item \alert{\False} ou \alert{boucle infiniment} si le mot n'appartient pas au langage
      \end{itemize}
    \item \structure{Contre-exemple :} {\footnotesize$G = \langle \{a\}, \{S\}, S, \{S \rightarrow a, SS \rightarrow S\}\rangle$\\
      $\{aa\} \leftarrow \{aS, Sa\} \leftarrow \{SS\} \leftarrow \{SSS\} \leftarrow \{SSSS\} \leftarrow ...$}
      \begin{center}
        Ces langages sont \alert{indécidables}, mais \alert{récursivement énumérables}.
      \end{center}
    \end{itemize}      
  \end{block}

\end{frame}
 

\endgroup
