% SPDX-License-Identifier: CC-BY-SA-4.0
% Author: Matthieu Perrin
% Part: 
% Section: 
% Sub-section: 
% Frame: 

\begingroup


\begin{frame}{Grammaire et langage contextuels}
  \begin{block}{Définition -- Grammaire contextuelle}
    Une grammaire \alert{$\langle \Sigma, \Gamma, S, R \rangle$} est dite \structure{contextuelle} si toutes ses règles sont :
    
    \begin{itemize}
    \item soit de la forme $\alert{g \cdot N \cdot d \rightarrow g \cdot \alpha \cdot d}$, avec

      \begin{center}
      \begin{tabular}{ccll}
      \alert{$g$}      & $\in$ & $(\Sigma \cup \Gamma)^\star$ & \structure{le contexte gauche}, \\
      \alert{$d$}      & $\in$ & $(\Sigma \cup \Gamma)^\star$ & \structure{le contexte droit}, \\
      \alert{$N$}      & $\in$ & $\Gamma$                    & un non-terminal, \\
      \alert{$\alpha$} & $\in$ & $(\Sigma \cup \Gamma)^+$    & un mot \alert{non-vide}. \\
      \end{tabular}
      \end{center}
      
%      \begin{description}
%      \item[\alert{$g$}] $\in (\Sigma \cup \Gamma)^\star$ : \structure{le contexte gauche},
%      \item[\alert{$d$}] $\in (\Sigma \cup \Gamma)^\star$ : \structure{le contexte droit},
%      \item[\alert{$N$}] $\in \Gamma$ : un non-terminal,
%      \item[\alert{$\alpha$}] $\in (\Sigma \cup \Gamma)^+$ : un mot \alert{non-vide}.
%      \end{description}
    Lire : \og{} $N$ peut se réécrire en $\alpha$, à condition d'être entre $g$ et $d$ \fg.
    \item soit la règle $S \rightarrow \varepsilon$, si $S$ n'apparaît jamais à droite d'une règle
    \end{itemize}
  \end{block}

  \begin{block}{Définition -- Langage contextuel}
    Un langage est dit \structure{contextuel} s'il existe une grammaire contextuelle qui l'engendre.
  \end{block}
\end{frame}
\endgroup
