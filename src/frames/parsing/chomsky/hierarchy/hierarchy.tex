% SPDX-License-Identifier: CC-BY-SA-4.0
% Author: Matthieu Perrin
% Part: 
% Section: 
% Sub-section: 
% Frame: 

\begingroup


\begin{frame}{Hiérarchie de chomsky}
  \small

  \centering
  \scalebox{.95}{
    \begin{tabular}{|c|c|c|c|c|}
      \hline
      Type
      &
      Grammaire
      &
      Langage
      &
      Automate
      &
      Déterminisable
      \\
      \hline
      \hline
      0 
      &
      \tabularcell[1.7cm]{Non-restreinte\\ $\alpha A \beta \rightarrow \gamma$}
      &
      \tabularcell[2cm]{Récursivement énumérable}
      &
      \tabularcell[2cm]{Machine de Turing}
      &
      \tabularcell[2cm]{Oui, mais complexité inconnue\\ \footnotesize $\textsc{p} \stackrel{?}{=} \textsc{np}$}
      \\
      \hline
      1
      &
      \tabularcell[1.7cm]{Contextuelle\\ $\alpha A \beta \rightarrow \alpha \gamma \beta$ \\ $\gamma\neq\varepsilon$}
      &
      \tabularcell[2cm]{Contextuel}
      &
      \tabularcell[2cm]{Automate \\linéairement borné}
      &
      \tabularcell[2cm]{Inconnu\\ \footnotesize $\textsc{nspace}(\mathcal{O}(n))$ \\ $\stackrel{?}{=}$\\
        $\textsc{dspace}(\mathcal{O}(n))$}
      \\
      \hline
      2
      &
      \tabularcell[1.7cm]{Algébrique\\ $A \rightarrow \beta$}
      &
      \tabularcell[2cm]{Algébrique}
      &
      \tabularcell[2cm]{Automate à pile non-déterministe}
      &
      \tabularcell[2cm]{Impossible}
      \\
      \hline
      3
      &
      \tabularcell[1.7cm]{Rationnelle\\ $A \rightarrow a B | b$}
      &
      \tabularcell[2cm]{Rationnel}
      &
      \tabularcell[2cm]{Automate fini}
      &
      \tabularcell[2cm]{Oui, exponentiel en nombre d'états}
      \\
      \hline
    \end{tabular}
  }
\end{frame}

\endgroup
