% SPDX-License-Identifier: CC-BY-SA-4.0
% Author: Matthieu Perrin
% Part: 
% Section: 
% Sub-section: 
% Frame: 

\begingroup

\begin{frame}{Classification des grammaires}
  On dit qu'une grammaire $G = \langle \Sigma, \Gamma, S, R \rangle$ \structure{est de type} $i \in \{0, 1, 2, 3\}$ si :
  \begin{description}
  \item[Type 0 :] $G$ est grammaire \structure{non-containte}\\
    \begin{itemize}
    \item $R \subset (\Sigma\cup \Gamma)^\star \cdot \Gamma \cdot (\Sigma\cup \Gamma)^\star) \times (\Sigma \cup \Gamma)^\star$ 
    \item[\example{Exemple :}] $a B c \rightarrow c B a$
    \end{itemize}
  \item[Type 1 :] $G$ est grammaire \structure{contextuelle}\\
    \begin{itemize}
    \item $\forall r\in R, r =  S \rightarrow \varepsilon \lor \exists g, d, \gamma \in (\Sigma \cup \Gamma \setminus \{S\})^\star,$ \\ $\exists A\in \Gamma, \gamma\neq \varepsilon \land r = g A d \rightarrow g \gamma d$
    \item[\example{Exemple :}] $a B c \rightarrow ac B ac | acac$ 
    \end{itemize}
  \item[Type 2 :] $G$ est grammaire \structure{algébrique}\\
    \begin{itemize}
    \item $R \subset \Gamma \times (\Sigma \cup \Gamma)^\star$ 
    \item[\example{Exemple :}] $S \rightarrow a S b | \varepsilon$ 
    \end{itemize}

  \item[Type 3 :] $G$ est grammaire \structure{rationnelle droite}
    \begin{itemize}
    \item $R \subset \Gamma \times (\Sigma \cdot \Gamma^?)^?$ 
    \item[\example{Exemple :}] $S \rightarrow aS | b | \varepsilon$ 
    \end{itemize}

    ou $G$ est grammaire \structure{rationnelle gauche}
    \begin{itemize}
    \item $R \subset \Gamma \times (\Gamma^? \cdot \Sigma)^?$ 
    \item[\example{Exemple :}] $S \rightarrow Sa | b | \varepsilon$ 
    \end{itemize}
  \end{description}
\end{frame}

\endgroup
