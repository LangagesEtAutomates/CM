% SPDX-License-Identifier: CC-BY-SA-4.0
% Author: Matthieu Perrin
% Part: 
% Section: 
% Sub-section: 
% Frame: 

\begingroup

\begin{frame}{Classification des langages}

  Le \structure{type d'un langage $L$} est le plus grand $i \in \{0, 1, 2, 3\}$ (si un tel $i$ existe) \\
  tel qu'il existe une grammaire de type $i$ qui engendre $L$. 

    \vspace{-1mm}
  \begin{minipage}{.5\textwidth}
    \begin{block}{Types de grammaires}
      \centering\scalebox{.6}{\begin{tikzpicture}
          \fill[structure!20] (0.00,0.75) ellipse(3.0cm and 2.0cm);
          \fill[alert!10] (0.00,1.00) ellipse(2.0cm and 1.0cm);
          \fill[structure!10] (0.00,0.00) ellipse(2.0cm and 1.0cm);
          \fill[structure!20] (0.00,0.25) ellipse(1.0cm and 0.5cm);

          \begin{scope}
            \clip (0.00,1.00) ellipse(2.0cm and 1.0cm);
            \fill[structure!40!alert!15] (0.00,0.00) ellipse(2.0cm and 1.0cm);
            \fill[structure!50!alert!25] (0.00,0.25) ellipse(1.0cm and 0.5cm);
          \end{scope}
          
          \draw[structure] (0.00,0.75) ellipse(3.0cm and 2.0cm);
          \draw[alert] (0.00,1.00) ellipse(2.0cm and 1.0cm);
          \draw[structure] (0.00,0.00) ellipse(2.0cm and 1.0cm);
          \draw[structure] (0.00,0.25) ellipse(1.0cm and 0.5cm);

          
          \draw (0,0.20) node{3 : rationnelle};
          \draw (0,-0.5) node{2 : algébrique};
          \draw (0,1.50) node{1 : contextuelle};
          \draw (0,2.25) node{0 : non-restreinte};
      \end{tikzpicture}}
    \end{block}
  \end{minipage}%
  \begin{minipage}{.5\textwidth}
    \begin{block}{Types de langages}
      \centering\scalebox{.6}{\begin{tikzpicture}
          \draw[structure, fill=structure!10] (1.5,0.75) ellipse(4.0cm and 2.0cm);
          \draw[structure, fill=structure!20] (1.0,0.50) ellipse(3.0cm and 1.5cm);
          \draw[structure, fill=structure!10] (0.5,0.25) ellipse(2.0cm and 1.0cm);
          \draw[structure, fill=structure!20] (0.0,0.00) ellipse(1.0cm and 0.5cm);

          \draw (0.0,0.0) node{3 : rationnel};
          \draw (0.5,0.75) node{2 : algébrique};
          \draw (1.0,1.5) node{1 : contextuel};
          \draw (1.5,2.25) node{0 : récursivement énumérable};
      \end{tikzpicture}}
    \end{block}
  \end{minipage}

    \vspace{-1mm}
  \begin{exampleblock}{Exemple}
    \vspace{-2mm}
    \begin{enumerate}
    \item $G_1 = \langle \{a,b\},\{S,A\},S,R=\{ S \rightarrow  bA  \;|\;  a, bA \rightarrow a\}\rangle$\\
      \begin{itemize}
      \item  \example{pas type 3 ni 2} car $bA \rightarrow a\in R$ et $bA\not\in \Gamma$
      \item  \example{pas type 1} car $bA \rightarrow a\in R$
      \item  \example{type 0}
      \end{itemize}
    \item $G_2 = (\{a,b\},\{S\},S,\{S \rightarrow a\})$
      \begin{itemize}
      \item \example{type 3} car $S\in \Gamma$ et $a\in \Sigma^+$
      \end{itemize}
    \end{enumerate}
    $\mathcal{L}(G_1) = \mathcal{L}(G_2) = \{a\}$ est de \example{type 3}.
  \end{exampleblock}
\end{frame}

\endgroup
