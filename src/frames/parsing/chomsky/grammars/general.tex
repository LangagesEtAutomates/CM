% SPDX-License-Identifier: CC-BY-SA-4.0
% Author: Matthieu Perrin
% Part: 
% Section: 
% Sub-section: 
% Frame: 

\begingroup


\begin{frame}{Généralisation du concept de grammaire}
  \begin{block}{Définition -- Grammaire non-restreinte}
    \vspace{2mm}
    Une \structure{grammaire non-restreinte} est un quadruplet \alert{$\langle \Sigma, \Gamma, S, R \rangle$} tel que :
    \begin{description}
    \item[\alert{$\Sigma$}] alphabet : \structure{l'ensemble des terminaux}
    \item[\alert{$\Gamma$}] alphabet tel que $\Sigma \cap \Gamma = \emptyset$ : \structure{l'ensemble des non-terminaux}
    \item[\alert{$S$}] $\in \Gamma$ : \structure{l'axiome} (ou symbole initial)
    \item[\alert{$R$}] $\subseteq ((\Sigma \cup \Gamma)^\star \cdot \Gamma \cdot (\Sigma \cup \Gamma)^\star) \times (\Sigma \cup \Gamma)^\star$ : \\\structure{l'ensemble des règles de production}
    \end{description}

    \vspace{3mm}
    Une \structure{règle de production} est un couple \alert{$\langle \alpha, \beta \rangle \in R$}, noté $\alert{\alpha \rightarrow \beta}$ tel que :
    \begin{description}
    \item[\alert{$\alpha$}] $\in (\Sigma \cup \Gamma)^\star \cdot \Gamma \cdot (\Sigma \cup \Gamma)^\star$ : \structure{membre gauche}
    \item[\alert{$\beta$}] $\in (\Sigma \cup \Gamma)^\star$ : \structure{membre droit}
    \end{description}
  \end{block}
  \begin{block}{Remarque -- Grammaire algébrique}
    Une grammaire $\langle \Sigma, \Gamma, S, R \rangle$ est algébrique si, et seulement si,
    
    \vspace{-3mm}  $$\forall \alpha \rightarrow \beta \in R, \alpha \in \Gamma^1.$$
  \end{block}
\end{frame}

\endgroup
