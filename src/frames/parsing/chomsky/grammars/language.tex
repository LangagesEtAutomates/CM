% SPDX-License-Identifier: CC-BY-SA-4.0
% Author: Matthieu Perrin
% Part: 
% Section: 
% Sub-section: 
% Frame: 

\begingroup

\begin{frame}{Langage engendré par une grammaire non-restreinte}
  Soient \alert{$\langle \Sigma, \Gamma, S, R \rangle$} une grammaire non-restreinte, et $u, v \in (\Sigma \cup \Gamma)^\star$. 
  \begin{itemize}
  \item On dit que \structure{$u$ se dérive directement en $v$}, noté $\alert{u \vdash v}$, si :
    $$\alert{\exists \structure{x}, \structure{y} \in (\Sigma \cup \Gamma)^\star, \exists \example{\alpha \rightarrow \beta} \in R,
    u = \structure{x} \cdot \example{\alpha} \cdot \structure{y} \land v = \structure{x} \cdot \example{\beta} \cdot \structure{y}}$$

  \item On note parfois \structure{$x \alpha y \vdash_{\alpha \rightarrow \beta} x \beta y$} pour indiquer la règle utilisée.

  \item On dit que \structure{$u$ se dérive en $v$} si \alert{$u \vdash^\star v$},\\
    où \alert{$\vdash^\star$ est la fermeture transitive et réflexive de $\vdash$}.

  \item On dit qu'un mot $u \in \Sigma^\star$ est \structure{généré} par $G$ s'il existe une dérivation $S \vdash^\star u$, appelée \structure{génération} de $u$. 
  \item Le langage $\alert{\mathcal{L}(G)}$ des mots générés par $G$ est appelé le \structure{langage engendré} par $G$ :
  \end{itemize}
  $$\alert{\mathcal{L}(G) = \{u\in\Sigma^\star | S \vdash^\star u \}}.$$
\end{frame}

\endgroup
