% SPDX-License-Identifier: CC-BY-SA-4.0
% Author: Matthieu Perrin
% Part: 
% Section: 
% Sub-section: 
% Frame: 

\begingroup

\begin{frame}{Graphe de dérivation}

  Soit \alert{$G = \langle \Sigma, \Gamma, S, R \rangle$} une grammaire non-restreinte.

  \begin{block}{Graphe de dérivation}
    En général, une dérivation ne peut plus être représentée par un arbre.

    Dans ce cas, on représente parfois une génération par un graphe. 
  \end{block}

  \begin{exampleblock}{Exemple}

      \vspace{3mm}
      \example{$G = \left\langle \{a, b, c\}, \{S, B\}, S, \left\{\begin{array}{rcl} S &\rightarrow & bS \,|\, bA \\ bA &\rightarrow & aa\end{array}\right\} \right\rangle$}.\\


      \vspace{5mm}
      Génération $S \vdash bS \vdash bbA \vdash baa$ :


      \vspace{-10mm}
      \hspace\fill
      \begin{tikzpicture}
          \draw[rounded corners, example, fill=example!20] (4.7,2.7) rectangle (7.3,3.3);
          
          \draw(5.0,5) node{$S$};
          \draw(4.0,4) node{$b$};
          \draw(6.0,4) node{$S$};
          \draw(5.0,3) node{$b$};
          \draw(7.0,3) node{$A$};
          \draw(5.0,2) node{$a$};
          \draw(7.0,2) node{$a$};

          \draw[-latex] (4.8,4.8) -- (4.2,4.2);
          \draw[-latex] (5.2,4.8) -- (5.8,4.2);

          \draw[-latex] (5.8,3.8) -- (5.2,3.2);
          \draw[-latex] (6.2,3.8) -- (6.8,3.2);

          \draw[-latex] (5.2,2.8) -- (6.8,2.2);
          \draw[-latex] (6.8,2.8) -- (5.2,2.2);

          \draw[-latex] (5.0,2.8) -- (5.0,2.2);
          \draw[-latex] (7.0,2.8) -- (7.0,2.2);
          
      \end{tikzpicture}
  \end{exampleblock}
\end{frame}

\endgroup
