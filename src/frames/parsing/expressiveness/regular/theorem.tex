% SPDX-License-Identifier: CC-BY-SA-4.0
% Author: Matthieu Perrin
% Part: 
% Section: 
% Sub-section: 
% Frame: 

\begingroup

\begin{frame}{Lien entre langages rationnels et algébriques}
  
  \begin{block}{Théorème -- Lien entre langages rationnels et algébriques}
    Tout langage rationnel sur un alphabet $\Sigma$ est algébrique sur $\Sigma$ :
    $$\alert{\textsc{rat}_\Sigma \subsetneq \textsc{alg}_\Sigma}$$
  \end{block}

  \vspace{-2mm}
  \begin{block}{Démonstration}
    \begin{itemize}
    \item Introduction des grammaires rationnelles
      \begin{itemize}
      \item Preuve de l'équivalence avec les automates finis
      \end{itemize}
    \item L'inclusion est stricte, car $\{a^n b^n \mid  n\in \mathbb{N} \} \in \textsc{alg}_\Sigma \setminus \textsc{rat}_\Sigma$
    \end{itemize}
  \end{block}

  \begin{exampleblock}{Exemple}
    Le langage \example{$a^\star b^\star$} est engendré par
    $$\example{\left\langle \{a, b\}, \{S, A\}, S, \left\{ \begin{array}{@{\,}r@{~\rightarrow~}l@{\,\mid\,}l@{\,}}
        S & aS & A\\
        A & bA & \varepsilon
      \end{array}\right\}\right\rangle}$$
  \end{exampleblock}
  
\end{frame}

\endgroup
