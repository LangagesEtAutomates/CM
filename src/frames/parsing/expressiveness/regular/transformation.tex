% SPDX-License-Identifier: CC-BY-SA-4.0
% Author: Matthieu Perrin
% Part: 
% Section: 
% Sub-section: 
% Frame: 

\begingroup

\begin{frame}{Grammaires pour les langages rationnels}

  Un langage $L$ est rationnel ssi $L$ est engendré par une grammaire :\\
  \vspace{2mm}
  \begin{minipage}{.5\textwidth}
    \begin{itemize}
    \item linéaire à gauche
    \item rationnelle à gauche 
    \end{itemize}
  \end{minipage}%
  \begin{minipage}{.5\textwidth}
    \begin{itemize}
    \item linéaire à droite 
    \item rationnelle à droite 
    \end{itemize}
  \end{minipage}
  
  \begin{block}{De linéaire à droite à rationnelle à droite}
    \begin{itemize}
    \item Transformation des règles similaire à la forme normale de Chomsky
    \end{itemize}
    
    \footnotesize $$\example{
      \left\{\begin{array}{@{\,}l@{~\rightarrow~}l@{\,}}
      S & abA \mid A \\
      A & c
      \end{array}\right.
      ~\Rightarrow~
      \left\{\begin{array}{@{\,}l@{~\rightarrow~}l@{\,}}
      S & abA \mid c \\
      A & c
      \end{array}\right.
      ~\Rightarrow~
      \left\{\begin{array}{@{\,}l@{~\rightarrow~}l@{\,}}
      S & aB \mid c \\
      A & c         \\
      B & bA
      \end{array}\right.
    }$$
  \end{block}

  \begin{block}{De rationnelle à gauche à rationnelle à droite}
    \begin{itemize}
    \item On passe par le miroir : $\mathcal{L}(G)^{\textsc{r}} = \mathcal{L}(\langle \Sigma, \Gamma, S, \{ \langle g, d^{\textsc{r}} \rangle \mid g \rightarrow d\} \rangle)$.
    \end{itemize}
    
    \footnotesize $$\example{
      \left\{\begin{array}{@{\,}r@{~\rightarrow~}l@{\,\mid\,}l@{\,}}
      S & Aa\\
      A & Aa & Bb\\
      B & Bb & \varepsilon
      \end{array}\right.
      ~\xRightarrow{\textsc{r}}~
      \left\{\begin{array}{@{\,}r@{~\rightarrow~}l@{\,\mid\,}l@{\,}}
      S & aA\\
      A & aA & bB\\
      B & bB & \varepsilon
      \end{array}\right.
      ~\Rightarrow~
      \begin{tikzpicture}[automaton, baseline=(c)]
        \state[initial below]   (S) at (0.3, 0.5) {$S$}; 
        \state                  (A) at (1.0, 1.0) {$A$}; 
        \state[accepting ]      (B) at (1.0, 0.0) {$B$}; 
        \path (S) edge             node     {$a$} (A);
        \path (A) edge[loop right] node     {$a$} (A);
        \path (A) edge             node (c) {$b$} (B);
        \path (B) edge[loop right] node     {$b$} (B);
      \end{tikzpicture}
      ~\xRightarrow{\textsc{r}}~
      \begin{tikzpicture}[automaton, baseline=(c)]
        \state[accepting]   (S) at (0.3, 0.5) {$S$}; 
        \state              (A) at (1.0, 1.0) {$A$}; 
        \state[initial]     (B) at (1.0, 0.0) {$B$}; 
        \path (A) edge             node[swap] {$a$} (S);
        \path (A) edge[loop right] node {$a$} (A);
        \path (B) edge             node[swap] (c) {$b$} (A);
        \path (B) edge[loop right] node {$b$} (B);
      \end{tikzpicture}
      ~\Rightarrow~
      \left\{\begin{array}{@{\,}r@{~\rightarrow~}l@{\,\mid\,}l@{\,}}
      B & bB & bA\\
      A & aA & aS\\
      S & \varepsilon
      \end{array}\right.
    }$$
  \end{block}
  
\end{frame}

\endgroup
