% SPDX-License-Identifier: CC-BY-SA-4.0
% Author: Matthieu Perrin
% Part: 
% Section: 
% Sub-section: 
% Frame: 

\begingroup

\begin{frame}{Lemme de pompage}

  \vspace{-5mm}
  \begin{block}{Question -- Expressivité des grammaires algébriques}
    \vspace{-2mm}
    \begin{itemize}
    \item Comment montrer qu'un langage $L$ n'est pas algébrique ? 
      \begin{enumerate}
      \item Si $L$ était algébrique, il vérifierait le \structure{lemme de pompage} 
      \item Or, $L$ ne vérifie pas le \structure{lemme de pompage} 
      \item Donc, par l'absurde, $L$ ne peut pas être algébrique
      \end{enumerate}
    \end{itemize}
  \end{block}
  
  \vspace{-1mm}
  \begin{block}{Rappel -- Lemme de l'étoile}
    \vspace{-1mm}
    $\forall\Sigma, \forall L\in \textsc{rat}_\Sigma, \exists N\in \mathbb{N}, \forall u\in L, |u| \ge N \Rightarrow \exists \structure{x, y, z}\in \Sigma^\star,$

    \vspace{1mm}
    \begin{minipage}{.4\textwidth}
      \begin{itemize}
      \item $u = \structure{x\cdot y \cdot z}$
      \item $\structure{y}\neq \varepsilon$
      \end{itemize}
    \end{minipage}%
    \begin{minipage}{.5\textwidth}
      \begin{itemize}
      \item $|\structure{x\cdot y}| \le N$
      \item $\forall i \in \mathbb{N}, x\cdot \structure{y^i} \cdot z \in L$
      \end{itemize}
    \end{minipage}
  \end{block}

  \vspace{-1mm}
  \begin{alertblock}{Lemme -- Lemme de pompage (ou d'itération)\footnote{\scriptsize Y. Bar-Hillel, M. Perles, E. Shamir. \textit{On formal properties of simple phrase structure grammars.} 1961}}
    \vspace{-1mm}
    $\forall\Sigma, \forall L\in \textsc{alg}_\Sigma, \exists N\in \mathbb{N}, \forall u\in L, |u| \ge N \Rightarrow \exists \alert{v, w, x, y, z}\in \Sigma^\star,$
    
    \vspace{1mm}
    \begin{minipage}{.4\textwidth}
      \begin{itemize}
      \item $\alert{u = v\cdot w\cdot x \cdot y \cdot z}$
      \item $\alert{w\cdot y} \neq \varepsilon$
      \end{itemize}
    \end{minipage}%
    \begin{minipage}{.5\textwidth}
      \begin{itemize}
      \item $|\alert{w\cdot x \cdot y}| \le N$
      \item $\forall i \in \mathbb{N}, v\cdot \alert{w^i} \cdot x \cdot \alert{y^i} \cdot z \in L$
      \end{itemize}
    \end{minipage}
  \end{alertblock}

  \begin{description}
  \item[Remarque :] Si $L$ est rationnel, on peut avoir $v = w = \varepsilon$.
  \end{description}

\end{frame}

\endgroup
