% SPDX-License-Identifier: CC-BY-SA-4.0
% Author: Matthieu Perrin
% Part: 
% Section: 
% Sub-section: 
% Frame: 

\begingroup

\begin{frame}{Exemple illustratif : grammaire française}

  \begin{itemize}
  \item Une grammaire est une collection de règles grammaticales du type :\vspace{1mm}
    \begin{center}
      \example{\og Une phrase est formée d'un sujet, et d'un groupe verbal \fg}\vspace{-2mm}
      $$\example{\mathit{Phrase} \rightarrow \mathit{Sujet}~\mathit{GV}}$$
      \begin{tikzpicture}[tree, x=2cm, tree node internal/.append style={structure,}, tree node leave/.append style={alert,},]
        \tree                  {$\mathit{Phrase}$}{
          \tree                {$\mathit{Sujet}$}{
            \tree              {$\mathit{Groupe~nominal}$}{
              \tree[yshift=-1] {Jean}{}
            }                   
          }                     
          \tree                {$\mathit{Groupe~verbal}$}{
            \tree              {$\mathit{Verbe}$}{
              \tree[yshift=-1] {mange}{}
            }                   
            \tree              {$\mathit{Groupe~nominal}$}{
              \tree            {$\mathit{Determinant}$}{
                \tree          {une}{}
              }                 
              \tree            {$\mathit{Nom~commun}$}{
                \tree          {pomme}{}
              }
            }
          }
        }
      \end{tikzpicture}
    \end{center}
  \item Les grammaires décrivent des mots et des langages par \structure{réécriture}.
  \item À la différence des définitions rationnelles, elles autorisent la \structure{récursivité}.
  \end{itemize}

\end{frame}

\endgroup
