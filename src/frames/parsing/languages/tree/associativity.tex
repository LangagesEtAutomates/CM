% SPDX-License-Identifier: CC-BY-SA-4.0
% Author: Matthieu Perrin
% Part: 
% Section: 
% Sub-section: 
% Frame: 

\begingroup

\begin{frame}{Associativité des opérateurs binaires}

  \onBlock[top=-4mm]{Définition -- Associativité}{
    L'\structure{associativité} d'un opérateur $\oplus$ précise dans quel ordre évaluer

    \vspace{-2mm}
    $$\alert{x \oplus y \oplus z}.$$
    
    \begin{itemize}
    \item\vspace{-1mm} $\oplus$ est \structure{associatif à gauche} quand :
      \hspace\fill \alert{$x \oplus y \oplus z$ signifie $(x \oplus y) \oplus z$}\\
      \example{Exemple : \lstinline{4 - 2 - 1} signifie \texttt{(4-2)-1}}
    \item\vspace{1mm} $\oplus$ est \structure{associatif à droite} quand :
      \hspace\fill \alert{$x \oplus y \oplus z$ signifie $x \oplus (y \oplus z)$}\\
      \example{Exemple : \lstinline{x = y = 0;} signifie \texttt{x = (y = 0);}}
    \end{itemize}
  }

  \onBlock[anchor=north, y=0mm, left=.5\textwidth]{Associativité à gauche :}{
    \centering
    \begin{tikzpicture}[tree, y=7mm, tree node/.append style={structure,},]
      \tree[name=root] {$A$}{
        \tree {$A$}{
          \tree[xshift=-1] {$A$}{
            \tree[xshift=1] {$B$}{}
          }
          \tree {\alert{$\oplus$}}{}
          \tree {$B$}{}
        }
        \tree[xshift=-1] {\alert{$\oplus$}}{}
        \tree[xshift=-1] {$B$}{}
      }
      \node[alert] at ([yshift=7mm]root) {$A \, \rightarrow \, A \oplus B \mid B$};
    \end{tikzpicture}
  }

  \onBlock[anchor=north, y=0mm, right=.5\textwidth]{Associativité à droite :}{
    \centering
    \begin{tikzpicture}[tree, y=7mm, tree node/.append style={structure,},]
      \tree[name=root] {$A$}{
        \tree[xshift=1] {$B$}{}
        \tree[xshift=1] {\alert{$\oplus$}}{}
        \tree {$A$}{
          \tree {$B$}{}
          \tree {\alert{$\oplus$}}{}
          \tree[xshift=1] {$A$}{
            \tree[xshift=-1] {$B$}{}
          }
        }
      }
      \node[alert] at ([yshift=7mm]root) {$A \, \rightarrow \, B \oplus A \mid B$};
    \end{tikzpicture}
  }
  
\end{frame}

\endgroup
