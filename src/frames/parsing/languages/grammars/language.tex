% SPDX-License-Identifier: CC-BY-SA-4.0
% Author: Matthieu Perrin
% Part: 
% Section: 
% Sub-section: 
% Frame: 

\begingroup

\begin{frame}{Langages algébriques}
  
  \begin{block}{Définitions -- Génération et langage engendré}
    Soient \alert{$G = \langle \Sigma, \Gamma, S, \rightarrow \rangle$} une grammaire algébrique et \alert{$u \in \Sigma^\star$} un mot sur $\Sigma$. 
    \begin{itemize}
    \item Une \structure{génération de $u$ par $G$} est une dérivation \alert{$S \vdash^\star u$} à partir de $S$.
    \item On dit que \structure{$u$ est généré par $G$} s'il existe une génération de $u$ par $G$. 
    \item Le \structure{langage engendré} par $G$, $\alert{\mathcal{L}(G)}$, contient les mots générés par $G$ :
      $$\alert{\mathcal{L}(G) = \left\{u\in\Sigma^\star \,\middle\mid\, S \vdash^\star u \right\}}.$$
    \item Un langage engendré par une grammaire algébrique est dit \structure{algébrique}.
    \item L'ensemble des langages algébriques sur $\Sigma$ est noté $\alert{\textsc{alg}_\Sigma}$.
    \end{itemize}
  \end{block}

  \begin{exampleblock}{Exemple}
    Soit \example{$G = \left\langle \{a, b, c\}, \{S, B\}, S, \left\{\begin{array}{@{\,}r@{~\rightarrow~}c@{\,\mid\,}c@{}}
    S & aSb & B \\
    B & cB & \varepsilon
    \end{array}\right\} \right\rangle$}. On a :
    \begin{itemize}
    \item \example{$aacbb \in \mathcal{L}(G)$}, car $S \vdash aSb \vdash aaSbb \vdash aaBbb \vdash aacBbb \vdash aacbb$.
    \item $\mathcal{L}(G) = \{a^n c^m b^n \mid m, n \in \mathbb{N}\}$ est algébrique. 
    \end{itemize}
  \end{exampleblock}

\end{frame}

\endgroup
