% SPDX-License-Identifier: CC-BY-SA-4.0
% Author: Matthieu Perrin
% Part: 
% Section: 
% Sub-section: 
% Frame: 

\begingroup

\begin{frame}{Exemples de grammaires et de langages algébriques}

  \begin{enumerate}
  \item \structure{Langage des parenthèses imbriquées}\vspace{2mm}
    \begin{itemize}
    \item $\mathcal{L}(\langle\{\alert{a}, \alert{b}\},\{S\},S,\{S \rightarrow \alert{a}S\alert{b} \;|\; \varepsilon\}\rangle) = \{\alert{a}^n\alert{b}^n, n\geq 0\}$\\
    \item $\mathcal{L}(\langle\{\alert{(}, \, \alert{)}\},\{S\},S,\{S \rightarrow \alert{(}S\alert{)}\, \;|\;\varepsilon \}\rangle) = \{\alert{(}^n\, \alert{)}^n, n\geq 0\}$
    \end{itemize}
    
  \item \structure{Langage de Dyck}\vspace{2mm}

    chaque parenthèse ouvrante correspond à une parenthèse fermante
    \begin{itemize}
    \item sur un couple de parenthèses, $\Sigma = \{\alert{(}, \alert{)}\}$\\
      $\mathcal{L}(\langle \Sigma ,\{S\},S,\{S \rightarrow S\alert{(}S\alert{)} \;|\; \varepsilon \}\rangle)$
    \item sur $n$ couples de parenthèses, $\Sigma_n = \{\alert{(_1}, \alert{)_1}, \ldots, \alert{(_n}, \alert{)_n}\}$\\
      $\mathcal{L}(\langle \Sigma ,\{S\},S,\{S \rightarrow S\alert{(_1}S\alert{)_1} \;|\; \ldots \;|\;S\alert{(_n}S\alert{)_n} \;|\; \varepsilon \}\rangle)$
    \end{itemize}
    \vspace{2mm}
    
  \item \structure{Expressions bien parenthésées}\vspace{2mm}
    
    \begin{itemize}
    \item $\mathcal{L}(\langle \{\alert{(}, \alert{)}, \alert{a}\} ,\{S\},S,\{S \rightarrow \alert{(}S\alert{)} \;|\; SS \;|\; a\}\rangle)$
    \end{itemize}
    \vspace{2mm}
    
  \item Syntaxe des langages de programmation : C, C++, Java...\vspace{2mm}
    
  \item Formats de données : HTML, XML, CSV...\vspace{2mm}
    
  \item Formules logiques, mathématiques...
  \end{enumerate}

\end{frame}

\endgroup
