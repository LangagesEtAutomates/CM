% SPDX-License-Identifier: CC-BY-SA-4.0
% Author: Matthieu Perrin
% Part: 
% Section: 
% Sub-section: 
% Frame: 

\begingroup

\begin{frame}{Dérivations}

  \begin{block}{Définitions -- Dérivation}
    Soient \structure{$\langle \Sigma, \Gamma, S, \rightarrow \rangle$} une grammaire algébrique et \structure{$u, v \in (\Sigma \cup \Gamma)^\star$} deux mots. 

    \begin{itemize}
    \item On dit que \structure{$u$ se dérive directement en $v$}, noté $\alert{u \vdash v}$, si :

      $$\alert{\exists x, y, A, \alpha \in (\Sigma \cup \Gamma)^\star,~
        \example{A \rightarrow \alpha} \land u = \structure{x} \cdot \example{A} \cdot \structure{y} \land v = \structure{x} \cdot \example{\alpha} \cdot \structure{y}}$$

    \item On note parfois \structure{$x \example{A} y \alert{\vdash_{A \rightarrow \alpha}} x \example{\alpha} y$} pour indiquer la règle utilisée.

    \item On dit que \structure{$u$ se dérive en $v$} si \alert{$u \vdash^\star v$},\\
      où \alert{$\vdash^\star$ est la fermeture transitive et réflexive de $\vdash$}.

    \end{itemize}
  \end{block}

  \begin{exampleblock}{Exemple}
    Soit \example{$G = \left\langle \{a, b, c\}, \{S, B\}, S, \left\{\begin{array}{@{\,}r@{~\rightarrow~}c@{\,\mid\,}c@{}}
    S & aSb & B \\
    B & cB & \varepsilon
    \end{array}\right\} \right\rangle$}. On a :

    \vspace{2mm}
    \begin{itemize}
    \item $\structure{a}\example{S}\structure{b} \vdash_{S \rightarrow aSb} \structure{a}\example{aSb}\structure{b} \vdash aaBbb \vdash aacBbb \vdash aaccBbb$
    \item $aSb \vdash^\star aSb \vdash^\star aaccBbb$
    \end{itemize}
  \end{exampleblock}

\end{frame}

\endgroup
