% SPDX-License-Identifier: CC-BY-SA-4.0
% Author: Matthieu Perrin
% Part: 
% Section: 
% Sub-section: 
% Frame: 

\begingroup

\begin{frame}{Généralisation des automates : machines de Turing}
  
  \tfExampleBlock[top=-3mm]{Exemple : reconnaissance de $\{a^n b^n c^n \mid n>0\}$}{}
  
  \tf[y=17mm] {
    \begin{smTape}[size=9mm]
      \smCell{$\#$}
      \smCell{\alt<-1>{$a$}{$A$}}  \smHead<1|handout>    \smHeadFrom<9>{1}
      \smCell{\alt<-10>{$a$}{$A$}} \smHeadFrom<2,10>{-1} \smHeadFrom<8>{1}  \smHeadFrom<13>{4}
      \smCell{\alt<-3>{$b$}{$B$}}  \smHeadFrom<3>{-1}    \smHeadFrom<7>{1}
      \smCell{\alt<-11>{$b$}{$B$}} \smHeadFrom<4>{-1}    \smHeadFrom<11>{-2}\smHeadFrom<6>{1}
      \smCell{\alt<-5>{$c$}{$C$}}  \smHeadFrom<5>{-1}  
      \smCell{\alt<-12>{$c$}{$C$}} \smHeadFrom<12>{-2}   \smHeadFrom<15>{1}
      \smCell{$\#$}                \smHeadFrom<14>{-5} 
    \end{smTape}
  }
  
  \tf[bottom=10mm] {
    \begin{tikzpicture}[smAutomaton]
      \smState[\smInitialAbove \smAlert<1,10|handout>] (0) at (2.5, 2.5) {0};
      \smState[                \smAlert<2,3,11>      ] (1) at (2.5, 0.0) {1};
      \smState[                \smAlert<4,5,12>      ] (2) at (0.0, 0.0) {2};
      \smState[                \smAlert<6-9,13>      ] (3) at (0.0, 2.5) {3};
      \smState[                \smAlert<14>          ] (4) at (5.0, 2.5) {4};
      \smState[\smAccepting    \smAlert<15>          ] (5) at (7.5, 2.5) {5};
      
      \smPath[\smAlert<2,11>]   (0) edge             node {\smTMtransR{a}{A}} (1);
      \smPath[\smAlert<4,12>]   (1) edge             node {\smTMtransR{b}{B}} (2);
      \smPath[\smAlert<6,13>]   (2) edge             node {\smTMtransL{c}{C}} (3);
      \smPath[\smAlert<10,14>]  (3) edge             node {\smTMtransR{A}{A}} (0);
      \smPath[\smAlert<14>]     (0) edge             node {\smTMtransR{B}{B}} (4);
      \smPath[\smAlert<15>]     (4) edge             node {\smTMtransL{\#}{\#}} (5);
      \smPath[\smAlert<3,11>]   (1) edge[loop right] node {\smAlign{\smTMtransR{a}{a}\smTMtransR{B}{B}}} (1);
      \smPath[\smAlert<5,12>]   (2) edge[loop left ] node {\smAlign{\smTMtransR{b}{b}\smTMtransR{C}{C}}} (2);
      \smPath[\smAlert<7-9,13>] (3) edge[loop left ] node {\smAlign{\smTMtransL{a}{a}\smTMtransL{b}{b}\smTMtransL{B}{B}\smTMtransL{C}{C}}} (3);
      \smPath[\smAlert<14>]     (4) edge[loop below] node {\smAlign{\smTMtransR{B}{B}\smTMtransR{C}{C}}} (4);
    \end{tikzpicture}
  }

  \tf[text,bottom] {\footnotesize
    \begin{description}
    \item[$0 \xrightarrow{\transD{a}{A}} 1$ :] si on lit $a$ dans l'état $0$, écrire $A$, aller en $1$ et se déplacer à droite\vspace{-2mm}
    \item[$2 \xrightarrow{\transG{c}{C}} 3$ :] si on lit $c$ dans l'état $2$, écrire $C$, aller en $3$ et se déplacer à gauche
    \end{description}
  }

\end{frame}

\endgroup
