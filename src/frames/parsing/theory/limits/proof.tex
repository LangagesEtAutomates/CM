% SPDX-License-Identifier: CC-BY-SA-4.0
% Author: Matthieu Perrin
% Part: 
% Section: 
% Sub-section: 
% Frame: 

\begingroup

\begin{frame}{Démonstration du lemme de pompage}
  
  \tf<-5>[text, top]{
    \example{On veut démontrer :}\\
    $\structure{\forall\Sigma, \forall L\in \textsc{alg}_\Sigma,} \alert{\exists N\in \mathbb{N},} \structure{\forall u\in L, |u| \ge N} \Rightarrow \alert{\exists v, w, x, y, z\in \Sigma^\star},$
    $u = v\cdot w\cdot x \cdot y \cdot z \land w\cdot y \neq \varepsilon \land | w\cdot x \cdot y| \le N \land \structure{\forall i \in \mathbb{N}}, v\cdot w^i \cdot x \cdot y^i \cdot z \in L$
  }
  
  \tf<2-5>[text,y=19mm]{
    \structure{Rappel :} Pour le lemme de l'étoile, $N =$ nombre d'états \alert{fini} dans l'automate.
  }

  \tf<3>[text, y=15mm, anchor=north]{
    \structure{Idée 1 :}
    \begin{itemize}
    \item $L$ est reconnu par une grammaire $\langle \Sigma, \Gamma, S, R \rangle$, avec \alert{$|\Gamma|$ fini}.\\
    \item Soit un arbre de dérivation avec \structure{au moins $|\Gamma|+1$ n\oe uds internes}.\\
    \item Un non-terminal $T$ est utilisé au moins deux fois.
    \end{itemize}
  }
  
  \tf<4,5>[text, y=15mm, anchor=north]{
    \structure{Idée 2 :}
    \begin{itemize}
    \item $L$ est reconnu par une grammaire $\langle \Sigma, \Gamma, S, R \rangle$, avec \alert{$|\Gamma|$ fini}.\\
    \item Soit un arbre de dérivation \structure{de hauteur au moins $|\Gamma|$} (plus 1 feuille).\\
    \item Un non-terminal $T$ est utilisé deux fois \alert{dans la même branche}.
    \end{itemize}
  }

  \tf<3-5>[bottom, x=-.25\textwidth]{
    \begin{tikzpicture}[anchor=center]
      \fill[structure!20]  (10.0,10.00) -- ( 8.0,8.00) -- (12.0,8.00);
      \fill[alert!20] (10.0, 9.33) -- ( 8.8,8.00) -- (11.2,8.00);
      \fill[structure!20]  (10.0, 8.66) -- ( 9.6,8.00) -- (10.4,8.00);

      \draw (10,10) node{$S$};
      \draw (10,9.33) node{$T$};
      \draw (10,8.66) node{$T$};

      \draw[dashed] (10.0,9.8 ) -- (10.0,9.53);
      \draw[dashed] (10.0,9.13) -- (10.0,8.86);
      
      \draw[dashed] ( 9.800,9.80) -- ( 8.0,8);
      \draw[dashed] ( 9.800,9.13) -- ( 8.8,8);
      \draw[dashed] ( 9.875,8.46) -- ( 9.6,8);
      \draw[dashed] (10.125,8.46) -- (10.4,8);
      \draw[dashed] (10.200,9.13) -- (11.2,8);
      \draw[dashed] (10.200,9.80) -- (12.0,8);

      \draw ( 8.5,8.1) node{$v$};
      \draw ( 9.25,8.1) node{$w$};
      \draw (10.0,8.1) node{$x$};
      \draw (10.75,8.1) node{$y$};
      \draw (11.5,8.1) node{$z$};

      \uncover<3>{
        \draw   (10,7.5) node{$\forall i\in \mathbb{N}, S \rightarrow v T z \rightarrow v w^i x y^i z$};
      }
      \uncover<4,5>{
        \draw [decorate, decoration={brace, amplitude=5pt}] (12.2,10) -- (12.2,8) node[midway,xshift=7mm]{$|\Gamma|+1$};
        \draw [decorate, decoration={brace, amplitude=5pt}] (12,7.9) -- (8,7.9) node[midway,yshift=-4mm]{$N = ?$};
      }
    \end{tikzpicture}
  }    

  \tf<3>[bottom, x=.25\textwidth]{
    \begin{tikzpicture}[anchor=center]
      \fill[structure!20]  (10.0,10) -- ( 8.0,8) -- (12.0,8);
      \fill[alert!20] ( 9.5, 9) -- ( 8.8,8) -- ( 9.6,8);
      \fill[alert!20] (10.5, 9) -- (11.2,8) -- (10.4,8);

      \draw (10,10) node{$S$};
      \draw (9.5,9) node{$T$};
      \draw (10.5,9) node{$T$};

      \draw[dashed] ( 9.8,9.8) -- ( 8.0,8);
      \draw[dashed] ( 9.9,9.8) -- ( 9.6,9.2);
      \draw[dashed] (10.1,9.8) -- (10.4,9.2);
      \draw[dashed] (10.2,9.8) -- (12.0,8);

      \draw[dashed] ( 9.37,8.8) -- ( 8.8,8);
      \draw[dashed] ( 9.52,8.8) -- ( 9.6,8);

      \draw[dashed] (10.48,8.8) -- (10.4,8);
      \draw[dashed] (10.63,8.8) -- (11.2,8);

      \draw ( 8.5,8.1) node{$v$};
      \draw ( 9.25,8.1) node{$w$};
      \draw (10.0,8.1) node{$x$};
      \draw (10.75,8.1) node{$y$};
      \draw (11.5,8.1) node{$z$};

      \draw (10,7.5) node{$v y x w z \in L$...};
    \end{tikzpicture}
  }

  \tf<5>[x=.3\textwidth, width=.6\textwidth, bottom=5mm]{
    \begin{itemize}
    \item Supposons $G$ en FN Chomsky.\\
    \item L'arbre de dérivation est binaire.\\
    \item Pour une hauteur $h$, au plus $2^h$ feuilles.\\
    \item Donc si $N > 2^K$, la hauteur est $> K$.\\
    \item Il suffit de prendre $N = 2^{|\Gamma|} +1$.
    \end{itemize}
  }

  \tf<6|handout>[text]{
    \structure{Soient $\Sigma$ un alphabet et $L\in \textsc{alg}_\Sigma$}.

    $L$ est reconnu par une grammaire $\langle \Sigma, \Gamma, S, R \rangle$ en forme quadratique.
    
    \alert{Posons $N = 2^\Gamma + 1$}. \structure{Soit $u \in L$ tel que $|u| \ge N$}.

    L'arbre de dérivation de $u$ est de hauteur $ \ge |\Gamma| + 1$, sans compter les feuilles.

    \begin{center}
      \begin{tikzpicture}
        \fill[structure!20]  (10.0,10.00) -- ( 8.0,8.00) -- (12.0,8.00);
        \fill[alert!20] (10.0, 9.33) -- ( 8.8,8.00) -- (11.2,8.00);
        \fill[structure!20]  (10.0, 8.66) -- ( 9.6,8.00) -- (10.4,8.00);

        \draw (10,10) node{$S$};
        \draw (10,9.33) node{$T$};
        \draw (10,8.66) node{$T$};

        \draw[dashed] (10.0,9.8 ) -- (10.0,9.53);
        \draw[dashed] (10.0,9.13) -- (10.0,8.86);
        
        \draw[dashed] ( 9.800,9.80) -- ( 8.0,8);
        \draw[dashed] ( 9.800,9.13) -- ( 8.8,8);
        \draw[dashed] ( 9.875,8.46) -- ( 9.6,8);
        \draw[dashed] (10.125,8.46) -- (10.4,8);
        \draw[dashed] (10.200,9.13) -- (11.2,8);
        \draw[dashed] (10.200,9.80) -- (12.0,8);

        \draw ( 8.5,8.1) node{$v$};
        \draw ( 9.25,8.1) node{$w$};
        \draw (10.0,8.1) node{$x$};
        \draw (10.75,8.1) node{$y$};
        \draw (11.5,8.1) node{$z$};
      \end{tikzpicture}
    \end{center}
    
    Une branche passe deux fois par $T \in \Gamma$ dans les $|\Gamma| + 1$ derniers étages.

    \alert{Posons $v, w, x, y, z$} tq $u$ généré par $S \rightarrow^\star v T z \rightarrow^\star v w T y z \rightarrow^\star v w x y z$. On a :

    \vspace{1mm}
    \begin{enumerate}
    \item \structure{$u = v\cdot w\cdot x \cdot y \cdot z$ :} par définition de $v, w, x, y, z$
    \item \structure{$w\cdot y \neq \varepsilon$ :} $T \rightarrow^\star \varepsilon T \varepsilon$ est impossible en forme normale de Chomsky
    \item \structure{$| w\cdot x \cdot y| \le N$ :} l'arbre de dérivation $T \rightarrow^\star w x y$ est de hauteur au plus $\Gamma$
    \item \structure{$\forall i \in \mathbb{N}, v\cdot w^i \cdot x \cdot y^i \cdot z \in L$ :} il suffit d'appliquer $i$ fois $T \rightarrow^\star w T y$.
    \end{enumerate}
  }

  
\end{frame}


\endgroup
