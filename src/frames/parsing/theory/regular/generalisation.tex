% SPDX-License-Identifier: CC-BY-SA-4.0
% Author: Matthieu Perrin
% Part: 
% Section: 
% Sub-section: 
% Frame: 

\begingroup

\begin{frame}{Généralisation : degré d'une grammaire}
  
  Soit $G = \langle \Sigma, \Gamma, S, R \rangle$ une grammaire algébrique.\\
  \begin{block}{Définition -- Degré d'une grammaire}
    Le \structure{degré} de $G$, noté $\alert{d(G)}$,
    est le nombre maximal de non-terminaux dans le membre droit d'une règle de $R$. 
  \end{block}

  \begin{exampleblock}{Exemples}
    \begin{enumerate}
    \item Les grammaires de \alert{degré 1} sont appelées \structure{linéaires}
      \begin{description}
      \item[Attention] \alert{$S \rightarrow aSb \mid \varepsilon$} est linéaire, mais ni à droite ni à gauche
      \end{description}
    \item Les grammaires de \alert{degré 2} sont appelées \structure{quadratiques}
      \begin{description}
      \item[Théorème] Toute grammaire algébrique peut être transformée en une grammaire quadratique équivalente 
        %      \item[Idée] Remplacer \structure{$N \rightarrow A B C$} par \structure{$N \rightarrow A X$} et \structure{$X \rightarrow B C$}
      \item[Preuve] La forme normale de Chomsky en est un exemple
      \end{description}
    \end{enumerate}
  \end{exampleblock}
\end{frame}


\endgroup
