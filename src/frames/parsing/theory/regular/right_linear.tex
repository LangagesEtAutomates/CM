% SPDX-License-Identifier: CC-BY-SA-4.0
% Author: Matthieu Perrin
% Part: 
% Section: 
% Sub-section: 
% Frame: 

\begingroup

\begin{frame}{Grammaires linéaires et rationnelles droites}
  Soit \alert{$G = \langle \Sigma, \Gamma, S, R \rangle$} une grammaire algébrique. 

  \vspace{2mm}
  \begin{block}{Définition -- Grammaire linéaire droite}
    On dit que $G$ est \structure{linéaire droite} si tous les membres droits de ses règles
    contiennent au plus un non-terminal, situé à droite de tous les terminaux :

    {\small $$
      \structure{\forall r\in R, \exists a_1, ..., a_n \in \Sigma, \exists A, B \in \Gamma, r = \alert{A \rightarrow a_1 ... a_n B} \lor r = \alert{A \rightarrow a_1 ... a_n}}.
      $$}
  \end{block}
  
  \vspace{2mm}
  \begin{block}{Définition -- Grammaire rationnelle droite}
    On dit que $G$ est \structure{rationnelle droite} si $G$ est linéaire droite et tous les membres droits de ses règles
    contiennent un unique terminal, ou sont $\varepsilon$ :

    {\small $$
      \structure{\forall r\in R, \exists a \in \Sigma, \exists A, B \in \Gamma, r = \alert{A \rightarrow a B} \lor r = \alert{A \rightarrow a} \lor r = \alert{A \rightarrow \varepsilon}}.
      $$}
  \end{block}
\end{frame}

\endgroup
