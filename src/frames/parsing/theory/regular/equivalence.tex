% SPDX-License-Identifier: CC-BY-SA-4.0
% Author: Matthieu Perrin
% Part: 
% Section: 
% Sub-section: 
% Frame: 

\begingroup

\begin{frame}{Équivalence grammaires rationnelles / automates}
  
  \begin{block}{Théorème -- Générations des grammaires rationnelles}
    Soit $\Sigma$ un alphabet et $L \in \mathscr{P}(\Sigma^\star)$ un langage. \\
    $L$ est généré par une grammaire rationnelle droite ssi L est rationnel. 
  \end{block}
  
  \uncover<2->{\structure{Démonstration :}}
  
  \begin{enumerate}
  \item<2-> Supposons que \structure{$L$ rationnel}. $L$ est reconnu par un AFD $A = \langle \Sigma, Q, \{i\}, F, \mu \rangle$. 
 
    Soit $R = \begin{array}[t]{rclcll}
      & \{ & A \rightarrow aB &|& \langle A, a, B\rangle \in \mu & \}\\
      \cup & \{ & A \rightarrow a\phantom{B} &|& \exists f\in F, \langle A, a, f\rangle \in \mu & \}.
    \end{array}$\\
    Alors \alert{$L$ est généré par la grammaire rationnelle droite $\langle \Sigma, Q, i, R \rangle$.}
  \item<3-> Supposons que \structure{$L$ est généré par $G = \langle \Sigma, \Gamma, S, R \rangle$ rationnelle droite.}\\
    Soient $F \notin \Gamma$ et $\mu = \begin{array}[t]{rclcll}
      & \{ & \langle A, a, B\rangle &|& A \rightarrow aB \in R &\}\\
      \cup & \{ & \langle A, a, F\rangle &|& A \rightarrow a\phantom{B} \in R &\}.
    \end{array}$\\
    Alors l'AFN $\langle \Sigma, \Gamma \cup \{F\}, \{S\}, \{F\}, \mu \rangle$ reconnaît $L$, donc \alert{$L$ rationnel}.
  \end{enumerate}

  
 
  \uncover<2->{\example{Exemple :}}
 
  \uncover<2->{%
    \begin{minipage}{.2\textwidth}
      \scalebox{.7}{\begin{tikzpicture}[shorten >=1pt,node distance=1.5cm,on grid,auto]
          \node [state, initial, initial text=] (S)                     {$S$}; 
          \node [state, accepting]              (A) [above right=of S]  {$A$}; 
          \node [state, accepting]              (B) [below right=of S]  {$B$}; 
          \path [->]    (S) edge node {$a$} (A);
          \path [->]    (S) edge node {$b$} (B);
          \path [->]    (A) edge[loop right, looseness=5] node {$a$} (A);
          \path [->]    (B) edge[loop right, looseness=5] node {$b$} (B);
      \end{tikzpicture}}
    \end{minipage}%
    \begin{minipage}{.55\textwidth}
      $$
      \Rightarrow \hspace{3mm}
      \left\{\begin{array}{rcl}
      S & \rightarrow & aA | bB | a | b\\
      A & \rightarrow & aA | a\\
      B & \rightarrow & bB | b\\
      \end{array}\right.
      \hspace{3mm} \uncover<3->{\Rightarrow}
      $$
  \end{minipage}}%
  \uncover<3->{%
    \begin{minipage}{.25\textwidth}
      \scalebox{.7}{\begin{tikzpicture}[shorten >=1pt,node distance=1.5cm,on grid,auto]
          \node [state, initial, initial text=] (S)                     {$S$}; 
          \node [state]                         (A) [above right=of S]  {$A$}; 
          \node [state]                         (B) [below right=of S]  {$B$}; 
          \node [state, accepting]              (F) [below right=of A]  {$F$}; 
          \path [->]    (S) edge node {$a$} (A);
          \path [->]    (S) edge node {$b$} (B);
          \path [->]    (A) edge node {$a$} (F);
          \path [->]    (B) edge node {$b$} (F);
          \path [->]    (S) edge node {$a, b$} (F);
          \path [->]    (A) edge[loop right, looseness=5] node {$a$} (A);
          \path [->]    (B) edge[loop right, looseness=5] node {$b$} (B);
      \end{tikzpicture}}
  \end{minipage}}

\end{frame}
\endgroup
