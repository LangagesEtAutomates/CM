% SPDX-License-Identifier: CC-BY-SA-4.0
% Author: Matthieu Perrin
% Part: 
% Section: 
% Sub-section: 
% Frame: 

\begingroup

\begin{frame}{Grammaires linéaires et rationnelles}

  Soit \alert{$G = \langle \Sigma, \Gamma, S, \rightarrow \rangle$} une grammaire linéaire. 

  \begin{block}{Définition -- Grammaires linéaires à droite ou à gauche}
    \begin{itemize}
    \item $G$ est dite \structure{linéaire à droite} si ses règles sont de la forme :
      \begin{center}
        $\alert{A \rightarrow u \cdot B}$ \quad ou\quad $\alert{A \rightarrow u}$ \quad avec \structure{$A, B \in \Gamma$} et \structure{$u\in \Sigma^\star$}
      \end{center}
    \item $G$ est dite \structure{linéaire à gauche} si ses règles sont de la forme :
      \begin{center}
        $\alert{A \rightarrow B \cdot u}$ \quad ou\quad $\alert{A \rightarrow u}$ \quad avec \structure{$A, B \in \Gamma$} et \structure{$u\in \Sigma^\star$}
      \end{center}
    \end{itemize}
  \end{block}

  \begin{block}{Cas particulier -- Grammaire rationnelles}
    \begin{itemize}
    \item $G$ est dite \structure{rationnelle à droite} si ses règles sont de la forme :
      \begin{center}
        $\alert{A \rightarrow a \cdot B}$ \quad ou\quad $\alert{A \rightarrow a}$ \quad ou\quad $\alert{A \rightarrow \varepsilon}$
        \quad avec \structure{$A, B \in \Gamma$} et \structure{$a\in \Sigma$}
      \end{center}
    \item $G$ est dite \structure{rationnelle à gauche} si ses règles sont de la forme :
      \begin{center}
        $\alert{A \rightarrow B \cdot a}$ \quad ou\quad $\alert{A \rightarrow a}$ \quad ou\quad $\alert{A \rightarrow \varepsilon}$
        \quad avec \structure{$A, B \in \Gamma$} et \structure{$a\in \Sigma$}
      \end{center}
    \item $G$ est dite \structure{rationnelle} si elle est \structure{rationnelle à droite ou à gauche}
    \end{itemize}
  \end{block}
  
\end{frame}

\endgroup
