% SPDX-License-Identifier: CC-BY-SA-4.0
% Author: Matthieu Perrin
% Part: 
% Section: 
% Sub-section: 
% Frame: 

\begingroup

\begin{frame}{Degré d'une grammaire}
  
  Soit $G = \langle \Sigma, \Gamma, S, \rightarrow \rangle$ une grammaire algébrique\\
  \begin{block}{Définition -- Degré d'une grammaire}
    \begin{itemize}
    \item Le \structure{degré} de $G$, noté $\alert{d(G)}$,
      est le nombre maximal de non-terminaux dans le membre droit d'une règle de $\rightarrow$
    \item $G$ est \structure{linéaire} si son degré est au plus 1
      \begin{center}
        \example{Exemple : $S \rightarrow aSb \mid \varepsilon$}
      \end{center}
    \item $G$ est \structure{quadratique} si son degré est au plus 2
      \begin{center}
        \example{Exemple : $S \rightarrow SS \mid (S) \mid \varepsilon$}
      \end{center}
    \end{itemize}
  \end{block}

  \begin{block}{Théorème -- Grammaires quadratique}
    Tout langage algébrique est engendré par une grammaire quadratique

    \vspace{3mm}\structure{Démonstration : } \\Toute grammaire algébrique peut être mise sous Forme Normale de Chomsky
  \end{block}

\end{frame}

\endgroup
