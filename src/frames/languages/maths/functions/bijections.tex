% SPDX-License-Identifier: CC-BY-SA-4.0
% Author: Matthieu Perrin
% Part: 
% Section: 
% Sub-section: 
% Frame: 

\begingroup


\begin{frame}{Bijection}
  Soient $E$ et $F$ deux ensembles, et $f\in \mathscr{F}(E, F)$ une fonction de $E$ dans $F$.
  \begin{block}{Définition -- Bijection}
    \begin{itemize}
    \item $f$ est une \structure{bijection} si $\alert{\forall y\in F, \existsunique x\in E, \langle x, y \rangle \in f}$. 
    \item Si $f$ est une bijection, on note $f^{-1} = \{\langle y, x \rangle | \langle x, y \rangle \in f\}$. 
    \item Si $f$ est une bijection, alors $f^{-1}$ est une bijection de $F$ dans $E$. 
    \end{itemize}
  \end{block}
  
  \begin{exampleblock}{Exemple}
    Les deux fonctions suivantes sont des bijections, et $f = g^{-1}$.
    $$
    f \eqdef \left\{\begin{array}{rcl}
    \mathbb{Z} &\rightarrow& \mathbb{Z}\\
    n &\mapsto & n+1
    \end{array}\right.
    \hspace{1cm}
    g \eqdef \left\{\begin{array}{rcl}
    \mathbb{Z} &\rightarrow& \mathbb{Z}\\
    n &\mapsto & n-1
    \end{array}\right.
    $$
  \end{exampleblock}

\end{frame}

\endgroup
