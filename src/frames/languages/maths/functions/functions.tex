% SPDX-License-Identifier: CC-BY-SA-4.0
% Author: Matthieu Perrin
% Part: 
% Section: 
% Sub-section: 
% Frame: 

\begingroup

\begin{frame}{Fonction}
  Soient $E$ et $F$ deux ensembles.

  \begin{block}{Définition -- Fonction}
    \begin{itemize}
    \item Une \structure{fonction de $E$ dans $F$} est une relation $f \in \mathscr{P}(E \times F)$ qui vérifie la propriété suivante :
      $\alert{\forall x\in E, \existsunique y\in F, \langle x, y \rangle \in f}$
    \item L'\structure{ensemble des fonctions} de $E$ dans $F$ est noté $\alert{\mathscr{F}(E, F)}$.

      $$\alert{\mathscr{F}(E, F) \eqdef \{f\in \mathscr{P}(E \times F) | \forall x\in E, \existsunique y\in F, \langle x, y \rangle \in f\}}$$

    \item Pour $f\in \mathscr{F}(E, F)$ et $x\in E$,
      on note $\alert{f(x)}$ \structure{l'unique $y$ tel que $\langle x, y \rangle \in f$}.

    \item Pour $f\in \mathscr{F}(E, F)$ et $E' \in \mathscr{P}(E)$, on note
      
      $$\alert{f(E') \eqdef \{f(x) | x\in E'\} \eqdef \{y\in F | \exists x\in E', y = f(x)\}}$$

    \item \structure{Définition par une expression :} si on a une expression $\mathit{expr}[x]$ telle que, 
      $\forall x\in E, \mathit{expr}[x] \in F$, on peut définir une fonction : 

      $$\alert{\left\{\begin{array}{rcl}
      E &\rightarrow& F \\
      x &\mapsto& \mathit{expr}[x] \\
      \end{array}\right.  \eqdef \{c \in E\times F | \exists x\in E, c = \langle x, \mathit{expr}[x] \rangle\}}$$
    \end{itemize}
  \end{block}
  
\end{frame}

\endgroup
