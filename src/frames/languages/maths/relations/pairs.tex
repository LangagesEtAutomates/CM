% SPDX-License-Identifier: CC-BY-SA-4.0
% Author: Matthieu Perrin
% Part: 
% Section: 
% Sub-section: 
% Frame: 

\begingroup

\begin{frame}{Couple et produit cartésien}
  \begin{block}{Définition}
    \begin{itemize}
    \item Un \structure{couple} $\langle a, b \rangle$ est une \alert{collection ordonnée de deux éléments} $a$ et $b$
    \item Deux couples sont égaux si leurs éléments sont égaux deux à deux 
      \vspace{-2mm}
      $$\forall a \forall b \forall c \forall d, \alert{\langle a, b \rangle =  \langle c, d \rangle \Leftrightarrow a = c \land b = d}$$
    \item Généralisation : un \structure{$n$-uplet} est une \alert{collection ordonnée de $n$ éléments}
    \item Le \structure{produit cartésien} de deux ensembles $A$ et $B$, noté \structure{$A\times B$}, est \alert{l'ensemble des couples formé d'un élément de $A$ et d'un élément de $B$}
      \vspace{-2mm}
      $$\forall A \forall B \forall a \forall b, \alert{\langle a, b \rangle \in A \times B \Leftrightarrow a \in A \land b \in B}$$
    \end{itemize}
  \end{block}
  \vspace{-2mm}
  \begin{exampleblock}{Exemple}
    Soient $A = \{1, 2, 3\}$ et $B = \{5, 6\}$. On a
    \vspace{-2mm}
    $$A\times B = \{\langle 1, 5 \rangle, \langle 1, 6 \rangle, \langle 2, 5 \rangle, \langle 2, 6 \rangle, \langle 3, 5 \rangle, \langle 3, 6 \rangle\}$$ 
  \end{exampleblock}
\end{frame}
\endgroup
