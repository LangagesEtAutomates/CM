% SPDX-License-Identifier: CC-BY-SA-4.0
% Author: Matthieu Perrin
% Part: 
% Section: 
% Sub-section: 
% Frame: 

\begingroup

\begin{frame}{Relation d'équivalence}
  \small 

  Soit $\bowtie$ une relation binaire homogène sur un ensemble $E$.
  \begin{block}{Définition -- Relation d'équivalence}
    $\bowtie$ est une \structure{relation d'équivalence} si elle est \alert{réflexive}, \alert{transitive} et \alert{symétrique}.
  \end{block}

  \begin{block}{Définition -- Classe d'équivalence}
    La \structure{classe d'équivalence} d'un élément $x \in E$, notée $\alert{[x]_{\bowtie}}$ est l'ensemble des éléments de $E$ en relation avec $x$ :
    $$\alert{[x]_{\bowtie} \eqdef \{y\in E | y \bowtie x\}}$$
  \end{block}

  \begin{block}{Définition -- Ensemble quotient}
    L'\structure{ensemble quotient} de $E$ par $\bowtie$, noté $\alert{E/_{\bowtie}}$, est l'ensemble des classes d'équivalence des éléments de $E$.

    $$\alert{ E/_{\bowtie} \eqdef \bigcup_{x\in E}\left\{[x]_{\bowtie}\right\}}$$
  \end{block}
\end{frame}

\endgroup
