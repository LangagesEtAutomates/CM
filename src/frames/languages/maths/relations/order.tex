% SPDX-License-Identifier: CC-BY-SA-4.0
% Author: Matthieu Perrin
% Part: 
% Section: 
% Sub-section: 
% Frame: 

\begingroup

\begin{frame}{Relation d'ordre}
  Soit $\bowtie$ une relation binaire homogène sur un ensemble $E$.

  \begin{block}{Définition -- Relation d'ordre}
    $\bowtie$ est une \structure{relation d'ordre} si elle est \alert{réflexive}, \alert{transitive} et \alert{antisymétrique}.
  \end{block}
  \begin{exampleblock}{Exemple}
    Soit $E$ un ensemble. 
    Montrons que $\subseteq$ est une relation d'ordre sur $\mathscr{P}(E)$.
    \begin{description}
    \item[Réflexivité]  Montrons que \alert{$\forall A\in \mathscr{P}(E), A\subseteq A$}
      \begin{itemize}
      \item Soit \structure{$A \in \mathscr{P}(E)$}. Montrons que $\alert{A\subseteq A}$.
      \item Soit $x\in A$. On a $x\in A$.
      \end{itemize}
    \item[Transitivité] Montrons que \alert{$\forall A, B, C \in \mathscr{P}(E), A \subseteq B \land B \subseteq C \Rightarrow A \subseteq C$}
      \begin{itemize}
      \item Soient $\structure{A, B, C \in \mathscr{P}(E)}$, tels que $\structure{A \subseteq B}$ et $\structure{B \subseteq C}$. 
      \item Soit $x \in A$. Donc $x\in B$, donc $x\in C$. 
      \end{itemize}
    \item[Antisymétrie] Montrons que \alert{$\forall A,B \in \mathscr{P}(E), A \subseteq B \land B \subseteq A \Rightarrow A=B$}
      \begin{itemize}
      \item Soient $\structure{A, B \in \mathscr{P}(E)}$, tels que $\structure{A \subseteq B}$ et $\structure{B \subseteq A}$. 
      \item Par double inclusion, on a $A=B$. 
      \end{itemize}
    \end{description}
  \end{exampleblock}
\end{frame}
\endgroup
