% SPDX-License-Identifier: CC-BY-SA-4.0
% Author: Matthieu Perrin
% Part: 
% Section: 
% Sub-section: 
% Frame: 

\begingroup

\begin{frame}{Récurrence forte}
  \small

  \structure{Principe de la récurrence forte.}\\
  Soit $P$ un prédicat s'appliquant sur tous les nombres entiers. On a : 

  $$\structure{(\alert{\forall k\in \mathbb{N}, (\forall i < k,  P(i)) \Rightarrow P(k)}) \Rightarrow \alert{\forall n\in \mathbb{N}, P(n)}}$$

  \pause
  \vspace{3mm}
  \example{Exemple :} \example{$\displaystyle P(n) \eqdef n\ge 2 \Rightarrow n$ est un produit de nombres premiers}.\\

  \vspace{2mm}
  \example{Récurrence :} Montrons \alert{$\alert{\forall k\in \mathbb{N}, (\forall i < k,  P(i)) \Rightarrow P(k)}$}.\\
  \uncover<3->{
    Soit \structure{$k\in \mathbb{N}$}. Supposons \structure{$\forall i < k,  P(i)$}. Montrons \alert{$P(k)$}.
    \begin{itemize}
    \item[\example{\myRec}] Cas 1 : $k$ est premier. (\example{Couvre le cas $k=2$}.)
      \begin{itemize}
      \item[\example{\myRec}] Donc $k = k$, donc $P(k)$.
      \end{itemize}
    \item[\example{\myRec}] Cas 2 : $k$ n'est pas premier. 
      \begin{itemize}
      \item[\example{\myRec}] Il existe $a, b < k$ tels que $k=ab$. 
      \item[\example{\myRec}] Par l'hypothèse de récurrence,  $a = a_1 a_2 ... a_{x}$ et $b = b_1 b_2 ... b_{y}$.
      \item[\example{\myRec}] Donc $k = a b = a_1 a_2 ... a_{x} b_1 b_2 ... b_{y}$, donc $P(k)$.
      \end{itemize}
    \end{itemize}
  }
  \example{Conclusion :} Donc par récurrence forte, on a bien $\alert{\forall n\in \mathbb{N}, P(n)}$
\end{frame}

\endgroup
