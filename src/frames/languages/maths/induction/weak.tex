% SPDX-License-Identifier: CC-BY-SA-4.0
% Author: Matthieu Perrin
% Part: 
% Section: 
% Sub-section: 
% Frame: 

\begingroup

\begin{frame}{Récurrence simple}
  \small

  \structure{Principe de la récurrence simple.}\\
  Soit $P$ un prédicat s'appliquant sur tous les nombres entiers. On a : 

  $$\structure{(\alert{P(0)} \land \alert{\forall k\in \mathbb{N}, P(k) \Rightarrow P(k+1)}) \Rightarrow \alert{\forall n\in \mathbb{N}, P(n)}}$$

  \pause
  \vspace{3mm}
  \example{Exemple :} 
  considérons le prédicat $\example{\displaystyle P(n) \eqdef \sum_{i = 1}^n i = \frac{n(n+1)}{2}}$.\\

  \begin{description}
  \item[\example{Initialisation :}] Montrons \alert{$P(0)$}.
    \uncover<3->{
      \begin{itemize}
      \item[\example{\myRec}] On a bien $\sum_{i = 1}^0 i = 0 = \frac{0 (0+1)}{2}$.
      \end{itemize}
    }
  \item[\example{Hérédité :}] Montrons \alert{$\forall k\in \mathbb{N}, P(k) \Rightarrow P(k+1)$}.
    \uncover<4->{
      \begin{itemize}
      \item[\example{\myRec}] Soit \structure{$k\in \mathbb{N}$}. Supposons \structure{$P(k)$}. Montrons \alert{$P(k+1)$}.
      \item[\example{\myRec}] $\begin{array}[t]{rcll}
        \sum_{i = 1}^{k+1} i & = & { \sum_{i = 1}^{k} i} + (k+1) & \text{\footnotesize Isolement de $k+1$}\\
        & = &\frac{k(k+1)}{2} + \frac{2(k+1)}{2}  & \text{\footnotesize hypothèse de récurrence}\\
        & = &\frac{(k+1)(k+2)}{2}  & \text{\footnotesize factorisation de $\frac{k+1}{2}$}\\
      \end{array}$
      \item[\example{\myRec}] Donc $P(k+1)$, donc $\forall k\in \mathbb{N}, P(k) \Rightarrow P(k+1)$.
      \end{itemize}
    }
  \item[\example{Conclusion :}] Donc par récurrence simple, on a bien $\alert{\forall n\in \mathbb{N}, P(n)}$
  \end{description}
\end{frame}

\endgroup
