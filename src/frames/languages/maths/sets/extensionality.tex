% SPDX-License-Identifier: CC-BY-SA-4.0
% Author: Matthieu Perrin
% Part: 
% Section: 
% Sub-section: 
% Frame: 

\begingroup


\begin{frame}{Extensionalité}
  \begin{block}{Définition -- égalité entre ensembles}
    Deux ensembles sont \structure{égaux} s'ils contiennent les mêmes éléments : 
    $$\alert{\forall A \forall B, (A=B) \Leftrightarrow (\forall x, x \in A \Leftrightarrow x\in B)}$$
  \end{block}

  \vspace{-3mm}
  \begin{exampleblock}{\structure{Preuve par double inclusion}}
    Pour montrer que \alert{$A=B$}, il suffit de montrer \alert{$A \subseteq B$ et $B\subseteq A$}

    \pause
    \begin{itemize}
    \item \example{Exemple :} montrons que \example{$\{1, 2, 3\} = \{3, 3, 1, 2\}$}
      \begin{enumerate}
      \item \example{Mq. $\{1, 2, 3\} \subseteq \{3, 3, 1, 2\}$}, c'est-à-dire \structure{$\forall x\in \{1, 2, 3\}, x \in \{3, 3, 1, 2\}$} \\
        Soit $x \in \{1, 2, 3\}$. %Mq. $x \in \{3, 3, 1, 2\}$ \\%(soit $x$, supposons $x\in \{1, 2, 3\}$)\\
        Par définition de $\{1, 2, 3\}$, $x=1 \lor x=2 \lor x=3$\\
        Il y a 3 cas possibles : 
        \begin{enumerate}
        \item Si $x=1$, alors $x=3 \lor x=3 \lor x=1 \lor x=2$, donc $x\in \{3, 3, 1, 2\}$
        \item Si $x=2$, alors $x=3 \lor x=3 \lor x=1 \lor x=2$, donc $x\in \{3, 3, 1, 2\}$
        \item Si $x=3$, alors $x=3 \lor x=3 \lor x=1 \lor x=2$, donc $x\in \{3, 3, 1, 2\}$
        \end{enumerate}
        Dans tous les cas, on a bien $x\in \{3, 3, 1, 2\}$
      \item \example{Mq. $\{3, 3, 1, 2\} \subseteq \{1, 2, 3\}$} : preuve similaire
      \end{enumerate}
      Donc, par double inclusion, on a bien $\{1, 2, 3\} = \{3, 3, 1, 2\}$.
    \end{itemize}
  \end{exampleblock}
\end{frame}
\endgroup
