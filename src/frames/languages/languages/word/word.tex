% SPDX-License-Identifier: CC-BY-SA-4.0
% Author: Matthieu Perrin
% Part: 
% Section: 
% Sub-section: 
% Frame: 

\begingroup

\begin{frame}{Mot}

  Soit $\Sigma$ un alphabet et $n\in \mathbb{N}$.
  \begin{itemize}
  \item Un \structure{mot} de longueur $n$ sur $\Sigma$ est une \alert{fonction de $\llbracket n\rrbracket$ dans $\Sigma$}.
  \item L'ensemble des mots de longueur $n$ sur $\Sigma$ est \alert{$\Sigma^n = \mathscr{F}(\llbracket n\rrbracket, \Sigma)$}.
  \item L'ensemble des mots sur $\Sigma$, noté \structure{$\Sigma^\star$}, est :
    $$\alert{\displaystyle \Sigma^\star \eqdef \bigcup_{n\in \mathbb{N}} \Sigma^n = \Sigma^0 \cup \Sigma^1 \cup \Sigma^2 \cup ...}$$
  \item L'ensemble des mots \structure{non-vides} sur $\Sigma$ est
    $\alert{\Sigma^+ \eqdef \bigcup_{n \neq 0 } \Sigma^n}$
  \end{itemize}

  \begin{exampleblock}{Exemples}
    Soit $\Sigma = \{a, b, ..., z\}$

    \begin{minipage}[t]{.5\textwidth}
      \begin{itemize}
      \item $\varepsilon = \left\{\begin{array}{ccc}
        \llbracket 0\rrbracket & \rightarrow & \Sigma\\
        x & \mapsto & \_
      \end{array}\right. \in \Sigma^0$
      \item $\text{``a''} = \left\{\begin{array}{ccc}
        \llbracket 1\rrbracket & \rightarrow & \Sigma\\
        x & \mapsto & \text{`a'}
      \end{array}\right. \in \Sigma^1$
      \end{itemize}
    \end{minipage}%
    \begin{minipage}[t]{.5\textwidth}
      \begin{itemize}
      \item $\text{``bon''} = \left\{\begin{array}{ccc}
        \llbracket 3\rrbracket& \rightarrow & \Sigma\\
        1 & \mapsto & \text{`b'}\\
        2 & \mapsto & \text{`o'}\\
        3 & \mapsto & \text{`n'}\\
      \end{array}\right. \in \Sigma^3$
      \end{itemize}
    \end{minipage}%
    \begin{itemize}
    \item Par abus de notation, on fera l'amalgame entre $\Sigma$ et $\Sigma^1$. 
    \end{itemize}
    
  \end{exampleblock}
\end{frame}

\endgroup
