% SPDX-License-Identifier: CC-BY-SA-4.0
% Author: Matthieu Perrin
% Part: 
% Section: 
% Sub-section: 
% Frame: 

\begingroup


\begin{frame}{Objet d'étude}
  On s'intéresse aux langages \alert{formels} :
  \begin{itemize}
  \item modélisation mathématique (abstraite) de la notion de langage
  \end{itemize}
  
  \begin{block}{Définition -- Alphabet et symbole}
    Un \structure{alphabet} est un \alert{ensemble fini non vide}\\
    Ses éléments sont appelés \structure{symboles}
  \end{block}
  \begin{block}{Définition -- Mot}
    Un \structure{mot} sur un alphabet $\Sigma$ est une \alert{suite finie de symboles} de $\Sigma$
  \end{block}
  \begin{block}{Définition -- Langage}
    Un \structure{langage} sur un alphabet $\Sigma$ est un \alert{ensemble de mots} sur $\Sigma$
  \end{block}
  \begin{alertblock}{Remarque}
    La \alert{théorie des langages formels} s'ancre dans la \alert{théorie des ensembles}
  \end{alertblock}
\end{frame}

\endgroup
