% SPDX-License-Identifier: CC-BY-SA-4.0
% Author: Matthieu Perrin
% Part: 
% Section: 
% Sub-section: 
% Frame: 

\begingroup

\begin{frame}{Exemple : test de primalité}
  \begin{block}{Représentation décimale des nombres}
    Soit $\Sigma = \{\text{`0'}, \text{`1'}, \text{`2'}, ..., \text{`9'}\}$ l'ensemble des chiffres.
    
    Pour un entier $n \in \mathbb{N}$, on note \structure{$(n)_{10}$} sa représentation décimale. 
    
    \begin{description}
    \item[\alert{Attention :}] `3' (chiffre) $\neq$ ``3'' (représentation décimale) $\neq$ 3 (nombre)
    \end{description}
    
    \begin{itemize}
    \item $\mathcal{L}_{\mathbb{N}} = \{(n)_{10} | n \in \mathbb{N}\} = \{\text{``0''},\text{``1''},\text{``2''},\text{``3''}, \text{``4''}, \text{``5''}, ...\}$
    \item<2-> $\mathcal{L}_{\mathbb{P}} = \{(n)_{10} | n \in \mathbb{N} \land n \text{ est premier}\} = \{\text{``2''},\text{``3''},\text{``5''}, \text{``7''}, \text{``11''}, ...\}$
    \end{itemize}
  \end{block}
  
  \vspace{3mm} 
  
  \begin{exampleblock}{Exercice}
    \begin{description}
    \item[Question :] $\text{``1547''} \in \mathcal{L}_{\mathbb{N}}$ ?
    \item<2->[Question :] $\text{``1547''} \in \mathcal{L}_{\mathbb{P}}$ ?
    \item<3->[Indication :] $1547 = 7 \times 13 \times 17$ 
    \end{description}
  \end{exampleblock}
  
  \uncover<3->{
    \centering
    \alert{Pour savoir si $(n)_{10}\in \mathcal{L}_{\mathbb{P}}$, il faut et il suffit de savoir si $n$ est premier }
  }  
\end{frame}

\endgroup
