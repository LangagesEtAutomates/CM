% SPDX-License-Identifier: CC-BY-SA-4.0
% Author: Matthieu Perrin
% Part: 
% Section: 
% Sub-section: 
% Frame: 

\begingroup


\begin{frame}{Exemple : représentation d'un nombre réel}
  \begin{block}{Langage représentant le calcul des décimales de $\pi$}
    On étend la notation \structure{$(n)_{10}$} aux nombres $n$ décimaux. 
    \begin{itemize}
    \item Pour $x\in \mathbb{R}$, on note $\alert{\mathcal{L}_{x}}$ le langage des troncatures de $x$
      $$\alert{\mathcal{L}_{x} \eqdef \left\{\left(\frac{\left\lfloor 10^n x \right\rfloor}{10^n} \right)_{10} \Big| n \in \mathbb{N}\right\}}$$
    \item Par exemple, $\example{\mathcal{L}_{\pi} = \left\{\text{``3''}, \text{``3.1''}, \text{``3.14''}, \text{``3.141''}, \text{``3.1415''}, ...   \right\}}$
    \end{itemize}
  \end{block}
  \pause
  \vspace{-2mm}
  \begin{alertblock}{Remarques}
    \vspace{-1mm}
    \begin{itemize}
    \item L'ensemble des langages est indénombrable
      \begin{itemize}
      \item Au moins autant de langages $\alert{\mathcal{L}_{x}}$ que de réel $\alert{x}$
      \end{itemize}
    \item Tout langage est dénombrable
      \begin{itemize}
      \item Par exemple, le langage des mots sur $\{1, 2\}$ est inclus dans $\alert{\mathcal{L}_{\mathbb{N}}}$
      \end{itemize}
    \item Aucun langage ne peut exprimer tous les langages 
      \begin{itemize}
      \item Même les langages de programmation $\rightarrow$ \alert{problèmes indécidables}
      \item Même le langage des mathématiques ~$\rightarrow$ \alert{problèmes indéfinissables}
      \end{itemize}
    \end{itemize}
  \end{alertblock}
\end{frame}

\endgroup
