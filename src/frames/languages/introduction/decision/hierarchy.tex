% SPDX-License-Identifier: CC-BY-SA-4.0
% Author: Matthieu Perrin
% Part: 
% Section: 
% Sub-section: 
% Frame: 

\begingroup


\begin{frame}{Hiérarchie des langages}
  \begin{block}{Formalismes de description d'un langage}
    \begin{itemize}
    \item Comment représenter un langage \structure{infini} de façon \structure{finie} ?
    \item Il doit avoir une \alert{structure} interne
    \item La ``\structure{complexité}'' de sa structure entraine la ``\structure{difficulté}'' à le décider
    \end{itemize}
  \end{block}
  
  \vspace{5mm}
  \centering\scalebox{.6}{\begin{tikzpicture}
      \draw[structure, fill=structure!20] (3.0,1.50) ellipse(7.0cm and 3.5cm);
      \draw[structure, fill=structure!10] (2.5,1.25) ellipse(6.0cm and 3.0cm);
      \draw[structure, fill=structure!20] (2.0,1.00) ellipse(5.0cm and 2.5cm);
      \draw[structure, fill=structure!10] (1.5,0.75) ellipse(4.0cm and 2.0cm);
      \draw[structure, fill=structure!20] (1.0,0.50) ellipse(3.0cm and 1.5cm);
      \draw[structure, fill=structure!10] (0.5,0.25) ellipse(2.0cm and 1.0cm);
      \draw[structure, fill=structure!20] (0.0,0.00) ellipse(1.0cm and 0.5cm);
      
      \draw (0.0,0.0) node{fini};
      \draw (0.5,0.75) node{rationnel};
      \draw (1.0,1.5) node{algébrique};
      \draw (1.5,2.25) node{contextuel};
      \draw (2.0,3.0) node{decidable};
      \draw (2.5,3.75) node{récursivement énumérable};
      \draw (3.0,4.5) node{définissable};
  \end{tikzpicture}}
\end{frame}


\endgroup
