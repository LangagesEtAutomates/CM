% SPDX-License-Identifier: CC-BY-SA-4.0
% Author: Matthieu Perrin
% Part: 
% Section: 
% Sub-section: 
% Frame: 

\begingroup

\begin{frame}{Problème binaire et problème de décision}
  
  Un \structure{problème binaire} est une question dont la question est \alert{oui} ou \alert{non} en fonction de ses \structure{entrées}.
  
  \begin{exampleblock}{Exemples}
    \vspace{-2mm}
    \begin{itemize}
    \item \example{Primalité :} un entier est-il premier ?  
    \item \example{Classification :} une image donnée contient-elle un chat ? 
    \item \example{Base de données :} une requête donnée retourne-t-elle des résultats ?  
    \item \example{Sudoku :} une grille donnée peut-elle être remplie ? 
    \item \example{SAT :} une formule en forme normale conjonctive est-elle satisfiable ? 
    \item \example{Preuve :} étant donné un énoncé de théorème, peut-on le démontrer ? 
    \item \example{Décision :} étant donnés un mot $u$ et un langage $L$, a-t-on $u\in L$ ? 
    \end{itemize}
  \end{exampleblock}
  \pause
  \begin{alertblock}{Observations}
    \vspace{-2mm}
    \begin{itemize}
    \item En informatique, les entrées sont toujours des suites de 0 et de 1
    \item \alert{Les langages sont une métaphore des problèmes binaires}
    \item On peut généraliser aux problèmes qui ne sont pas binaires
    \end{itemize}
  \end{alertblock}
  
\end{frame}

\endgroup
